\begin{table}[!htb]
	\caption{Highlights of visual inspection}
	\label{ch04_visualinspection01}
%	\resizebox{\columnwidth}{!}{%
	{\scriptsize

\begin{tabular}{c|p{3cm}|p{9.5cm}}
	\hline
	No. & Items & Remarks \\ 
	\hline
	
%	THIS IS THE SAMPLE FROM VILLAMOR (FOR REFERENCE)
%	1 & Modified pipe design to fit space limitations & Limitation in space prompted compromise in pipe design and not did not follow recommended piping arrangements and instead may promote turbulent flow profiles at pump suction and discharge that can accelerate pump and fittings wear \\ 
%	2 & Type of pipe and fitting support & Anchor type piping support makes pipe system rigid. \\ 
%	3 & Paint and protective coating of outdoor pipe & Portion of pipe extending outside of pump house and are experiencing surface deterioration. Protective coating and paint falling off exposing some bare metal which are already display corrosion. \\ 
%	4 & Local corrosion on indoor pipes & Some indoor pipes have torn paint films and exposed bare metals which display corrosion \\ 
%	5 & Valve Leaks & Minor leaks that still contributes to deterioration of parts due to corrosion \\ 
%	6 & Pressure gauges & Discrepant measurements between dial and digital gauges up to 10 psi \\ 
%	&  & Deterioration of information tags making them unreadable \\ 
%	&  & Superimposed tags which make it obscure \\ 
%	7 & Motion actuated lighting & Interview with the maintenance team revealed that the motion actuated lighting sometimes causes slight nausea due to dim lighting when repair. The minute motions of repair are sometimes not enough to actuate the lights and thus interrupt the work \\ 




	\hline
\end{tabular}


\end{table}