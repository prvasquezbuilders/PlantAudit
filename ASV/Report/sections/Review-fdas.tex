\subsection{Fire protection and safety (FDAS) audit}
Audit on FDAS has been conducted following sequences

\paragraph{Step 1: Assign (1) one person on the Fire Alarm Control Panel  to operate / accept  the fire alarm activation and another group/person to conduct spraying on the device, communicate using two way radio.}

\paragraph{Step 2: Conduct spray of smoke detector tester (SOLO brand or any) directly on the smoke detector device for not more than 1 sec, repeat action until detector is activated. Note : If detector fails to respond after 3 tries,  device will declared faulty (Figure \ref{ch02_fdas}-a).}

\paragraph{Step 3: Hear and visually check strobe light and sounder every time you activated the smoke detectors.}

\paragraph{Step 4: Remove device and clean, allow particles to disperse. Then return to socket (Figure \ref{ch02_fdas}-b).}

\paragraph{Step 5: Check that strobe light is functioning/ blinking after returning. Note original status if no light is visible. Check that the control panel breaker feeding the device is reset}

\paragraph{Step 6: Repeat steps 2, 3, and 4 on different locations until all the devices are tested.}

\paragraph{Step 7: Conduct testing for manual call point /manual pull station by pressing the device, hear if the alarm bell / buzzer is activate after you trigger the device}

\paragraph{Step 8: Check bells and buzzer audibility.}

\paragraph{Step 9: Return Manual Call Point /Manual Pull Station on stand by position. Repeat it on all device.}

\paragraph{Step 10: Make a record for the fault device.}

\paragraph{Step 11: Record the status of FACP and reset the panel until the fault clear on trouble.}

\paragraph{Step 12: Conduct closing of activities to all concerned .}
