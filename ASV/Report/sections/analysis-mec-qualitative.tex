% QUALITATIVE
%\begin{document}
\section{Qualitative and Operational Analysis}
\label{42}
\subsection{Facts and Data}

Summary of facts and data concerning operational and overall plan reliability is presented in this subsection.

\subsubsection{Operation Scenario}

\begin{itemize}
	\item Pump station is already 17 years in operation
	
	\item 2 pumps (10hp) for Southvale but only 1 in operation at any one time. Pumps switch every 24 hours.
	
	\item 2 pumps (15hp) for Sonera but only 1 pump in operation at any one time. 
	\begin{itemize}
		\item [$\circ$] P1 operates from 6PM to 6AM.
		\item [$\circ$] P2 operates from 6AM to 6PM.
		\item [$\circ$] P1 is used during low demand period because it cannot maintain pressure during high demand period.
	\end{itemize}
	\item Maintain 50psi 24 hours for all distribution lines.
\end{itemize}


\subsubsection{Spares Policy}

\begin{itemize}
	\item Since only 1 pump is used, the other acts as spare.
\end{itemize}

\subsubsection{Emergency Situation (loss of electrical power from Meralco)}
\begin{itemize}
	\item Auto-start of pumps
\end{itemize}

\subsubsection{Maintenance}
\begin{itemize}
	\item For operational problems, operator will call Control Center to report problem. 
	\item Control Center to send contractor within 1 to 2 hours.
	\item Maintenance contractors conduct a weekly visit to do some maintenance activities.
\end{itemize}

\subsubsection{Current (Reliability) Problems}
\begin{itemize}
	\item None.
\end{itemize}

\subsection{Way forward/Recommendations}
In order to ensure the PS to provide adequate level of services around the clock, it is important to establish a good operational scheme that allows optimization of use of pumps to reduce breakdown and to conserve energy. A summary of the observations and their corresponding recommendations to be considered are:

\begin{itemize}
	\item There is enough flexibility in the system to allow for the smaller pumps to fail.  The larger spare 15hp pump can supply the requirements. However, if P2 of the 15hp pumps fails, the system cannot maintain the 50psi requirements and there will be a loss of water to some customers. First, overhaul Sonera Service Pump1 to upgrade its performance equal to its pair Pump2. Then, consider an additional storage pump3 (15hp) to increase the reliability of the whole system (including the Sonera system).
	
	\item Consider a dedicated duty and a dedicated spare set-up for the pumps.  If this is not acceptable, then consider doing a much longer switch of the storage pumps.  Currently, it is being switched daily.  This allows for almost an equal rate of deterioration between the two pumps and if one pump fails due to age-related component failure, the other one is close to a similar failure which may occur before the first pump is fully repaired.  It is suggested that the switch happen once a month or even every 3 months.
	
	\item In place of the longer switching cycle (e.g. every 3 months), there should be a corresponding maintenance program for the standby pump for both booster and storage.
	
	\item Know what maintenance activities are done weekly and how the contractors/Maynilad use the information gathered to predict equipment failures.  
	
	\item Develop a more structured discipline in applying routine maintenance work process to ensure that maintenance tasks are given the proper priority in terms of mitigation measures and avoid unplanned shutdown of critical pumps in operation.
	
\end{itemize}

Aside from the above recommendations, we also generate a list of recommendations based on the RCM methodology. This is presented at at the end of the document (refer to Appendix )\ref{app_maintenance}). The list shall be considered as a living program, which requires continuously improvement as part of the total quality management system (refer to Deming cycle presented in GHD's technical proposal).

%\end{document}