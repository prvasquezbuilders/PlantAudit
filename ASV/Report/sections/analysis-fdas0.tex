\section{Fire protection and safety (FDAS) audit} \label{ch04fdas}

%VILLAMOR Pump Station was surveyed for fire protection and safety on 10/30/2018 by the GHD/Alternative Power Solutions Inc. inspection Team. The summary below identifies the major deficiencies that were identified during the inspection. The purpose of this survey was to identify significant fire safety issues and to provide recommendations for remediation based on applicable standards specified by the GHD/APSI. The scope of this initial fire safety inspection was limited to the review and identification of major fire safety issues. The inspection did not include identification of minor deficiencies, which will be further addressed as part of follow-up inspections. Table 1 summarizes the major fire safety issues identified during the inspection. Recommendations have been provided to address each issue. An implementation schedule shall be developed by GHD to remediate each of the findings. The specific timing of improvements, including any requested extensions due to design / installation constraints, shall be submitted to the GHD for approval


\subsection{Fire alarm and detection system} \label{ch04fdas01}
\subsubsection{Data and analysis}
Summary of data and information from FDAS audit is presented in Table \ref{ch04_fdas01} with visual images on as-found devices and panels (Figure \ref{ch04_fig_fdas01}).

\begin{table}[!h]
	\caption{FDAS data highlights 01.}
	\label{ch04_fdas01}
	{\scriptsize
\begin{tabular}{c|p{6.5cm}|c|p{5.5cm}}
	\hline
	No. & Assets & Status & Remarks \\ 
\hline
A & Visual check of the fire alarm control panel & 0 & No panel \\ 
1 & Panel Status, installed and location area & 0 & No panel \\ 
2 & Power indicator lamp operational & 0 & No panel \\ 
3 & Devices properly indicated and marked & 0 & No panel \\ 
4 & Panel clear from trouble indicators & 0 & No panel \\ 
5 & Lamp test indicator operational & 0 & No panel \\ 
6 & Zones properly indicated and marked & 0 & No panel \\ 
7 & Check if it’s connected to sprinkler system & 0 & No panel \\ 
\hline
B & Checking of installed devices &  &  \\ 
1 & Check floor plan lay-out and location of the device if accessible/easy to access & 0 & No as-built plan \\ 
2 & Heat detectors and / or smoke detectors indicator lamp functioning & 0 & No indicator lamp for battery operated device \\ 
3 & Heat detectors and / or smoke detectors indicator lamp functioning & 0 & No final test \\ 
4 & Pull station locations acceptable & 1 & Accessible \\ 
5 & Bells and buzzers operated correctly & 0 & For verification \\ 
6 & Bells and buzzers audibility & 0 & For verification \\ 
7 & Strobe lights locations are acceptable & 0 & No strobe lights \\ 
8 & Strobe light operated correctly & 0 & No device \\ 
9 & Are Fire alarm zones (areas) clearly marked & 0 & N/A \\ 
10 & Is there a maintenance and service contract for the fire alarm system & 0 & N/A \\ 
11 & Does the Fire Alarm System smoke detector, heat detector, manual call point , horn and strobe light working and  have a current inspection tag & 1 & MCP and bell have inspection tags \\ 
12 & Is the fire alarm system if full working order & 0 & No FACP panel \\ 
\hline
\end{tabular}

	}
\end{table}

\begin{figure}[!h]
	\includegraphics[scale=1.7]{figures/ch04_fig_fdas01} \\
	\caption{As-found devices and panels}
	\label{ch04_fig_fdas01} 
\end{figure}



\begin{table}[!h]
	\caption{FDAS data highlights 02.}
	\label{ch04_fdas011}
	{\footnotesize
		\begin{tabular}{c|p{4cm}|c|p{7cm}}
			\hline
			No. & Assets/Description & Status & Remarks \\ 
			\hline
			1 & Evacuation plan & 1 & Ground floor and 2nd floor \\ 
			2 & Fire extinguishers & 1 & Green (HCFC); Red (Dry Chemical); FEX signages without FEX. As per plan total of 16 \\ 
			3 & Fire exits & 1 & Indicated \\ 
			4 & Fire hose cabine & 1 & with inspection tag 8/3/2018 \\ 
			5 & Fire sprinkler system & 1 & no \\ 
			6 & Fire sprinkler tank & 1 & 3 units as per inspection \\ 
			7 & Emergency exit signages & 1 & BIGLITE brand, signages no power supply \\ 
			8 & Emergency lights & 1 & FIREFLY brand, last inspection 8/30/2-18 \\ 
			9 & PPE cabinet & 1 & with inspection tag \\ 
			\hline
		\end{tabular}	
	}
\end{table}

\begin{figure}[!h]
	\begin{minipage}[b]{0.5\linewidth}
		\centering
		\includegraphics[width=\textwidth]{figures/ch04_fig_safety01}
		\caption*{(a - 1st floor)}
		%		\label{ch02_fdas01}
	\end{minipage}
	\hspace{0.05cm}
	\begin{minipage}[b]{0.5\linewidth}
		\centering
		\includegraphics[width=\textwidth]{figures/ch04_fig_safety02}
		\caption*{(b -2nd floor)}
		%		\label{ch02_fdas02}
	\end{minipage}
	\hspace{0.05cm}
\begin{minipage}[b]{0.5\linewidth}
	\centering
	\includegraphics[width=\textwidth]{figures/ch04_fig_safetygen}
	\caption*{(c -Genset room)}
	%		\label{ch02_fdas02}
\end{minipage}
	\caption{Existing evacuation plan}
	\label{ch04_fig_safety01}
\end{figure}


\begin{figure}[!h]
	
	\begin{minipage}[b]{0.22\linewidth}
		\centering
		\includegraphics[width=\textwidth]{figures/ch04_fig_safety03}
		\caption*{(a - Emergency light)}
		%	\label{ch02_fdas03}
	\end{minipage}
	\hspace{0.03cm}
	\begin{minipage}[b]{0.22\linewidth}
		\centering
		\includegraphics[width=\textwidth]{figures/ch04_fig_safety04}
		\caption*{(b-Exit door)}
		%	\label{ch02_fdas03}
	\end{minipage}
	\hspace{0.03cm}
	\begin{minipage}[b]{0.22\linewidth}
		\centering
		\includegraphics[width=\textwidth]{figures/ch04_fig_safety05}
		\caption*{(c- Cabinet)}
		%	\label{ch02_fdas03}
	\end{minipage}
	\hspace{0.03cm}
	\begin{minipage}[b]{0.22\linewidth}
		\centering
		\includegraphics[width=\textwidth]{figures/ch04_fig_safety06}
		\caption*{(d - Fire hose)}
		%	\label{ch02_fdas03}
	\end{minipage}
	\hspace{0.03cm}
	\begin{minipage}[b]{0.22\linewidth}
		\centering
		\includegraphics[width=\textwidth]{figures/ch04_fig_safety07}
		\caption*{(e - HFCF FEX)}
		%	\label{ch02_fdas03}
	\end{minipage}
	\hspace{0.03cm}
	\begin{minipage}[b]{0.22\linewidth}
		\centering
		\includegraphics[width=\textwidth]{figures/ch04_fig_safety08}
		\caption*{(f-Dry chemical FEX)}
		%	\label{ch02_fdas03}
	\end{minipage}
	\hspace{0.03cm}
	\begin{minipage}[b]{0.5\linewidth}
		\centering
		\includegraphics[width=\textwidth]{figures/ch04_fig_safety09}
		\caption*{(g - Exit signage)}
		%	\label{ch02_fdas03}
	\end{minipage}
	\caption{Existing safety devices}
	\label{ch04_fig_safety02}
\end{figure}


\subsubsection{Recommendations}

The findings/facts and results of the audit are presented in Table \ref{ch05_tbl_fdas01}. Visual images of assets are shown in Figure \ref{ch05_fig_fdas01}. 

Highlights are

\begin{itemize}
	\item \textbf{Smoke Detectors 01/02/04/05/06/07/08:} Battery operated smoke detector should have audible chirping sound to indicate that device needs battery replacement. However, since there was no response from SD to the tests conducted, this means that the battery is already drained. Removal of device from base for cleaning did not show any improvement on the device. Hence device is declared not functioning;

	\item \textbf{Manual Call Point 01:} Manual call point should have an audible alarm response after pulling the cover. Hence this device is declared functioning;

	\item \textbf{Manual Call Point 02:} Manual call point did not respond with an audible alarm after pushing the button. Hence this device is declared not functioning; 

	\item \textbf{Bell 01:} Bell \#01 responded to MCP01  with an audible alarm after pulling the cover. Hence this device is declared functioning;

	\item \textbf{Bell 02:} Bell 02 did not responded to MCP2  with an audible alarm after pushing the button. Hence this device is declared not functioning.


\end{itemize}

\begin{table}[!h]
	\caption{FDAS analysis.}
	\label{ch05_tbl_fdas01}
	{\scriptsize
		\begin{tabular}{l|l|l|l|p{5cm}|p{5cm}}
\hline
No. & Assets & CS & IT & Facts & Remarks \\ 
\hline
1 & SM 01 & 0 & 4 & dust inside and outside & No response to push and hold test \\ 
&  &  &  & no chirping sound indicating that the battery does not provide adequate LOS & Repeat clean \\ 
&  &  &  & Perform push and hold to test & After 3x sprays, still no response \\ 
&  &  &  & Spray Max 3 times &  \\ 
\hline
2 & SM 02 & 0 & 4 & Same as SM01 & Same as SM 01 \\ 
\hline
3 & SM 03 & 1 & 1 & dust inside and outside & With audible alarm response to Push and hold test   \\ 
&  &  &  & With chirping sound. This indicates that battery needs replacement & Repeat clean \\ 
&  &  &  & Perform push and hold to test & 3X Repeat spray with audible alarm response in the device \\ 
&  &  &  & Spray Max 3 times &  \\ 
\hline
4 & SM 04 & 0 & 4 & Same as SM01 & Same as SM 01 \\ 
\hline
4 & SM 05 & 0 & 4 & Same as SM01 & Same as SM 01 \\ 
\hline
6 & SM 06 & 0 & 4 & Same as SM01 & Same as SM 01 \\ 
\hline
7 & SM 07 & 0 & 4 & Same as SM01 (No light signal indication (orange color maintained light) & Same as SM01 \\ 
\hline
8 & SM 08 & 0 & 4 & Same as SM01 (No light signal indication (orange color maintained light) & Same as SM01 \\ 
\hline
7 & MCP 01 & 1 & 1 & Pull the MCP  cover & Alarm response audible after pulling  the cover \\ 
\hline
8 & MCP 02 & 0 & 4 & Push the MCP  button & No response after pushing the button \\ 
\hline
9 & Bell 01 & 0 & 4 & Pull cover of MCP\#1 & With response/audible sound activated on Manual call point (MCP)\#1 \\ 
\hline
10 & Bell 02 & 0 & 4 & push button of MCP\#2 & No response or activation  on Manual call point (MCP) \#2 \\ 
\hline
		\end{tabular}
		
	}
\end{table}



\begin{figure}[!h]
	
	\begin{minipage}[b]{0.22\linewidth}
		\centering
		\includegraphics[width=\textwidth]{figures/ch05_fdas_sd01}
		\caption*{a - Smoker detector 01}
		%	\label{ch02_fdas03}
	\end{minipage}
	\hspace{0.03cm}
	\begin{minipage}[b]{0.22\linewidth}
		\centering
		\includegraphics[width=\textwidth]{figures/ch05_fdas_sd02}
		\caption*{b - Smoker detector 02}
		%	\label{ch02_fdas03}
	\end{minipage}
	\hspace{0.03cm}
	\begin{minipage}[b]{0.22\linewidth}
		\centering
		\includegraphics[width=\textwidth]{figures/ch05_fdas_sd03}
		\caption*{c - Smoker detector 03}
		%	\label{ch02_fdas03}
	\end{minipage}
	\hspace{0.03cm}
	\begin{minipage}[b]{0.22\linewidth}
		\centering
		\includegraphics[width=\textwidth]{figures/ch05_fdas_sd04}
		\caption*{d - Smoker detector 04}
		%	\label{ch02_fdas03}
	\end{minipage}
	\hspace{0.03cm}
	\begin{minipage}[b]{0.22\linewidth}
		\centering
		\includegraphics[width=\textwidth]{figures/ch05_fdas_sd05}
		\caption*{e - Smoker detector 05}
		%	\label{ch02_fdas03}
	\end{minipage}
	\hspace{0.03cm}
	\begin{minipage}[b]{0.22\linewidth}
		\centering
		\includegraphics[width=\textwidth]{figures/ch05_fdas_sd06}
		\caption*{f - Smoker detector 06}
		%	\label{ch02_fdas03}
	\end{minipage}
	\hspace{0.03cm}
\begin{minipage}[b]{0.22\linewidth}
	\centering
	\includegraphics[width=\textwidth]{figures/ch05_fdas_sd07}
	\caption*{f - Smoker detector 07}
	%	\label{ch02_fdas03}
\end{minipage}
	\hspace{0.03cm}
\begin{minipage}[b]{0.22\linewidth}
	\centering
	\includegraphics[width=\textwidth]{figures/ch05_fdas_sd08}
	\caption*{f - Smoker detector 08}
	%	\label{ch02_fdas03}
\end{minipage}
	\hspace{0.03cm}
	\begin{minipage}[b]{0.22\linewidth}
		\centering
		\includegraphics[width=\textwidth]{figures/ch05_fdas_mcp01}
		\caption*{g - Manual call point 01}
		%	\label{ch02_fdas03}
	\end{minipage}
	\hspace{0.03cm}
	\begin{minipage}[b]{0.22\linewidth}
		\centering
		\includegraphics[width=\textwidth]{figures/ch05_fdas_mcp02}
		\caption*{h - Manual call point 02}
		%	\label{ch02_fdas03}
	\end{minipage}
	\hspace{0.03cm}
	\begin{minipage}[b]{0.22\linewidth}
		\centering
		\includegraphics[width=\textwidth]{figures/ch05_fdas_buzzer01}
		\caption*{i - bell 01}
		%	\label{ch02_fdas03}
	\end{minipage}
	\hspace{0.03cm}
	\begin{minipage}[b]{0.22\linewidth}
		\centering
		\includegraphics[width=\textwidth]{figures/ch05_fdas_buzzer02}
		\caption*{j - bell 02}
		%	\label{ch02_fdas03}
	\end{minipage}
	\caption{FDAS assets}
	\label{ch05_fig_fdas01}
\end{figure}

In brief, FDAS of the station is not provide adequate level of services mainly due to:
\begin{itemize}
\item Most of the smoke detector devices, manual call point, buzzer and the FACP were not functioning. These were established  during the conducted testing of FDAS and devices;

%\item Lacking smoke / heat detector devices at substation room and genset room, engineer’s office, pantry and guardhouse. In every close room there should be a device installed to detect heat or smoke. 

%\item Lightning protection system is not in place	

\end{itemize}

\paragraph{\underline{Short term recommendations}}

\begin{itemize}
\item Test the batteries for voltage output;

\item Replace existing batteries for smoke detectors that are not functioning with fresh 9V Battery super heavy duty (Carbon-zinc type or Alkaline ) with 0\% Mercury;

\item After replacement of battery, push test/ “hush” button. This will decrease the sensitivity for approximately 8 minutes. During this time, the RED LED will flush every 10 seconds. This will indicate that the device is functioning;

\item Dust can clog your smoke alarms. Battery-powered smoke alarms should be cleaned by opening the cover of the alarm and gently vacuuming the inside with a soft bristle brush;

\item Maintain smoke detectors by testing and manual push to ascertain functionality every week as per instruction on the device by the manufacturer.

\end{itemize}


\paragraph{\underline{Long term recommendations}}

\begin{itemize}
	\item [$\checkmark$] Replace the system with newer and addressable type FACP to determine exact location of fire as it happens;
	
	\item[$\checkmark$] Additional smoke detector devices are required in engineer’s office, guardhouse, pump area;
	
	\item[$\checkmark$] Additional heat detector devices are required in pantry, genset room, substation room;
	
	\item[$\checkmark$] Additional strobe with sounder and call point  at the genset and substation room since these rooms are located in separate buildings and for safety reasons for sounding alarm at the instant that there is fire;
	
	\item [$\checkmark$]Annual Inspection, Testing and Maintenance;
	
	\item [$\checkmark$] Replace existing bell to strobe light  with sounder and call point  at the booster pump area and additional strobe light with sounder and call point at genset room. Genset room is located in separate building and for safety reasons for sounding alarm at the instant that there is fire;
	
\end{itemize}
Note that the FACP will no longer communicate with existing bell and is not compatible

\paragraph{\underline{System Testing }}

FDAS shall be subjected to the following tests conforming to the Philippine Electronics Code of 2014 and Philippine Electrical Code of 2017
%\renewcommand{\labelitemi}{$\checkmark$}
\begin{itemize}%[label={\checkmark}]
	\item [$\checkmark$] Testing of insulation resistance and continuity of wires;
	\item [$\checkmark$] Verification of installed devices;
	\item [$\checkmark$] Operation and response of FDAS;
	\item [$\checkmark$] Testing the operation of initiating devices;
	\item [$\checkmark$] Measuring sound pressure level generated by notification devices;
\end{itemize}


\paragraph{\underline{Records }}

Every FDAS system shall keep the following documentations
%\renewcommand{\labelitemi}{$\checkmark$}
\begin{itemize}%[label={\checkmark}]
	\item [$\checkmark$] A complete set of operation and maintenance manuals of the manufacturer covering all equipment used in the system;
	\item [$\checkmark$] A complete set of as-built drawings;
	\item [$\checkmark$] A written sequence of operation;
	\item [$\checkmark$] Record of completion and results of every inspection, testing and maintenance;
	\item [$\checkmark$] Record of components within the database.
\end{itemize}



\subsection{Lighting protection system} \label{ch04fdas02}
\subsubsection{Data and analysis}
No lightning protection was installed for this PS.

\subsubsection{Recommendations}

Refer to the conceptual design in Chapter \ref{Chapter6}

\paragraph{\underline{Short term Recommendations}}

Plan for the installation of a new lightning protection system 
%\renewcommand{\labelitemi}{$\checkmark$}
\begin{itemize}%[label={\checkmark}]
	\item [$\checkmark$] the LPS conforms to the design and is based on the  standard;
	\item [$\checkmark$] all components of the LPS are in good condition and capable of performing their designed functions, and that there is no corrosion.
\end{itemize}


\paragraph{\underline{Long term Recommendations}}

Plan for the installation of a new lightning protection system 
%\renewcommand{\labelitemi}{$\checkmark$}
\begin{itemize}%[label={\checkmark}]
	\item [$\checkmark$] According to the standard, an inspection should be undertaken during the construction of the structure, after the installation, after alterations or repairs, and when it is known that the structure has been struck by lightning;
	\item [$\checkmark$] It is also recommended that inspections take place “periodically at such intervals as determined with regard to the nature of the structure to be protected”, taking into account the local environment, such as corrosive soils and corrosive atmospheric conditions and the type of protection measures employed;
	\item [$\checkmark$]The inspection comprises checking the technical documentation, visual inspections and test measurements;
	\item [$\checkmark$]Prepare an inspection guide to facilitate the inspection process containing sufficient information on the installation and its components, tests methods and previous inspection/test data;	
	\item [$\checkmark$]During the visual inspection, the following should be checked;	
	\begin{itemize}
		\item [-] the deterioration and corrosion of air-termination elements, conductors and connections
		\item [-]	the corrosion of earth electrodes
		\item [-]	the earthing resistance value for the earth-termination system
		\item [-]	the condition of connections, equipotential bonding and fixings.
		
	\end{itemize}
	
	\item [$\checkmark$] For those parts of an earthing system and bonding network not visible for inspection, tests of electrical continuity should be performed;
	
	\item [$\checkmark$] An inspection report should be prepared detailing the status of the system, any deviations from the technical documentation and the results of any measurements undertaken. Any obvious faults should also be reported.
\end{itemize}

No lightning protection system is 100\% effective. A system designed in compliance with the standard does not guarantee immunity from damage. Lightning protection is an issue of statistical probabilities and risk management. A system designed in compliance with the standard should statistically reduce the risk to below a pre-determined threshold. The IEC 62305-2 risk management process provides a framework for this analysis. An effective lightning protection system needs to control a variety of risks. While the current of the lightning flash creates a number of electrical hazards, thermal and mechanical hazards also need to be addressed. 

Risk to persons (and animals) include: 

\begin{itemize}
\item Direct flash;
\item  Step potential ;
\item Touch potential ;
\item  Side flash ;
\item Secondary effects

\begin{itemize}
	
	 \item[-]  asphyxiation from smoke or injury due to fire 
	\item [-] structural dangers such as falling masonry from  point of strike 
	\item [-] unsafe conditions such as water ingress from roof  penetrations causing electrical or other hazards,  failure or malfunction of processes, equipment and  safety systems

\end{itemize}
\end{itemize}




Risk to structures \& internal equipment include: 

\begin{itemize}
\item Fire and/or explosion triggered by heat of lightning flash,  its attachment point or electrical arcing of lightning  current within structures ;
\item  Fire and/or explosion triggered by ohmic heating of  conductors or arcing due to melted conductors;
\item Punctures of structure roofing due to plasma heat  at lightning point of strike ;
\item Failure of internal electrical and electronic systems ;
\item Mechanical damage including dislodged materials at  point of strike.
\end{itemize}










\subsection{Ground-Fault circuit interrupter (GFCI) or electric leakage circuit breaker (ELCB) or Residual circuit devices (RCD)} \label{ch04fdas03}
\subsubsection{Data and analysis}
No ground fault circuit interrupter (GFCI) or earth leakage Circuit breaker (ELCB) protection was installed in the panel for FDAS for this PS.
\subsubsection{Recommendations}
Refer to the conceptual design in Chapter \ref{Chapter6}
\subsection{Electrical safety and protective devices} \label{ch04fdas04}
\subsubsection{Data and analysis}
Facts obtained from inspection are presented in Table \ref{ch05_tbl_fdassafe01} with indicative figures for each devices presented in Figure \ref{ch05_fig_fdassafety01}.

\begin{table}[!htb]
	\caption{Protective devices.}
	\label{ch05_tbl_fdassafe01}
	{\scriptsize
\begin{tabular}{c|p{3cm}|c|c|p{4cm}|p{4cm}}
	\hline
	No. & Assets & CS & IT & Facts & Remarks \\ 
\hline
1 & Stairway going to 2nd floor & 0 & 4 & No adequate illumination & Unsafe \\ 
&  &  &  & Steep stairway & May cause tripping and downfall \\ 
\hline
2 & Stairway going to rooftop & 0 & 4 & Safety Cage shield for climbing is too high.  It do not protect personnel on the onset of climbing after 6 feet. Safety cage should start at 6 feet  & Unsafe \\ 
&  &  &  &  & May cause fall \\ 
\hline
3 & Construction near entrance & 0 & 4 & Unfinished construction left unattended and without restoring pathway & Unsafe \\ 
&  &  &  &  & May cause tripping if unnoticed \\ 
\hline
4 & Fire Extinguisher at the entrance & 0 & 4 & Fire extinguisher placed beside an obstruction & Not readily accessible \\ 
&  &  &  &  & Fire Extinguisher partly hidden by boxes \\ 
\hline
5 & Emergency Light- Ground floor & 0 & 4 & Green indicator & Emergency Light does not show when was the last inspection due to lost tag \\ 
&  &  &  & Without inspection tag &  \\ 
\hline
6 & Fire Extinguisher - ground floor tag 01 & 0 & 4 & No updated & Unsafe \\ 
&  &  &  & Last update is 30/Aug/2018 & Failure to inspect can make it ineffective  \\ 
\hline
7 & Fire Extinguisher - ground floor tag 02 & 0 & 4 & No tag & FEX workability/functionality should be determined \\ 
&  &  &  &  &  \\ 
\hline
8 & Fire Extinguisher - Pump Area & 0 & 4 & No tag & FEX workability/functionality should be determined \\ 
&  &  &  &  &  \\ 
\hline
9 & Cooking pot-Pump Area & 0 & 4 & Cooking activity being done in Pump Area & Unsafe \\ 
&  &  &  &  & Cooking activity should be done in designated Area like pantry \\ 
\hline
10 & Isolation Transfer Room - Ground floor   & 0 & 4 & Chair obstruction in front of Switchgear & Unsafe \\ 
&  &  &  &  & Switchgear shall be free from all obstruction. \\ 
\hline
11 & Chairs stores at the back of MCC & 0 & 4 & Chair obstruction on the rear of the Motor control Starter & Unsafe \\ 
&  &  &  &  & Switchgear shall be free from all obstruction \\ 
\hline
12 & Emergency Light - Pump Area & 0 & 4 & With Power indicator & Inspection tag says EL is ok \\ 
&  &  &  & Not charging &  \\ 
\hline
13 & First Aid Kit- Engineer's  Area & 0 & 4 & First Aid kit not found only with signage & Unsafe \\ 
&  &  &  &  & Area should be dedicated to First Aid Alone and not to mix with other office materials \\ 
\hline
\end{tabular}
}
\end{table}


\begin{figure}[!htb]
	\begin{minipage}[b]{0.22\linewidth}
		\centering
		\includegraphics[width=\textwidth]{figures/ch05_fdas_safety01}
		\caption*{a - stairway going to 2nd floor}
		%	\label{ch02_fdas03}
	\end{minipage}
	\hspace{0.03cm}
	\begin{minipage}[b]{0.22\linewidth}
		\centering
		\includegraphics[width=\textwidth]{figures/ch05_fdas_safety02}
		\caption*{b - stairway going to rooftop of Building}
		%	\label{ch02_fdas03}
	\end{minipage}
	\hspace{0.03cm}
	\begin{minipage}[b]{0.22\linewidth}
		\centering
		\includegraphics[width=\textwidth]{figures/ch05_fdas_safety03}
		\caption*{c - Construction near entrance gate}
		%	\label{ch02_fdas03}
	\end{minipage}
	\hspace{0.03cm}
	\begin{minipage}[b]{0.22\linewidth}
		\centering
		\includegraphics[width=\textwidth]{figures/ch05_fdas_safety04}
		\caption*{d - Fire Extinguisher at the entrance}
		%	\label{ch02_fdas03}
	\end{minipage}
	\hspace{0.03cm}
	\begin{minipage}[b]{0.22\linewidth}
		\centering
		\includegraphics[width=\textwidth]{figures/ch05_fdas_safety05}
		\caption*{e - Emergency light ground floor}
		%	\label{ch02_fdas03}
	\end{minipage}
	\hspace{0.03cm}
	\begin{minipage}[b]{0.22\linewidth}
		\centering
		\includegraphics[width=\textwidth]{figures/ch05_fdas_safety06}
		\caption*{f - Fire Extinguisher – ground floor 01}
		%	\label{ch02_fdas03}
	\end{minipage}
	\hspace{0.03cm}
	\begin{minipage}[b]{0.22\linewidth}
		\centering
		\includegraphics[width=\textwidth]{figures/ch05_fdas_safety07}
		\caption*{g - Fire Extinguisher – ground floor 02}
		%	\label{ch02_fdas03}
	\end{minipage}
	\hspace{0.03cm}
\begin{minipage}[b]{0.22\linewidth}
	\centering
	\includegraphics[width=\textwidth]{figures/ch05_fdas_safety08}
	\caption*{h - Fire Extinguisher – pump area}
	%	\label{ch02_fdas03}
\end{minipage}
	\hspace{0.03cm}
\begin{minipage}[b]{0.22\linewidth}
	\centering
	\includegraphics[width=\textwidth]{figures/ch05_fdas_safety09}
	\caption*{i - Cooking pot}
	%	\label{ch02_fdas03}
\end{minipage}
	\hspace{0.03cm}
\begin{minipage}[b]{0.22\linewidth}
	\centering
	\includegraphics[width=\textwidth]{figures/ch05_fdas_safety10}
	\caption*{j - Isolation transfer room}
	%	\label{ch02_fdas03}
\end{minipage}
	\hspace{0.03cm}
\begin{minipage}[b]{0.22\linewidth}
	\centering
	\includegraphics[width=\textwidth]{figures/ch05_fdas_safety11}
	\caption*{k - Chair at the back of the MCC}
	%	\label{ch02_fdas03}
\end{minipage}
	\hspace{0.03cm}
\begin{minipage}[b]{0.22\linewidth}
	\centering
	\includegraphics[width=\textwidth]{figures/ch05_fdas_safety12}
	\caption*{l - emergency light - pump area}
	%	\label{ch02_fdas03}
\end{minipage}

	\hspace{0.03cm}
\begin{minipage}[b]{0.22\linewidth}
	\centering
	\includegraphics[width=\textwidth]{figures/ch05_fdas_safety13}
	\caption*{m - First aid kid}
	%	\label{ch02_fdas03}
\end{minipage}

	\caption{Protective devices.}
	\label{ch05_fig_fdassafety01}
\end{figure}

\subsubsection{Recommendations}
Based on the status of devices, recommendations are with intervention types shown in Table \ref{ch05_tbl_fdassafe01}. Chapter \ref{Chapter6} further illustrates the recommendation with the conceptual design.
%\subsection{Recommendations}
