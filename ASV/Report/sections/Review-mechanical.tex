\subsection{Mechanical Audit}
\label{232}
The Mechanical Tests to be conducted are enumerated and discussed hereunder including their background and applications, standards used if applicable, and the equipment to be used. During testing, the following are the assumptions and considerations:

\begin{itemize}
\item The operation of the pumps cannot be interrupted (at day time when demand is high).
\item The valve settings then cannot be adjusted to produce different flow rates.
\end{itemize}

\subsubsection{Structural Inspection for Pump Discharge and Suction Line}

This activity measures the current thickness of the existing pipelines at the pump vicinity using ultrasonic thickness gauging. The flow regime especially around the elbow and possibly corrosion and scaling conditions are to be predicted from the measurements of this test.

Following procedures will be executed

\paragraph{Step 1: Locate and mark testing points}
At a minimum of two (2) meters away from the pump intake/discharge flange, the test points shall be marked at 3, 6, 9 and 12 o’clock positions and at one (1) meter interval along the pipes, additional test sections with same set points shall be added as long as available beneath the immediate ground level.


%The test points shall also consider elbows on probable points of most thinning from turbulent water flow.

\paragraph{Step 2: Prepare test point surfaces}
\begin{itemize}
\item	Wipe the surface free of dirt (no need to remove paint)
\item	Using a chalkstone (erasable), mark x on the test point
\end{itemize}

\paragraph{Step 3: Apply sufficient couplant on test point surface}
\begin{itemize}
\item	Use petroleum jelly/Vaseline as couplant
\end{itemize}

\paragraph{Step 4: Set transducer probe on test point}

\paragraph{Step 5: Read and record value as indicated on module display} 


\paragraph{Step 6: Clean test point after reading}

\subsubsection{Unit Flow Measurement}
The activity measures pump capacities. Pump efficiency is then calculated using the measured values.  

\paragraph{Step 1: Locate Sensor Position Point Area and mark all points to be taken.}

\begin{itemize}
	\item Observe required offset distance from fittings/pump to consider the fully developed flow. At least 10 times the diameter distance away from the suction/discharge of the pump if applicable, otherwise consider at least 2D distance away from the fittings. This is to ensure the flow will be stable and fully developed for flow measurement accuracy %(Figure \ref{ch02_flowmeasurement01}).
	\item Otherwise, test at near turbulent zones and consider normalizing the flow. 
\end{itemize}

In particular, the headers can be chosen as set points for flow measurement. (Figures \ref{fig_ch02_ufm} - a to d).

\begin{figure}[ht]
	\begin{minipage}[b]{0.225\linewidth}
		\centering
		\includegraphics[width=\textwidth]{figures/fig_ch02_ufm3}
		\caption*{a}
		%		\label{ch02_flowmeasurement01}
	\end{minipage}
	\hspace{0.05cm}
	\begin{minipage}[b]{0.225\linewidth}
		\centering
		\includegraphics[width=\textwidth]{figures/fig_ch02_ufm4}
		\caption*{b}
		%		\label{ch02_flowmeasurement02}
	\end{minipage}
	\hspace{0.05cm}
	\begin{minipage}[b]{0.225\linewidth}
		\centering
		\includegraphics[width=\textwidth]{figures/fig_ch02_ufm1}
		\caption*{c}
		%		\label{ch02_flowmeasurement02}
	\end{minipage}
	\hspace{0.05cm}
	\begin{minipage}[b]{0.225\linewidth}
		\centering
		\includegraphics[width=\textwidth]{figures/fig_ch02_ufm2}
		\caption*{d}
		%		\label{ch02_flowmeasurement02}
	\end{minipage}
	\caption{UFM testing points}
	\label{fig_ch02_ufm}	
\end{figure}


\paragraph{Step 2: Pipe Specification Input on the Flow Meter.}
\begin{itemize}
\item Identify nominal pipe size with equivalent parameters such as schedule designation, equivalent thickness, OD, and etc.
\item Input outside diameter.
\item Input pipe thickness.
\item Input pipe material (carbon steel).
\item Input pipe medium (water).

\end{itemize}

\paragraph{Step 3: Prepare test point surfaces}
\begin{itemize}
	\item Clean the surface of pipe with a sandpaper and steel brush or any suitable abrasive materials, exposing the base metal.
	
\end{itemize}

\paragraph{Step 4: Install transducers at set points}
\begin{itemize}
	\item Apply enough couplant that it covers transducers sensors to ensure an acoustically conductive connection to the pipe. Also apply couplant on the test point surface %(Figure \ref{ch02_flowmeasurement03}).
	\item Clamp the transducers at the side of pipe using metal chains, straps or mounting rails Observing proper spacing and alignment. Note flow direction and install transducers at either 0 or 45 degrees, whichever would give more stable reading %(Figure \ref{ch02_flowmeasurement04}, Figure \ref{ch02_flowmeasurement05}, Figure \ref{ch02_flowmeasurement06})
	\item Wait for the module to display “System Normal” before reading. Inspect set-up for any fault and properly reinstall if signal is poor/low (no reading)
\end{itemize}


\paragraph{Step 5: Data gathering}
Read and record all necessary data measurement by the equipment, (i.e. flow, fluid velocity, sound velocity, Reynolds number, etc.) 

\begin{figure}[!htb]
	\includegraphics[scale=1.3]{figures/fig_ch02_flowmeasurement07} \\
	\caption{UFM Measurement Display}
	\label{ch02_flowmeasurement07} 
\end{figure}

\paragraph{Step 6: Remove transducers and restore paints}
Remove the transducers and restore the surface of pipe after measurement.

\subsubsection{Suction and Discharge Pressure Measurement}
The activity measures each pump suction and discharge pressure. The pump efficiency is then calculated using the measured values.

\paragraph{Step 1: Disassembly of existing Pressure Gauge}
\begin{itemize}
\item Inspect for any leaks or unusual noise before proceeding: If anything is detected, report immediately to the operator;
\item 	Close gate valve located before the pressure gauge and wait for the pressure reading to drop;
\item 	Remove the pressure gauge: (1) Hold the adapter steady with one wrench and the grip the stationary socket of the pressure gauge with another; (2) Loosen the pressure gauge then remove it.
\end{itemize}

\paragraph{Step 2: Installing the Pressure Gauge}
\begin{itemize}
\item Prepare the connections: (1) Clean the connections before installing; (2) Put Teflon tape on the pressure connection of the gauge;
\item Install the pressure gauge: (1) Mount the pressure gauge on the adapter then hand tighten the arrangement; (2) Further tighten the assembly using a pair of wrenches: hold the adapter steady with one wrench and the grip the stationary socket of the pressure gauge with another; (3) Tighten the assembly;
\item Inspect the assembly again.
\end{itemize}
\paragraph{Step 3: Reading the pressure}
\begin{itemize}
\item Slowly open the gate valve: Observe any leaks or unusual noise;
\item Measurement: (1) Wait until reading is stable; (2) Record the pressure as indicated.
\end{itemize}

\paragraph{Step 4: Restoring the earlier gauge}
\begin{itemize}
\item Inspect for any leaks or unusual noise before proceeding: If anything is detected, report immediately to the operator;
\item 	Close gate valve located before the pressure gauge and wait for the pressure reading to drop;
\item Remove the pressure gauge: (1) Hold the adapter steady with one wrench and the grip the stationary socket of the pressure gauge with another; (2) Loosen the pressure gauge then remove it;
\item 	Prepare the connections: (1) Clean the connections before installing; (2) Put Teflon tape on the pressure connection of the gauge;
\item 	Install the pressure gauge;
\item 	Mount the pressure gauge on the adapter then hand tighten the arrangement;
\item 	Further tighten the assembly using a pair of wrenches: hold the adapter steady with one wrench and the grip the stationary socket of the pressure gauge with another;
\item 	Tighten the assembly.
\end{itemize}
\subsubsection{Parameters}
Parameters was recorded using visual inspection form, interview questionnaire, and testing results. Main parameters are listed, but not limited to, in the Table \ref{ch02_tbl_parameter}. Raw data is enclosed in the Appendix.

\begin{table}[h]
	\caption{Main parameters to be collected.}
	\label{ch02_tbl_parameter}
	{\footnotesize
\begin{tabular}{l|l|l}
	\hline
	Parameters & \multicolumn{1}{c|}{Symbol} & Remarks \\ 
	\hline
	Pipe thickness Gauge & \multicolumn{1}{c|}{t} & \multicolumn{1}{c}{mm} \\ 
	Pump Capacity & \multicolumn{1}{c|}{Q} & \multicolumn{1}{c}{Gpm/cmh} \\ 
	Suction Pressure  & \multicolumn{1}{c|}{Ps} & \multicolumn{1}{c}{mH2O} \\ 
	Discharge Pressure & \multicolumn{1}{c|}{Pd} & \multicolumn{1}{c}{mH2O} \\ 
	Vibration Data  & \multicolumn{1}{c|}{-} & \multicolumn{1}{c}{-} \\ 
	Head & \multicolumn{1}{c|}{H} & \multicolumn{1}{c}{mH2O} \\ 
	Efficiency  & \multicolumn{1}{c|}{e} & \multicolumn{1}{c}{\%} \\ 
		\hline
	\end{tabular}		
	}
\end{table}

