%{\color{red}

\section{Energy management audit}
\label{46}
\subsection{Production and power data}
Production data for this station has been recorded in excel files. Each file represents a month with 24 hours of daily records. Maynilad provided this set of data of 2018 per GHD' request. Initial verification on this set was conducted with following conclusions

\begin{itemize}
	
	\item The structure of data is not homogeneous with numerical errors. This problem is due to the fact that excel file is not suitable for recording a large volume of data, particularly cells are not set up to reject string and value outside the lower and upper bounds.
	content.
\end{itemize}

%When excluding the data of 2012 and 2013, the set used for compilation has following statistics

In order to compile such a data set, it is not possible with manual inputting, instead, GHD has developed a hybrid program consisting of Visual Basic (VBA) Code and MySQL code for fast compilation. VBA code is used to add header, fill up missing information in excel file, and ignore rows and columns that should not exist with regard to database structure. MySQL codes are used to eliminate measurement errors and bring together all individual files to one file that allows statistical analysis with R.
\subsection{Measurement errors}
Following measurement errors are with the provided excel files
\begin{itemize}
\item String/text values are found numerous in columns that shall be only numerical values;
\item Extreme values are found numerous;
\item Negative values are found in many places that shall only be positive
\end{itemize}
%\subsection{Summary of statistics}
\subsection{Data compilation for analysis}
Out of all recorded attributes, useful attributes that can be used for energy audit are total production per hour and total power consumption per hour. There is no record on production and power consumption for individual pump.

%After data filtering, data correction, and compilation, the obtained set of data includes 43,740 records (equivalent to 1458 days in total 5 years). Final data set is saved in MySQL server.

\subsection{Analysis}

As a matter of fact, power consumption of a PS is mostly contributed by the operation of pumps. Thus, the audit has been centralized on 
\begin{itemize}
	\item Analyzing given production and power consumption data to understand the trend and establish a benchmark ratio of production vs power for future audit and management;
	\item Evaluating other part of the audit such as pump efficiency and reliability in order to derive better intervention program that will eventually beneficial to the Client to maintain a benchmark level of power consumption against the production. 
\end{itemize}

Figure \ref{ch05_fig_energy_correlation} shows the statistical correlation between production and power. It can be seen from the correlation graph and correlation value that there is very weak correlation among these two values (coefficient is 0.118). A careful inspection on data reveals that data has been recorded inappropriately. It could be possible that the meters were not provide adequate level of services.

%This infers more or less that there is less breakdown of components of this station. This conclusion is also supported by the fact that there has been little historical record on both preventive and corrective intervention of this station.

\begin{figure}[!htb]
	\includegraphics[scale=0.6]{figures/ch05_fig_energy_correlation} \\
	\caption{Correlation between production and power consumption}
	\label{ch05_fig_energy_correlation} 
\end{figure}

Figure \ref{ch05_fig_energy_analysis} shows a trend in time series production, power, and its ration in 2018. Whilst, Figure \ref{ch05_fig_energy_distribution} presents the distributions.


\begin{figure}[!htb]
	\includegraphics[width=\textwidth]{figures/ch05_fig_energy_analysis} \\
	\caption{Time series analysis}
	\label{ch05_fig_energy_analysis} 
\end{figure}

\begin{figure}[!htb]
	\begin{minipage}[b]{0.3\linewidth}
		\centering
		\includegraphics[width=\textwidth]{figures/distribution_production}
		\caption*{a - Production}
		%		\label{pagcorlocation}
	\end{minipage}
	\hspace{0.05cm}
	\begin{minipage}[b]{0.3\linewidth}
		\centering
		\includegraphics[width=\textwidth]{figures/distribution_power}
		\caption*{b - Power}
		%		\label{ch01_pumpgallery}
	\end{minipage}
	\hspace{0.05cm}
	\begin{minipage}[b]{0.3\linewidth}
		\centering
		\includegraphics[width=\textwidth]{figures/distribution_ratio}
		\caption*{c - Ratio}
		%		\label{ch01_pumpgallery}
	\end{minipage}
	\caption{Density distribution}
	\label{ch05_fig_energy_distribution}
\end{figure}

It can be seen from the time series and distribution graphs that the values of the three parameters are randomly distribute without clear trends. This explains the reason why the value of correlation is about 0.3. The shapes of distribution graphs have some distortion at their tails. In normal operation condition, it is expected that when there is high energy consumed, there will be higher production. However, in many cases, the production was lower than average but the energy was not proportional decreasing. This issue could be a result of low water level in the reservoir causing more power as the pumps were still in need of operations or even more. In order to minimize this impact, it is important to maintain the water level in the reservoir chambers from the source.



%As the production decreases and power increase, the ratio keeps increasing over time. 


Interpretation from these graphs can be summarized as follows

\begin{itemize}
	\item There is almost low level of correlation between the production and power, which has been contributed by a certain abnormal trend in the graphs which might due to the failure of reading devices;
%	\item There has been a few number of peaks at which the ratio between power and production were significantly high compared to average value. These peaks seem to be repeated at least one in a year. The reason causing that peaks are unknown;
%	\item There are too much outliers on the recorded data. THis data set is not reliable for a complete energy audit;
	
%	\item There was a high peak in ratio in the first quarter of 2018. The ratio has reached between 300 and 400 for 3 months period, inferring an abnormal operation.
	%\item Power pressure of suction
	\item Average daily production could be in between of 0.45 to 0.6 MLD;
	\item Average daily power consumption could be in the range from 30 to 35;
	\item Average ratio between power and production is between 60 to 70. It is believed that the benchmark ratio should be at least equal to the average or better. Optimistically, the benchmark ratio could be 70.
\end{itemize}

\subsection{Recommendation}
In order to operate the PS in a manner that is energy efficient, it is advisable to 

\begin{itemize}
\item Establish an optimal operation scheme;
\item Establish a benchmark energy efficiency ratio for continuous monitoring and reporting. This ratio shall become a Key Performance Indicator (KPI) used for managerial purpose. GHD suggests to first fix the issue of data recording and compilation for getting reliable set of data. This can be done from now to the end of 2019 for another round of energy audit.
\end{itemize}
%}