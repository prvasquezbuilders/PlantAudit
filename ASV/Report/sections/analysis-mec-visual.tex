%\begin{document}

\section{Visual Inspection on Pipe, valves, fittings, supports, expansions, and appurtenances}
\label{ch04mech02}

\subsection{Highlights}
\label{ch04mech02_highlight}

Visual inspection data on pipes, valves, fittings, supports, expansions, and appurtenances is highlighted in Table \ref{ch043_tbl_visualinspectionHL}.

\begin{table}[!htb]
	\caption{Highlights of visual inspection}
	\label{ch043_tbl_visualinspectionHL}
%	\resizebox{\columnwidth}{!}{%
	{\scriptsize

\begin{tabular}{c|p{3cm}|p{9.5cm}}

\hline
No. & Items & Remarks \\ 
\hline
1 & Existing suction pipe and fittings & Suction line too short and is jam packed with fittings. Does not promote good flow development. Intake water will be turbulent and is not desired \\ 
2 & Discharge piping and fittings & Pressure gauge near pump discharge is prefered for measuring head of indivual pump \\ 
2 & Pump vibration isolation  & Does not appropriately serve its function to isolate vibration from building and reduce noise \\ 
3 & As built difference & Actual system contain many differences from provided copy of old as-built including valve positions and pipe design \\ 
4 & Pump foundation block & Unitary block with multiple slots and may  \\ 
5 & Space & Modifications for improvement are possible because of available space inside pump room \\ 
6 & Instrumentation and monitoring & Pump instrumentation do not include PLC and other important parameters typically displayed by PLC not monitored \\ 
\hline

\end{tabular}

	}%}
\end{table}

Visual inspections are supported with the photos taken at particular locations/positions in question.

The pump station is situated in Ayala Southvale subdivision where the available space is quite ample relative to the other stations. As earlier mentioned, the station services two areas namely, Southvale and Sonera. Two 10HP long coupled pumps (Figure \ref{fig_ch01_intro_psr_loc} - a) deliver to Southvale and another two 15HP close coupled pumps (Figure \ref{fig_ch01_intro_psr_loc} - b) deliver water for Sonera area. In contrast to other pumping stations, the pumps and motors installed have relatively smaller capacity. 

Machine support

The four booster pump units in this station are bolted onto spring-mounted baseplates (Figure \ref{fig_ch043_spring_mounted_baseplates}) typical of small Goulds pumps installations. Designed to allow displacement under applied loads associated with thermal expansion of piping, such as those employed in small chilled water or hot water systems, said spring-mounted baseplates typically sit or slide on a smooth floor. However, the station designers choose to bolt the spring mounts onto concrete blocks, which defeats the purpose of the spring-mount design. (Refer to Goulds Pumps Installation, Operation and Maintenance Instructions.)

\begin{figure}[!htb]
	\begin{minipage}[b]{0.3\linewidth}
		\centering
		\includegraphics[width=\textwidth]{figures/fig_ch043_spring_mounted_baseplates1}
		\caption*{a - Southvale service pump baseplate}
		%		\label{pagcorlocation}
	\end{minipage}
	\hspace{0.05cm}
	\begin{minipage}[b]{0.3\linewidth}
		\centering
		\includegraphics[width=\textwidth]{figures/fig_ch043_spring_mounted_baseplates4}
		\caption*{b - Sonera service pump baseplate}
		%		\label{ch01_pumpgallery}
	\end{minipage}
	\hspace{0.05cm}
	\begin{minipage}[b]{0.3\linewidth}
		\centering
		\includegraphics[width=\textwidth]{figures/fig_ch043_pump_vibisolator2}
		\caption*{c - Deflection Spring Mounting Assembly}
		%		\label{ch01_pumpgallery}
	\end{minipage}
	\caption{Spring Mounted Baseplates}
	\label{fig_ch043_spring_mounted_baseplates}
\end{figure}

If possible, it is best to inquire the station designers to re-examine the assumptions or reasons behind this unusual mounting design.

On another point of view, assemblies on the pump mounting such as spring isolators are for isolating or reducing the damaging structure vibration and annoying vibration induced noise produced by the mechanical equipment. This is designed for modern buildings with which vibration and noise can become the major source of occupant complaint. Also, installation of vibration isolators is for the purpose of protecting the building from micro-vibrations which could potentially lead to structural failures.

Special spring isolators are important and appropriate for industries that have sensitive process machining, where more damping is required and less motion can be tolerated due to certain special requirements, precision for example. Special cases such as wind loads will require isolators for pumps or fans of cooling towers and condensers. None of the conditions stated above do not fit the situation for the pump station. Vibration isolators are not necessary for the pump system. Furthermore, springs might have deteriorated with time and might need to be removed and the equipment foundation and mounting be refurbished. 
%See conceptual design for details of proposed machine foundation and mounting.

Pipe design and flow profile 
Fittings neary the pump intake will cause flow disturbances and turbulence (Figure \ref{fig_ch043_suction_fittings}) 

\begin{figure}[!htb]
	\begin{minipage}[b]{0.5\linewidth}
		\centering
		\includegraphics[width=\textwidth]{figures/fig_ch043_suction_fittings1}
		\caption*{a -Southvale service pump suction}
		%		\label{pagcorlocation}
	\end{minipage}
	\hspace{0.05cm}
	\begin{minipage}[b]{0.5\linewidth}
		\centering
		\includegraphics[width=\textwidth]{figures/fig_ch043_suction_fittings2}
		\caption*{b - Sonera service pump suction}
		%		\label{ch01_pumpgallery}
	\end{minipage}
	\caption{Suction side}
	\label{fig_ch043_suction_fittings}
\end{figure}


It is important to observe good flow development for water as it enters the pump. As is always noted, a developed water inflow helps in the reducing vibration, cavitation, noise and many other problems that will occur in long term operation of the pump. In this regard, it is best to reinstall the pump and lengthen the suction pipe and reposition them. Follow recommended straight pipe provisions, resize the existing pipe diameters and refurbish the machine mounting and foundation to improve pump performance. on the other end the foundation blocks instead. Since in actual, the jockey pumps are not really needed, the foundation block for pump 1 might as well be refurbished. Even then, these changes are only quick fixes and do not really involve massive overhaul or redesign of the whole system but will definitely help improve the operating conditions. These are made more expedient with the results of the vibration analyzer tests presented in Section \ref{45}.



%The foundation blocks are longer than the length of the pump and motor couple for those installed on the main storage pump slots. 
%See conceptual design for pipe rerouting and corresponding adjustments on the pump system.
%It is not beneficial to install y-strainers on the nearby suction of the pump as it will disturb the flow development and increase turbulence of water entering the pump. Currently, this fitting is used by the operator to free air in the system during priming/starting Pumps 1 and 2. Pumps 3 and 4 have different suction piping. 

%The conceptual design proposes offsetting the pump and a provision of straight pipe for the offset. This will allow the installation of air valve and will remove the need for the y-strainer and further improve flow profile. 

Parameter monitoring opportunities. The longer suction run will also allow use of flow meters for individual pumps. Current spots for tapping flow meter probes are not very convenient for both device accuracy and technician access. 

(Note that these recommendations are not oriented towards total plant redesign but only for addressing existing plant problems and providing possible solution as the consultant see fit. If total plant redesign is desired, it will have to be done in a separate contract and will likely disregard the most if not all of the current problems stated.)

Other pertinent observations include the following:

\begin{itemize}
	\item  The motors lack frame ground wires (Figure \ref{fig_ch043_mtrgrounding}). The motor frames should be bonded to the station grounding bus, If there are any, the grounding design/policies for this station should be reviewed.
	\item Defective valves – valves have already locked up either due to age and/or corrosion (Figure \ref{fig_ch043_valve1A}).
	\item Valve not easily accessible - located above head level and require ladder to access (Figure \ref{fig_ch043_valvesabovehead})
	\item Clogged tapping point -  during testing, clogged pressure tapping points and pressure gauges on the same line were found.
	\item Inconveniently placed gauges (Figure \ref{gauges_at_header})
	\item Cluttered Electric wires (Figure \ref{fig_ch043_wireclutters})
	\item Minor Deteriorations (Figure \ref{otherdamages})
	\item Exhaust fan not easily accessible for cleaning or repair (Figure \ref{fig_ch043_exfan})
	\item Inconvenient blank nameplate (Figure \ref{fig_ch043_nonameplate})
	
\end{itemize}

\begin{figure}[!htb]
	\begin{minipage}[b]{0.3\linewidth}
		\centering
		\includegraphics[width=\textwidth]{figures/fig_ch043_mtrgrounding}
		\caption{no  frame ground wires}
		\label{fig_ch043_mtrgrounding}
	\end{minipage}
	\hspace{0.05cm}
	\begin{minipage}[b]{0.3\linewidth}
		\centering
		\includegraphics[width=\textwidth]{figures/fig_ch043_valve1A}
		\caption{defective valve}
		\label{fig_ch043_valve1A}
	\end{minipage}
	\hspace{0.05cm}
	\begin{minipage}[b]{0.3\linewidth}
		\centering
		\includegraphics[width=\textwidth]{figures/fig_ch043_valvesabovehead}
		\caption{over-head valve location}
		\label{fig_ch043_valvesabovehead}
	\end{minipage}
\end{figure}

\begin{figure}[!htb]
	\begin{minipage}[b]{0.3\linewidth}
		\centering
		\includegraphics[width=\textwidth]{figures/fig_ch043_gauges1}
		\caption*{a}
	\end{minipage}	
	\hspace{0.05cm}
	\begin{minipage}[b]{0.3\linewidth}
		\centering
		\includegraphics[width=\textwidth]{figures/fig_ch043_gauges2}
		\caption*{b}
	\end{minipage}
	
	\caption{Different gauges tapped at header}
	\label{gauges_at_header}
\end{figure}

\begin{figure}
	\begin{minipage}[b]{0.3\linewidth}
		\centering
		\includegraphics[width=\textwidth]{figures/fig_ch043_tank_crack}
		\caption*{a}
		%		\label{pagcorlocation}
	\end{minipage}
	\hspace{0.05cm}
	\begin{minipage}[b]{0.3\linewidth}
		\centering
		\includegraphics[width=\textwidth]{figures/fig_ch043_reservoir_leakage}
		\caption*{b}
		%		\label{ch01_pumpgallery}
	\end{minipage}
	
	\caption{Other deterioration}
	\label{otherdamages}
\end{figure}	



\begin{figure}[!htb]	
	\begin{minipage}[b]{0.3\linewidth}
		\centering
		\includegraphics[width=\textwidth]{figures/fig_ch043_exfan}
		\caption{exhaust fan}
		\label{fig_ch043_exfan}
	\end{minipage}
	\hspace{0.05cm}
	\begin{minipage}[b]{0.3\linewidth}
		\centering
		\includegraphics[width=\textwidth]{figures/fig_ch043_nonameplate}
		\caption{blank nameplate}
		\label{fig_ch043_nonameplate}
	\end{minipage}
	\hspace{0.05cm}
	\begin{minipage}[b]{0.3\linewidth}
		\centering
		\includegraphics[width=\textwidth]{figures/fig_ch043_wireclutters}
		\caption{wire clutters}
		\label{fig_ch043_wireclutters}
	\end{minipage}
\end{figure}



It is acknowledged that the station is already old and will suffer from issues typical to deterioration with time. However, it is necessary to do refurbishments to meet desired level of performance of the station.

As built discrepancy

Differences in as-builts are observed for the machine foundation. Initially, there was only one discharge header where all pumps would deliver water (Figure \ref{fig_ch043_asbuilt_layout} - a). However, there are two headers, one to Southvale and the other to Sonera. Old as-built show that there are 7 slots with 2 outer jockey pumps and the rest in between the main storage pumps (Figure \ref{fig_ch043_asbuilt_layout} - b). However, there are only 5 slots for main storage pumps and 1 slot for the jockey pump. Installed on the jockey pump slot is Southvale service pump1. Pumps 2 to 4 are installed on the main storage pump slots. Corresponding piping adjustments then follows.

\begin{figure}[!htb]
	\begin{minipage}[b]{0.5\linewidth}
		\centering
		\includegraphics[width=\textwidth]{figures/fig_ch043_asbuilt_layout1}
		\caption*{a - Pumps and pipings}
		%		\label{pagcorlocation}
	\end{minipage}
	\hspace{0.05cm}
	\begin{minipage}[b]{0.3\linewidth}
		\centering
		\includegraphics[width=\textwidth]{figures/fig_ch043_asbuilt_layout2}
		\caption*{b - Pump slots}
		%		\label{ch01_pumpgallery}
	\end{minipage}
	\caption{Old as-built plan}
	\label{fig_ch043_asbuilt_layout}
\end{figure}
%\end{document}