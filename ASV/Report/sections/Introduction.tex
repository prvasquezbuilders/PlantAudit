% Chapter 1

\chapter{Introduction} % Write in your own chapter title
\label{Chapter1}
%\lhead{Chapter 1. \emph{Introduction}} % Write in your own chapter title to set the page header
%%%

\section{General introduction} \label{namdario}
The station is located in Ayala Alabang, Muntinlupa (refer to Figure \ref{fig_ch01_intro_psr_loc} - a and Figure \ref{fig_ch01_intro_psr_loc} - b). The station services two areas namely, Southvale and Sonera. Two 10HP long coupled pumps (Figure \ref{fig_ch01_intro_pump_gallery} - a) deliver to Southvale and another two 15HP close coupled pumps (Figure \ref{fig_ch01_intro_pump_gallery} - b) deliver water for Sonera area. Built in 1985, this is one of the stations with the smallest average power consumption of only 7,841 kW per month.

\begin{figure}[!htb]
	\begin{minipage}[b]{0.5\linewidth}
		\centering
		\includegraphics[ height=3.75cm, width=7.5cm]{figures/fig_ch01_intro_psr_loc2}
		\caption*{a - [$14^{\circ}23^{'}55.86.74^{''}N, 121^{\circ}0^{'}5.98^{''}E$]}
		%		\label{fig_ch01_intro_psr_loc1}
	\end{minipage}
	\hspace{0.05cm}
	\begin{minipage}[b]{0.5\linewidth}
		\centering
		\includegraphics[width=\textwidth]{figures/fig_ch01_intro_psr_loc3}
		\caption*{b - Pump House}
		%		\label{ch01_pumpgallery}
	\end{minipage}
	\caption{ASV PSR]}
	\label{fig_ch01_intro_psr_loc}
\end{figure}

\begin{figure}[!htb]
	\begin{minipage}[b]{0.5\linewidth}
		\centering
		\includegraphics[ height=4.2cm, width=7.5cm]{figures/fig_ch01_intro_pump_gallery1}
		\caption*{a - Southvale service pumps}
		%		\label{pagcorlocation}
	\end{minipage}
	\hspace{0.05cm}
	\begin{minipage}[b]{0.5\linewidth}
		\centering
		\includegraphics[width=\textwidth]{figures/fig_ch01_intro_pump_gallery2}
		\caption*{b - Sonera service pumps}
		%		\label{ch01_pumpgallery}
	\end{minipage}
	\caption{Pump gallery}
	\label{fig_ch01_intro_pump_gallery}
\end{figure}


This PS has been included by the Client in the first cluster of pump stations for systems audit, benchmark establishment and asset management.

The Client has therefore awarded GHD and its sub-consultants (RB Sanchez and APSI) to conduct a plant audit project with an expectation to establish rigorous asset management framework based on reliability study and to determine optimal intervention program for the next five (5) years.

%%%
\section{Objectives}
%%%%%%%%%%%
The objectives of this work are as follows
\begin{itemize}
	\item To evaluate the current operating condition of PS as compared to the original design intent and to provide recommendations for improving the operational efficiency and lowering operating cost;
	\item To be able to determine an optimal intervention program for the PS in the next 5 years with reference to the recommendations from the assessment and audit based on life cycle cost; and equipment efficiency study whether the equipment is subjected to replacement or repair. These equipment are:
	\begin{itemize}
		\item[$\circ$] Pumps;
		\item[$\circ$] Motors;
		\item[$\circ$] Generators;
		\item[$\circ$] Electrical System and Protective Device;
		\item[$\circ$] Substation (Transformer, Switchgears);
		\item[$\circ$] MCC (VFDs, Soft starters, Circuit Breakers, and Protective Devices);
		\item[$\circ$] Motorize Valves.	
	\end{itemize}
\end{itemize}

\section{Scope of Work}
Scope of Work (SOW) has been defined in the Contract Agreement and be in compliance with the GHD technical and financial proposal and the agreements made during a number of project meetings (refer to minutes of meeting of the project). 

%{\color{red}
%IT IS IMPORTANT TO NOTE THAT THIS REPORT ONLY CONTAINS THE ANALYSIS AND RECOMMENDATION FOR FDAS AND ELECTRICAL AUDIT. THE REPORT DOES NOT INCLUDE A SECTION ON INTEGRITY TEST, WHICH SHALL BE INCLUDED IN THE FUTURE REVISION. INTEGRITY TEST HAS BEEN DEFERRED DUE TO OPERATIONAL CONSTRAINT. 
%
%PRELIMINARY REPORT ON MECHANCIAL TESTS HAS BEEN SUBMITTED TO THE CLIENT IN REVISION NUMBER OP18REFCS03-GHD-ASV-REP-G002A.
%}

\section{Limitations}
Results of the study with analysis, conclusion, and recommendations are only within the scope of work and agreements, and particularly under the following major constraints:
\begin{itemize}
\item Operational constraint: It was not possible to shutdown the entire PS for visual inspection of assets, particularly mechanical assets;
\item Incomplete historical data: It was a matter of fact that Maynilad has not established an asset management system, thus data regarding historical intervention is limited and incomplete, leading to non-optimal reliability analysis; 
\end{itemize}



\section{Glossaries}
Following glossaries are defined and used in the report:

\paragraph{\textbf{Level of Services (LOS)}}
A Level of Services (LOS) is any value or expectation of asset managers and beneficiers regarding the functionality and serviability of an asset of a system of assets.

\paragraph{\textbf{Intervention}}
Intervention is a generic and global term used to refer to non-physical and physical activities on assets. It encompasses do-nothing, or do somethings like repair, maintenance, rehabilitation, renewal, investment, and inspection and testing.

\paragraph{\textbf{Corrective Intervention (CI)}}
A Corrective Intervention (CI) is an intervention executed without proper and systematic plan. An CI is often incurred by failure/breakdown of assets. In most of cases, it incurs significant negative impacts (e.g. cost to repair, disruption of service, loss in revenue).

\paragraph{\textbf{Preventive Intervention (PI)}}
A Preventive Intervention (PI) is an intervention executed with proper and systematic plan. Note that an PI is executed on asset that is still working but not provide adequate level of services.

\paragraph{\textbf{Intervention Type}}
An Intervention Type (IT) is a specific and well-defined type of work/task that can be executed on/for an asset (e.g. replacement of a bearing for a pump).

\paragraph{\textbf{Intervention Strategy (IS)}}
An Intervention Strategy (IS) is a set/collection of intervention types.

\paragraph{\textbf{Intervention Program (IP)}}
An Intervention Program (IP) is a set/collection of intervention strategies for one asset or more than one assets of the same system.

\paragraph{\textbf{Work Program (WP)}}
A Work Program (WP) is an execution program consisting of Intervention Program and management program (e.g. project management, procurement) that shall be implemented in order to realize/actualize the Intervention Program.


