\chapter{Data and Analysis} \label{Chapter4}
%\lipsum[1-1]
\section{Qualitative and Operational Analysis}
\label{42}
\subsection{Facts and Data}
Summary of facts and data concerning operational and overall plan reliability is presented in this subsection.

\subsubsection{Normal Operation Scenario}
\begin{itemize}
\item 6 booster pumps are available but only 5 are in operation. The booster pumps deliver to the Cavite side of the distribution system including parts of Pasay. Distribution pipe size is 1400mm. 
\item A separate system for the Reclamation Area is supplied by the storage pumps. The line size is 700mm. Only one storage pump is required for the Reclamation Area distribution system. The other pump serves as standby. 
\item At 10pm-12mn, the MOV to the Reclamation Area distribution system is closed and the 2 storage pumps are used to refill the 2 storage tanks. During this period, the individual consumers are using their own stored water supply for internal use. Normal operations resumes at 12mn.



\end{itemize}

\subsubsection{High Demand Scenario}

\begin{itemize}
	\item all 6 booster pumps are in operation
\end{itemize}

\subsubsection{Low Demand Scenario}

\begin{itemize}
	\item only 5 booster pumps are running
\end{itemize}

\subsubsection{Spares Policy}

\begin{itemize}
	\item The booster pumps do not have a spare during high demand scenario. If 1 pump will be under maintenance during this period, there will be a drop in pressure and some of the consumers in the outermost section of the distribution system will experience low pressure or will not get any water. 
	\item The storage pumps also do not have a spare during the refilling phase. If a failure of one of the storage pumps occurs, the refilling duration will be much longer to complete.
\end{itemize}

\subsubsection{Emergency Situation (loss of electrical power from Meralco)}
\begin{itemize}
	\item Immediately after the power is cut, one of the 2 gensets automatically start and makes electric power available for the booster and/or supply pumps. The operator resets all the pumps before restarting each manually. This may take between 60 seconds to 5 minutes before all the pumps provide enough pressure/flow to the system.
\end{itemize}

\subsubsection{Maintenance}
\begin{itemize}
	\item There is no structured maintenance program in the facilities. The operator makes rounds of the pumps and if something unusual is observed, a text or email is sent to the Control Center for scheduling of maintenance check and repairs.
	\item There is also a maintenance team visiting the site regularly with specific lists of tasks and responsibilities.
\end{itemize}

\subsubsection{Others}
\begin{itemize}
	\item Water Leakages are found in the engineering room during rains.
\end{itemize}

\subsection{Recommendations}
In order to ensure the PS to provide adequate level of services around the clock, it is important to establish a good operational scheme that allows optimization of utilization of pumps to reduce breakdown and conserve energy. A summary of major recommendations to be considered are
\begin{itemize}
\item Adding an additional booster pump to improve the overall reliability, particularly for the major distribution line (1'400mm). This additional pump will allows switching of pump usage, for instance, with one pump undergoing major repairs, the arrangement will not allow for another pump to fail, otherwise, there will be a major in pressure with only 5 booster pumps running during peak demands;
\item Adding an additional storage pump to increase the reliability of the reclamation area distribution system (700 mm line);
\item Establish a scheme to consider a dedicated duty and a dedicated spare set-up for the pumps. If this is not acceptable, then consider doing a much longer switch of the storage pumps. Currently, it is being switched daily to supply 700mm distribution system. This allows for almost an equal rate of deterioration between the two pumps and if one pump fails due to age-related component failure, the other one is close to a similar failure which may occur before the first pump is fully repaired. It is suggested that the switch happen once a month or even every 3 months;
\item In place of the longer switching cycle (e.g. every 3 months), there should be a corresponding maintenance program for the standby pump for both booster and storage;

\item Develop a more structured discipline in applying routine maintenance work process to ensure that maintenance tasks are given the proper priority in terms of mitigation measures and avoid unplanned shutdown of critical pumps in operation.
\end{itemize}

Aside from the above recommendations, we also generate a list of recommendations based on the RCM methodology. The list shall be considered as a living program, which requires continuously improvement as part of the total quality management system (refer to Deming cycle presented in GHD's technical proposal).


\section{Pump discharge and suction pipe - thickness} \label{ch04mech01}
\subsection{Data and measurement}
Thickness data on discharge and suction pipes of pumps is presented in Table \ref{ch04_tbl_thickness02} and Table \ref{ch04_tbl_thickness03} .

\begin{table}[h]
	\caption{Thickness data - Booster Pumps (mm).}
	\label{ch04_tbl_thickness02}
	{\footnotesize
\begin{tabular}{l|c|c|c|c|c|c|c|c}
	\hline
	Asset & Position & \multicolumn{7}{c}{Distance} \\ 
	\cline{3-9}
	&  & \multicolumn{4}{c|}{Suction} & \multicolumn{3}{c}{Discharge} \\ 
	\cline{3-9}
	&  & 2.5m & 3.5m & 4.5m & Elbow & 4m & 5m & Elbow \\ 
	\hline
	BP1 & 12 & 4.68 & 4.68 & 4.66 & 4.59 & 4.60 & 4.63 & 3.94 \\ 
	& 3 & 4.62 & 4.75 & 4.79 & 4.48 & 4.69 & 4.67 & 3.92 \\ 
	& 6 & 4.69 & 4.74 & 4.69 & 4.63 & 4.73 & 4.60 & 4.64 \\ 
	& 9 & 4.71 & 4.74 & 4.71 & - & 4.71 & 4.72 & - \\ 
	\hline
	BP2 & 12 & 4.68 & 4.41 & 4.69 & 4.66 & 4.73 & 4.92 & 4.93 \\ 
	& 3 & 4.76 & 4.67 & 4.80 & 4.32 & 4.91 & 4.22 & 4.98 \\ 
	& 6 & 4.69 & 4.46 & 4.76 & 4.69 & 4.70 & 4.94 & 4.62 \\ 
	& 9 & 4.65 & 4.78 & 4.69 & - & 4.88 & 4.90 & - \\ 
	\hline
	BP3 & 12 & 4.61 & 4.67 & 4.65 & 4.69 & 4.71 & 4.69 & 4.62 \\ 
	& 3 & 4.66 & 4.69 & 4.75 & 4.72 & 4.76 & 4.79 & 4.73 \\ 
	& 6 & 4.68 & 4.63 & 4.66 & 4.67 & 4.68 & 4.70 & 4.61 \\ 
	& 9 & 4.65 & 4.69 & 4.64 & - & 4.67 & 4.74 & - \\ 
	\hline
	BP4 & 12 & 4.63 & 4.77 & 4.75 & 4.67 & 4.91 & 4.98 & 4.37 \\ 
	& 3 & 4.65 & 4.75 & 4.70 & 4.69 & 4.90 & 4.92 & 4.97 \\ 
	& 6 & 4.75 & 4.54 & 4.71 & 4.83 & 4.93 & 4.95 & 4.75 \\ 
	& 9 & 4.62 & 4.67 & 4.58 & - & 4.98 & 4.92 & - \\ 
	\hline
	BP5 & 12 & 4.80 & 4.69 & 4.49 & 4.73 & 4.72 & 4.68 & 4.76 \\ 
	& 3 & 4.71 & 4.56 & 4.71 & 4.67 & 4.86 & 4.65 & 4.76 \\ 
	& 6 & 4.75 & 4.75 & 4.60 & 4.67 & 4.75 & 4.64 & 4.86 \\ 
	& 9 & 4.66 & 4.65 & 4.67 & - & 4.67 & 4.74 & - \\ 
	\hline
	BP6 & 12 & 4.64 & 4.66 & 4.77 & 4.09 & 4.95 & 4.96 & 4.92 \\ 
	& 3 & 4.65 & 4.74 & 4.72 & 4.46 & 4.93 & 4.87 & 4.98 \\ 
	& 6 & 4.66 & 4.77 & 4.71 & 4.69 & 4.86 & 4.88 & 4.60 \\ 
	& 9 & 4.75 & 4.63 & 4.70 & - & 4.81 & 4.79 &  \\ 
	\hline
\end{tabular}

	}
\end{table}


\begin{table}[h]
	\caption{Thickness data - Storage Pumps (mm).}
	\label{ch04_tbl_thickness03}
	{\footnotesize
\begin{tabular}{l|c|c|c|c|c|c|c|c|l|l}
	\hline
	Asset & Position & \multicolumn{9}{c}{Distance} \\ 
	\cline{3-11}
	&  & \multicolumn{4}{c|}{Suction} & \multicolumn{5}{c}{Discharge} \\ 
	\cline{3-11}
	&  & 3m & 5m & 6m & Elbow & 4m & 6m & 8m & \multicolumn{1}{c|}{9m} & \multicolumn{1}{c}{Elbow} \\ 
	\hline
	SP1 & 12 &  &  &  & 4.84 & 4.78 & 4.76 & 4.74 & \multicolumn{1}{c|}{4.77} & \multicolumn{1}{c}{4.61} \\ 
	& 3 &  &  &  & 4.54 & 4.61 & 4.78 & 4.64 & \multicolumn{1}{c|}{4.72} & \multicolumn{1}{c}{4.77} \\ 
	& 6 &  &  &  & 4.4 & 4.25 & 4.81 & 4.8 & \multicolumn{1}{c|}{4.64} & \multicolumn{1}{c}{4.66} \\ 
	& 9 &  &  &  &  & 4.73 & 4.88 & 4.65 & \multicolumn{1}{c|}{4.72} & \multicolumn{1}{c}{-} \\ 
	\hline
	SP2 & 12 & 5.4 & 5.11 & 5.35 & 5.43 & 4.6 & 4.75 & 4.97 & \multicolumn{1}{c|}{4.74} & \multicolumn{1}{c}{4.91} \\ 
	& 3 & 5.28 & 5.12 & 5.3 & 5.41 & 4.62 & 4.78 & 4.78 & \multicolumn{1}{c|}{4.66} & \multicolumn{1}{c}{4.64} \\ 
	& 6 & - & 5.09 & 5.45 & 5.4 & 4.43 & 4.18 & 4.73 & \multicolumn{1}{c|}{4.75} & \multicolumn{1}{c}{4.51} \\ 
	& 9 & - & 5.2 & 5.22 & - & 4.4 & 4.75 & 4.45 & \multicolumn{1}{c|}{4.58} & \multicolumn{1}{c}{-} \\ 
	\hline
\end{tabular}

	}
\end{table}

In the table, the positions and the distances for the Ultrasonic Thickness Gauging (UTG) are referred to Figure \ref{ch04_fig_utgbp} and Figure \ref{ch04_fig_utgsp}.

\begin{figure}[!htb]
	\includegraphics[scale=1.3]{figures/ch04_fig_utgbp} \\
	\caption{Positions and distances of UTG - Booster Pump}
	\label{ch04_fig_utgbp} 
\end{figure}

\begin{figure}[!htb]
	\includegraphics[scale=1.3]{figures/ch04_fig_utgsp} \\
	\caption{Positions and distances of UTG - Storage Pump}
	\label{ch04_fig_utgsp} 
\end{figure}


%\begin{table}[h]
%	\caption{Thickness data (mm).}
%	\label{thicknessdata}
%	{\footnotesize
%	\begin{tabular}{l|l|l|l|l}
%		\hline
%		Pumps & \multicolumn{2}{c|}{Suction} & \multicolumn{2}{c}{Discharge} \\ 
%		\cline{2-5}
%		& \multicolumn{1}{c|}{Design} & \multicolumn{1}{c|}{Actual} & \multicolumn{1}{c|}{Design } & \multicolumn{1}{c}{Actual} \\ 
%		\hline
%		BP1 & \multicolumn{1}{c|}{} & \multicolumn{1}{c|}{4.98} & \multicolumn{1}{c|}{} & \multicolumn{1}{c}{3.92} \\ 
%		BP2 & \multicolumn{1}{c|}{} & \multicolumn{1}{c|}{4.32} & \multicolumn{1}{c|}{} & \multicolumn{1}{c}{4.22} \\ 
%		BP3 & \multicolumn{1}{c|}{} & \multicolumn{1}{c|}{4.61} & \multicolumn{1}{c|}{} & \multicolumn{1}{c}{4.61} \\ 
%		BP4 & \multicolumn{1}{c|}{} & \multicolumn{1}{c|}{4.54} & \multicolumn{1}{c|}{} & \multicolumn{1}{c}{4.37} \\ 
%		BP5 & \multicolumn{1}{c|}{} & \multicolumn{1}{c|}{4.49} & \multicolumn{1}{c|}{} & \multicolumn{1}{c}{4.64} \\ 
%		BP6 & \multicolumn{1}{c|}{} & \multicolumn{1}{c|}{4.09} & \multicolumn{1}{c|}{} & \multicolumn{1}{c}{4.60} \\ 
%		SP1 & \multicolumn{1}{c|}{} & \multicolumn{1}{c|}{4.40} & \multicolumn{1}{c|}{} & \multicolumn{1}{c}{4.25} \\ 
%		SP2 & \multicolumn{1}{c|}{} & \multicolumn{1}{c|}{5.09} & \multicolumn{1}{c|}{} & \multicolumn{1}{c}{4.18} \\ 
%		\hline
%	\end{tabular}
%			
%	}
%\end{table}

%Detailed measurement data is provided in appendix \ref{appthicknesss}.
%\textcolor{red}{RB Sanchez to write here the summary of raw data collected from visual inspection and testing. Tables shall be used as much as we can. Note that no analysis in this session. This session is purely the high level presentation of data. Raw data can be linked as an Appendix}

\subsection{Analysis} 
This section provides analysis/discussion on estimation of minimum allowable thickness of pipes and statistics around the measured data collected during inspection and testings.

\subsubsection{Statistics} \label{ch05_statistics}
A summary on statistics regarding the measured thickness for booster pumps and storage pumps is presented in Table \ref{ch05_tbl_thicknesssta}.
\begin{table}[h]
	\caption{Summary of statistics - thickness.}
	\label{ch05_tbl_thicknesssta}
	{\footnotesize
\begin{tabular}{l|l|l|l|l|l|l|l|l}
	\hline
	Statistics & \multicolumn{2}{c|}{Booster suction} & \multicolumn{2}{c|}{Booster discharge} & \multicolumn{2}{c|}{Storage suction} & \multicolumn{2}{c}{Storage discharge} \\ 
	\cline{2-9}
	& \multicolumn{1}{c|}{Straight} & \multicolumn{1}{c|}{Elbow} & \multicolumn{1}{c|}{Straight} & \multicolumn{1}{c|}{Elbow} & \multicolumn{1}{c|}{Straight} & \multicolumn{1}{c|}{Elbow} & \multicolumn{1}{c|}{Straight} & \multicolumn{1}{c}{Elbow} \\ 
	\hline
	Min & \multicolumn{1}{c|}{4.410} & \multicolumn{1}{c|}{4.090} & \multicolumn{1}{c|}{4.220} & \multicolumn{1}{c|}{3.920} & \multicolumn{1}{c|}{5.090} & \multicolumn{1}{c|}{4.400 } & \multicolumn{1}{c|}{4.180} & \multicolumn{1}{c}{4.510 } \\ 
	1st Qua. & \multicolumn{1}{c|}{4.650} & \multicolumn{1}{c|}{4.600} & \multicolumn{1}{c|}{4.690} & \multicolumn{1}{c|}{4.612} & \multicolumn{1}{c|}{5.140} & \multicolumn{1}{c|}{4.615 } & \multicolumn{1}{c|}{4.617} & \multicolumn{1}{c}{4.617} \\ 
	Median & \multicolumn{1}{c|}{4.690} & \multicolumn{1}{c|}{4.670} & \multicolumn{1}{c|}{4.755} & \multicolumn{1}{c|}{4.740 } & \multicolumn{1}{c|}{5.250 } & \multicolumn{1}{c|}{5.120} & \multicolumn{1}{c|}{4.730} & \multicolumn{1}{c}{4.650} \\ 
	Mean & \multicolumn{1}{c|}{4.681} & \multicolumn{1}{c|}{4.608} & \multicolumn{1}{c|}{4.782} & \multicolumn{1}{c|}{4.664} & \multicolumn{1}{c|}{5.252} & \multicolumn{1}{c|}{5.003} & \multicolumn{1}{c|}{4.670} & \multicolumn{1}{c}{4.683 } \\ 
	3rd Qua. & \multicolumn{1}{c|}{4.740} & \multicolumn{1}{c|}{4.690} & \multicolumn{1}{c|}{4.910 } & \multicolumn{1}{c|}{4.905} & \multicolumn{1}{c|}{5.338} & \multicolumn{1}{c|}{5.407} & \multicolumn{1}{c|}{4.772} & \multicolumn{1}{c}{4.742 } \\ 
	Max & \multicolumn{1}{c|}{4.800} & \multicolumn{1}{c|}{4.830 } & \multicolumn{1}{c|}{4.980} & \multicolumn{1}{c|}{4.980} & \multicolumn{1}{c|}{5.450} & \multicolumn{1}{c|}{5.430} & \multicolumn{1}{c|}{4.970} & \multicolumn{1}{c}{4.910} \\ 
	\hline
\end{tabular}
	}
\end{table}

Table \ref{ch05_tbl_thicknessextra} shows the summary of statistics for individual pump with extrados thickness calculated.
\begin{table}[h]
	\caption{Summary of statistics - thickness (individual pump).}
	\label{ch05_tbl_thicknessextra}
	{\footnotesize
\begin{tabular}{l|l|l|l|l|l|l|l|l}
	\hline
	\multicolumn{1}{c|}{Assets} & \multicolumn{4}{c|}{Suction (mm)} & \multicolumn{4}{c}{Discharge (mm)} \\ 
	\cline{2-9}
	\multicolumn{1}{c|}{} & \multicolumn{1}{c|}{Min} & \multicolumn{1}{c|}{Mean} & \multicolumn{1}{c|}{Extrados} & \multicolumn{1}{c|}{Max} & \multicolumn{1}{c|}{Min} & \multicolumn{1}{c|}{Mean} & \multicolumn{1}{c|}{Extrados} & \multicolumn{1}{c}{Max} \\ 
	\hline
	\multicolumn{1}{c|}{BP1} & \multicolumn{1}{c|}{4.480} & \multicolumn{1}{c|}{4.677} & \multicolumn{1}{c|}{4.608} & \multicolumn{1}{c|}{4.790} & \multicolumn{1}{c|}{3.920} & \multicolumn{1}{c|}{4.532} & \multicolumn{1}{c|}{4.346} & \multicolumn{1}{c}{4.730} \\ 
	\multicolumn{1}{c|}{BP2} & \multicolumn{1}{c|}{4.320} & \multicolumn{1}{c|}{4.647} & \multicolumn{1}{c|}{4.554} & \multicolumn{1}{c|}{4.800} & \multicolumn{1}{c|}{4.220} & \multicolumn{1}{c|}{4.794} & \multicolumn{1}{c|}{4.836} & \multicolumn{1}{c}{4.980} \\ 
	\multicolumn{1}{c|}{BP3} & \multicolumn{1}{c|}{4.610} & \multicolumn{1}{c|}{4.671} & \multicolumn{1}{c|}{4.680} & \multicolumn{1}{c|}{4.750} & \multicolumn{1}{c|}{4.610} & \multicolumn{1}{c|}{4.700} & \multicolumn{1}{c|}{4.672} & \multicolumn{1}{c}{4.790} \\ 
	\multicolumn{1}{c|}{BP4} & \multicolumn{1}{c|}{4.540} & \multicolumn{1}{c|}{4.742} & \multicolumn{1}{c|}{4.742} & \multicolumn{1}{c|}{4.830} & \multicolumn{1}{c|}{4.370} & \multicolumn{1}{c|}{4.871} & \multicolumn{1}{c|}{4.796} & \multicolumn{1}{c}{4.980} \\ 
	\multicolumn{1}{c|}{BP5} & \multicolumn{1}{c|}{4.490} & \multicolumn{1}{c|}{4.650} & \multicolumn{1}{c|}{4.674} & \multicolumn{1}{c|}{4.800} & \multicolumn{1}{c|}{4.640} & \multicolumn{1}{c|}{4.735} & \multicolumn{1}{c|}{4.756} & \multicolumn{1}{c}{4.860} \\ 
	\multicolumn{1}{c|}{BP6} & \multicolumn{1}{c|}{4.090} & \multicolumn{1}{c|}{4.534} & \multicolumn{1}{c|}{4.643} & \multicolumn{1}{c|}{4.770} & \multicolumn{1}{c|}{4.600} & \multicolumn{1}{c|}{4.868} & \multicolumn{1}{c|}{4.882} & \multicolumn{1}{c}{4.980} \\ 
	\multicolumn{1}{c|}{SP1} & \multicolumn{1}{c|}{4.400} & \multicolumn{1}{c|}{4.593} & \multicolumn{1}{c|}{4.593} & \multicolumn{1}{c|}{4.840} & \multicolumn{1}{c|}{4.250} & \multicolumn{1}{c|}{4.701} & \multicolumn{1}{c|}{4.710} & \multicolumn{1}{c}{4.880} \\ 
	\multicolumn{1}{c|}{SP2} & \multicolumn{1}{c|}{5.090} & \multicolumn{1}{c|}{5.289} & \multicolumn{1}{c|}{5.398} & \multicolumn{1}{c|}{5.450} & \multicolumn{1}{c|}{4.180} & \multicolumn{1}{c|}{4.644} & \multicolumn{1}{c|}{4.754} & \multicolumn{1}{c}{4.970} \\ 
	\hline
\end{tabular}

	}
\end{table}

Figures \ref{ch05_thickness_suction} and \ref{ch05_thickness_dicharge} show comparative graphs of min, mean, and max values of thickness.


\begin{figure}[!htb]
	\begin{minipage}[b]{0.5\linewidth}
		\centering
		\includegraphics[width=\textwidth]{figures/ch05_thickness_suction}
		\caption*{a - Straight} 
		%		\label{ch05_thickness_suction}
	\end{minipage}
	\hspace{0.05cm}
	\begin{minipage}[b]{0.5\linewidth}
		\centering
		\includegraphics[width=\textwidth]{figures/ch05_thickness_suctione}
		\caption*{b - Elbow} 
		%		\label{ch05_thickness_suction_e}
	\end{minipage}
	\caption{Suction line}
	\label{ch05_thickness_suction}
\end{figure}

\begin{figure}[!htb]
	\begin{minipage}[b]{0.5\linewidth}
		\centering
		\includegraphics[width=\textwidth]{figures/ch05_thickness_discharge}
		\caption*{a - Straight} 
		%		\label{ch05_thickness_suction}
	\end{minipage}
	\hspace{0.05cm}
	\begin{minipage}[b]{0.5\linewidth}
		\centering
		\includegraphics[width=\textwidth]{figures/ch05_thickness_dischargee}
		\caption*{b - Elbow} 
		%		\label{ch05_thickness_suction_e}
	\end{minipage}
	\caption{Discharge line}
	\label{ch05_thickness_dicharge}
\end{figure}

Followings are generic interpretation by examining the tables and graphs
\begin{itemize}
\item Mean value of thickness is above 4.6 mm;
\item Mean and median values are close, inferring a confidence on having less heterogeneity, i.e. distribution of thickness around the pipe is more or less homogeneous;
\item Thickness at elbow is less than that of the straight line;
\item THickness of storage line is likely to be higher than that of the discharge line;
\item BP1 has a value of 3.92 as min at the elbow, which requires attention from time to time.
\end{itemize}

\paragraph{\underline{BP1}}
\begin{itemize}
	\item Suction Piping System - The extrados thickness is less than the remaining thickness values measured.
	\item 	Discharge Piping System - The average extrados thickness is 4.346 mm, with a minimum of 3.94 at the entry area to the center of the extrados. Localized thinning is observed to be higher at the upper half of the discharge line (as seen in 4m mark, 12-3-9 o'clock position). Then extends to the elbow entry up to the center of extrados. This may be caused by possible cavitation carried over from the pump to the discharge side and then amplified by the backflows in the elbow.
\end{itemize}

\paragraph{\underline{BP2}}
\begin{itemize}
	\item Suction Piping System - The extrados thickness is less than the remaining thickness values measured. The 6-o'clock value at 3.5 meter is the backflow/eddie zone of the elbow, thus assuming to have high backflow rate in the suction.
	\item	Discharge Piping System - The flow from the pump enters the elbow at the lower half (6 o'clock of 4m mark) and swirls to the 3 o'clock position reference to the pump flange. It continued swirling to the sides and the exit extrados of the elbow. This is caused by the disturbance caused by the fittings in between.
\end{itemize}

\paragraph{\underline{BP3}}
\begin{itemize}
	\item Suction Piping System - The extrados thickness is approximately equal to all measured thickness. Also, it is seen in the 6-o'clock of 3.5 and 4.5m mark that it is slightly thinner that of its 12-o'clock positions. This may indicate a high backflow of water thus creating higher turbulency as a result. The high turbulency of water causes the coverage of the localized thinning larger extending up to the 2.5m mark from the pump flange.
	\item	Discharge Piping System - The flow enters the lower half of the intrados entry. The flow then continued at the extrados area with a considerate backflow at the intrados. The average extrados thickness indicates also that the extrados is thinner when compared to the rest of the elbow thickness.
\end{itemize}

\paragraph{\underline{BP4}}
\begin{itemize}
	\item Suction Piping System - The extrados thickness is slightly higher to all measured thickness. This indicates the flow of water in this pipe enters the elbow at between 9-o'clock position and 12-o'clock position with high backflow rate and eddies forming at the intrados area (6-o'clock of elbows). This pattern creates a swirling turbulent flow that results to larger localized wall thinning coverage extending to the upper portion of the pipe.
	\item	Discharge Piping System - The data indicates that the flow through the elbow based on thinning is that the flow enters the elbow entry and extending to the intrados area of the elbow, with small backflow rate (as seen in 3-6-9 o'clock position at 5m mark). It is then flows through the extrados exit again resulting to thinning with 4.75 mm thickness measured.
\end{itemize}

\paragraph{\underline{BP5}}
\begin{itemize}
	\item Suction Piping System - The extrados thickness is less than the remaining thickness values measured. The 3-o'clock thinning at 3.5m mark may indicate that the water flows from extrados to the sides, that causes water swirling entering the diffusers and pumps.
	\item	Discharge Piping System - The elbow middle extrados is where the localized thinning area occur at this pipe section. However, the average extrados thickness is almost the same compared to the rest of pipe measurements.
\end{itemize}

\paragraph{\underline{BP6}}
\begin{itemize}
	\item Suction Piping System - The extrados thickness is less than the remaining thickness values measured. The 4.09 mm thickness reading may indicate that the water flows faster in the upper section of the elbow exit and partially swirls extending to the 2.5m mark, thus having a uniform localized wall thinning at the right-half of the straight pipe.
	\item	Discharge Piping System - The thinnest part of the discharge pipeline is at the exit elbow extrados. However, the thickness is not critical compared to the other elbows. The component to be monitored is the elbow, yet is not critical.
\end{itemize}

\paragraph{\underline{SP1}}
\begin{itemize}
	\item Suction Piping System - The thickness data for suction is only at the extrados zone of the elbow. However, this shows the common thinning effect at elbows. The thinning measurement show that the flow inside the pipe is high that the water makes contact between the central extrados area extending to the exit extrados of the pipe. Noting that the points are of the same central angles, the thinning difference between the points, 0.30 mm and 0.10 mm shows faster thinning at the exit extrados of the elbow.
	\item	Discharge Piping System - This pipe is long enough to make the flow developed and not too turbulent before entering the elbow. However, the bottom (6 o'clock position) the 4m mark from the pipe flange has considerable localized thinning and is extended to the sides of the pipe. The thinning continued at the sides of the elbow entry (3-6 o'clock) of the 8 and upper half of the 9m mark. The thinning also indicates a backflow in the elbow's intrados extending to the elbow exit.
\end{itemize}

\paragraph{\underline{SP2}}
\begin{itemize}
	\item Suction Piping System - This pipe line as seen in the actual and in the plan is a larger pipe compared to that of the previous pipes. This pipe is not critical in terms of the current thickness.
	\item	Discharge Piping System - The discharge side of the pipe indicates possible cavitation is carried over. It is observed from the data having a uniform thinning at the pipe circumference. It shows that the water flows more at the left side (6-9 o'clock position reference to the pipe flange), then continued to the bottom of the pipe at 6m mark. The flow enters the elbow at the sides with considerable thinning due to backflow (as seen in the 3-9 o'clock at 9m mark). The thinning greatly eroded the exit part of the elbow.
\end{itemize}



\subsubsection{Assumptions}
Following assumptions are used in calculating the required thickness of pipe
\begin{itemize}
\item Maximum Working Head – based on the design drawings and pump nameplate;
\item Pipe Material – assume pipe material is ASTM A570 Grade 33 (market available material for spiral welded pipe);
\item Design Guide – basis used for the simulated calculation is AWWA  Manual M11 – Steel Pipe, A Guide for Design and Installation, 4th Edition. Statement for corrosion allowance is located at Chapter 4, which states \textit{"At one time, it was a general practice to add a fixed, rule-of-thumb thickness to the pipe wall as a corrosion allowance. This was not an applicable solution in the water work field, where standard for coating and lining materials and procedures exists. The design shall be made for the required wall-thickness pipe as determined by the loads imposed, then linings, coatings, and cathodic protection selected to provide the necessary corrosion protection"};
\item Thickness calculation will be based on the internal pressure. External pressure will not be considered because much of the discharge line is not buried.
\item 	Surge Pressure was not considered since there are surge protection along the line. 
\item 	This document will only consider the calculation of the minimum thickness along the discharge line since this is the part of the system where maximum pressure is experience.
\end{itemize}
\subsubsection{Limitations}
As confirmed by Maynilad, there is no available data regarding the design report. Design assumptions herein may be different from what was used by the designer/contractor of this station.

This document will not be able to provide the corrosion/degradation factor of the pipe since there is no available historical data on the thickness of the pipe.

\subsubsection{Parameter values for thickness estimation}
In order to estimate the minimum allowance thickness for pipes in straight line considering material handling ($t_{mh}$) and maximum internal pressure based on AWWA M11 ($t_{sph}$), following equations are used, respectively:

\begin{eqnarray}
&& t_{mh} = \frac{\Phi}{\delta} \label{ch05thickness01}
\end{eqnarray}

\begin{eqnarray}
&& t_{sp} = \frac{\epsilon\times P_{max} \times \Phi}{2 \times S_e} \label{ch05thickness02}
\end{eqnarray}
where $P_{max}$ is maximum internal pressure

\begin{eqnarray}
&& P_{max} = \frac{\rho_{H_2O} \times g \times H_{max}}{1000} \label{ch05thickness03}
\end{eqnarray}

In order to estimate the minimum allowance thickness for pipes at elbows (Miter Bend), only maximum internal pressure is considered:

\begin{eqnarray}
&& t_{mb} = \frac{P_{max} \times \Phi}{2 \times S_e} \times \left[ 1 + \frac{\Phi}{(3 \times R)-(1.5 \times \Phi_d)}\right]\times \epsilon \label{ch05thickness04}
\end{eqnarray}

Paramater values used for computation are given in Table \ref{ch05_tbl_thicknesscalc}

\begin{table}[h]
	\caption{Parameter values for thickness estimation.}
	\label{ch05_tbl_thicknesscalc}
	{\footnotesize
\begin{tabular}{p{4cm}|c|c|c|c|p{4cm}}
	\hline
	Parameters & Symbol & Unit & \multicolumn{2}{c|}{Pumps} & Remarks \\ 
	&  &  & Booster & Storage &  \\ 
	\hline
	Discharge diameter & $\Phi$ & $mm$ & 600 & 600 &  \\ 
	Max flow rate & $Q_{max}$ & $m^3/s$ & 0.64 & 0.53 &  \\ 
	Max pump head & $H_{max}$ & $m$ & 40 & 50 & based on name plate \\ 
	Yield strength of material & $S_y$ & $MPa$ & 227.5 & 227.5 & ASTM A570 Grade 33, spiral welded pipe based on AWWA C200 \\ 
	Allowable stress & $S_e$ & MPa & 113.75 & 113.75 &  \\ 
	Density of water & $\rho_{H_2O}$ & $kg/m^3$ & 1000 & 1000 &  \\ 
	Gravity constant & $g$ & $m/s^2$ & 9.81 & 9.81 &  \\ 
	Safety factor & $\epsilon$ &  & 2 & 2 &  \\ 
	Bulk modulus of compressibility of liquid & $k$ & $Pa$ & 2.1E+09 & 2.1E+09 &  \\ 
	Young's modulus of elasticity of pipe wall & $E$ & $Pa$ & 2.1E+11 & 2.1E+11 &  \\ 
	Radius of Elbow & $R$ & $mm$ & 800 & 800 &  \\ 
	Empirical constant & $\delta$ &  & 288 & 288 &  \\ 
	\hline
\end{tabular}
	}
\end{table}

\subsubsection{Required thickness}

Results of computation for minimum allowable thickness for booster pumps and storage pumps are given in Table \ref{ch05_tbl_thicknesscalcresult}.

\begin{table}[h]
	\caption{Minimum thickness allowance.}
	\label{ch05_tbl_thicknesscalcresult}
	{\footnotesize
		\begin{tabular}{l|l|p{3cm}|p{3cm}|p{3cm}}
			\hline
			Pumps & \multicolumn{1}{c|}{Internal pressure  (Mpa)} & \multicolumn{3}{c}{Minimum allowable thickness (mm)} \\ 
			\cline{3-5}
			& \multicolumn{1}{c|}{$P_{max}$} & \multicolumn{1}{c|}{$t_{mh}$} & \multicolumn{1}{c|}{$t_{sp}$} & \multicolumn{1}{c}{$t_{mb}$} \\ 
			\hline
			Booster & \multicolumn{1}{c|}{0.392} & \multicolumn{1}{c|}{2.080} & \multicolumn{1}{c|}{2.070} & \multicolumn{1}{c}{2.900} \\ 
			Storage & \multicolumn{1}{c|}{0.491} & \multicolumn{1}{c|}{2.080} & \multicolumn{1}{c|}{2.590} & \multicolumn{1}{c}{3.620} \\ 
			\hline
		\end{tabular}
		
	}
\end{table}

If comparing these values with the measured values of thickness shown in subsection \ref{ch05_statistics}, it can be concluded that current thickness of pipes at both suction and discharge still provide adequate level of services as the measured value is about 4.60 mm on average while the required values for booster pumps are less than 3 mm and for storage pump is less than 3.62 mm. 

However, it is important to note that required thickness at the elbow of storage pumps is 3.62 mm, which is not so far off from measured value of 4.60 mm. Especially, elbow section is significant important and shall not be at risk.

%\subsubsection{Deterioration prediction}
\subsubsection{Deterioration}
Given the lack of design data, precise design thickness is unknown, following assumptions are made

\begin{itemize}
\item Maximum measured thickness is considered to be the design thickness;
\item Pipe has been in operation for 9 years since 2010;
\item Deterioration rate is to follow linear function.
\end{itemize}

%Condition states used for pipe are defined in Table \ref{ch04:csthickness}

%\begin{table}[h]
%	\caption{Condition state definition - Pipe thickness.}
%	\label{ch04:csthickness}
%	{\footnotesize
%\begin{tabular}{l|l|l}
%	\hline
%	\multicolumn{1}{c|}{CS} & \multicolumn{2}{c}{Pipe thickness (mm)} \\ 
%	\cline{2-3}
%	\multicolumn{1}{c|}{} & Booster pump & Storage pump \\ 
%	\hline
%	\multicolumn{1}{c|}{1} & \multicolumn{1}{c|}{(4.564 - 4.980]} & (5.084 - 5.450] \\ 
%	\multicolumn{1}{c|}{2} & \multicolumn{1}{c|}{(4.184 - 4.564]} & (4.718 - 5.084] \\ 
%	\multicolumn{1}{c|}{3} & \multicolumn{1}{c|}{(3.732 - 4.184]} & (4.352 - 4.718] \\ 
%	\multicolumn{1}{c|}{4} & \multicolumn{1}{c|}{(3.316 - 3.732]} & (3.986 - 4.352] \\ 
%	\multicolumn{1}{c|}{5} & \multicolumn{1}{c|}{<=3.316} & \multicolumn{1}{c}{<=3.986} \\ 
%	\hline
%\end{tabular}
%	}
%\end{table}

%Note that the range of condition states is mapped from maximum observed thickness to minimum allowable thickness of pipes for booster pumps and storage pumps. 

A simplest way to predict the remaining duration till thickness of pipe reaching alarming level (minimum allowable thickness). This prediction is shown in Table \ref{ch04_thickness_predict}


\begin{figure}[!htb]
	\includegraphics[scale=0.6]{figures/ch04_thickness_predict} \\
	\caption{Thickness prediction}
	\label{ch04_thickness_predict} 
\end{figure}

Inferences from reading the figure are

\begin{itemize}
\item thickness of booster pipes, particularly at the elbow, will reach its minimum allowable thickness at years 18 (or in 2027);
\item thickness of storage pipes, particularly at the elbow, will reach its minimum allowable thickness at years 14 (or in 2023).
\end{itemize}

%It seems critical for pipes of storage pumps as the remaining years till reaching the minimum allowable thickness of 3.62 mm is about 4 years from now.  










\subsubsection{Recommendations}
Given the current thickness of pipe the lack of design information, it is advisable to 
\begin{itemize}
\item Not perform any major intervention on the pipes;
\item Keep regular testing on exact locations using the same type of UTG devices. It is important for Maynilad to establish a testing regime for obtaining thickness at exact same location over time (e.g. every year). Information obtained from testing will be then used to compute deterioration rate based on thickness value;
\item Establish an approach to inspect/test the thickness of underground pipe, which is considered to be more vulnerable to leakage and corrosion on external wall;
\item The elbows in the suction and the discharge piping systems must be monitored regularly especially BP1, BP2, BP5 and BP6.
\item	It is recommended to have a profiling of the piping systems above and below the ground in order to have a baseline in the analysis of the Maynilad Piping System. In order to have a profiling of pipe thickness at differential time T, additional measurement at similar locations shall be conducted periodically, behavior can then be monitored.
\item Perform coating regularly the pipe to prevent possible corrosion/errosion and damage that cause by external factors and surrounding condition;
\item Since the location is reclaimed area, it is advisable to set up a regime to regular inspection for the use of cathodic protection for all underground piping lines.
\end{itemize}

\section{Visual Inspection on Pipe, valves, fittings, supports, expansions, and appurtenances} \label{ch04mech02}
%\textcolor{red}{RB Sanchez to write here the summary of raw data collected from visual inspection and testing. Tables shall be used as much as we can. Note that no analysis in this session. This session is purely the high level presentation of data. Raw data can be linked as an Appendix}

\subsection{Highlights} \label{ch04mech02_highlight}

Visual inspection data on pipes, valves, fittings, supports, expansions, and appurtenances is highlighted in Table \ref{ch04_visualinspection01}.

\begin{table}[!htb]
	\caption{Highlights of visual inspection}
	\label{ch04_visualinspection01}
%	\resizebox{\columnwidth}{!}{%
	{\scriptsize
\begin{tabular}{c|p{2cm}|c|p{9.5cm}|l}
	\hline
	No. & Items & CS & Remarks & Ref \\ 
	\hline
	1 & Valve Leaks &  & Water leakages for moving parts of checked valves of pumps. Evident of prolonged leaks are shown by local corrosion and accumulation of water pools around valves's vincinity &  \\ 
	2 & Pressure gauges &  & Discrepancy in reading between the dial and digital gauges &  \\ 
	&  &  & Defective bourdon (dial) pressure gauges were found at suction side. No pressure gauses installed near pump suction nozzle and discharge flange inferring no ability to immediately read the head pressure. &  \\ 
	&  &  & Excessive deterioration/fading of tags making them unreadable &  \\ 
	&  &  & Some tags have been superimposed with recent data written by inappropriate markers that cause difficulty in reading. &  \\ 
	&  &  & Some tags found with inconsistency of data &  \\ 
	3 & Alignment bolts &  & without washers &  \\ 
	4 & Grounding cables &  & Not found for any motor &  \\ 
	5 & Vibration monitoring probes &  & Some were found disconnected or/and untended &  \\ 
	6 & Spare pumps &  & Susceptible to false brinelling (e.g. The spare pump located near BP1 may experience minor vibrations due to its location near the almost continuously operating pumps. There is no observed intermittent rotating of the pump shaft by the operators during the visit) &  \\ 
	7 & Piping stability/settlement &  & Possible execcisve level of movement leading to weakness of tensile/compressive strength of materials/fixtures. The entire pump house is suffering from ground settlement and that the alignments of the pumps and the corresponding fittings have been compromised. Mitigating supports have been installed but their functions are doubtful due to observed obscurities. Interviews with operators also confirm serious damages to pumps and increased vibration during operation because of the significant piping movements &  \\ 
	&  &  & Wall opening provisions for pipe movement due to ground settlement. These local tear downs leave the wall unaesthetic and inconsistently enclosed as some are covered with deteriorating plywood. Some round bars are left protruding and can pause danger &  \\ 
	&  &  & Crooked and/or drawn concrete saddle strap bolts. Some of the fastening area on the concrete saddle display small to large cracks &  \\ 
	&  &  & Evidences of axial suction piping movements &  \\ 
	&  &  & Space gap between saddle arc and pipe &  \\ 
	&  &  & Doubtful piping supports. The counter action of the U supports (both round and flat type) are doubtful. The U supports function to counter the water thrust on the elbow as immense water volumes pass thru it. The bolts are not turned to tighten the flat bar U support. Space between the U supports and the pipes are observed &  \\ 
	8 & Motion actuated lighting &  & Interview with the maintenance team revealed that the motion actuated lighting sometimes causes slight nausea due to dim lighting when repair. The minute motions of repair are sometimes not enough to actuate the lights and thus interrupt the work &  \\ 
	\hline
\end{tabular}
	}%}
\end{table}

Visual inspections are supported with the photos taken at particular locations/positions in questions.

It was realized that the entire pump house is suffering from ground settlement and that the alignments of pumps and corresponding fittings have been compromised. For example, as can be shown in Figure \ref{ch04_settlement}-a, the portion of wall through which the pipe passes thru has been torn down partially to provide allowance for settlement as the pipe levels going down.

\begin{figure}[!htb]
	\begin{minipage}[b]{0.5\linewidth}
		\centering
		\includegraphics[width=\textwidth]{figures/ch04_settlement}
		\caption*{a - wall opening}
	%	\label{pagcorlocation}
	\end{minipage}
	\hspace{0.05cm}
	\begin{minipage}[b]{0.5\linewidth}
		\centering
		\includegraphics[width=\textwidth]{figures/ch04_settlement01}
		\caption*{b - Support damage}
	%	\label{ch01_pumpgallery}
	\end{minipage}
	\hspace{0.05cm}
\begin{minipage}[b]{0.5\linewidth}
	\centering
	\includegraphics[width=\textwidth]{figures/ch04_settlement02}
	\caption*{c - Gap between saddle support and pipe}
	%	\label{ch01_pumpgallery}
\end{minipage}
	\hspace{0.05cm}
\begin{minipage}[b]{0.5\linewidth}
	\centering
	\includegraphics[width=\textwidth]{figures/ch04_fig_visual03}
	\caption*{d - common leakage}
	%	\label{ch01_pumpgallery}
\end{minipage}
\caption{Ground settlement impacts and leakages}
\label{ch04_settlement}
\end{figure}

Damages to the concrete saddle supports appearing as cracks and crooked or completely drawn bolts have been observed as shown in Figure \ref{ch04_settlement}-b. Furthermore, gaps are also observed between the concrete saddle supports and the straight pipe beneath as shown in \ref{ch04_settlement}-c.  

For the booster pumps, there are two saddle supports located near to each other and are positioned just outside the pump house after the suction elbows as shown in Figure \ref{ch04_fig_visual1}-a. For the storage pumps, the three saddles are located farther, one before the discharge elbow and  the other still outside the pump house and just after the check valve.

Piping movements parallel to the central axis also have been observed. As shown in Figure \ref{ch04_fig_visual1}-b, a gap has been observed between several counter supports and the elbow.  The counter supports are supposed to handle elbow thrust as immense water volumes pass thru the elbow. 

\begin{figure}[!htb]
	\begin{minipage}[b]{0.5\linewidth}
		\centering
		\includegraphics[width=\textwidth]{figures/ch04_fig_visual01}
		\caption*{a - Piping supports}
		%	\label{pagcorlocation}
	\end{minipage}
	\hspace{0.05cm}
	\begin{minipage}[b]{0.5\linewidth}
		\centering
		\includegraphics[width=\textwidth]{figures/ch04_fig_visual02}
		\caption*{b - Piping counter supports}
		%	\label{ch01_pumpgallery}
	\end{minipage}
	\caption{Impacts from supporting system}
	\label{ch04_fig_visual1}
\end{figure}

A common observation is that corrosion appears where there are leaks exist. Figure \ref{ch04_settlement}-d shows water leakage of BP6 check valve and surrounding and SP1 check valve.

Several dial pressure gauges were not functioning (Figure \ref{ch04_fig_visual2}-a). Also it was observed that some suction pressure gauges do not have a vacuum range and are only able to measure positive pressures.  Moreover, it was observed that the readings of dial gauges and digital pressure gauges do not match with differences up to 10 psi as shown in Figure \ref{ch04_fig_visual2}-b.


\begin{figure}[!htb]
	\begin{minipage}[b]{0.5\linewidth}
		\centering
		\includegraphics[width=\textwidth]{figures/ch04_fig_visual04}
		\caption*{a - Defective}
		%	\label{pagcorlocation}
	\end{minipage}
	\hspace{0.05cm}
	\begin{minipage}[b]{0.5\linewidth}
		\centering
		\includegraphics[width=\textwidth]{figures/ch04_fig_visual05}
		\caption*{b - Digital and dial pressure gause discrepancy}
		%	\label{ch01_pumpgallery}
	\end{minipage}
	\caption{Defective/discrepancy pressure gauges}
	\label{ch04_fig_visual2}
\end{figure}

The component tags were also inspected during the visits. As observed, some of the tags are unprotected and erased. Further, some tags are inconsistent in content. Different brands of pressure gauges were installed at similar corresponding tapping points. It was observed that the alignment bolts for the pumps are washerless and that nuts are used instead, as shown in Figure \ref{ch04_fig_visual3}-a. Also, there were no frame grounding found for all the pumps as seen in Figure \ref{ch04_fig_visual3}-b.

\begin{figure}[!htb]
	\begin{minipage}[b]{0.5\linewidth}
		\centering
		\includegraphics[width=\textwidth]{figures/ch04_fig_visual06}
		\caption*{a - Motor alignment bolts}
		%	\label{pagcorlocation}
	\end{minipage}
	\hspace{0.05cm}
	\begin{minipage}[b]{0.5\linewidth}
		\centering
		\includegraphics[width=\textwidth]{figures/ch04_fig_visual07}
		\caption*{b - Motor alignment supportsy}
		%	\label{ch01_pumpgallery}
	\end{minipage}
	\caption{Alignment impacts}
	\label{ch04_fig_visual3}
\end{figure}


%Values of CSs presented in the table are determined based on both generic definition of CSs as presented in Table \ref{ch03:cs} and specifically in the Table \ref{ch04:cs}.


%\begin{longtable}{|p{2.5cm}|p{8cm}|p{3cm}|}
%	\caption{Deliverable plan} \label{tab:long} \\
%	
%	\hline \multicolumn{1}{|l|}{\textbf{Items}} & \multicolumn{1}{l|}{\textbf{Task description}} & \multicolumn{1}{l|}{\textbf{Resources}} \\ \hline 
%	\endfirsthead
%	
%	\multicolumn{3}{c}%
%	{{\bfseries \tablename\ \thetable{} -- continued from previous page}} \\
%	\hline \multicolumn{1}{|l|}{\textbf{Items}} & \multicolumn{1}{l|}{\textbf{Task description}} & \multicolumn{1}{l|}{\textbf{Resources}} \\ \hline 
%	\endhead
%	\hline \multicolumn{3}{|r|}{{Continued on next page}} \\ \hline
%	\endfoot
%	\hline \hline
%	\endlastfoot
%	Phase 1 & Current state review &  \\ \hline 
%	Define Status Quo & Project Kick-off through a 3-hour kick off meeting with Maynilad team to review and revise important set of documents (e.g. project plan, objectives, workshop schedule, core team, existing prominent data) & GHD and Maynilad \\ \hline 
%	review & Review and analysis existing prominent data and knowledge, recommend a set of tests to be conducted to further identify the reliability and efficiency of equipment and facilities. & GHD \\ \hline 
%	Stakeholder Engagement Workshop 1 & Run the Workshop 1 to validate the Status Quo and define tasks for the next step (e.g. a concrete list of tests for equipment and facilities) & GHD with Maynilad participants \\ \hline 
%	Deliverable & Report, providing Status Quo, identify and highlight consolidated gaps and challenges & GHD with Maynilad review and approval \\ \hline 
%	Phase 2 & Database development for Data acquisition purposes, System description, and Condition State definition &  \\ \hline 
%	Select program & Define a suitable database program to be implemented for data collection and validation (e.g. PostgreSQL, MS access, or MS excel) & GHD \\ \hline 
%	Data acquisition & Import and migrate data in different formats (e.g. flat file, excel) to the selected database program & GHD \\ \hline 
%	Condition state definition & Identify system description and develop an appropriate set of condition states representing either physical condition and operational condition of items, components, sub-system, and system. & GHD \\ \hline 
%	Stakeholder Engagement Workshop 2 & Run the Workshop 2 to present up-to-date development status for data acquisition, system description, and condition state definition & GHD with Maynilad participants \\ \hline 
%	Deliverable & Report on data acquisition, system description and conditional state definition & GHD with Maynilad review and approval \\ \hline 
%	Phase 3 & System engineering and operational analysis (Modelling) &  \\ \hline 
%	Qualitative risk analysis & Study on risks that can be described as a combination of intensity and consequence. Perform interview to extract useful information learn from end users, engineers and line managers of Maynilad & GHD \\ \hline 
%	Quantitative risk analysis & Conduct reliability study on existing data (incl. estimation for failure rate, reliability, efficiency, availability, maintainability,  & GHD \\ \hline 
%	Operation study & Review and record various parameters on technical and financial operation of items, components, sub-system, and system (e.g. corrective and preventive intervention costs, energy consumption, labor consumption, spare parts) & GHD \\ \hline 
%	Stakeholder Engagement Workshop 3 & Run the Workshop 3 to present up-to-date development status for risk and reliability analysis and operation study & GHD with Maynilad participants \\ \hline 
%	Deliverable & Report on reliability study and operational study & GHD with Maynilad review and approval \\ \hline 
%	Phase 4 & Evaluation &  \\ \hline 
%	Life cycle cost analysis & Perform LCC analysis for items with different set of preventive intervention strategies & GHD \\ \hline 
%	Benchmarking & Benchmark for the optimal set of intervention strategies that yields the minimum LCC, whilst satisfying the requirements of Maynilad & GHD \\ \hline 
%	Consolidation & Consolidate the optimal intervention strategies to form the optimal preventive intervention program for 5 years implementation. & GHD \\ \hline 
%	Stakeholder Engagement Workshop 4 & Run the Workshop 4 to present the results of life cycle cost analysis and optimal preventive intervention program & GHD with Maynilad participants \\ \hline 
%	Deliverable & Report on life cycle cost analysis and optimal preventive intervention program for 5 years plan & GHD with Maynilad review and approval \\ \hline 
%	Phase 5 & Design &  \\ \hline 
%	Design & Perform 4 steps of detailed design for the purpose of procurement and installation & GHD \\ \hline 
%	Stakeholder Engagement Workshop 5 & Run the Workshop 5 to present the results of the detailed design works and conduct value engineering for selection of optimal design if required& GHD with Maynilad participants \\ \hline 
%	Deliverables & Calculation sheet, modelling, basic design report, draft and final versions of all reports and drawings associated with detailed design. & GHD with Maynilad review and approval \\ \hline 
%	Phase 6 & Tender package preparation &  \\ \hline 
%	Writing tender documents & Write and compile documents for tender package (e.g. instruction to bidders, scope of works, contract, program, functional guarantee) & GHD \\ \hline 
%	Stakeholder Engagement Workshop 6 & Run the Workshop 6 to present the content of the tender package & GHD with Maynilad participants \\ \hline 
%	Deliverable & Draft version and final version of the tender package  & GHD with Maynilad review and approval \\ \hline 
%	Finalization &  &  \\ \hline 
%	Stakeholder Engagement Workshop 7 & Run the Workshop 7 to transfer the knowledge to Maynilad team & GHD with Maynilad participants \\ \hline 
%	Deliverable & Report on knowledge transfer workshop  & GHD with Maynilad review and approval \\ \hline 
%	
%\end{longtable}

\subsection{Visual inspection data}
Visual inspection data on assets are summarized in tables of this section. %and also in the Appendix \ref{appvisualinspectionmech} with pictures.

%\paragraph{\textbf{BP1}}

\begin{table}[!htb]
	\caption{Visual inspection data - BP1}
	\label{ch04_visualinspectionbp1}
%	\resizebox{\columnwidth}{!}{%
		{\scriptsize
\begin{tabular}{c|l|c|p{12cm}}
	\hline
	No. & Items & CS & Remarks \\ 
	\hline
	1 & SBV & - & Below ground; not inspected,  \\ 
	2 & FJ & - & Below ground; not inspected, \\ 
	3 & SE & 1 & Mitered elbows were used (not radius elbows) \\ 
	4 & PIPE1 & 2 & Pipe not consistently cylindrical; Pipe made from very short pieces of pipes welded diagonally and circumferentially. Slump areas due to counteraction of steel strap when pipe imbalance during ground settlement develop slump areas. \\ 
	5 & ECR & 1 & Eccentric reducers used appropriately \\ 
	6 & FJ & 3 & Flexible joint adjusts to ground settlement for pump and casing, Joint statues need be checked \\ 
	7 & CS1 & 4 & Concrete saddles display crack propagations. Rigid steel straps resist pipe movement during ground settlement because of misalignment and crook fastening bolts. Space/gap between saddle seat and pipe exist. \\ 
	8 & CS2 & 4 & Concrete saddles display crack propagations. Rigid steel straps resist pipe movement during ground settlement because of misalignment and crook fastening bolts. Space/gap between saddle seat and pipe exist. U counter supports faultily installed. Fastening bolts are not tighten/used at all. \\ 
	9 & CCR & 1 & Concentric reducers were used appropriately \\ 
	10 & CV & 2 & Water leakage and corrosion observed at the shaft of the balancer \\ 
	11 & EJ & 3 & Outer stud rods may short circuit expansion action of the inner joint \\ 
	12 & DBV & 1 & Valve and actuator working properly  \\ 
	13 & DE & 1 & Thickness is acceptable and no sign of corrosion or paint deterioration \\ 
	14 & PAD & 1 & No crack propagation observed \\ 
	15 & CS3 & 1 & No crack propagation observed; CV sits well on the support  \\ 
	\hline
\end{tabular}
	}
\end{table}

\begin{table}[!h]
	\caption{Visual inspection data - BP2}
	\label{ch04_visualinspectionbp2}
%	\resizebox{\columnwidth}{!}{%
		{\scriptsize
\begin{tabular}{c|l|c|p{12cm}}
\hline
No. & Items & CS & Remarks \\ 
\hline
1 & SBV & - & Below ground; not inspected,  \\ 
2 & FJ & - & Below ground; not inspected, \\ 
3 & SE & 1 & Mitered elbows were used (not radius elbows) \\ 
4 & PIPE1 & 2 & Pipe not consistently cylindrical; Pipe made from very short pieces of pipes welded diagonally and circumferentially. Slump areas due to counteraction of steel strap when pipe imbalance during ground settlement develop slump areas. \\ 
5 & ECR & 1 & Eccentric reducers used appropriately \\ 
6 & FJ & 4 & Flexible joint adjusts to ground settlement for pump and casing, Joint statues need be checked \\ 
7 & CS1 & 4 & Concrete saddles display crack propagations. Rigid steel straps resist pipe movement during ground settlement because of misalignment and so uproot and/or crook fastening bolts. Space/gap between saddle seat and pipe exist. \\ 
8 & CS2 & 4 & Concrete saddles display crack propagations. Rigid steel straps resist pipe movement during ground settlement because of misalignment and so uproot and/or crook fastening bolts. Space/gap between saddle seat and pipe exist. U counter supports faultily installed. Fastening bolts are not tighten/used at all. \\ 
9 & CCR & 1 & Concentric reducers were used appropriately \\ 
10 & CV & 1 & No water leakage and local corrosion is minimal \\ 
11 & EJ & 4 & Outer stud rods short circuit expansion action of the inner joint \\ 
12 & DBV & 1 & Valve and actuator working properly  \\ 
13 & DE & 1 & Thickness is acceptable and no sign of corrosion or paint deterioration \\ 
14 & PAD & 1 & No crack propagation observed \\ 
15 & CS3 & 1 & No crack propagation observed; CV sits well on the support  \\ 
\hline
\end{tabular}
	}
\end{table}

\begin{table}[!h]
	\caption{Visual inspection data - BP3}
	\label{ch04_visualinspectionbp3}
%	\resizebox{\columnwidth}{!}{%
		{\scriptsize
\begin{tabular}{c|l|c|p{12cm}}
\hline
No. & Items & CS & Remarks \\ 
\hline
1 & SBV & - & Below ground; not inspected,  \\ 
2 & FJ & - & Below ground; not inspected, \\ 
3 & SE & 1 & Mitered elbows were used (not radius elbows) \\ 
4 & PIPE1 & 2 & Pipe not consistently cylindrical; Pipe made from very short pieces of pipes welded diagonally and circumferentially. Slump areas due to counteraction of steel strap when pipe imbalance during ground settlement develop slump areas. \\ 
5 & ECR & 1 & Eccentric reducers used appropriately \\ 
6 & FJ & 4 & Flexible joint adjusts to ground settlement for pump and casing, Joint statues need be checked \\ 
7 & CS1 & 4 & Concrete saddles display crack propagations. Rigid steel straps resist pipe movement during ground settlement because of misalignment and so uproot and/or crook fastening bolts. Space/gap between saddle seat and pipe exist. \\ 
8 & CS2 & 4 & Concrete saddles display crack propagations. Rigid steel straps resist pipe movement during ground settlement because of misalignment and so uproot and/or crook fastening bolts. Space/gap between saddle seat and pipe exist. U counter supports faultily installed. Fastening bolts are not tighten/used at all. \\ 
9 & CCR & 1 & Concentric reducers were used appropriately \\ 
10 & CV & 1 & No water leakage and local corrosion is minimal \\ 
11 & EJ & 4 & Outer stud rods short circuit expansion action of the inner joint \\ 
12 & DBV & 1 & Valve and actuator working properly  \\ 
13 & DE & 1 & Thickness is acceptable and no sign of corrosion or paint deterioration \\ 
14 & PAD & 1 & No crack propagation observed \\ 
15 & CS3 & 1 & No crack propagation observed; CV sits well on the support  \\ 
\hline
\end{tabular}
	}
\end{table}


\begin{table}[!h]
	\caption{Visual inspection data - BP4}
	\label{ch04_visualinspectionbp4}
%	\resizebox{\columnwidth}{!}{%
		{\scriptsize
\begin{tabular}{c|l|c|p{12cm}}
\hline
No. & Items & CS & Remarks \\ 
\hline
1 & SBV & - & Below ground; not inspected,  \\ 
2 & FJ & - & Below ground; not inspected, \\ 
3 & SE & 1 & Mitered elbows were used (not radius elbows) \\ 
4 & PIPE1 & 2 & Pipe not consistently cylindrical; Pipe made from very short pieces of pipes welded diagonally and circumferentially. Slump areas due to counteraction of steel strap when pipe imbalance during ground settlement develop slump areas. \\ 
5 & ECR & 1 & Eccentric reducers used appropriately \\ 
6 & FJ & 4 & Flexible joint adjusts to ground settlement for pump and casing, Joint statues need be checked \\ 
7 & CS1 & 4 & Concrete saddles display crack propagations. Rigid steel straps resist pipe movement during ground settlement because of misalignment and so uproot and/or crook fastening bolts. Space/gap between saddle seat and pipe exist. \\ 
8 & CS2 & 4 & Concrete saddles display crack propagations. Rigid steel straps resist pipe movement during ground settlement because of misalignment and so uproot and/or crook fastening bolts. Space/gap between saddle seat and pipe exist. U counter supports faultily installed. Fastening bolts are not tighten/used at all. \\ 
9 & CCR & 1 & Concentric reducers were used appropriately \\ 
10 & CV & 2 & Water leakage and corrosion observed at the shaft of the balancer \\ 
11 & EJ & 4 & Outer stud rods short circuit expansion action of the inner joint \\ 
12 & DBV & 1 & Valve and actuator working properly  \\ 
13 & DE & 1 & Thickness is acceptable and no sign of corrosion or paint deterioration \\ 
14 & PAD & 1 & No crack propagation observed \\ 
15 & CS3 & 1 & No crack propagation observed; CV sits well on the support  \\ 
\hline
\end{tabular}
	}
\end{table}

\begin{table}[!h]
	\caption{Visual inspection data - BP5}
	\label{ch04_visualinspectionbp5}
%	\resizebox{\columnwidth}{!}{%
		{\scriptsize
\begin{tabular}{c|l|c|p{12cm}}
\hline
No. & Items & CS & Remarks \\ 
\hline
1 & SBV & - & Below ground; not inspected,  \\ 
2 & FJ & - & Below ground; not inspected, \\ 
3 & SE & 1 & Mitered elbows were used (not radius elbows) \\ 
4 & PIPE1 & 2 & Pipe not consistently cylindrical; Pipe made from very short pieces of pipes welded diagonally and circumferentially. Slump areas due to counteraction of steel strap when pipe imbalance during ground settlement develop slump areas. \\ 
5 & ECR & 1 & Eccentric reducers used appropriately \\ 
6 & FJ & 4 & Flexible joint adjusts to ground settlement for pump and casing, Joint statues need be checked \\ 
7 & CS1 & 4 & Concrete saddles display crack propagations. Rigid steel straps resist pipe movement during ground settlement because of misalignment and so uproot and/or crook fastening bolts. Space/gap between saddle seat and pipe exist. \\ 
8 & CS2 & 4 & Concrete saddles display crack propagations. Rigid steel straps resist pipe movement during ground settlement because of misalignment and so uproot and/or crook fastening bolts. Space/gap between saddle seat and pipe exist. U counter supports faultily installed. Fastening bolts are not tighten/used at all. \\ 
9 & CCR & 1 & Concentric reducers were used appropriately \\ 
10 & CV & 1 & No water leakage and local corrosion is minimal \\ 
11 & EJ & 4 & Outer stud rods short circuit expansion action of the inner joint \\ 
12 & DBV & 1 & Valve and actuator working properly  \\ 
13 & DE & 1 & Thickness is acceptable and no sign of corrosion or paint deterioration \\ 
14 & PAD & 1 & No crack propagation observed \\ 
15 & CS3 & 1 & No crack propagation observed; CV sits well on the support  \\ 
\hline
\end{tabular}
	}
\end{table}

\begin{table}[!h]
	\caption{Visual inspection data - BP6}
	\label{ch04_visualinspectionbp6}
%	\resizebox{\columnwidth}{!}{%
		{\scriptsize
\begin{tabular}{c|l|c|p{12cm}}
\hline
No. & Items & CS & Remarks \\ 
\hline
1 & SBV & - & Below ground; not inspected,  \\ 
2 & FJ & - & Below ground; not inspected, \\ 
3 & SE & 1 & Mitered elbows were used (not radius elbows) \\ 
4 & PIPE1 & 2 & Pipe not consistently cylindrical; Pipe made from very short pieces of pipes welded diagonally and circumferentially. Slump areas due to counteraction of steel strap when pipe imbalance during ground settlement develop slump areas. \\ 
5 & ECR & 1 & Eccentric reducers used appropriately \\ 
6 & FJ & 4 & Flexible joint adjusts to ground settlement for pump and casing, Joint statues need be checked \\ 
7 & CS1 & 4 & Concrete saddles display crack propagations. Rigid steel straps resist pipe movement during ground settlement because of misalignment and so uproot and/or crook fastening bolts. Space/gap between saddle seat and pipe exist. \\ 
8 & CS2 & 4 & Concrete saddles display crack propagations. Rigid steel straps resist pipe movement during ground settlement because of misalignment and so uproot and/or crook fastening bolts. Space/gap between saddle seat and pipe exist. U counter supports faultily installed. Fastening bolts are not tighten/used at all. \\ 
9 & CCR & 1 & Concentric reducers were used appropriately \\ 
10 & CV & 2 & Water leakage and corrosion observed at the shaft of the balancer \\ 
11 & EJ & 4 & Outer stud rods short circuit expansion action of the inner joint \\ 
12 & DBV & 1 & Valve and actuator working properly  \\ 
13 & DE & 1 & Thickness is acceptable and no sign of corrosion or paint deterioration \\ 
14 & PAD & 1 & No crack propagation observed \\ 
15 & CS3 & 1 & No crack propagation observed; CV sits well on the support  \\ 
\hline
\end{tabular}			
		}
\end{table}


\begin{table}[!h]
	\caption{Visual inspection data - SP1}
	\label{ch04_visualinspectionsp1}
%	\resizebox{\columnwidth}{!}{%
		{\scriptsize
\begin{tabular}{c|l|c|p{12cm}}
\hline
No. & Items & CS & Remarks \\ 
\hline
1 & SBV & 1 & Valve working properly  \\ 
2 & EJ & 4 & Outer stud rods short circuit expansion action of the inner joint \\ 
3 & ECR & 1 & Eccentric reducers used appropriately \\ 
4 & PAD & 1 & No crack propagation observed; Base is dirty and lubricating sludge drops from the drive end of the motor \\ 
5 & FJ & 3 & Flexible joint adjusts to ground settlement for pump and casing, however misalignment between suction nozzle and pipe centerline needs to be checked. \\ 
6 & CCR &  & Concentric reducers were used appropriately \\ 
7 & CV & 2 & Water leakage and corrosion observed at the shaft of the balancer \\ 
8 & PIPE2 & 2 & Pipe shape is good; Corrosion and paint removal at pipe portion that seats with saddle show accelerated deterioration comapred to other portions.  \\ 
9 & DE & 1 & No disfigurement or paint deterioration \\ 
10 & FJ & - & Below ground; not inspected \\ 
11 & DBV & - & Below ground; not inspected, \\ 
12 & CS3 & 1 & No crack propagation observed; CV sits well on the support; Refurbishing needed on portion that catches the water leakage from CV \\ 
13 & CS4 & 4 & Concrete saddles display crack propagations. Rigid steel straps resist pipe movement during ground settlement because of misalignment and so uproot and/or crook fastening bolts. Space/gap between saddle seat and pipe exist. Counter supports not properly installed with either no fasteners or nut-tightened bolts \\ 
14 & CS5 & 4 & Concrete saddles display crack propagations. Rigid steel straps resist pipe movement during ground settlement because of misalignment and so uproot and/or crook fastening bolts. Space/gap between saddle seat and pipe exist. \\ 
\hline
\end{tabular}
	}
\end{table}


\begin{table}[!h]
	\caption{Visual inspection data - SP2}
	\label{ch04_visualinspectionsp2}
%	\resizebox{\columnwidth}{!}{%
		{\scriptsize
\begin{tabular}{c|l|c|p{12cm}}
\hline
No. & Items & CS & Remarks \\ 
\hline
1 & SBV & 1 & Valve working properly  \\ 
2 & EJ & 4 & Outer stud rods short circuit expansion action of the inner joint \\ 
3 & ECR & 1 & Eccentric reducers used appropriately \\ 
4 & PAD & 1 & No crack propagation observed; Base is dirty and lubricating sludge drops from the drive end of the motor \\ 
5 & FJ & 3 & Flexible joint adjusts to ground settlement for pump and casing, however misalignment between suction nozzle and pipe centerline needs to be checked. \\ 
6 & CCR &  & Concentric reducers were used appropriately \\ 
7 & CV & 2 & Water leakage and corrosion observed at the shaft of the balancer \\ 
8 & PIPE2 & 2 & Pipe shape is good; Corrosion and paint removal at pipe portion that seats with saddle show accelerated deterioration comapred to other portions.  \\ 
9 & DE & 1 & No disfigurement or paint deterioration \\ 
10 & FJ & - & Below ground; not inspected \\ 
11 & DBV & - & Below ground; not inspected, \\ 
12 & CS3 & 1 & No crack propagation observed; CV sits well on the support; Refurbishing needed on portion that catches the water leakage from CV \\ 
13 & CS4 & 4 & Concrete saddles display crack propagations. Rigid steel straps resist pipe movement during ground settlement because of misalignment and so uproot and/or crook fastening bolts. Space/gap between saddle seat and pipe exist. Counter supports not properly installed with either no fasteners or nut-tightened bolts \\ 
14 & CS5 & 4 & Concrete saddles display crack propagations. Rigid steel straps resist pipe movement during ground settlement because of misalignment and so uproot and/or crook fastening bolts. Space/gap between saddle seat and pipe exist. \\ 
\hline
\end{tabular}
	}
\end{table}

\subsection{Observation and recommendations}
\begin{itemize}
\item Pumps were installed with poorly fabricated and/or fitted pipe concrete saddles and steel straps, which provide insufficient support. These flaws, along with the natural reaction forces between the pipe sections and said supports, possibly aggravated by soil settlement, have created obvious visible gaps within the load bearing areas, as well as fractures in the concrete saddles and the steel anchor bolts. Instances of crevice corrosion were found in some contact areas. 
\paragraph{\underline{Recommendations}}
\begin{itemize}
	\item [$\checkmark$] Short term: clear away oxide scales, especially among crevices, and recoat affected areas; 
	\item[$\checkmark$]  Medium term: install properly engineered piping supports, equipped with anti-corrosion shields; install cathodic protection on the pipelines.
\end{itemize}

\item Station designers relied on foam inserts along penetrations in the front concrete wall to provide flexible piping support, which resulted in numerous wall cracks that are currently being repaired. Even if repairs on the said wall are completed, support for the affected piping is insufficient because wall penetrations are not designed for such a purpose. 

\paragraph{\underline{Recommendations}}
\begin{itemize}
	\item [$\checkmark$] Isolate piping from the concrete wall. 
\end{itemize}

\item Pumps have little to no isolation from hydrodynamic forces. Strain caused by movement at the inlet and outlet pipe sections are transmitted to pump casings. Moreover, hydrodynamic forces, along with the static weight of said pipe sections, including the flexible coupling and the water contained therein, are transmitted to the casing due to lack of proper piping support upstream or downstream of the pump.

\paragraph{\underline{Recommendations}}
\begin{itemize}
	\item [$\checkmark$] Install proper isolation and support upstream and downstream of the pumps. 
\end{itemize}

\item Poor installation and/or alignment practices allowed the use of spacer-nuts to compensate for short jack screws and washerless anchor bolts in BP6.

\paragraph{\underline{Recommendations}}
\begin{itemize}
	\item [$\checkmark$] Implement proper quality control of shaft alignment jobs, whether done in-house or outsourced. 
\end{itemize}

\item All motors lack frame grounding. Assuming the motor power supplies were grounded at the motor control center (MCC), an assumptions that still needs to be verified, current may pass through the bearings should the motor shaft be grounded whilst the motor frame is ungrounded, which may lead to bearing damage, especially among variable speed drive (VFD) units. Bonding of non-current carrying metal components (e.g., motor frames) to the ground system is necessary to create an equipotential plane between the concrete floor and plant personnel who may risk electrocution should said parts become energized.

\paragraph{\underline{Recommendations}}
\begin{itemize}
	\item [$\checkmark$] Bond motor frames to the ground system; review present grounding policies/procedures.  
\end{itemize}


\item Possible inner cavitation/corrosion: inner parts of pipes, particularly at the elbow sections, might have already been sufferred from cavitation/corrosion. 

\paragraph{\underline{Recommendations}}
\begin{itemize}
	\item [$\checkmark$] Develop a long term testing regime that allows partial shutdown to conduct visual inspection inside (e.g. by CCTV) for checking cavitation/corrosion, manufacturing defects for steel plat or during spiral welded pipe fabrication.
\end{itemize}


\end{itemize}




%Aside from the CSs observed as presented in the table, we also learn from the inspection that the entire pump house has been suffering from ground settlement and that the alignments of the pumps and the corresponding fittings have been compromised. As shown in Figure \ref{ch04_settlement}, the portion of wall through which the pipe passes thru has been torn down partially to provide allowance for settlement as the pipe levels down.  

%\begin{figure}[!htb]
%	\includegraphics[scale=1.5]{figures/ch04_settlement} \\
%	\caption{Wall opening provision for ground settling}
%	\label{ch04_settlement} 
%\end{figure}

%Damages to the concrete saddle supports appearing as cracks and crooked or completely drawn bolts have been observed as shown in Figure \ref{ch04_settlement01}. 
%
%\begin{figure}[!htb]
%	\includegraphics[scale=1.5]{figures/ch04_settlement01} \\
%	\caption{Support damages due to pipe movement and ground settling}
%	\label{ch04_settlement01} 
%\end{figure}
%
%Furthermore, gaps are also observed between the concrete saddle supports and the straight pipe beneath as shown in Figure \ref{ch04_settlement02}.  
%\begin{figure}[!htb]
%	\includegraphics[scale=1.5]{figures/ch04_settlement02} \\
%	\caption{Support damages due to pipe movement and ground settling}
%	\label{ch04_settlement02} 
%\end{figure}
%
%For the booster pumps, there are two saddle supports located near to each other and are positioned just outside the pump house after the suction elbows as shown in Figure \ref{ch04_piping_supports01}. For the storage pumps, the three saddles are located farther, one before the discharge elbow and   the other still outside the pump house and just after the check valve.
%
%\begin{figure}[!htb]
%	\includegraphics[scale=1.5]{figures/ch04_piping_supports01} \\
%	\caption{Support damages due to pipe movement and ground settling}
%	\label{ch04_piping_supports01} 
%\end{figure}


\section{Pump efficiency} \label{ch04mech03}
Data on flow and head were measured for each pump. However, there was no measured data of motor/pump assembly regarding power ratings of all pumps. This was due to the fact that electrical audit is not part of the scope of work.

GHD/RBSanchez did verify on the provided electrical audit to see if there is available power rating for individual motor/pump assembly. However, the electrical audit does not include measured data for individual pump. Thus, cannot be used as a reference to more or less correlate with measured flow and head for computation of pump efficiency as desired.

\subsection{Unit flow measurement} \label{ch04mech04}
Data on measured flow (Q) was recorded with min and max values is shown in Table \ref{ch04_tbl_flow01} for each pumps. %Raw data is provided in the Appendix \ref{appflow}.

\begin{table}[!h]
	\caption{Unit flow measurement (cubic meter per second - cms).}
	\label{ch04_tbl_flow01}
	{\footnotesize
\begin{tabular}{c|c|c|c|c|c|l}
	\hline
	Assets & $\Phi$  & \multicolumn{4}{c|}{Flow Q (cms)} & Remarks \\ 
	\cline{3-6}
	& (mm) & Design & \multicolumn{3}{c|}{Measure} &  \\ 
	\cline{4-6}
	&  &  & Min & Max & Ave. &  \\ 
	\hline
	BP1 & 700 & 0.6365 & 0.3407 & 0.3533 & 0.3470  &  \\ 
	BP2 & 700 & 0.6365 & 0.3028 & 0.3343 & 0.3186  &  \\ 
	BP3 & 700 & 0.6365 & 0.5930 & 0.6119 & 0.6024  &  \\ 
	BP4 & 700 & 0.6365 & 0.5110 & 0.5614 & 0.5362  &  \\ 
	BP5 & 700 & 0.6365 & 0.6939 & 0.7128 & 0.7034  &  \\ 
	BP6 & 700 & 0.6365 & 0.6119 & 0.6561 & 0.6340  &  \\ 
	SP1 & 600 & 0.5324 & 0.3583 & 0.3659 & 0.3621  &  \\ 
	SP2 & 600 & 0.5324 & 0.3785 & 0.3848 & 0.3817  &  \\ 
	\hline
\end{tabular}
	}
\end{table}

\subsection{Pressure measurement} \label{ch04pressure}

Data on measured flow (Q) was recorded with min and max values is shown in Table \ref{ch04_tbl_flow02} for each pumps. %Raw data is provided in the Appendix \ref{appflow}.

\begin{table}[!h]
	\caption{Head ($mH_2O$).}
	\label{ch04_tbl_flow02}
	{\footnotesize
\begin{tabular}{c|c|c|c|l}
	\hline
	Assets & \multicolumn{3}{c|}{Head (H - $mH_2O$)} & Remarks \\ 
	\cline{2-4}
	& Design & Discharge & Suction &  \\ 
	\hline
	BP1 & 40 & 42.1581 & 2.4592 &  \\ 
	BP2 & 40 & 42.1581 & 2.7403 &  \\ 
	BP3 & 40 & 42.1581 & 2.4592 &  \\ 
	BP4 & 40 & 40.7529 & 2.7403 &  \\ 
	BP5 & 40 & 42.1581 & 2.9511 &  \\ 
	BP6 & 40 & 40.7529 & 2.8105 &  \\ 
	SP1 & 50 & 37.8088 & 0.7026 &  \\ 
	SP2 & 50 & 36.5371 & 0.7026 &  \\ 
	\hline
\end{tabular}
	}
\end{table}


\subsection{Efficiency}

Pump efficiency is computed based on the flow/head measurement and the assumed value of power rating (Table \ref{ch05_tbl_efficiency}). 
\begin{table}[!h]
	\caption{Pump efficiency (\%).}
	\label{ch05_tbl_efficiency}
	{\footnotesize
\begin{tabular}{c|c|c|c|c|c|l|l}
	\hline
	Assets & Flow & Head & Input Power & Water Power & \multicolumn{3}{c}{Efficiency (\%)} \\ 
	\cline{6-8}
	& $m^3/s$ & $mH_2O$ & kW & kW & Tested & Design & Diff. \\ 
	\hline
	BP1 & 0.347 & 41.68 & 181 & 142 & 78.60 & \multicolumn{1}{c|}{85.67} & \multicolumn{1}{c}{-7.07} \\ 
	BP2 & 0.319 & 41.4 & 181 & 129 & 71.70 & \multicolumn{1}{c|}{85.67} & \multicolumn{1}{c}{-13.97} \\ 
	BP3 & 0.602 & 41.68 & 283 & 246 & 86.90 & \multicolumn{1}{c|}{85.67} & \multicolumn{1}{c}{1.23} \\ 
	BP4 & 0.536 & 40 & 283 & 210 & 74.20 & \multicolumn{1}{c|}{85.67} & \multicolumn{1}{c}{-11.47} \\ 
	BP5 & 0.703 & 41.19 & 283 & 242 & - & \multicolumn{1}{c|}{85.67} & \multicolumn{1}{c}{} \\ 
	BP6 & 0.634 & 39.93 & 283 & 248 & 87.60 & \multicolumn{1}{c|}{85.67} & \multicolumn{1}{c}{1.93} \\ 
	SP1 & 0.362 & 45.33 & 185 & 161 & 87.00 & \multicolumn{1}{c|}{86.89} & \multicolumn{1}{c}{0.11} \\ 
	SP2 & 0.382 & 44.06 & 185 & 165 & 89.10 & \multicolumn{1}{c|}{86.89} & \multicolumn{1}{c}{2.21} \\ 
	\hline
\end{tabular}
	}
\end{table}

It is important to note that the values of input power is an assumed values which might not perfectly reflect the actual value in actual situation. This assumption is a limitation of the study, particularly for this station, since the electrical audit was carried out without actual records on power for individual pump. The design pump efficiencies are taken from the test report of the KSB \cite{KSB2010}. 

In Table \ref{ch05_tbl_efficiency}, efficiency for BP5 is not estimated due to the fact that the measured flow is beyond the maximum allowable flow of the pump. This infers that the pump has been operated inappropriately pursuant to its design specification. As a consequence, the operation of the pump might have already incurred more power than it should be. If this scheme of operation is continues, highly likely that the failure probability will increase. 

Figure \ref{ch04_efficiencycurves}-a, and \ref{ch04_efficiencycurves}-b presents the efficiency curves for booster pumps and storage pumps, respectively. The curves are created based on the recorded data provided in the test record of KSB \cite{KSB2010}. 


\begin{figure}[!htb]
	\begin{minipage}[b]{0.5\linewidth}
		\centering
		\includegraphics[width=\textwidth]{figures/ch04_fig_efficiency01}
		\caption*{a - Booster pumps}% \label{peter1}
	\end{minipage}
	\hspace{0.05cm}
	\begin{minipage}[b]{0.5\linewidth}
		\centering
		\includegraphics[width=\textwidth]{figures/ch04_fig_efficiency02}
		\caption*{b -Storage Pumps} %\label{peter2}
	\end{minipage}
		\caption{Efficiency curves}
		\label{ch04_efficiencycurves}
\end{figure}

Figure \ref{ch04_efficiencycurves}-a shows that BP1 and BP2 deviates away from the curve. In this case, BP1 and BP2 possibly operates at underflow condition, and at higher friction loss. The pump operates at lower efficiency based on the plot curve, and backed up to have an efficiency of 78.6\% and 71.7\%, respectively. This also constitutes to the “fair” condition of the pump with possible cavitation at the suction head. Moreover, BP3, BP5, and BP6 have efficiencies higher than 85\%. As shown at the shaded region, these pumps operate under the 10\% tolerance BEP range, thereby operating at considerably good operation point although BP5 has overflow-measured flow. In the contrary, these pumps were diagnosed to have a “fair” pump condition state. This is due to pump possible cavitation from the suction and carried throughout the discharged side of the pumps. In the other hand, BP4 has an efficiency of 74.2\% and deviates slightly away to the curve with the operating point of BP4 is under the 20\% tolerance BEP range. The pump operates at lower head and possibly in underflow condition. The pump have a diagnosis of having a good pump health that means lower vibration and possible cavitation occurred / occurring within pump impellers

Figure \ref{ch04_efficiencycurves}-b shows that the operating point of SP1 and SP2 deviates away from the pump curve. This may possibly means that the pump is operating at higher friction head, and either overflow and underflow. Operating points of both pumps are within the 20\% tolerance range (shaded region) based on the GPM* (best GPM point), and has slight increase in pressure head. This can be associated in why is their efficiency is still above 80\% even operating at deviated operation conditions. In the other hand, it would affect the system performance when operated over time, thereby having an initial diagnosis of “fair” pump health based on the vibration analysis standpoint. 

Generally, VFD controlled-pumps deviates greatly away to the curve than the fixed-speed pumps (BP3 to BP6). This could possibly means that the operating duty points of the pumps when controlled by VFD is out of the best efficiency range of the pumps, thereby may incur higher power consumptions. 

\paragraph{\underline{Recommendations}}

\begin{itemize}
\item	Proper operating condition shall be established for the VFD Pumps in a given time to avoid them operating beyond their BEP for a long period of time. Also include the proper combinations of running pumps in a given time.
\item	Multiple tests and conditions is still recommended to have a holistic approach on the pump assessment.
\item	Measurement of important pump performance parameters shall also be included in the modifications (refer to the conceptual design in Chapter \ref{Chapter6}).
\end{itemize}


%\section{Electrical Audit}\label{43}
%\subsection{Visual inspection} \label{ch04elec01}
%\textcolor{blue}{APSI to write here the summary of raw data collected from visual inspection and testing. Tables shall be used as much as we can. Note that no analysis in this session. This session is purely the high level presentation of data. Raw data can be linked as an Appendix}
%\subsection{Short circuit calculation} \label{ch04elec02}
%\textcolor{blue}{APSI to write here the summary of raw data collected from visual inspection and testing. Tables shall be used as much as we can. Note that no analysis in this session. This session is purely the high level presentation of data. Raw data can be linked as an Appendix}
%\subsection{Voltage drop calculation} \label{ch04elec03}
%\textcolor{blue}{APSI to write here the summary of raw data collected from visual inspection and testing. Tables shall be used as much as we can. Note that no analysis in this session. This session is purely the high level presentation of data. Raw data can be linked as an Appendix}
%\subsection{Protection coordination study} \label{ch04elec04}
%\textcolor{blue}{APSI to write here the summary of raw data collected from visual inspection and testing. Tables shall be used as much as we can. Note that no analysis in this session. This session is purely the high level presentation of data. Raw data can be linked as an Appendix}
%\subsection{Harmonic analysis} \label{ch04elec05}
%\textcolor{blue}{APSI to write here the summary of raw data collected from visual inspection and testing. Tables shall be used as much as we can. Note that no analysis in this session. This session is purely the high level presentation of data. Raw data can be linked as an Appendix}
%\subsection{Power quality} \label{ch04elec06}
%\textcolor{blue}{APSI to write here the summary of raw data collected from visual inspection and testing. Tables shall be used as much as we can. Note that no analysis in this session. This session is purely the high level presentation of data. Raw data can be linked as an Appendix}
%\subsection{Grounding system} \label{ch04elec07}
%\textcolor{blue}{APSI to write here the summary of raw data collected from visual inspection and testing. Tables shall be used as much as we can. Note that no analysis in this session. This session is purely the high level presentation of data. Raw data can be linked as an Appendix}
%\subsection{Asset registry} \label{ch04elec08}
%\textcolor{blue}{APSI to write here the summary of raw data collected from visual inspection and testing. Tables shall be used as much as we can. Note that no analysis in this session. This session is purely the high level presentation of data. Raw data can be linked as an Appendix}
%\subsection{Electrical system design analysis}\label{ch04elec09}
%\textcolor{blue}{APSI to write here the summary of raw data collected from visual inspection and testing. Tables shall be used as much as we can. Note that no analysis in this session. This session is purely the high level presentation of data. Raw data can be linked as an Appendix}
%
%\subsection{Electrical integrity system} \label{ch04elec11}
%\textcolor{blue}{APSI to write here the summary of raw data collected from visual inspection and testing. Tables shall be used as much as we can. Note that no analysis in this session. This session is purely the high level presentation of data. Raw data can be linked as an Appendix}
%\subsection{Outdoor electrical equipment} \label{ch04elec12}
%\textcolor{blue}{APSI to write here the summary of raw data collected from visual inspection and testing. Tables shall be used as much as we can. Note that no analysis in this session. This session is purely the high level presentation of data. Raw data can be linked as an Appendix}



\section{Vibration and structural assessment}
\label{45}
\subsection{Measurement and spectrum reading}
Rotating equipment generate vibration waveforms that are mathematical functions of machine dynamics, such as speed, alignment, and rotor balance, among others. Vibration analysis entails measurement and analysis of the amplitude of vibration at certain frequencies to gather useful information relating to the accuracy of shaft alignment and balance, the physical condition of bearings, and the possible effects of structural issues; in the case of Maynilad, the problem of impeller possible cavitation is an added and serious concern.

Three main parameters are measured to determine the severity or amplitude of vibration; namely: displacement, velocity and acceleration. Along with temperature, the vibration level is a primary indicator of the physical condition of a machine. As a generally rule, higher vibration levels indicate greater defects. 

Rotating speeds below 600 rpm (10 Hz) generate minimal acceleration, moderate velocity, but relatively high displacement. Hence, shaft displacement is a critical parameter for slow speed rotors, such as steam turbines. Between 600 – 60000 rpm (10 - 1000 Hz) velocity and acceleration levels provide useful indications of the severity of defects. While velocity as a parameter may indicate the presence or relative magnitude of a problem, it makes no distinction as to the source or cause. This is where an FFT vibration analyzer comes in. A fast Fourier transform algorithm converts acceleration waveforms into functions of frequency in a way suitable-trained humans can distinguish the component sources or causes of the vibration.

By means of a OneProd Falcon high-resolution FFT analyzer equipped with tri-axial accelerometer with a linear frequency range of 2Hz-30kHz, vibration spectral readings were taken from four bearing locations in each motor-pump unit. Analysis and results are summarized as follows:

\subsection{Data and analysis}

Raw data of vibration measurement is provided in separately digital format. The raw data of each pump is used to generate a set of graphs provided in Appendix \ref{app_vibrationdata}.

Analytical results on vibration are with the Appendix \ref{app_vibrationdata}. A summary of grading for each pump is given in Table \ref{ch05_tbl_vibration}.

\begin{table}[!h]
	\caption{Pump vibration condition state.}
	\label{ch05_tbl_vibration}
	{\footnotesize
\begin{tabular}{l|l|l|l}
	\hline
	\multicolumn{1}{c|}{Assets} & Operational issues detected & \multicolumn{2}{c}{Condition} \\ 
	\cline{3-4}
	\multicolumn{1}{c|}{} &  & \multicolumn{1}{c|}{Motor} & \multicolumn{1}{c}{Pump} \\ 
	\hline
	\multicolumn{1}{c|}{BP1} & shaft misalignment & \multicolumn{1}{c|}{2} & \multicolumn{1}{c}{3} \\ 
	\multicolumn{1}{c|}{} & impeller possible cavitation & \multicolumn{1}{c|}{} & \multicolumn{1}{c}{} \\ 
	\multicolumn{1}{c|}{} & low lubrication at pump inboard (IB) bearing & \multicolumn{1}{c|}{} & \multicolumn{1}{c}{} \\ 
	\hline
	\multicolumn{1}{c|}{BP2} & early stage fault  & \multicolumn{1}{c|}{2} & \multicolumn{1}{c}{3} \\ 
	\multicolumn{1}{c|}{} & low lubrication at pump outboard (OB) bearing & \multicolumn{1}{c|}{} & \multicolumn{1}{c}{} \\ 
	\hline
	\multicolumn{1}{c|}{BP3} & misalignment & \multicolumn{1}{c|}{2} & \multicolumn{1}{c}{3} \\ 
	\multicolumn{1}{c|}{} & impeller possible cavitation & \multicolumn{1}{c|}{} & \multicolumn{1}{c}{} \\ 
	\hline
	\multicolumn{1}{c|}{BP4} & vibration in all bearings were within acceptable levels & \multicolumn{1}{c|}{2} & \multicolumn{1}{c}{2} \\ 
	\hline
	\multicolumn{1}{c|}{BP5} & impeller possible cavitation & \multicolumn{1}{c|}{2} & \multicolumn{1}{c}{3} \\ 
	\multicolumn{1}{c|}{} & low lubrication at pump outboard (OB) bearing & \multicolumn{1}{c|}{} & \multicolumn{1}{c}{} \\ 
	\hline
	\multicolumn{1}{c|}{BP6} & impeller possible cavitation & \multicolumn{1}{c|}{2} & \multicolumn{1}{c}{3} \\ 
	\multicolumn{1}{c|}{} & low lubrication at pump outboard (OB) bearing & \multicolumn{1}{c|}{} & \multicolumn{1}{c}{} \\ 
	\hline
	\multicolumn{1}{c|}{SP1} & shaft misalignment & \multicolumn{1}{c|}{2} & \multicolumn{1}{c}{3} \\ 
	\multicolumn{1}{c|}{} & impeller possible cavitation & \multicolumn{1}{c|}{} & \multicolumn{1}{c}{} \\ 
	\hline
	\multicolumn{1}{c|}{SP2} & impeller possible cavitation & \multicolumn{1}{c|}{2} & \multicolumn{1}{c}{3} \\ 
	\hline
\end{tabular}
	}
\end{table}
It is note that the CS 2, and 3 shown in Table \ref{ch05_tbl_vibration} infers good and fair, respectively \footnote{The CS is slightly different from that defines in Table \ref{ch03:cs}}. 

It can be seen from Table \ref{ch05_tbl_vibration}, vibration on motor is with CS 2 inferring that they are still operating in acceptance level of vibration (good). However, vibration on pump is mostly with CS 3 (fair), except for BP4. Problems found from observation and vibration analysis are mainly due to shaft misalignment, impeller possible cavitation, and low lubrication at pump inboard (IB) and outboard (OB) bearing. 

\subsection{Recommendations}
Recommendations are shown in Table \ref{ch05_tbl_vibrationre}

\begin{table}[!h]
	\caption{Recommendation to reduce vibration.}
	\label{ch05_tbl_vibrationre}
	{\footnotesize
		\begin{tabular}{l|l|l|l|c}
\hline
\multicolumn{1}{c|}{Assets} & \multicolumn{2}{c|}{Condition} & Recommendations & IT \\ 
\cline{2-3}
\multicolumn{1}{c|}{} & \multicolumn{1}{c|}{Motor} & \multicolumn{1}{c|}{Pump} & \multicolumn{1}{c|}{} &  \\ 
\hline
\multicolumn{1}{c|}{BP1} & \multicolumn{1}{c|}{2} & \multicolumn{1}{c|}{3} & Align motor and pump shafts per manufacturer specifications & 2 \\ 
\multicolumn{1}{c|}{} & \multicolumn{1}{c|}{} & \multicolumn{1}{c|}{} & Monitor and/or mitigate possible cavitation progress &  \\ 
\multicolumn{1}{c|}{} & \multicolumn{1}{c|}{} & \multicolumn{1}{c|}{} & \multicolumn{1}{c|}{Follow manufacturer prescribed dynamic operation to prevent impeller damage} &  \\ 
\multicolumn{1}{c|}{} & \multicolumn{1}{c|}{} & \multicolumn{1}{c|}{} & Regrease pump IB bearing &  \\ 
\hline
\multicolumn{1}{c|}{BP2} & \multicolumn{1}{c|}{2} & \multicolumn{1}{c|}{3} & Regrease pump OB bearing & 2 \\ 
\multicolumn{1}{c|}{} & \multicolumn{1}{c|}{} & \multicolumn{1}{c|}{} & Monitor vibration regularly to assess trend &  \\ 
\hline
\multicolumn{1}{c|}{BP3} & \multicolumn{1}{c|}{2} & \multicolumn{1}{c|}{3} & Align motor and pump shafts per manufacturer's specifications &  \\ 
\multicolumn{1}{c|}{} & \multicolumn{1}{c|}{} & \multicolumn{1}{c|}{} & Monitor and/or mitigate possible cavitation progress &  \\ 
\multicolumn{1}{c|}{} & \multicolumn{1}{c|}{} & \multicolumn{1}{c|}{} & Follow manufacturer prescribed dynamic operation to prevent impeller damage &  \\ 
\multicolumn{1}{c|}{} & \multicolumn{1}{c|}{} & \multicolumn{1}{c|}{} & Monitor vibration regularly to assess trend &  \\ 
\hline
\multicolumn{1}{c|}{BP4} & \multicolumn{1}{c|}{2} & \multicolumn{1}{c|}{2} & Monitor vibrations regularly & 2 \\ 
\hline
\multicolumn{1}{c|}{BP5} & \multicolumn{1}{c|}{2} & \multicolumn{1}{c|}{3} & Monitor and/or mitigate possible cavitation progress &  \\ 
\multicolumn{1}{c|}{} & \multicolumn{1}{c|}{} & \multicolumn{1}{c|}{} & Follow manufacturer prescribed dynamic operation to prevent impeller damage &  \\ 
\multicolumn{1}{c|}{} & \multicolumn{1}{c|}{} & \multicolumn{1}{c|}{} & Regrease pump bearings &  \\ 
\hline
\multicolumn{1}{c|}{BP6} & \multicolumn{1}{c|}{2} & \multicolumn{1}{c|}{3} & Monitor and/or mitigate possible cavitation progress & 2 \\ 
\multicolumn{1}{c|}{} & \multicolumn{1}{c|}{} & \multicolumn{1}{c|}{} & \multicolumn{1}{c|}{Follow manufacturer prescribed dynamic operation to prevent impeller damage} &  \\ 
\multicolumn{1}{c|}{} & \multicolumn{1}{c|}{} & \multicolumn{1}{c|}{} & Regrease pump bearings &  \\ 
\hline
\multicolumn{1}{c|}{SP1} & \multicolumn{1}{c|}{2} & \multicolumn{1}{c|}{3} & Align motor and pump shafts per manufacturer's specifications  & 2 \\ 
\multicolumn{1}{c|}{} & \multicolumn{1}{c|}{} & \multicolumn{1}{c|}{} & Monitor and/or mitigate possible cavitation progress &  \\ 
\multicolumn{1}{c|}{} & \multicolumn{1}{c|}{} & \multicolumn{1}{c|}{} & \multicolumn{1}{c|}{Follow manufacturer prescribed dynamic operation to prevent impeller damage} &  \\ 
\hline
\multicolumn{1}{c|}{SP2} & \multicolumn{1}{c|}{2} & \multicolumn{1}{c|}{3} & Monitor and/or mitigate possible cavitation progress & 2 \\ 
\multicolumn{1}{c|}{} & \multicolumn{1}{c|}{} & \multicolumn{1}{c|}{} & Follow manufacturer prescribed dynamic operation to prevent impeller damage &  \\ 
\hline
\end{tabular}

	}
\end{table}

A common problem that occurs to all pumps are issues concerning possible misalignment that can affect the vibration of pumps now and in the future. Some proof of issues are with figures shown in subsection \ref{ch04mech02_highlight}. 

\section{Energy management audit}
\label{46}

\subsection{Production and power data}
Production data for this station has been recorded in excel files. Each file represents a month with 24 hours of daily records. Maynilad provided this set of data from 2012 to 2018 per GHD' request. Initial verification on this set was conducted with following conclusions

\begin{itemize}
	\item Data of 2012 and 2013 is not usable due to its incompatibility to the later data. Fundamentally, we found a great number of errors on this data. In addition, the data itself is incomplete and only reflects aggregate value, which makes impossible to compare with the later set;
	\item The structure of data is not homogeneous with many numerical errors. This problem is due to the fact that excel file is not suitable for recording a large volume of data, particularly cells are not set up to reject string and value outside the lower and upper bounds.
	content.
\end{itemize}

When excluding the data of 2012 and 2013, the set used for compilation has following statistics

In order to compile such a huge data set, it is not possible with manual inputting, instead, GHD has developed a hybrid program consisting of Visual Basic (VBA) Code and MySQL code for fast compilation. VBA code is used to add header, fill up missing information in excel file, and ignore rows and columns that should not exist with regard to database structure. MySQL codes are used to eliminate measurement errors and bring together all individual files to one file that allows statistical analysis with R.
\subsection{Measurement errors}
Following measurement errors are with the provided excel files
\begin{itemize}
\item String/text values are found numerous in columns that shall be only numerical values;
\item Extreme values are found numerous;
\item Negative values are found in many places that shall only be positive
\end{itemize}
%\subsection{Summary of statistics}
\subsection{Data compilation for analysis}
Out of all recorded attributes, useful attributes that can be used for energy audit are total production per hour and total power consumption per hour. There is no record on production and power consumption for individual pump.

After data filtering, data correction, and compilation, the obtained set of data includes 27,625 records. Final data set is saved in MySQL server.

\subsection{Analysis}

As a matter of fact, power consumption of a PS is mostly contributed by the operation of pumps. Thus, the audit has been centralized on 
\begin{itemize}
	\item Analyzing given production and power consumption data to understand the trend and establish a benchmark ratio of production vs power for future audit and management;
	\item Evaluating other part of the audit such as pump efficiency and reliability in order to derive better intervention program that will eventually beneficial to the Client to maintain a benchmark level of power consumption against the production. 
\end{itemize}

Figure \ref{ch05_fig_energy_correlation} shows the statistical correlation between production and power. It can be seen from the correlation graph and correlation value that there is little correlation among these two values. This is against the  hypothesis that when production increase, power consumption shall also increase. However, it is not the case as observed from the graph. This infers that the pump system might have incurred a certain level of deterioration leading to its low reliability over time. 

\begin{figure}[!htb]
	\includegraphics[scale=0.6]{figures/ch05_fig_energy_correlation} \\
	\caption{Correlation between production and power consumption}
	\label{ch05_fig_energy_correlation} 
\end{figure}

Figure \ref{ch05_fig_energy_production} shows a trend in time series production since 2014. It can be seen from the graph that the production has kept decreasing slightly over the year. 

\begin{figure}[!htb]
	\includegraphics[scale=0.6]{figures/ch05_fig_energy_production} \\
	\caption{Time series production/hour}
	\label{ch05_fig_energy_production} 
\end{figure}

Figure \ref{ch05_fig_energy_power} shows a trend in time series power consumption since 2014. It can be seen from the graph that the power has kept increasing over the year. 

\begin{figure}[!htb]
	\includegraphics[scale=0.6]{figures/ch05_fig_energy_power} \\
	\caption{Time series power/hour}
	\label{ch05_fig_energy_power} 
\end{figure}
Figure \ref{ch05_fig_energy_ratio} shows time series of ratio between power and production. As the production decreases and power increase, the ratio keeps increasing over time. 

\begin{figure}[!htb]
	\includegraphics[scale=0.6]{figures/ch05_fig_energy_ratio} \\
	\caption{Time series ratio between production and power}
	\label{ch05_fig_energy_ratio} 
\end{figure}

Interpretation from these graphs can be summarized as follows

\begin{itemize}
	\item Efficiency of the pump system has decreased due to more frequent breakdown of pumps;
	\item Pumps might have been operated in a non-optimal operational scheme/sequence;
	%\item Power pressure of suction
\end{itemize}

\subsection{Recommendation}
In order to operate the PS in a manner that is energy efficient, it is advisable to 

\begin{itemize}
\item Establish an optimal operation scheme;
\item Establish a benchmark energy efficiency ratio for continuous monitoring and reporting. This ratio shall become a Key Performance Indicator (KPI) used for managerial purpose.
\end{itemize}


\section{Workplace environment management}
\label{47}

\subsection{Temperature and relative humidity}
\subsubsection{Data}
Data concerning the temperature and relative humidity is presented in Table \ref{ch04_tbl_wem01}. Data was measured at targeted points shown in Figure \ref{ch02_wem01}. Raw data is with the site inspection reports, which will be provided to the Client separately. Persuant to ASHRAE standard, the recommended ranges for temperature and humidity are [72 - 80 $^\circ F$] and [45 - 60 \%], respectively.

\begin{table}[!htb]
	\caption{Temperature and relative humidity.}
	\label{ch04_tbl_wem01}
	{\footnotesize
\begin{tabular}{c|l|c|c|c|c}
	\hline
	Points & Description of points & \multicolumn{2}{c|}{Temperature ($^\circ F$)} & \multicolumn{2}{c}{Humidity (\%)} \\ 
	\cline{3-6}
	&  & Actual & Range & Actual & Range \\ 
	\hline
	& A. Inside pump house (outside the office) &  &  &  &  \\ 
	1 & Between BP5 and BP6 away from pump & 91.04 & 72 - 80 & 59.80 & 45 - 60 \\ 
	2 & Between BP5 and BP6 & 95.72 & 72 - 80 & 53.30 & 45 - 60 \\ 
	3 & Between BP5 and BP4 away from pump & 92.12 & 72 - 80 & 53.40 & 45 - 60 \\ 
	4 & Between BP5 and BP4 & 93.38 & 72 - 80 & 52.70 & 45 - 60 \\ 
	5 & Between BP4 and BP3 away from pump & 93.20 & 72 - 80 & 56.30 & 45 - 60 \\ 
	6 & Between BP4 and BP3 & 96.26 & 72 - 80 & 52.30 & 45 - 60 \\ 
	7 & Between BP3 and BP2 away from pump & 94.28 & 72 - 80 & 53.30 & 45 - 60 \\ 
	8 & Between BP3 and BP2 & 95.00 & 72 - 80 & 52.40 & 45 - 60 \\ 
	9 & Between BP2 and BP1 away from pump & 93.74 & 72 - 80 & 56.90 & 45 - 60 \\ 
	10 & Between BP2 and BP1 & 96.08 & 72 - 80 & 53.40 & 45 - 60 \\ 
	11 & Between SP2 and BP1 away from pump & 93.38 & 72 - 80 & 53.90 & 45 - 60 \\ 
	12 & Between SP2 and BP1 & 96.44 & 72 - 80 & 52.40 & 45 - 60 \\ 
	13 & Between SP2 and SP1 away from pump & 93.56 & 72 - 80 & 56.90 & 45 - 60 \\ 
	14 & Between SP2 and SP1 & 94.10 & 72 - 80 & 56.70 & 45 - 60 \\ 
	& Average & 94.16 &  & 54.55 &  \\ 
	\hline
	& B. Outside pump house &  &  &  &  \\ 
	15 & Near Reservoir & 84.44 & 72 - 80 & 64.73 & 45 - 60 \\ 
	16 & Back of the Office & 89.42 & 72 - 80 & 58.97 & 45 - 60 \\ 
	17 & Near Guard House & 88.46 & 72 - 80 & 59.40 & 45 - 60 \\ 
	18 & Near Diesel Tank & 86.66 & 72 - 80 & 62.33 & 45 - 60 \\ 
	& Average & 87.25 &  & 61.36 &  \\ 
	\hline
	& C. Vicinity &  &  &  &  \\ 
	19 & Near Diesel Tank, outside vicinity & 91.28 & 72 - 80 & 56.57 & 45 - 60 \\ 
	20 & Near Guard House, outside vicinity & 90.56 & 72 - 80 & 56.47 & 45 - 60 \\ 
	& Average & 90.92 &  & 56.52 &  \\ 
	\hline
	21 & D. Office & 82.16 & 72 - 80 & 48.80 & 45 - 60 \\ 
	\hline
\end{tabular}

	}
\end{table}

\subsubsection{Analysis}
As can be seen from Table \ref{ch04_tbl_wem01}, it is obvious that the temperatures inside the pump house at every measurement points are significant higher than the maximum value of the recommended range (80 F). The average value is 94.16 F. The higher values of temperature compared to the range have also been observed for points outside the pump house, in the vicinity, and even inside the office. 

Regarding the relative humidity, recorded data was within the recommended range inside the pump house and in the office (54.55 \% and 48.80 \%, respectively). The humidity value outside the pump house (61.36\%) is slighly higher than the maximum value in the recommended range (60.00\%). However, it can be understandable given the the fact that it is directly affected by the ambient temperature.

As a matter of fact, temperature and humidity is highly correlated and as per XXX, the recommended combination of temperature and humidity shall be within the comfortable zone as shown in Figure \ref{ch04_fig_wem01}.

\begin{figure}[!htb]
	\includegraphics[scale=2]{figures/ch04_fig_wem01} \\
	\caption{ASHRAE standard 55 : Summer Comfort Zone}
	\label{ch04_fig_wem01} 
\end{figure}

\subsubsection{Recommendations}
In order to reduce the negative impacts from high temperature, particularly inside the pump house, the Client shall consider

\begin{itemize}
	\item To establish a good daily monitoring, exercise, and management considering ergonomic and health and occupational activities (e.g. appropriate time window for break in designated resting area);
	\item To execute physical intervention to reduce temperature can be with improving ventilation system by natural mean (e.g. installation of weather proofed louvers). This will be reflected in the conceptual design in Chapter \ref{Chapter6}.
\end{itemize}


\subsection{Air quality}\label{aq01}

\subsubsection{Data and analysis}
Data concerning the air quality is presented in Table \ref{ch04_tbl_wem03} with value of PM2.5 measured in ppm. Data was measured at targeted points shown in Figure \ref{ch02_wem01}. Raw data is with the site inspection reports, which will be provided to the Client separately. Persuant to currently applied standard, the recommended safe ranges for PM2.5 is in [0-35].

\begin{table}[h]
	\caption{Air quality - PM2.5 (ppm).}
	\label{ch04_tbl_wem03}
	{\footnotesize
\begin{tabular}{c|l|c}
	\hline
	Point & Description of the Point Location & PM2.5 \\ 
	\hline
	& A. Inside pump house (outside the office) &  \\ 
	1 & Between BP5 and BP6 away from pump & 16.00 \\ 
	2 & Between BP5 and BP6 & 16.00 \\ 
	3 & Between BP5 and BP4 away from pump & 12.00 \\ 
	4 & Between BP5 and BP4 & 15.00 \\ 
	5 & Between BP4 and BP3 away from pump & 12.00 \\ 
	6 & Between BP4 and BP3 & 15.00 \\ 
	7 & Between BP3 and BP2 away from pump & 13.00 \\ 
	8 & Between BP3 and BP2 & 15.00 \\ 
	9 & Between BP2 and BP1 away from pump & 14.00 \\ 
	10 & Between BP2 and BP1 & 14.00 \\ 
	11 & Between SP2 and BP1 away from pump & 12.00 \\ 
	12 & Between SP2 and BP1 & 14.00 \\ 
	13 & Between SP2 and SP1 away from pump & 15.00 \\ 
	14 & Between SP2 and SP1 & 14.00 \\ 
	& Average & 14.07 \\ 
	\hline
	& B. Outside pump house &  \\ 
	15 & Near Reservoir & 11.00 \\ 
	16 & Back of the Office & 12.00 \\ 
	17 & Near Guard House & 12.00 \\ 
	18 & Near Diesel Tank & 13.00 \\ 
	& Average & 12.00 \\ 
	\hline
	& C. Vicinity &  \\ 
	19 & Near Diesel Tank, outside vicinity & 14.00 \\ 
	20 & Near Guard House, outside vicinity & 14.00 \\ 
	& Average & 14.00 \\ 
	\hline
	21 & Office & 11.00 \\ 
	\hline
\end{tabular}
	}
\end{table}

Values of PM2.5 are all within the range of acceptance, inferring no issue with the quality of air. 


\subsubsection{Recommendations}
Though there is no issue with the air quality, it is anticipated that future problem can incur with a certain low probability, a better management approach is to ensure that all activities/tasks to be executed within the premise of the PS to follow strictly safety and environmental regulation. For example, all employees and staff to wear appropriate dust-proofed masks when working with activities that potentially incurs dusts or other harmful particles.

%\subsection{Hazards}\label{aq02}
%\textcolor{red}{RB Sanchez to write here the summary of raw data collected from visual inspection and testing. Tables shall be used as much as we can. Note that no analysis in this session. This session is purely the high level presentation of data. Raw data can be linked as an Appendix}
\subsection{Illumination}\label{aq03}
\subsubsection{Data and analysis}
Data concerning the illumination is presented in Table \ref{ch04_tbl_wem04} with the LUX value. Data was measured at targeted points shown in Figure \ref{ch02_wem01}. Raw data is with the site inspection reports, which will be provided to the Client separately. Persuant to RULE 1075.4 of DOLE-OSH standard \cite{DOLE2016}, the recommended minimum for LUX is in 100.

\begin{table}[h]
	\caption{Illumination (x 100 LUX).}
	\label{ch04_tbl_wem04}
	{\footnotesize
\begin{tabular}{c|l|c|c|c|c}
	\hline
	Point & Description of the points & \multicolumn{3}{c|}{Trials} & Ave. \\ 
	\cline{3-5}
	&  & 1 & 2 & 3 &  \\ 
	\hline
	& A. Inside pump house (outside the office) &  &  &  &  \\ 
	1 & Between BP5 and BP6 away from pump & 15.02 & 12.67 & 15.50 & 14.40 \\ 
	2 & Between BP5 and BP6 & 13.89 & 12.64 & 11.74 & 12.76 \\ 
	3 & Between BP5 and BP4 away from pump & 8.86 & 9.37 & 9.21 & 9.15 \\ 
	4 & Between BP5 and BP4 & 10.79 & 10.50 & 10.40 & 10.56 \\ 
	5 & Between BP4 and BP3 away from pump & 9.43 & 10.21 & 10.09 & 9.91 \\ 
	6 & Between BP4 and BP3 & 10.45 & 10.50 & 9.84 & 10.26 \\ 
	7 & Between BP3 and BP2 away from pump & 10.73 & 10.62 & 9.96 & 10.44 \\ 
	8 & Between BP3 and BP2 & 11.15 & 10.50 & 10.37 & 10.67 \\ 
	9 & Between BP2 and BP1 away from pump & 9.48 & 9.60 & 9.76 & 9.61 \\ 
	10 & Between BP2 and BP1 & 10.52 & 11.13 & 10.51 & 10.72 \\ 
	11 & Between SP2 and BP1 away from pump & 8.97 & 7.98 & 8.01 & 8.32 \\ 
	12 & Between SP2 and BP1 & 10.57 & 10.78 & 10.41 & 10.59 \\ 
	13 & Between SP2 and SP1 away from pump & 6.43 & 5.43 & 5.83 & 5.90 \\ 
	14 & Between SP2 and SP1 & 5.66 & 5.44 & 5.34 & 5.48 \\ 
	& Average & 10.14 & 9.81 & 9.78 & 9.91 \\ 
	\hline
	& B. Outside pump house &  &  &  &  \\ 
	15 & Near Reservoir & 135.70 & 154.10 & 160.20 & 150.00 \\ 
	16 & Back of the Office & 259.00 & 263.00 & 264.00 & 262.00 \\ 
	17 & Near Guard House & 170.60 & 164.40 & 163.90 & 166.30 \\ 
	18 & Near Diesel Tank & 145.70 & 132.70 & 124.00 & 134.13 \\ 
	& Average & 177.75 & 178.55 & 178.03 & 178.11 \\ 
	\hline
	& C. Vicinity &  &  &  &  \\ 
	19 & Near Diesel Tank, outside vicinity & 268.00 & 260.00 & 274.00 & 267.33 \\ 
	20 & Near Guard House, outside vicinity & 259.00 & 269.00 & 263.00 & 263.67 \\ 
	& Average & 263.50 & 264.50 & 268.50 & 265.50 \\ 
	21 & D. Office & 321.00 & 326.00 & 339.00 & 328.67 \\ 
	\hline
\end{tabular}

	}
\end{table}

The average lightings inside the Pump House (1,000 Lux) where activities are being conducted are 10x higher than the minimum Illumination Level Acceptable Value (100 Lux).

Illumination is provided by natural means through the Skylights and is augmented by Motion-activated Lighting Systems. At night time, the light provided by Lighting System is for a particular zone only or place where there are motions.

\subsubsection{Recommendations}

\begin{itemize}
	\item	Use artificial lighting equipment when accessing and conducting activities requiring detailed output at darker specific areas especially at night because the existing lighting systems cannot provide adequate lighting or when deemed necessary.
	\item	Such artificial supplementary lightings shall be especially designed for the specific tasks and provided with shading or diffusing devices to prevent glare.
	\item	Periodic cleaning of Skylights and glass windows should be implemented to ensure they are kept clean at all times
	
\end{itemize}


\subsection{Industrial ventilation}\label{aq04}
\subsubsection{Data and analysis}

There are two ways of ventilation available, natural and mechanical. Natural ventilation is possible through the entrance door in front of Guard House (Door 1) and the other one beside the office (Door 2) which are both always open. Other door, situated at the farthest side of SP1 (Door 3) can also be used as a means of natural ventilation but it is normally closed and not utilized (Figure \ref{ch04_fig_ventilation01}). Thus, mechanical ventilation is the main option as the plant installed supply and exhaust fan for the pump house running most of the time, specifically during hot dry days and when there are on-going maintenance activities. An air conditioned office is also available for the operator monitoring the plant daily operation.

\begin{figure}[h]
	\includegraphics[scale=2]{figures/ch04_fig_ventilation01} \\
	\caption{Existing ventilation layout}
	\label{ch04_fig_ventilation01} 
\end{figure}

It was realized that natural ventilation through Door 1 and Door 2 is not sufficient to attain the minimum air changes requirement of the Pump House and so mechanical ventilation is utilized most of the time.


\subsubsection{Recommendations}

\begin{itemize}
\item Install weather-proof louver at the Pump House. This is to increase the air change inside the pump house by natural ventilation. The purpose is to utilize the Natural wind around the area and at the same time, lessen the Power Consumption of the Supply and Exhaust Fan. 
\end{itemize}

If the PS is rehabilitated with the recommendation, eventually, it will contribute to the reduction of hazards associated with air quality and energy/heat. The recommendation is also reflected in the conceptual design provided in later chapter.


\subsection{Housekeeping}\label{aq05}
\subsubsection{Documentation}
Following problems are the facts:

\begin{itemize}
\item Current documentation practice is heavily dominated with paper based system, which follows the current practice in Maynilad. There is a large amount/collection of papers that recorded past activities but is of no use and beneficial if data cannot be transformed into digital format for time series analysis, which is an essential part of asset management practice;

\item No proper filing/library system with standardized coding rule that will provide convenience for operators/users to timely find appropriate documents;

\item Daily operation data is crucial information for future analysis but it is recorded in excel based file without relational tables, which makes it from hard to impossible for data compilation, filtering, and mining. Many past data has been recorded with outliers and incorrect data types. 

\end{itemize}

 





\subsubsection{Waste management and environmental control}
There is no issue with regard to waste management and environmental control as confirmed by the checklist shown in Table \ref{ch05_tbl_housekeeping}

\begin{table}[h]
	\caption{Waste management.}
	\label{ch05_tbl_housekeeping}
	{\footnotesize
	\begin{tabular}{l|c|c|c|l}
		\hline
		Description & Status & CS & IT & Remarks \\ 
		\hline
		- Sufficient waste segregation assets & yes & 1 & 1 &  \\ 
		- waste segregation policy & yes & 1 & 1 &  \\ 
		- Signage & yes & 1 & 1 &  \\ 
		- Genset emission control & yes &  & 1 &  \\ 
		\hline
	\end{tabular}	
	}
\end{table}

%The plant has its own waste segregation policy and an organized documentations procedure (evidenced of the arranged daily monitoring sheet). Standby generator set is situated outside the pump house where its exhaust gases through natural ventilation will not be able to penetrate the pump house.

\subsubsection{Office arrangement and ergonomic}
Table \ref{ch05_tbl_egornomic} shows the data concerning parameters associated with office arrangement and ergonomic. 

\begin{table}[!htb]
	\caption{Ergonomic.}
	\label{ch05_tbl_egornomic}
	{\scriptsize
\begin{tabular}{p{2cm}|l|c|p{8cm}}
	\hline
	Parameters & Sub-parameters & Status & Remarks \\ 
	\hline
	Posture & Head & 1 & Ceiling height is high enough to cause head injury while sitting or when standing. \\ 
	&  Neck & 1 & Neck posture is in good ergonomic condition. \\ 
	&  &  & Consider having an interval for fit-break to avoid neck muscles stiffening. \\ 
	& Back & 1 & Back posture while sitting is in good posture.  \\ 
	&  &  & Consider standing and doing fit-break exercises to relax spine.  \\ 
	& Hands/Wrist  & 0 & Proper hand positioning in the keyboard is not observed. \\ 
	&  &  & Wrist bending is seldom. \\ 
	& Feet & 1 & Feet position is in good posture. \\ 
	&  &  & Good clearance below worktables. \\ 
	& Eyes & 0 & The computer monitor is on eyelevel in a certain operator only. \\ 
	&  &  & Consider adjusting the monitor level comfortable to every operator. \\ 
	&  &  & Look away into distance in order to rest the eyes for every 10 minutes or so. \\ 
	\hline
	Equipment / Tool &  &  &  \\ 
	& Computer display & 0 & Not adjusted and the operator get used to its current setting. \\ 
	&  &  & Display brightness must be adjustable in the comfortability of the operator-in-charge. \\ 
	&  &  & Consider the use of anti-glare and blue light to reduce the possibility of eyestrain. \\ 
	& Keyboard & 1 & Keyboard position causes poor hand posture that can lead to injury at long exposure. \\ 
	& Mouse & 1 & Mouse usage is average due to monitoring. \\ 
	&  &  & Prolong usage may cause reduced blood flow leading to muscular injury. \\ 
	& Chair & 0 & Consider using ergonomic chair that is capable of back support, height, armrest adjustments. \\ 
	& Table & 0 & Consider use of ergonomic tables to adjust the height of the table in desired position easily without exerting much effort to adjust manually. \\ 
	& Files & 1 & Hard copy file system and location is well observed. Too high or too low file location may require a person to bend his body or force his hand to grip a file in an awkward posture. Frequent situation may lead to MSD. \\ 
	\hline
	Operations / Maintenance & Illumination & 0 & According to the maintenance team, the motion-activated light is not bright enough to complete their task efficiently at night. Moreover, the light has short on-off delay operation that means that the team must move more often to avoid the light to dim.  \\ 
	&  &  & Consider having a manual switch option to by-pass the motion sensors and le the light on while doing maintenance.  \\ 
	& Noise Exposure & 1 & Noise emitted by the machines in the pump station is high. Consider the use of proper ear protections to reduce the sound intensity. In offices, the sound intensity is acceptable.  \\ 
	& Temperature & 1 & Temperature in the pump station is not acceptable at long exposure. Consider cooling down the body temperature at the designated area (i.e. outside, office). \\ 
	\hline
	Facility / General Workplace & Layout & 1 & Layout of the pump station is well observed. Distance between pumps is acceptable for well maintenance movement.  \\ 
	& Height clearances & 1 & Height clearances from ceiling to head is very high. Chance of getting head injury is very low. \\ 
	\hline
\end{tabular}
	
	}
\end{table}



\subsubsection{Recommendations}
Followings are recommendations

\begin{itemize}
\item Development of a web-based database management system, with appropriate set of relational data tables to record operational data, power consumption data, and intervention data;

\item Development of documentation code and naming for appropriate filing and library/referencing;

\item Applying best practices with regard to ergonomic in combination with interior design and arrangement of office space.

\end{itemize}





\subsection{Noise}\label{aq06}
\subsubsection{Data and analysis}
Data concerning the noise is presented in Table \ref{ch04_tbl_wem02}. Data was measured at targeted points shown in Figure \ref{ch02_wem01}. Raw data is with the site inspection reports, which will be provided to the Client separately. Persuant to 

\begin{table}[h]
	\caption{Noise (dBA)}
	\label{ch04_tbl_wem02}
	\resizebox{\columnwidth}{!}{%
	{\scriptsize
\begin{tabular}{c|p{4cm}|ccc|ccc|ccc|c}
	\hline
	Point & Description of the Point Location & \multicolumn{10}{c}{Trials} \\ 
	\cline{3-12}
	&  & \multicolumn{3}{c|}{1} & \multicolumn{3}{c|}{2} & \multicolumn{3}{c|}{3} &  \\ 
	\cline{3-11}
	&  & Min & Ave. & max & Min & Ave. & Max & Min & Ave. & Max & Ave. \\ 
	\hline
	& A. Inside pump house (outside the office) &  &  &  &  &  &  &  &  &  &  \\ 
	1 & Between BP5 and BP6 away from pump & 91.30 & 94.40 & 97.50 & 91.50 & 93.55 & 95.60 & 91.30 & 93.15 & 95.00 & 93.70 \\ 
	2 & Between BP5 and BP6 & 90.00 & 92.10 & 94.20 & 90.50 & 92.85 & 95.20 & 90.20 & 92.80 & 95.40 & 92.58 \\ 
	3 & Between BP5 and BP4 away from pump & 91.30 & 93.30 & 95.30 & 91.90 & 93.60 & 95.30 & 90.50 & 92.45 & 94.40 & 93.12 \\ 
	4 & Between BP5 and BP4 & 90.60 & 93.00 & 95.40 & 90.80 & 93.25 & 95.70 & 90.60 & 93.00 & 95.40 & 93.08 \\ 
	5 & Between BP4 and BP3 away from pump & 91.60 & 93.25 & 94.90 & 91.80 & 93.55 & 95.30 & 91.30 & 93.30 & 95.30 & 93.37 \\ 
	6 & Between BP4 and BP3 & 91.10 & 93.20 & 95.30 & 91.50 & 93.35 & 95.20 & 91.90 & 93.50 & 95.10 & 93.35 \\ 
	7 & Between BP3 and BP2 away from pump & 92.20 & 93.90 & 95.60 & 92.50 & 93.90 & 95.30 & 92.10 & 94.05 & 96.00 & 93.95 \\ 
	8 & Between BP3 and BP2 & 91.20 & 92.60 & 94.00 & 90.70 & 92.35 & 94.00 & 90.80 & 92.40 & 94.00 & 92.45 \\ 
	9 & Between BP2 and BP1 away from pump & 91.70 & 93.65 & 95.60 & 91.70 & 93.45 & 95.20 & 92.10 & 93.45 & 94.80 & 93.52 \\ 
	10 & Between BP2 and BP1 & 90.40 & 92.70 & 95.00 & 90.20 & 92.55 & 94.90 & 90.90 & 93.00 & 95.10 & 92.75 \\ 
	11 & Between SP2 and BP1 away from pump & 93.50 & 94.85 & 96.20 & 92.40 & 94.55 & 96.70 & 92.00 & 94.10 & 96.20 & 94.50 \\ 
	12 & Between SP2 and BP1 & 90.70 & 92.95 & 95.20 & 90.60 & 92.85 & 95.10 & 90.60 & 92.70 & 94.80 & 92.83 \\ 
	13 & Between SP2 and SP1 away from pump & 92.10 & 94.30 & 96.50 & 91.90 & 93.85 & 95.80 & 92.20 & 94.00 & 95.80 & 94.05 \\ 
	14 & Between SP2 and SP1 & 91.20 & 93.35 & 95.50 & 90.80 & 92.95 & 95.10 & 90.80 & 91.95 & 93.10 & 92.75 \\ 
	& Average &  & 93.40 &  &  & 93.33 &  &  & 93.13 &  & 93.29 \\ 
	\hline
	& B. Outside pump house &  &  &  &  &  &  &  &  &  &  \\ 
	15 & Near Reservoir & 64.70 & 71.70 & 78.70 & 64.20 & 65.80 & 67.40 & 64.20 & 65.60 & 67.00 & 67.70 \\ 
	16 & Back of the Office & 71.50 & 77.25 & 83.00 & 71.40 & 74.90 & 78.40 & 71.80 & 72.60 & 73.40 & 74.92 \\ 
	17 & Near Guard House & 65.40 & 66.30 & 67.20 & 64.60 & 65.65 & 66.70 & 64.80 & 66.35 & 67.90 & 66.10 \\ 
	18 & Near Diesel Tank & 65.10 & 72.70 & 80.30 & 65.70 & 69.75 & 73.80 & 65.40 & 69.35 & 73.30 & 70.60 \\ 
	& Average &  & 71.99 &  &  & 69.03 &  &  & 68.48 &  & 69.83 \\ 
	\hline
	& C. Vicinity &  &  &  &  &  &  &  &  &  &  \\ 
	19 & Near Diesel Tank, outside vicinity & 65.70 & 69.50 & 73.30 & 66.00 & 68.35 & 70.70 & 65.50 & 72.35 & 79.20 & 70.07 \\ 
	20 & Near Guard House, outside vicinity & 66.80 & 71.75 & 76.70 & 67.00 & 71.75 & 71.00 & 66.90 & 71.75 & 76.00 & 71.07 \\ 
	& Average &  & 70.63 &  &  & 70.05 &  &  & 72.05 &  & 70.91 \\ 
	\hline
	21 & Office & 66.70 & 68.10 & 69.50 & 68.00 & 68.40 & 68.80 & 68.60 & 69.35 & 70.10 & 68.62 \\ 
	\hline
\end{tabular}
	}}
\end{table}


All pumps (6 Booster and 2 Storage Pumps) are running during the Sound Level Testing and so the reading closely represents the normal daily noise level inside the Plant. The average sound level inside the Pump House is 93 dBA, beyond the standard of less than 90 dbA. 

On the other hand, the sound level at locations – Inside Pump Station and Reservoir, Outside the Vicinity of Pump Station and inside the office are from 69 – 70 dBA, an acceptable value.

\subsubsection{Recommendations}

\begin{itemize}
	\item	Use protective hearing equipment when working inside the Pump House and have a scheduled break/rest at designated location. Shall not be exposed at such noise beyond 5 hours in a day.
	\item	Designate location inside the Plant with the minimum noise level - can be the Office, Inside Pump Station and Reservoir, and Outside the Vicinity of Pump Station, if below is not possible.
	\item	 Install Sound Attenuation Device (such as sound-absorbing wall panels and door seals) at the Office to reduce the current 69 dBA to ideal Office level of 50-55 dBA.
	
	\item Purchase a sufficient numbers of electronic noise canceling earphones.
\end{itemize}

\section{Fire protection and safety (FDAS) audit} \label{ch04fdas}
\subsection{Fire alarm and detection system} \label{ch04fdas01}
\subsubsection{Data and analysis}
Summary of data and information from FDAS audit is presented in Table \ref{ch04_fdas01} with visual images on as-found devices and panels (Figure \ref{ch04_fig_fdas01}).

\begin{table}[!h]
	\caption{FDAS data highlights.}
	\label{ch04_fdas01}
	{\scriptsize
\begin{tabular}{c|p{6.5cm}|c|p{6.5cm}}
	\hline
	No. & Assets & Status & Remarks \\ 
	\hline
	A & Visual check of the fire alarm control panel &  &  \\ 
	1 & Panel Status, installed and location area & 1 & Power up, located near Engineering office with light indicator \\ 
	2 & Power indicator lamp operational & 1 & Operational \\ 
	3 & Devices properly indicated and marked & 1 & Device installed \@ motor control room and 2nd floor, 3 meters in height \\ 
	4 & Panel clear from trouble indicators & 0 & For verification of fault \\ 
	5 & Lamp test indicator operational & 0 & For verification of fault \\ 
	6 & Zones properly indicated and marked & 1 & Conventional system \\ 
	7 & Check if it’s connected to sprinkler system & 0 & For verification on testing \\ 
	\hline
	B & Checking of installed devices &  &  \\ 
	1 & Check floor plan lay-out and location of the device if accessible/easy to access & 1 & No FDAS lay-out but devices are accessible \\ 
	2 & Heat detectors and / or smoke detectors indicator lamp functioning & 1 & For verification \\ 
	3 & Heat detectors and / or smoke detectors indicator lamp functioning & 0 & Not all indicator lamp \@ the device are functioning \\ 
	4 & Pull station locations acceptable & 0 & For Verification \\ 
	5 & Bells and buzzers operated correctly & 0 & For verification \\ 
	6 & Bells and buzzers audibility & 0 & For verification \\ 
	7 & Strobe lights locations are acceptable & 0 & Lacking devices at rooms \\ 
	8 & Strobe light operated correctly & 0 & For verification during testing \\ 
	9 & Are Fire alarm zones (areas) clearly marked & 0 &  \\ 
	10 & Is there a maintenance and service contract for the fire alarm system & 1 & None as per interview with the maintenance \\ 
	& Does the Fire Alarm System smoke detector, heat detector, manual call point , horn and strobe light working and  have a current inspection tag & 0 & None \\ 
	& Is the fire alarm system if full working order & 0 & For verification \\ 
	\hline
\end{tabular}

	}
\end{table}

\begin{figure}[!h]
	\includegraphics[scale=1.7]{figures/ch04_fig_fdas01} \\
	\caption{As-found devices and panels}
	\label{ch04_fig_fdas01} 
\end{figure}



\begin{table}[!h]
	\caption{FDAS data highlights.}
	\label{ch04_fdas011}
	{\footnotesize
		\begin{tabular}{c|p{4cm}|c|c|p{7cm}}
			\hline
			No. & Assets/Description & Status & CS & Remarks  \\ 
			\hline
			1 & Evacuation plan &   1 &  &    \\ 
			2 & Fire extinguishers &   1 &  & Green (HCFC)    \\ 
			&    &  &  & FEX signage without physical Fire Extingushers    \\ 
			&    &  &  & As per plan, there should be 6 FEX   \\ 
			3 & Fire exits   & 1 &  & With exit indication   \\ 
			4 & Fire hose cabine   & 1 &  & With inspection tag 8/30/18   \\ 
			5 & Fire sprinkler system   & 1 &  & No sprinkler system   \\ 
			6 & Emergency exit signages   & 1 &  & BIGLITE Brand   \\ 
			&    &  &  & Exit signage do no have power supply   \\ 
			7 & Emergency lights &   1 &  & Firefly brand. Last Inspection 8/30/18  \\ 
			&    &  &  & All emergency lights are with light indicator   \\ 
			8 & PPE cabinet &   1 &  & with latest tag inspection as of 8/30/18   \\ 
			\hline
		\end{tabular}	
	}
\end{table}

\begin{figure}[!h]
	\begin{minipage}[b]{0.5\linewidth}
		\centering
		\includegraphics[width=\textwidth]{figures/ch04_fig_safety01}
		\caption*{(a - 1st floor)}
		%		\label{ch02_fdas01}
	\end{minipage}
	\hspace{0.05cm}
	\begin{minipage}[b]{0.5\linewidth}
		\centering
		\includegraphics[width=\textwidth]{figures/ch04_fig_safety02}
		\caption*{(b -2nd floor)}
		%		\label{ch02_fdas02}
	\end{minipage}
	\caption{Existing evacuation plan}
	\label{ch04_fig_safety01}
\end{figure}


\begin{figure}[!h]
	
	\begin{minipage}[b]{0.22\linewidth}
		\centering
		\includegraphics[width=\textwidth]{figures/ch04_fig_safety03}
		\caption*{(a - Emergency light)}
		%	\label{ch02_fdas03}
	\end{minipage}
	\hspace{0.03cm}
	\begin{minipage}[b]{0.22\linewidth}
		\centering
		\includegraphics[width=\textwidth]{figures/ch04_fig_safety04}
		\caption*{(b-Exit door)}
		%	\label{ch02_fdas03}
	\end{minipage}
	\hspace{0.03cm}
	\begin{minipage}[b]{0.22\linewidth}
		\centering
		\includegraphics[width=\textwidth]{figures/ch04_fig_safety05}
		\caption*{(c- Cabinet)}
		%	\label{ch02_fdas03}
	\end{minipage}
	\hspace{0.03cm}
	\begin{minipage}[b]{0.22\linewidth}
		\centering
		\includegraphics[width=\textwidth]{figures/ch04_fig_safety06}
		\caption*{(d - Fire hose)}
		%	\label{ch02_fdas03}
	\end{minipage}
	\hspace{0.03cm}
	\begin{minipage}[b]{0.22\linewidth}
		\centering
		\includegraphics[width=\textwidth]{figures/ch04_fig_safety07}
		\caption*{(e - HFCF FEX)}
		%	\label{ch02_fdas03}
	\end{minipage}
	\hspace{0.03cm}
	\begin{minipage}[b]{0.22\linewidth}
		\centering
		\includegraphics[width=\textwidth]{figures/ch04_fig_safety08}
		\caption*{(f-Dry chemical FEX)}
		%	\label{ch02_fdas03}
	\end{minipage}
	\hspace{0.03cm}
	\begin{minipage}[b]{0.5\linewidth}
		\centering
		\includegraphics[width=\textwidth]{figures/ch04_fig_safety09}
		\caption*{(g - Exit signage)}
		%	\label{ch02_fdas03}
	\end{minipage}
	\caption{Existing safety devices}
	\label{ch04_fig_safety02}
\end{figure}


\subsubsection{Recommendations}

The findings/facts and results of the audit are presented in Table \ref{ch05_tbl_fdas01}. Visual images of assets are shown in Figure \ref{ch05_fig_fdas01}. 

Highlights are

\begin{itemize}
	\item \textbf{Smoke Detector 01:} Red indicator light should be visible after spraying of smoke tester.  Removal of device from base to Reset contact point and cleaning did not show any improvement on the device. Hence device is declared not functioning and there is communication failure between device and FACP panel;
	\item \textbf{Smoke Detector 02:} Smoke detector is still functioning but there a   is communication failure between device and FACP panel  since the  FACP did not detect the change of status of the device during testing; 
	\item \textbf{Smoke Detector 03:} Red indicator light should be visible after spraying of smoke tester.  Removal of device from base to Reset contact point and cleaning did not show any improvement on the device. Hence device is declared not functioning and there is communication failure between device and FACP panel;
	\item \textbf{Smoke Detector 04:} Red indicator light should be visible after spraying of smoke tester.  Removal of device from base to Reset contact point and cleaning did not show any improvement on the device. Hence device is declared not functioning and there is communication failure between device and FACP panel;
	\item \textbf{Smoke Detector 05:} Smoke detector is still functioning but there is a communication failure between device and FACP panel  since the  FACP did not detect the change of status of the device during testing;
	\item \textbf{Smoke Detector 06:} Red indicator light should be visible after spraying of smoke tester.  Removal of device from base to Reset contact point and cleaning did not show any improvement on the device. Hence device is declared not functioning and there is communication failure between device and FACP;
	\item \textbf{Manual Call Point 01:} Communication failure between Manual call point 01 and FACP;
	\item \textbf{Manual Call Point 02:} Communication failure between Manual call point 02 and FACP; 
	\item \textbf{Buzzer with strobe light 01:} Communication failure between Buzzer  01 and FACP;
	\item \textbf{Buzzer with strobe light 02:} Communication failure between Buzzer 02 and FACP;
	\item \textbf{Fire Alarm control Panel (FACP):} System failure of FDAS  and devices  due to communication failure and not functioning devices installed within the system.
\end{itemize}

\begin{table}[!h]
	\caption{FDAS analysis.}
	\label{ch05_tbl_fdas01}
	{\scriptsize
		\begin{tabular}{l|l|l|l|p{5cm}|p{5cm}}
			\hline
			No. & Assets & CS & IT & Facts & Remarks \\ 
			\hline
			1 & SM 01 & 0 & 4 & dust inside and outside & no response after repeat sprays \\ 
			&  &  &  & no light indication & Repeat clean \\ 
			&  &  &  & broken base & After 3x sprays, still no response \\ 
			&  &  &  & Spray Max 3 times &  \\ 
			\hline
			2 & SM 02 & 1 & 4 & dust inside and outside & no response after repeat sprays \\ 
			&  &  &  & With light indication (orange color maintained light) & Repeat clean \\ 
			&  &  &  & Spray Max 3 times & After 2x spray, light turned red but without response on the alarm panel \\ 
			&  &  &  &  & Removed and cleaned dirt and smoke particles \\ 
			&  &  &  &  & No light indicator after cleaning \\ 
			\hline
			3 & SM 03 & 0 & 4 & dust inside and outside & no response after repeat sprays \\ 
			&  &  &  & With light indicator (orange color maintained light) & Repeat clean \\ 
			&  &  &  & Spray Max 3 times & After 3x sprays, still no response \\ 
			&  &  &  &  & Removed and cleaned dirt and smoke particles \\ 
			&  &  &  &  & No light indicator after cleaning (reset) \\ 
			\hline
			4 & SM 04 & 0 & 4 & dust inside and outside & Same as SM 03 \\ 
			&  &  &  & With light indicator (orange color maintained light) &  \\ 
			&  &  &  & Spray Max 3 times &  \\ 
			\hline
			5 & SM 05 & 0 & 4 & Same as SM 04 & Same as SM 04 \\ 
			\hline
			6 & SM 06 & 0 & 4 & Same as SM 04 & Same as SM 04 \\ 
			\hline
			7 & MCP 01 & 0 & 4 & No response after pushing &  \\ 
			\hline
			8 & MCP 02 & 0 & 4 & No response after pushing &  \\ 
			\hline
			9 & Buzzer 01 & 0 & 4 & Spray SD /push MCP & No response/audible sound/ smoke  \\ 
			\hline
			10 & Buzzer 02 & 0 & 4 & Spray SD /push MCP & No response/audible sound/ smoke  \\ 
			\hline
			11 & FACP & 0 & 4 & Spray SD /push MCP & No response \\ 
			&  &  &  &  & No response on pushbutton, reset, evacuation, silence, mute buzzer or disabled/enabled \\ 
			&  &  &  &  & Remain light signal indicator for fire, power charger/fault. Fault zone 1\&3 \\ 
			\hline
		\end{tabular}
		
	}
\end{table}



\begin{figure}[!h]
	
	\begin{minipage}[b]{0.22\linewidth}
		\centering
		\includegraphics[width=\textwidth]{figures/ch05_fdas_sd01}
		\caption*{a - Smoker detector 01}
		%	\label{ch02_fdas03}
	\end{minipage}
	\hspace{0.03cm}
	\begin{minipage}[b]{0.22\linewidth}
		\centering
		\includegraphics[width=\textwidth]{figures/ch05_fdas_sd02}
		\caption*{b - Smoker detector 02}
		%	\label{ch02_fdas03}
	\end{minipage}
	\hspace{0.03cm}
	\begin{minipage}[b]{0.22\linewidth}
		\centering
		\includegraphics[width=\textwidth]{figures/ch05_fdas_sd03}
		\caption*{c - Smoker detector 03}
		%	\label{ch02_fdas03}
	\end{minipage}
	\hspace{0.03cm}
	\begin{minipage}[b]{0.22\linewidth}
		\centering
		\includegraphics[width=\textwidth]{figures/ch05_fdas_sd04}
		\caption*{d - Smoker detector 04}
		%	\label{ch02_fdas03}
	\end{minipage}
	\hspace{0.03cm}
	\begin{minipage}[b]{0.22\linewidth}
		\centering
		\includegraphics[width=\textwidth]{figures/ch05_fdas_sd05}
		\caption*{e - Smoker detector 05}
		%	\label{ch02_fdas03}
	\end{minipage}
	\hspace{0.03cm}
	\begin{minipage}[b]{0.22\linewidth}
		\centering
		\includegraphics[width=\textwidth]{figures/ch05_fdas_sd06}
		\caption*{f - Smoker detector 06}
		%	\label{ch02_fdas03}
	\end{minipage}
	\hspace{0.03cm}
	\begin{minipage}[b]{0.22\linewidth}
		\centering
		\includegraphics[width=\textwidth]{figures/ch05_fdas_mcp01}
		\caption*{g - Manual call point 01}
		%	\label{ch02_fdas03}
	\end{minipage}
	\hspace{0.03cm}
	\begin{minipage}[b]{0.22\linewidth}
		\centering
		\includegraphics[width=\textwidth]{figures/ch05_fdas_mcp02}
		\caption*{h - Manual call point 02}
		%	\label{ch02_fdas03}
	\end{minipage}
	\hspace{0.03cm}
	\begin{minipage}[b]{0.22\linewidth}
		\centering
		\includegraphics[width=\textwidth]{figures/ch05_fdas_buzzer01}
		\caption*{i - buzzer 01}
		%	\label{ch02_fdas03}
	\end{minipage}
	\hspace{0.03cm}
	\begin{minipage}[b]{0.22\linewidth}
		\centering
		\includegraphics[width=\textwidth]{figures/ch05_fdas_buzzer02}
		\caption*{j - buzzer 01}
		%	\label{ch02_fdas03}
	\end{minipage}
	\hspace{0.03cm}
	\begin{minipage}[b]{0.22\linewidth}
		\centering
		\includegraphics[width=\textwidth]{figures/ch05_fdas_facp}
		\caption*{k - FACP}
		%	\label{ch02_fdas03}
	\end{minipage}
	\caption{FDAS assets}
	\label{ch05_fig_fdas01}
\end{figure}

In brief, FDAS of the station is not provide adequate level of services mainly due to:
\begin{itemize}
\item Most of the smoke detector devices, manual call point, buzzer and the FACP were not functioning. These were established  during the conducted testing of FDAS and devices;

\item Lacking smoke / heat detector devices at substation room and genset room, engineer’s office, pantry and guardhouse. In every close room there should be a device installed to detect heat or smoke. 

%\item Lightning protection system is not in place	

\end{itemize}

\paragraph{\underline{Short term recommendations}}

\begin{itemize}
\item Troubleshoot wiring for output voltage signal and check fuse of panel to determine if FACP if is still functioning completely and working in coordination with the devices;

\item Troubleshoot wiring outside the panel to determine if there is non-continuity and check end of line resistor if it is still functioning;

\item Troubleshoot devices declared as non-functioning. Follow manufacturers procedure on troubleshooting.

\end{itemize}


\paragraph{\underline{Short term recommendations}}

\begin{itemize}
	\item [$\checkmark$] Troubleshoot wiring for output voltage signal and check fuse of panel to determine if FACP if is still functioning completely and working in coordination with the devices;
	
	\item[$\checkmark$] Troubleshoot wiring outside the panel to determine if there is non-continuity and check end of line resistor if it is still functioning;
	
	\item[$\checkmark$] Troubleshoot devices declared as non-functioning. Follow manufacturers procedure on troubleshooting.
	
\end{itemize}


\paragraph{\underline{Long term recommendations}}

\begin{itemize}
	\item [$\checkmark$] Replace the system with newer and addressable type FACP to determine exact location of fire as it happens;
	
	\item[$\checkmark$] Additional smoke detector devices are required in engineer’s office, guardhouse, pump area;
	
	\item[$\checkmark$] Additional heat detector devices are required in pantry, genset room, substation room;
	
	\item[$\checkmark$] Additional strobe with sounder and call point  at the genset and substation room since these rooms are located in separate buildings and for safety reasons for sounding alarm at the instant that there is fire;
	
	\item [$\checkmark$]Annual Inspection, Testing and Maintenance
\end{itemize}


\paragraph{\underline{System Testing }}

FDAS shall be subjected to the following tests conforming to the Philippine Electronics Code of 2014 and Philippine Electrical Code of 2017
%\renewcommand{\labelitemi}{$\checkmark$}
\begin{itemize}%[label={\checkmark}]
	\item [$\checkmark$] Testing of insulation resistance and continuity of wires;
	\item [$\checkmark$] Verification of installed devices;
	\item [$\checkmark$] Operation and response of FDAS;
	\item [$\checkmark$] Testing the operation of initiating devices;
	\item [$\checkmark$] Measuring sound pressure level generated by notification devices;
\end{itemize}


\paragraph{\underline{Records }}

Every FDAS system shall keep the following documentations
%\renewcommand{\labelitemi}{$\checkmark$}
\begin{itemize}%[label={\checkmark}]
	\item [$\checkmark$] A complete set of operation and maintenance manuals of the manufacturer covering all equipment used in the system;
	\item [$\checkmark$] A complete set of as-built drawings;
	\item [$\checkmark$] A written sequence of operation;
	\item [$\checkmark$] Record of completion and results of every inspection, testing and maintenance;
	\item [$\checkmark$] Record of components within the database.
\end{itemize}



\subsection{Lighting protection system} \label{ch04fdas02}
\subsubsection{Data and analysis}
No lightning protection was installed for this PS.

\subsubsection{Recommendations}

Refer to the conceptual design in Chapter \ref{Chapter6}

\paragraph{\underline{Short term Recommendations}}

Plan for the installation of a new lightning protection system 
%\renewcommand{\labelitemi}{$\checkmark$}
\begin{itemize}%[label={\checkmark}]
	\item [$\checkmark$] the LPS conforms to the design and is based on the  standard;
	\item [$\checkmark$] all components of the LPS are in good condition and capable of performing their designed functions, and that there is no corrosion.
\end{itemize}


\paragraph{\underline{Long term Recommendations}}

Plan for the installation of a new lightning protection system 
%\renewcommand{\labelitemi}{$\checkmark$}
\begin{itemize}%[label={\checkmark}]
	\item [$\checkmark$] According to the standard, an inspection should be undertaken during the construction of the structure, after the installation, after alterations or repairs, and when it is known that the structure has been struck by lightning;
	\item [$\checkmark$] It is also recommended that inspections take place “periodically at such intervals as determined with regard to the nature of the structure to be protected”, taking into account the local environment, such as corrosive soils and corrosive atmospheric conditions and the type of protection measures employed;
	\item [$\checkmark$]The inspection comprises checking the technical documentation, visual inspections and test measurements;
	\item [$\checkmark$]Prepare an inspection guide to facilitate the inspection process containing sufficient information on the installation and its components, tests methods and previous inspection/test data;	
	\item [$\checkmark$]During the visual inspection, the following should be checked;	
	\begin{itemize}
		\item [-] the deterioration and corrosion of air-termination elements, conductors and connections
		\item [-]	the corrosion of earth electrodes
		\item [-]	the earthing resistance value for the earth-termination system
		\item [-]	the condition of connections, equipotential bonding and fixings.
		
	\end{itemize}
	
	\item [$\checkmark$] For those parts of an earthing system and bonding network not visible for inspection, tests of electrical continuity should be performed;
	
	\item [$\checkmark$] An inspection report should be prepared detailing the status of the system, any deviations from the technical documentation and the results of any measurements undertaken. Any obvious faults should also be reported.
\end{itemize}

No lightning protection system is 100\% effective. A system designed in compliance with the standard does not guarantee immunity from damage. Lightning protection is an issue of statistical probabilities and risk management. A system designed in compliance with the standard should statistically reduce the risk to below a pre-determined threshold. The IEC 62305-2 risk management process provides a framework for this analysis. An effective lightning protection system needs to control a variety of risks. While the current of the lightning flash creates a number of electrical hazards, thermal and mechanical hazards also need to be addressed. 

Risk to persons (and animals) include: 

\begin{itemize}
\item Direct flash;
\item  Step potential ;
\item Touch potential ;
\item  Side flash ;
\item Secondary effects

\begin{itemize}
	
	 \item[-]  asphyxiation from smoke or injury due to fire 
	\item [-] structural dangers such as falling masonry from  point of strike 
	\item [-] unsafe conditions such as water ingress from roof  penetrations causing electrical or other hazards,  failure or malfunction of processes, equipment and  safety systems

\end{itemize}
\end{itemize}




Risk to structures \& internal equipment include: 

\begin{itemize}
\item Fire and/or explosion triggered by heat of lightning flash,  its attachment point or electrical arcing of lightning  current within structures ;
\item  Fire and/or explosion triggered by ohmic heating of  conductors or arcing due to melted conductors;
\item Punctures of structure roofing due to plasma heat  at lightning point of strike ;
\item Failure of internal electrical and electronic systems ;
\item Mechanical damage including dislodged materials at  point of strike.
\end{itemize}










\subsection{Ground-Fault circuit interrupter (GFCI) or electric leakage circuit breaker (ELCB) or Residual circuit devices (RCD)} \label{ch04fdas03}
\subsubsection{Data and analysis}
No ground fault circuit interrupter (GFCI) or earth leakage Circuit breaker (ELCB) protection was installed in the panel for FDAS for this PS.
\subsubsection{Recommendations}
Refer to the conceptual design in Chapter \ref{Chapter6}
\subsection{Electrical safety and protective devices} \label{ch04fdas04}
\subsubsection{Data and analysis}
Facts obtained from inspection are presented in Table \ref{ch05_tbl_fdassafe01} with indicative figures for each devices presented in Figure \ref{ch05_fig_fdassafety01}.

\begin{table}[!htb]
	\caption{Protective devices.}
	\label{ch05_tbl_fdassafe01}
	{\scriptsize
\begin{tabular}{c|l|c|c|p{3cm}|p{5cm}}
	\hline
	No. & Assets & CS & IT & Facts & Remarks \\ 
	\hline
	1 & Exit light 01 & 0 & 4 & Not function & Unsafe \\ 
	& (Fig. \ref{ch04_fig_safety02}-g) &  &  & No light indication & risk to the safety of people in the event of power a power outage \\ 
	\hline
	2 & Exit light 02 & 0 & 4 & Not function & Unsafe \\ 
	& (Fig. \ref{ch05_fig_fdassafety01}-h) &  &  & No light indication & risk to the safety of people in the event of power a power outage \\ 
	\hline
	3 & Exit light 03 & 0 & 4 &  & Unsafe \\ 
	& (Fig. \ref{ch04_fig_safety02}-b) &  &  &  & risk to the safety of people in the event of power a power outage \\ 
	\hline
	4 & Exit light 04 & 0 & 4 &  & Unsafe \\ 
	& (Fig. \ref{ch05_fig_fdassafety01}-i) &  &  &  & risk to the safety of people in the event of power a power outage \\ 
	\hline
	5 & Exit light 05 & 0 & 4 &  & Unsafe \\ 
	& (Fig. \ref{ch05_fig_fdassafety01}-j) &  &  &  & risk to the safety of people in the event of power a power outage \\ 
	\hline
	6 & Emergency light tag 01  & 0 & 4 & No updated & Unsafe \\ 
	& (Fig. \ref{ch04_fig_safety02}-a) &  &  & Last update is 30/Aug/2018 & Failure to inspect emergency light can make it ineffective  \\ 
	\hline
	7 & Emergency light tag 02 & 0 & 4 & No tag & Unsafe \\ 
	& (Fig. \ref{ch05_fig_fdassafety01}-a) &  &  &  & Failure to inspect emergency light can make it ineffective  \\ 
	\hline
	8 & Emergency light tag 03 & 0 & 4 & No tag & Unsafe \\ 
	& (Fig. \ref{ch05_fig_fdassafety01}-b) &  &  &  & Failure to inspect emergency light can make it ineffective  \\ 
	\hline
	9 & MCP\#1 & 0 & 4 & No updated & Unsafe \\ 
	& (Fig. \ref{ch05_fig_fdas01}-g) &  &  &  & Failure to inspect manual call point can make it ineffective in case of fire \\ 
	\hline
	10 & Fire extinguisher 01 & 0 & 4 & Missing & Unsafe \\ 
	& (Fig. \ref{ch05_fig_fdassafety01}-c) &  &  &  & Failure to replace or locate fire extinguisher can delay response to  \\ 
	\hline
	11 & Fire extinguisher 02 & 0 & 4 & Missing & Unsafe \\ 
	& (Fig. \ref{ch05_fig_fdassafety01}-d) &  &  &  & Failure to replace or locate fire extinguisher can delay response to  \\ 
	\hline
	12 & Fire extinguisher 03 & 0 & 4 & Missing & Unsafe \\ 
	& (Fig. \ref{ch05_fig_fdassafety01}-e) &  &  &  & Failure to replace or locate fire extinguisher can delay response to  \\ 
	\hline
	13 & Fire extinguisher 04 & 0 & 4 & No inspection tag & Unsafe \\ 
	& (Fig. \ref{ch05_fig_fdassafety01}-f) &  &  &  & Failure to inspect fire extinguisher can make it ineffective in case of fire. \\ 
	\hline
	14 & PPE cabinet tag & 0 & 4 & Not updated & Unsafe \\ 
	& (Fig. \ref{ch05_fig_fdassafety01}-g) &  &  & Last update is 7-30-2018 & Failure to inspect PPE cabinet may not ensure completeness of content and PPE may have damage that can make these equipment ineffective  \\ 
	\hline
\end{tabular}
}
\end{table}


\begin{figure}[!htb]
	\begin{minipage}[b]{0.22\linewidth}
		\centering
		\includegraphics[width=\textwidth]{figures/ch05_fdas_sd07}
		\caption*{a - Emergency light 2}
		%	\label{ch02_fdas03}
	\end{minipage}
	\hspace{0.03cm}
	\begin{minipage}[b]{0.22\linewidth}
		\centering
		\includegraphics[width=\textwidth]{figures/ch05_fdas_sd08}
		\caption*{b - Emergency light 3}
		%	\label{ch02_fdas03}
	\end{minipage}
	\hspace{0.03cm}
	\begin{minipage}[b]{0.22\linewidth}
		\centering
		\includegraphics[width=\textwidth]{figures/ch05_fdas_sd09}
		\caption*{c - Fire extinguisher 1}
		%	\label{ch02_fdas03}
	\end{minipage}
	\hspace{0.03cm}
	\begin{minipage}[b]{0.22\linewidth}
		\centering
		\includegraphics[width=\textwidth]{figures/ch05_fdas_sd10}
		\caption*{d - Fire extinguisher 2}
		%	\label{ch02_fdas03}
	\end{minipage}
	\hspace{0.03cm}
	\begin{minipage}[b]{0.22\linewidth}
		\centering
		\includegraphics[width=\textwidth]{figures/ch05_fdas_sd11}
		\caption*{e - Fire extinguisher 3}
		%	\label{ch02_fdas03}
	\end{minipage}
	\hspace{0.03cm}
	\begin{minipage}[b]{0.22\linewidth}
		\centering
		\includegraphics[width=\textwidth]{figures/ch05_fdas_sd12}
		\caption*{f - Fire extinguisher 4}
		%	\label{ch02_fdas03}
	\end{minipage}
	\hspace{0.03cm}
	\begin{minipage}[b]{0.22\linewidth}
		\centering
		\includegraphics[width=\textwidth]{figures/ch05_fdas_sd13}
		\caption*{g - PPE cabinet}
		%	\label{ch02_fdas03}
	\end{minipage}
	\hspace{0.03cm}
\begin{minipage}[b]{0.22\linewidth}
	\centering
	\includegraphics[width=\textwidth]{figures/ch05_fdas_sd14}
	\caption*{h - Exit light 02}
	%	\label{ch02_fdas03}
\end{minipage}
	\hspace{0.03cm}
\begin{minipage}[b]{0.22\linewidth}
	\centering
	\includegraphics[width=\textwidth]{figures/ch05_fdas_sd15}
	\caption*{i - Exit light 04}
	%	\label{ch02_fdas03}
\end{minipage}
	\hspace{0.03cm}
\begin{minipage}[b]{0.22\linewidth}
	\centering
	\includegraphics[width=\textwidth]{figures/ch05_fdas_sd16}
	\caption*{j - Exit light 05}
	%	\label{ch02_fdas03}
\end{minipage}
	\caption{Protective devices.}
	\label{ch05_fig_fdassafety01}
\end{figure}

\subsubsection{Recommendations}
Based on the status of devices, recommendations are with intervention types shown in Table \ref{ch05_tbl_fdassafe01}. Chapter \ref{Chapter6} further illustrates the recommendation with the conceptual design.
%\subsection{Recommendations}





