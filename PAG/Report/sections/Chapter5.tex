\chapter{Analysis and Discussion} % Write in your own chapter title
\label{Chapter5}
%\lipsum[1-1]
%\textbf{Keywords:} Weibull hazard function, optimal renewal interval, pipeline management.
\section{Mechanical Audit}
\label{42}
\subsection{Pump discharge and suction pipe - thickness} \label{ch05mech01}
%\textcolor{red}{RB Sanchez to write here the findings. Note that the findings and result and analysis. Try to insert picture and tables to details the findings. Any recommendation shall not be written here. This section is purely about findings, the facts and the cause of the problem if any}

With the limitations of conducting thickness measurement at one time only, the following are the observations/findings and recommendations for each pumping system.

\subsubsection{Discussion}

\paragraph{\textbf{BP1}}
\begin{itemize}
\item Suction Piping System - The extrados thickness is less than the remaining thickness values measured.
\item 	Discharge Piping System - The average extrados thickness is 4.346 mm, with a minimum of 3.94 at the entry area to the center of the extrados. Localized thinning is observed to be higher at the upper half of the discharge line (as seen in 4m mark, 12-3-9 o'clock position). Then extends to the elbow entry up to the center of extrados. This may be caused by the cavitation carried over from the pump to the discharge side and then amplified by the backflows in the elbow.
\end{itemize}

\paragraph{\textbf{BP2}}
\begin{itemize}
	\item Suction Piping System - The extrados thickness is less than the remaining thickness values measured. The 6-o'clock value at 3.5 meter is the backflow/eddie zone of the elbow, thus assuming to have high backflow rate in the suction.
	\item	Discharge Piping System - The flow from the pump enters the elbow at the lower half (6 o'clock of 4m mark) and swirls to the 3 o'clock position reference to the pump flange. It continued swirling to the sides and the exit extrados of the elbow. This is caused by the disturbance caused by the fittings in between.
\end{itemize}

\paragraph{\textbf{BP3}}
\begin{itemize}
	\item Suction Piping System - The extrados thickness is approximately equal to all measured thickness. Also, it is seen in the 6-o'clock of 3.5 and 4.5m mark that it is slightly thinner that of its 12-o'clock positions. This may indicate a high backflow of water thus creating higher turbulency as a result. The high turbulency of water causes the coverage of the localized thinning larger extending up to the 2.5m mark from the pump flange.
	\item	Discharge Piping System - The flow enters the lower half of the intrados entry. The flow then continued at the extrados area with a considerate backflow at the intrados. The average extrados thickness indicates also that the extrados is thinner when compared to the rest of the elbow thickness.
\end{itemize}

\paragraph{\textbf{BP4}}
\begin{itemize}
	\item Suction Piping System - The extrados thickness is slightly higher to all measured thickness. This indicates the flow of water in this pipe enters the elbow at between 9-o'clock position and 12-o'clock position with high backflow rate and eddies forming at the intrados area (6-o'clock of elbows). This pattern creates a swirling turbulent flow that results to larger localized wall thinning coverage extending to the upper portion of the pipe.
	\item	Discharge Piping System - The data indicates that the flow through the elbow based on thinning is that the flow enters the elbow entry and extending to the intrados area of the elbow, with small backflow rate (as seen in 3-6-9 o'clock position at 5m mark). It is then flows through the extrados exit again resulting to thinning with 4.75 mm thickness measured.
\end{itemize}

\paragraph{\textbf{BP5}}
\begin{itemize}
	\item Suction Piping System - The extrados thickness is less than the remaining thickness values measured. The 3-o'clock thinning at 3.5m mark may indicate that the water flows from extrados to the sides, that causes water swirling entering the diffusers and pumps.
	\item	Discharge Piping System - The elbow middle extrados is where the localized thinning area occur at this pipe section. However, the average extrados thickness is almost the same compared to the rest of pipe measurements.
\end{itemize}

\paragraph{\textbf{BP6}}
\begin{itemize}
	\item Suction Piping System - The extrados thickness is less than the remaining thickness values measured. The 4.09 mm thickness reading may indicate that the water flows faster in the upper section of the elbow exit and partially swirls extending to the 2.5m mark, thus having a uniform localized wall thinning at the right-half of the straight pipe.
	\item	Discharge Piping System - The thinnest part of the discharge pipeline is at the exit elbow extrados. However, the thickness is not critical compared to the other elbows. The component to be monitored is the elbow, yet is not critical.
\end{itemize}

\paragraph{\textbf{SP1}}
\begin{itemize}
	\item Suction Piping System - The thickness data for suction is only at the extrados zone of the elbow. However, this shows the common thinning effect at elbows. The thinning measurement show that the flow inside the pipe is high that the water makes contact between the central extrados area extending to the exit extrados of the pipe. Noting that the points are of the same central angles, the thinning difference between the points, 0.30 mm and 0.10 mm shows faster thinning at the exit extrados of the elbow.
	\item	Discharge Piping System - This pipe is long enough to make the flow developed and not too turbulent before entering the elbow. However, the bottom (6 o'clock position) the 4m mark from the pipe flange has considerable localized thinning and is extended to the sides of the pipe. The thinning continued at the sides of the elbow entry (3-6 o'clock) of the 8 and upper half of the 9m mark. The thinning also indicates a backflow in the elbow's intrados extending to the elbow exit.
\end{itemize}

\paragraph{\textbf{SP2}}
\begin{itemize}
	\item Suction Piping System - This pipe line as seen in the actual and in the plan is a larger pipe compared to that of the previous pipes. This pipe is not critical in terms of the current thickness.
	\item	Discharge Piping System - The discharge side of the pipe indicates cavitation is carried over. It is observed from the data having a uniform thinning at the pipe circumference. It shows that the water flows more at the left side (6-9 o'clock position reference to the pipe flange), then continued to the bottom of the pipe at 6m mark. The flow enters the elbow at the sides with considerable thinning due to backflow (as seen in the 3-9 o'clock at 9m mark). The thinning greatly eroded the exit part of the elbow.
\end{itemize}


\subsubsection{Recommendation}
\begin{itemize}
	\item The elbows in the suction and the discharge piping systems must be monitored regularly especially BP1, BP2, BP5 and BP6.
	\item	It is recommended to have a profiling of the piping systems above and below the ground in order to have a baseline in the analysis of the Maynilad Piping System. In order to have a profiling of pipe thickness at differential time T, additional measurement at similar locations shall be conducted periodically, behavior can then be monitored.
\end{itemize}

\subsection{Pipe, valves, fittings, supports, expansions, and appurtenances} \label{ch05mech02}

\textbf{Observation/Finding:}

Pumping station comprising two horizontal supply pumps and six horizontal booster pumps were installed with poorly fabricated and/or fitted pipe concrete saddles and steel straps, which provide insufficient support. These flaws, along with the natural reaction forces between the pipe sections and said supports, possibly aggravated by soil settlement, have created obvious visible gaps within the load bearing areas, as well as fractures in the concrete saddles and the steel anchor bolts. Instances of crevice corrosion were found in some contact areas. 

\textbf{Recommendations: }
\begin{itemize}
\item Short term: clear away oxide scales, especially among crevices, and recoat affected areas; 
\item  Medium term: install properly engineered piping supports, equipped with anti-corrosion shields; install cathodic protection on the pipelines.
\end{itemize}


\textbf{Observation/Finding:}

Station designers relied on foam inserts along penetrations in the front concrete wall to provide flexible piping support, which resulted in numerous wall cracks that are currently being repaired. Even if repairs on the said wall are completed, support for the affected piping is insufficient because wall penetrations are not designed for such a purpose. 

\textbf{Recommendation: }
\textbf{Recommendations: }
\begin{itemize}
	\item Isolate piping from the concrete wall..
\end{itemize}

\textbf{Observation/Finding:}

Pumps have little to no isolation from hydrodynamic forces. Strain caused by movement at the inlet and outlet pipe sections are transmitted to pump casings. Moreover, hydrodynamic forces, along with the static weight of said pipe sections, including the flexible coupling and the water contained therein, are transmitted to the casing due to lack of proper piping support upstream or downstream of the pump.

\textbf{Recommendation:}

\begin{itemize}
	\item Install proper isolation and support upstream and downstream of the pumps.
\end{itemize}

\textbf{Observation/Finding:}

Poor installation and/or alignment practices allowed the use of spacer-nuts to compensate for short jack screws and washerless anchor bolts in BP6.

\textbf{Recommendation: }
\begin{itemize}
	\item Implement proper quality control of shaft alignment jobs, whether done in-house or outsourced.
\end{itemize}

\textbf{Observation/Finding:}

All motors lack frame grounding. Assuming the motor power supplies were grounded at the motor control center (MCC), an assumptions that still needs to be verified, current may pass through the bearings should the motor shaft be grounded whilst the motor frame is ungrounded, which may lead to bearing damage, especially among variable speed drive (VFD) units. Bonding of non-current carrying metal components (e.g., motor frames) to the ground system is necessary to create an equipotential plane between the concrete floor and plant personnel who may risk electrocution should said parts become energized.

\textbf{Recommendations: }
\begin{itemize}
	\item Bond motor frames to the ground system; review present grounding policies/procedures. 
\end{itemize}

%\textcolor{red}{RB Sanchez to write here the findings. Note that the findings and result and analysis. Try to insert picture and tables to details the findings. Any recommendation shall not be written here. This section is purely about findings, the facts and the cause of the problem if any}
\subsection{Pump efficiency} \label{ch05mech03}
Pump efficiency is computed based on the flow/head measurement and the assumed value of power rating (Table \ref{ch05_tbl_efficiency}). 
\begin{table}[!h]
	\caption{Pump efficiency (\%).}
	\label{ch05_tbl_efficiency}
	{\footnotesize
\begin{tabular}{c|c|c|c|c|c}
	\hline
	Assets & Flow & Head & Input Power & Water Power & Efficiency \\ 
	& $m^3/s$ & $mH_2O$ & kW & kW & \% \\ 
	\hline
	BP1 & 0.35 & 41.68 & 181.00 & 142.00 & 78.60 \\ 
	BP2 & 0.32 & 41.40 & 181.00 & 129.00 & 71.70 \\ 
	BP3 & 0.60 & 41.68 & 283.00 & 246.00 & 86.90 \\ 
	BP4 & 0.54 & 40.00 & 283.00 & 210.00 & 74.20 \\ 
	BP5 & - & 41.19 & 283.00 & 242.00 & - \\ 
	BP6 & 0.63 & 39.93 & 283.00 & 248.00 & 87.60 \\ 
	SP1 & 0.36 & 45.33 & 185.00 & 161.00 & 87.00 \\ 
	SP2 & 0.38 & 44.06 & 185.00 & 165.00 & 89.10 \\ 
	\hline
\end{tabular}
	}
\end{table}

It is important to note that the values of input power is an assumed values which might not perfectly reflect the actual value in actual situation. This assumption is a limitation of the study, particularly for this station, since the electrical audit was carried out without actual records on power for individual pump. 



%\subsection{Unit flow measurement} \label{ch05mech04}

%\textcolor{red}{RB Sanchez to write here the findings. Note that the findings and result and analysis. Try to insert picture and tables to details the findings. Any recommendation shall not be written here. This section is purely about findings, the facts and the cause of the problem if any}
%\section{Electrical Audit}
%\label{43}
%
%\subsection{Visual inspection} \label{ch05elec01}
%\textcolor{blue}{APSI to write here the findings. Note that the findings and result and analysis. Try to insert picture and tables to details the findings. Any recommendation shall not be written here. This section is purely about findings, the facts and the cause of the problem if any}
%\subsection{Short circuit calculation} \label{ch05elec02}
%\textcolor{blue}{APSI to write here the findings. Note that the findings and result and analysis. Try to insert picture and tables to details the findings. Any recommendation shall not be written here. This section is purely about findings, the facts and the cause of the problem if any}
%\subsection{Voltage drop calculation} \label{ch05elec03}
%\textcolor{blue}{APSI to write here the findings. Note that the findings and result and analysis. Try to insert picture and tables to details the findings. Any recommendation shall not be written here. This section is purely about findings, the facts and the cause of the problem if any}
%\subsection{Protection coordination study} \label{ch05elec04}
%\textcolor{blue}{APSI to write here the findings. Note that the findings and result and analysis. Try to insert picture and tables to details the findings. Any recommendation shall not be written here. This section is purely about findings, the facts and the cause of the problem if any}
%\subsection{Harmonic analysis} \label{ch05elec05}
%\textcolor{blue}{APSI to write here the findings. Note that the findings and result and analysis. Try to insert picture and tables to details the findings. Any recommendation shall not be written here. This section is purely about findings, the facts and the cause of the problem if any}
%\subsection{Power quality} \label{ch05elec06}
%\textcolor{blue}{APSI to write here the findings. Note that the findings and result and analysis. Try to insert picture and tables to details the findings. Any recommendation shall not be written here. This section is purely about findings, the facts and the cause of the problem if any}
%\subsection{Grounding system} \label{ch05elec07}
%\textcolor{blue}{APSI to write here the findings. Note that the findings and result and analysis. Try to insert picture and tables to details the findings. Any recommendation shall not be written here. This section is purely about findings, the facts and the cause of the problem if any}
%\subsection{Asset registry} \label{ch05elec08}
%\textcolor{blue}{APSI to write here the findings. Note that the findings and result and analysis. Try to insert picture and tables to details the findings. Any recommendation shall not be written here. This section is purely about findings, the facts and the cause of the problem if any}
%\subsection{Electrical system design analysis} \label{ch05elec09}
%\textcolor{blue}{APSI to write here the findings. Note that the findings and result and analysis. Try to insert picture and tables to details the findings. Any recommendation shall not be written here. This section is purely about findings, the facts and the cause of the problem if any}
%\subsection{Electrical integrity system} \label{ch05elec11}
%\textcolor{blue}{APSI to write here the findings. Note that the findings and result and analysis. Try to insert picture and tables to details the findings. Any recommendation shall not be written here. This section is purely about findings, the facts and the cause of the problem if any}
%\subsection{Outdoor electrical equipment} \label{ch04elec12}
%\textcolor{blue}{APSI to write here the findings. Note that the findings and result and analysis. Try to insert picture and tables to details the findings. Any recommendation shall not be written here. This section is purely about findings, the facts and the cause of the problem if any}

\section{Fire protection and safety (FDAS) audit}
\label{ch05fdas}
\subsection{Fire alarm and detection system} \label{fdas01}
The findings/facts and results of the audit are presented in Table \ref{ch05_tbl_fdas01}. Visual images of assets are shown in Figure \ref{ch05_fig_fdas01}. 

Highlights are

\begin{itemize}
\item \textbf{Smoke Detector 01:} Red indicator light should be visible after spraying of smoke tester.  Removal of device from base to Reset contact point and cleaning did not show any improvement on the device. Hence device is declared not functioning and there is communication failure between device and FACP panel;
\item \textbf{Smoke Detector 02:} Smoke detector is still functioning but there a   is communication failure between device and FACP panel  since the  FACP did not detect the change of status of the device during testing; 
\item \textbf{Smoke Detector 03:} Red indicator light should be visible after spraying of smoke tester.  Removal of device from base to Reset contact point and cleaning did not show any improvement on the device. Hence device is declared not functioning and there is communication failure between device and FACP panel;
\item \textbf{Smoke Detector 04:} Red indicator light should be visible after spraying of smoke tester.  Removal of device from base to Reset contact point and cleaning did not show any improvement on the device. Hence device is declared not functioning and there is communication failure between device and FACP panel;
\item \textbf{Smoke Detector 05:} Smoke detector is still functioning but there is a communication failure between device and FACP panel  since the  FACP did not detect the change of status of the device during testing;
\item \textbf{Smoke Detector 06:} Red indicator light should be visible after spraying of smoke tester.  Removal of device from base to Reset contact point and cleaning did not show any improvement on the device. Hence device is declared not functioning and there is communication failure between device and FACP;
\item \textbf{Manual Call Point 01:} Communication failure between Manual call point 01 and FACP;
\item \textbf{Manual Call Point 02:} Communication failure between Manual call point 02 and FACP; 
\item \textbf{Buzzer with strobe light 01:} Communication failure between Buzzer  01 and FACP;
\item \textbf{Buzzer with strobe light 02:} Communication failure between Buzzer 02 and FACP;
\item \textbf{Fire Alarm control Panel (FACP):} System failure of FDAS  and devices  due to communication failure and not functioning devices installed within the system.
\end{itemize}



\begin{table}[!h]
	\caption{FDAS analysis.}
	\label{ch05_tbl_fdas01}
	{\footnotesize
\begin{tabular}{l|l|l|l|p{5cm}|p{5cm}}
	\hline
	No. & Assets & CS & IT & Facts & Remarks \\ 
	\hline
	1 & SM 01 & 0 & 4 & dust inside and outside & no response after repeat sprays \\ 
	&  &  &  & no light indication & Repeat clean \\ 
	&  &  &  & broken base & After 3x sprays, still no response \\ 
	&  &  &  & Spray Max 3 times &  \\ 
	\hline
	2 & SM 02 & 1 & 4 & dust inside and outside & no response after repeat sprays \\ 
	&  &  &  & With light indication (orange color maintained light) & Repeat clean \\ 
	&  &  &  & Spray Max 3 times & After 2x spray, light turned red but without response on the alarm panel \\ 
	&  &  &  &  & Removed and cleaned dirt and smoke particles \\ 
	&  &  &  &  & No light indicator after cleaning \\ 
	\hline
	3 & SM 03 & 0 & 4 & dust inside and outside & no response after repeat sprays \\ 
	&  &  &  & With light indicator (orange color maintained light) & Repeat clean \\ 
	&  &  &  & Spray Max 3 times & After 3x sprays, still no response \\ 
	&  &  &  &  & Removed and cleaned dirt and smoke particles \\ 
	&  &  &  &  & No light indicator after cleaning (reset) \\ 
	\hline
	4 & SM 04 & 0 & 4 & dust inside and outside & Same as SM 03 \\ 
	&  &  &  & With light indicator (orange color maintained light) &  \\ 
	&  &  &  & Spray Max 3 times &  \\ 
	\hline
	5 & SM 05 & 0 & 4 & Same as SM 04 & Same as SM 04 \\ 
	\hline
	6 & SM 06 & 0 & 4 & Same as SM 04 & Same as SM 04 \\ 
	\hline
	7 & MCP 01 & 0 & 4 & No response after pushing &  \\ 
	\hline
	8 & MCP 02 & 0 & 4 & No response after pushing &  \\ 
	\hline
	9 & Buzzer 01 & 0 & 4 & Spray SD /push MCP & No response/audible sound/ smoke  \\ 
	\hline
	10 & Buzzer 02 & 0 & 4 & Spray SD /push MCP & No response/audible sound/ smoke  \\ 
	\hline
	11 & FACP & 0 & 4 & Spray SD /push MCP & No response \\ 
	&  &  &  &  & No response on pushbutton, reset, evacuation, silence, mute buzzer or disabled/enabled \\ 
	&  &  &  &  & Remain light signal indicator for fire, power charger/fault. Fault zone 1\&3 \\ 
	\hline
\end{tabular}
	
	}
\end{table}



\begin{figure}[h]

	\begin{minipage}[b]{0.22\linewidth}
		\centering
		\includegraphics[width=\textwidth]{figures/ch05_fdas_sd01}
		\caption*{a - Smoker detector 01}
		%	\label{ch02_fdas03}
	\end{minipage}
	\hspace{0.03cm}
	\begin{minipage}[b]{0.22\linewidth}
		\centering
		\includegraphics[width=\textwidth]{figures/ch05_fdas_sd02}
		\caption*{b - Smoker detector 02}
		%	\label{ch02_fdas03}
	\end{minipage}
	\hspace{0.03cm}
	\begin{minipage}[b]{0.22\linewidth}
		\centering
		\includegraphics[width=\textwidth]{figures/ch05_fdas_sd03}
		\caption*{c - Smoker detector 03}
		%	\label{ch02_fdas03}
	\end{minipage}
	\hspace{0.03cm}
	\begin{minipage}[b]{0.22\linewidth}
		\centering
		\includegraphics[width=\textwidth]{figures/ch05_fdas_sd04}
		\caption*{d - Smoker detector 04}
		%	\label{ch02_fdas03}
	\end{minipage}
	\hspace{0.03cm}
	\begin{minipage}[b]{0.22\linewidth}
		\centering
		\includegraphics[width=\textwidth]{figures/ch05_fdas_sd05}
		\caption*{e - Smoker detector 05}
		%	\label{ch02_fdas03}
	\end{minipage}
	\hspace{0.03cm}
	\begin{minipage}[b]{0.22\linewidth}
		\centering
		\includegraphics[width=\textwidth]{figures/ch05_fdas_sd06}
		\caption*{f - Smoker detector 06}
		%	\label{ch02_fdas03}
	\end{minipage}
	\hspace{0.03cm}
	\begin{minipage}[b]{0.22\linewidth}
		\centering
		\includegraphics[width=\textwidth]{figures/ch05_fdas_mcp01}
		\caption*{g - Manual call point 01}
		%	\label{ch02_fdas03}
	\end{minipage}
	\hspace{0.03cm}
\begin{minipage}[b]{0.22\linewidth}
	\centering
	\includegraphics[width=\textwidth]{figures/ch05_fdas_mcp02}
	\caption*{h - Manual call point 02}
	%	\label{ch02_fdas03}
\end{minipage}
	\hspace{0.03cm}
\begin{minipage}[b]{0.22\linewidth}
	\centering
	\includegraphics[width=\textwidth]{figures/ch05_fdas_buzzer01}
	\caption*{i - buzzer 01}
	%	\label{ch02_fdas03}
\end{minipage}
	\hspace{0.03cm}
\begin{minipage}[b]{0.22\linewidth}
	\centering
	\includegraphics[width=\textwidth]{figures/ch05_fdas_buzzer02}
	\caption*{j - buzzer 01}
	%	\label{ch02_fdas03}
\end{minipage}
	\hspace{0.03cm}
\begin{minipage}[b]{0.22\linewidth}
	\centering
	\includegraphics[width=\textwidth]{figures/ch05_fdas_facp}
	\caption*{k - FACP}
	%	\label{ch02_fdas03}
\end{minipage}
	\caption{FDAS assets}
	\label{ch05_fig_fdas01}
\end{figure}



\subsection{Lighting protection system} \label{fdas02}
No lightning protection was installed for this PS. Thus, it is recommended to install a new lighting production system. This is reflected in the Conceptual Design presented in Chapter \ref{Chapter6}.
\subsection{Ground-Fault circuit interrupter (GFCI) or electric leakage circuit breaker (ELCB) or Residual circuit devices (RCD)} \label{fdas03}
No ground fault circuit interrupter, earth leakage circuit breaker (ELCB) or Residual circuit devices (RCD) are installed for this PS. Thus, it is recommended to install a new system. This is reflected in the Conceptual Design presented in Chapter \ref{Chapter6}.
%\subsection{Electrical safety and protective devices} \label{fdas04}
%\textcolor{blue}{APSI to write here the findings. Note that the findings and result and analysis. Try to insert picture and tables to details the findings. Any recommendation shall not be written here. This section is purely about findings, the facts and the cause of the problem if any}
\section{Vibration and structural assessment}
%\textcolor{red}{RB Sanchez to write here the findings. Note that the findings and result and analysis. Try to insert picture and tables to details the findings. Any recommendation shall not be written here. This section is purely about findings, the facts and the cause of the problem if any}
Analytical results on vibration are with the Appendix \ref{app_vibrationdata}. A summary of grading for each pump is given in Table \ref{ch05_tbl_vibration}.

\begin{table}[!h]
	\caption{Pump vibration condition state.}
	\label{ch05_tbl_vibration}
	{\footnotesize
\begin{tabular}{l|l|l|l|l}
	\hline
	\multicolumn{1}{c|}{Assets} & Operational issues detected & \multicolumn{2}{c}{Condition} & \multicolumn{1}{c}{IT} \\ 
	\cline{3-4}
	\multicolumn{1}{c|}{} &  & \multicolumn{1}{c|}{Motor} & \multicolumn{1}{c|}{Pump} & \multicolumn{1}{c}{} \\ 
	\hline
	\multicolumn{1}{c|}{BP1} & shaft misalignment & \multicolumn{1}{c|}{2} & \multicolumn{1}{c|}{3} & \multicolumn{1}{c}{} \\ 
	\multicolumn{1}{c|}{} & impeller cavitation & \multicolumn{1}{c|}{} & \multicolumn{1}{c|}{} & \multicolumn{1}{c}{} \\ 
	\multicolumn{1}{c|}{} & low lubrication at pump inboard (IB) bearing & \multicolumn{1}{c|}{} & \multicolumn{1}{c|}{} & \multicolumn{1}{c}{} \\ 
	\hline
	\multicolumn{1}{c|}{BP2} & early stage fault  & \multicolumn{1}{c|}{2} & \multicolumn{1}{c|}{3} & \multicolumn{1}{c}{} \\ 
	\multicolumn{1}{c|}{} & low lubrication at pump outboard (OB) bearing & \multicolumn{1}{c|}{} & \multicolumn{1}{c|}{} & \multicolumn{1}{c}{} \\ 
	\hline
	\multicolumn{1}{c|}{BP3} & misalignment & \multicolumn{1}{c|}{2} & \multicolumn{1}{c|}{3} & \multicolumn{1}{c}{} \\ 
	\multicolumn{1}{c|}{} & impeller cavitation & \multicolumn{1}{c|}{} & \multicolumn{1}{c|}{} & \multicolumn{1}{c}{} \\ 
	\hline
	\multicolumn{1}{c|}{BP4} & vibration in all bearings were within acceptable levels & \multicolumn{1}{c|}{2} & \multicolumn{1}{c|}{2} & \multicolumn{1}{c}{} \\ 
	\hline
	\multicolumn{1}{c|}{BP5} & impeller cavitation & \multicolumn{1}{c|}{2} & \multicolumn{1}{c|}{3} & \multicolumn{1}{c}{} \\ 
	\multicolumn{1}{c|}{} & low lubrication at pump outboard (OB) bearing & \multicolumn{1}{c|}{} & \multicolumn{1}{c|}{} & \multicolumn{1}{c}{} \\ 
	\hline
	\multicolumn{1}{c|}{BP6} & impeller cavitation & \multicolumn{1}{c|}{2} & \multicolumn{1}{c|}{3} & \multicolumn{1}{c}{} \\ 
	\multicolumn{1}{c|}{} & low lubrication at pump outboard (OB) bearing & \multicolumn{1}{c|}{} & \multicolumn{1}{c|}{} & \multicolumn{1}{c}{} \\ 
	\hline
	\multicolumn{1}{c|}{SP1} & shaft misalignment & \multicolumn{1}{c|}{2} & \multicolumn{1}{c|}{3} & \multicolumn{1}{c}{} \\ 
	\multicolumn{1}{c|}{} & impeller cavitation & \multicolumn{1}{c|}{} & \multicolumn{1}{c|}{} & \multicolumn{1}{c}{} \\ 
	\hline
	\multicolumn{1}{c|}{SP2} & impeller cavitation & \multicolumn{1}{c|}{2} & \multicolumn{1}{c|}{3} & \multicolumn{1}{c}{} \\ 
	\hline
\end{tabular}

	}
\end{table}
It is note that the CS 2, and 3 shown in Table \ref{ch05_tbl_vibration} infers good and fair, respectively \footnote{The CS is slightly different from that defines in Table \ref{ch03:cs}}. 

It can be seen from Table \ref{ch05_tbl_vibration}, vibration on motor is with CS 2 inferring that they are still operating in acceptance level of vibration (good). However, vibration on pump is mostly with CS 3 (faire), except for BP4. Problems found from observation and vibration analysis are mainly due to shaft misalignment, impeller cavitation, and low lubrication at pump inboard (IB) and outboard (OB) bearing. 

Recommendations are shown in Table \ref{ch05_tbl_vibrationre}

\begin{table}[!h]
	\caption{Recommendation to reduce vibration.}
	\label{ch05_tbl_vibrationre}
	{\footnotesize
\begin{tabular}{l|l|l|l|c}
	\hline
	\multicolumn{1}{c|}{Assets} & \multicolumn{2}{c}{Condition} & Recommendations & IT \\ 
	\cline{2-3}
	\multicolumn{1}{c|}{} & \multicolumn{1}{c|}{Motor} & \multicolumn{1}{c|}{Pump} & \multicolumn{1}{c|}{} &  \\ 
	\hline
	\multicolumn{1}{c|}{BP1} & \multicolumn{1}{c|}{2} & \multicolumn{1}{c|}{3} & Align motor and pump shafts per manufacturer specifications &  \\ 
	\multicolumn{1}{c|}{} & \multicolumn{1}{c|}{} & \multicolumn{1}{c|}{} & Monitor and/or mitigate cavitation progress &  \\ 
	\multicolumn{1}{c|}{} & \multicolumn{1}{c|}{} & \multicolumn{1}{c|}{} & \multicolumn{1}{c|}{Follow manufacturer prescribed dynamic operation to prevent impeller damage} &  \\ 
	\multicolumn{1}{c|}{} & \multicolumn{1}{c|}{} & \multicolumn{1}{c|}{} & Regrease pump IB bearing &  \\ 
	\hline
	\multicolumn{1}{c|}{BP2} & \multicolumn{1}{c|}{2} & \multicolumn{1}{c|}{3} & Regrease pump OB bearing &  \\ 
	\multicolumn{1}{c|}{} & \multicolumn{1}{c|}{} & \multicolumn{1}{c|}{} & Monitor vibration regularly to assess trend &  \\ 
	\hline
	\multicolumn{1}{c|}{BP3} & \multicolumn{1}{c|}{2} & \multicolumn{1}{c|}{3} & Align motor and pump shafts per manufacturer's specifications &  \\ 
	\multicolumn{1}{c|}{} & \multicolumn{1}{c|}{} & \multicolumn{1}{c|}{} & Monitor and/or mitigate cavitation progress &  \\ 
	\multicolumn{1}{c|}{} & \multicolumn{1}{c|}{} & \multicolumn{1}{c|}{} & Follow manufacturer prescribed dynamic operation to prevent impeller damage &  \\ 
	\multicolumn{1}{c|}{} & \multicolumn{1}{c|}{} & \multicolumn{1}{c|}{} & Monitor vibration regularly to assess trend &  \\ 
	\hline
	\multicolumn{1}{c|}{BP4} & \multicolumn{1}{c|}{2} & \multicolumn{1}{c|}{2} & Monitor vibrations regularly &  \\ 
	\hline
	\multicolumn{1}{c|}{BP5} & \multicolumn{1}{c|}{2} & \multicolumn{1}{c|}{3} & Monitor and/or mitigate cavitation progress &  \\ 
	\multicolumn{1}{c|}{} & \multicolumn{1}{c|}{} & \multicolumn{1}{c|}{} & Follow manufacturer prescribed dynamic operation to prevent impeller damage &  \\ 
	\multicolumn{1}{c|}{} & \multicolumn{1}{c|}{} & \multicolumn{1}{c|}{} & Regrease pump bearings &  \\ 
	\hline
	\multicolumn{1}{c|}{BP6} & \multicolumn{1}{c|}{2} & \multicolumn{1}{c|}{3} & Monitor and/or mitigate cavitation progress &  \\ 
	\multicolumn{1}{c|}{} & \multicolumn{1}{c|}{} & \multicolumn{1}{c|}{} & \multicolumn{1}{c|}{Follow manufacturer prescribed dynamic operation to prevent impeller damage} &  \\ 
	\multicolumn{1}{c|}{} & \multicolumn{1}{c|}{} & \multicolumn{1}{c|}{} & Regrease pump bearings &  \\ 
	\hline
	\multicolumn{1}{c|}{SP1} & \multicolumn{1}{c|}{2} & \multicolumn{1}{c|}{3} & Align motor and pump shafts per manufacturer's specifications  &  \\ 
	\multicolumn{1}{c|}{} & \multicolumn{1}{c|}{} & \multicolumn{1}{c|}{} & Monitor and/or mitigate cavitation progress &  \\ 
	\multicolumn{1}{c|}{} & \multicolumn{1}{c|}{} & \multicolumn{1}{c|}{} & \multicolumn{1}{c|}{Follow manufacturer prescribed dynamic operation to prevent impeller damage} &  \\ 
	\hline
	\multicolumn{1}{c|}{SP2} & \multicolumn{1}{c|}{2} & \multicolumn{1}{c|}{3} & Monitor and/or mitigate cavitation progress &  \\ 
	\multicolumn{1}{c|}{} & \multicolumn{1}{c|}{} & \multicolumn{1}{c|}{} & Follow manufacturer prescribed dynamic operation to prevent impeller damage &  \\ 
	\hline
\end{tabular}

	}
\end{table}


\section{Energy management audit}
As a matter of fact, power consumption of a PS is mostly contributed by the operation of pumps. Thus, the audit has been centralized on 
\begin{itemize}
\item Analyzing given production and power consumption data to understand the trend and establish a benchmark ratio of production vs power for future audit and management;
\item Evaluating other part of the audit such as pump efficiency and reliability in order to derive better intervention program that will eventually beneficial to the Client to maintain a benchmark level of power consumption agaisnt the production. 
\end{itemize}

Figure \ref{ch05_fig_energy_correlation} shows the statistical correlation between production and power. It can be seen from the correlation graph and correlation value that there is little correlation among these two values. This is agaisnt the  hypothesis that when production increase, power consumption shall also increase. However, it is not the case as observed from the grahp. THis infers that the pump system might have incurred a certain level of deterioration leading to its low reliability over time. 

\begin{figure}[!htb]
	\includegraphics[scale=0.5]{figures/ch05_fig_energy_correlation} \\
	\caption{Correlation between production and power consumption}
	\label{ch05_fig_energy_correlation} 
\end{figure}

Figure \ref{ch05_fig_energy_production} shows a trend in time series production since 2014. It can be seen from the graph that the production has kept decreasing slightly over the year. 

\begin{figure}[!htb]
	\includegraphics[scale=0.5]{figures/ch05_fig_energy_production} \\
	\caption{Time series production/hour}
	\label{ch05_fig_energy_production} 
\end{figure}

Figure \ref{ch05_fig_energy_power} shows a trend in time series power consumption since 2014. It can be seen from the graph that the power has kept increasing over the year. 

\begin{figure}[!htb]
	\includegraphics[scale=0.5]{figures/ch05_fig_energy_power} \\
	\caption{Time series power/hour}
	\label{ch05_fig_energy_power} 
\end{figure}
Figure \ref{ch05_fig_energy_ratio} shows time series of ratio between power and production. As the production decreases and power increase, the ratio keeps increasing over time. 

\begin{figure}[!htb]
	\includegraphics[scale=0.5]{figures/ch05_fig_energy_ratio} \\
	\caption{Time series ratio between production and power}
	\label{ch05_fig_energy_ratio} 
\end{figure}

Interpretation from these graphs can be summarized as follows

\begin{itemize}
\item Efficiency of the pump system has decreased due to more frequence breakdown of pumps;
\item Pumps might have been operated in a non-optimal operational scheme/sequence;
%\item Power pressure of suction
\end{itemize}

\section{Workplace environment management}
\subsection{Temperature and relative humidity}
As can be seen from Table \ref{ch04_tbl_wem01}, it is obvious that the temperatures inside the pump house at every measurement points are significant higher than the maximum value of the recommended range (80 F). The average value is 94.16 F. The higher values of temperature compared to the range have also been observed for points outside the pump house, in the vicinity, and even inside the office. 

Regarding the relative humidity, recorded data was within the recommended range inside the pump house and in the office (54.55 \% and 48.80 \%, respectively). The humidity value outside the pump house (61.36\%) is slighly higher than the maximum value in the recommended range (60.00\%). However, it can be understandable given the the fact that it is directly affected by the ambient temperature.

As a matter of fact, temperature and humidity is highly correlated and as per XXX, the recommended combination of temperature and humidity shall be within the comfortable zone as shown in Figure \ref{ch04_fig_wem01}.

\begin{figure}[!htb]
	\includegraphics[scale=2]{figures/ch04_fig_wem01} \\
	\caption{ASHRAE standard 55 : Summer Comfort Zone}
	\label{ch04_fig_wem01} 
\end{figure}

\paragraph{\textbf{Recommendations}}
In order to reduce the negative impacts from high temperature, particularly inside the pump house, the Client shall consider

\begin{itemize}
\item To establish a good daily monitoring, exercise, and management considering ergonomic and health and occupational activities (e.g. appropriate time window for break in designated resting area);
\item To execute physical intervention to reduce temperature can be with improving ventilation system by natural mean (e.g. installation of weather proofed louvers). This will be reflected in the conceptual design in Chapter \ref{Chapter6}.
\end{itemize}


%the average temperature readings inside the pump house, inside pump station and reservoir, and outside the vicinity of pump station are beyond the recommended value of 72-80 F, this is aggravated by the relative reading which is beyond the recommended value inside Pump Station and Reservoir and nearly falling beyond the range of 45-60 \% for the locations-inside the Pump House and Outside the vicinity of Pump Station. Inside office, on the other hand, registered a reading of 82 F, way closer to the recommended value and with a relative humidity of 49 \% - an ideal value. Note that Air Conditioning System is installed inside the Office thus the temperature can still be vary based on the thermal comfort needed by the Operators.






\subsection{Air quality}\label{ch05aq01}
%\textcolor{red}{RB Sanchez to write here the findings. Note that the findings and result and analysis. Try to insert picture and tables to details the findings. Any recommendation shall not be written here. This section is purely about findings, the facts and the cause of the problem if any}
%\subsection{Hazards}\label{ch05aq02}
%\textcolor{red}{RB Sanchez to write here the findings. Note that the findings and result and analysis. Try to insert picture and tables to details the findings. Any recommendation shall not be written here. This section is purely about findings, the facts and the cause of the problem if any}
\subsection{Illumination}\label{ch05aq03}
%\textcolor{red}{RB Sanchez to write here the findings. Note that the findings and result and analysis. Try to insert picture and tables to details the findings. Any recommendation shall not be written here. This section is purely about findings, the facts and the cause of the problem if any}
The average lightings inside the Pump House (1,000 Lux) where activities are being conducted are 10x higher than the minimum Illumination Level Acceptable Value (100 Lux).

Illumination is provided by natural means through the Skylights and is augmented by Motion-activated Lighting Systems. At night time, the light provided by Lighting System is for a particular zone only or place where there are motions.

\paragraph{\textbf{Recommendations}}

\begin{itemize}
\item	Use artificial lighting equipment when accessing and conducting activities requiring detailed output at darker specific areas especially at night because the existing lighting systems cannot provide adequate lighting or when deemed necessary.
\item	Such artificial supplementary lightings shall be especially designed for the specific tasks and provided with shading or diffusing devices to prevent glare.
\item	Periodic cleaning of Skylights and glass windows should be implemented to ensure they are kept clean at all times

\end{itemize}


\subsection{Industrial ventilation}\label{ch05aq04}
%\textcolor{red}{RB Sanchez to write here the findings. Note that the findings and result and analysis. Try to insert picture and tables to details the findings. Any recommendation shall not be written here. This section is purely about findings, the facts and the cause of the problem if any}
\subsection{Housekeeping}\label{ch05aq05}
%\textcolor{red}{RB Sanchez to write here the findings. Note that the findings and result and analysis. Try to insert picture and tables to details the findings. Any recommendation shall not be written here. This section is purely about findings, the facts and the cause of the problem if any}
\subsection{Noise}\label{ch05aq06}
%\textcolor{red}{RB Sanchez to write here the findings. Note that the findings and result and analysis. Try to insert picture and tables to details the findings. Any recommendation shall not be written here. This section is purely about findings, the facts and the cause of the problem if any}
All pumps (6 Booster and 2 Storage Pumps) are running during the Sound Level Testing and so the reading closely represents the normal daily noise level inside the Plant. The average sound level inside the Pump House is 93 dBA, beyond the standard of less than 90 dbA. 

On the other hand, the sound level at locations – Inside Pump Station and Reservoir, Outside the Vicinity of Pump Station and inside the office are from 69 – 70 dBA, an acceptable value.

\paragraph{\textbf{Recommendations}}

\begin{itemize}
\item	Use protective hearing equipment when working inside the Pump House and have a scheduled break/rest at designated location. Shall not be exposed at such noise beyond 5 hours in a day.
\item	Designate location inside the Plant with the minimum noise level - can be the Office, Inside Pump Station and Reservoir, and Outside the Vicinity of Pump Station, if below is not possible.
\item	 Install Sound Attenuation Device (such as sound-absorbing wall panels and door seals) at the Office to reduce the current 69 dBA to ideal Office level of 50-55 dBA.
\end{itemize}



