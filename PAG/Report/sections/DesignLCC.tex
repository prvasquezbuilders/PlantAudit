% Chapter 6
\chapter{Conceptual Design and Reliability Study} % Write in your own chapter title
\label{Chapter6}
%\lhead{Chapter 6. \emph{Conceptual Design}} % Write in your own chapter title to set the page header
%
%%%%%%%%%%%%%%%%%%%%%%%%%%%%
\section{Basis of Design}
%\label{61}
\subsection{Drawings}
A collection of as-built drawings are given in electronic files saved both in PDF and CAD formats.

As set of conceptual drawings is given in Appendix \ref{app_design_mech}.
\subsection{FDAS design}
\subsubsection{Fire alarm and detection system}
\paragraph{\underline{Design criteria}}
The conceptual design for FDAS has been developed based on findings/results of the audit (refer to subsections \ref{ch04fdas}, design criteria including required code of practice, and the required level of services.
\begin{itemize}
\item Individual components shall be compatible with each other and shall be approved and listed by institutions recognized by the relevant authority
\item The FDAS designer shall have the experience in the proper design, application, installation and testing of FDAS
\item If the total floor area is more than 8000m2, a semi-addressible system shall be used, otherwise, a conventional system maybe used.
\item  Automatic detection shall have a complete indoor coverage of building or facilities including all rooms, halls, etc.
\item  For smoke detectors, the performance characteristics of the detector and the area shall be taken into account when selecting smoke detectors. Smoke detectors shall not be installed in rooms with temperature below 5 degree centigrade, above 45 degree centigrade and with relative humidity above 93%.
\item  For heat detectors, temperature rating shall be set at least 11 degree centigrade above maximum expected temperature and is spaces not more than 7.5meters. It shall not be installed in locations where relative humidity is above 93 \% and if the ceiling is more than 4 meters.
\item  Beam-type smoke detector shall be used if ceilings are more than 6m in height and shall be kept clear of opaque obstacles at all times
\item  Manual detection is achieved through the manual activation of fire push or pull stations installed at a height of 1.4meters above floor and shall be easily seen and is accessible. It is usually colored red.
\item  Alarm shall be clearly audible throughout the floor and or/or building where they are installed. It shall have a minimum of 65 dbA or 10db higher than ambient room noise and a maximum of 115 dbA.
\item  Visual notification shall be used along with audible notification for areas where hearing protection is worn.
\end{itemize}

\paragraph{\underline{Design}}

%The general layout of the proposed conceptual design is shown in Figure \ref{ch06_fdas_design01}. Relevant drawings are enclosed in the Appendix \ref{app_design_fdas}.

%\begin{figure}[!htb]
%	\includegraphics[scale=2.5, angle = 90]{figures/ch06_fdas_design01} \\
%	\caption{Conceptual design - FDAS layout}
%	\label{ch06_fdas_design01}
%\end{figure}
\subsubsection{Lighting protection system}
\paragraph{\underline{Requirements}}

The design of a lightning protection system needs to: Intercept lightning flash (i.e. create a preferred point of strike) Conduct the lightning current to earth Dissipate current into the earth Create an equipotential bond to prevent hazardous potential differences between LPS, structure and internal elements/circuits

Following points shall be considered
\begin{itemize}
\item The lightning protection design shall be in accordance with the requirements of Article 2.90 Protection against Lightning , Philippine Electrical Code 2017 and NFPA 780, standard for the installation of lightning protection systems;

\item The building lightning protection system shall include

\begin{itemize}
\item[-] Roof mounted copper air terminals;
\item[-]	Ground rods of lightning protection shall shall be metal clad steel with a diameter of 20mm by 3 meters long connected triangularly with equal distances of 3 meters between two ground rods;
\item[-]	Down conductor, 50mm2 Bare copper wire and shall be placed in protective conduit (PVC);
\item[-]	Lightning protection shall have electronically controlled mechanical register which activates registration for every discharge;
\item[-]	clamps, conduits and auxiliary equipment as required for complete and operational lightning protection system. Materials shall be resistant or protected against  corrosion.

\end{itemize}

\item The company providing the design should have a minimum of 5 years experience and be well versed with Article 2.90 Protection against Lightning , Philippine Electrical Code 2017 and NFPA 780;
\item 	Lightning protection systems are designed specifically for the building or structures they are intended to protect;
\item 	The design is not only impacted by the shape and size, but also by building systems and structural components.

\end{itemize}

\paragraph{\underline{Installation}}
The installation shall be in accordance with the requirements set forth in Article 2.90 Protection against Lightning and NFPA 780 and installed in a neat and workmanlike manner


\subsubsection{Ground Electrode Installation and Common Bonding }

\paragraph{\underline{Ground electrode installation and common building}}
\begin{itemize}
\item Ground electrodes shall be installed in accordance with NFPA 780 and Article 2.90 of PEC 2017.
\item 	Common bonding between all building electrode systems shall be installed in accordance with NFPA 780 and PEC 2017
\item 	Maintain horizontal or vertical runs of ground wires and ensures that all bends have at least 200 mm radius and angle of any bend shall not be less than 90 degrees.

\item  	Lightning carrier cable and down conductor shall be supported every 1.50 meters on center using fabricated copper clamps, bolted to roof slab with plastic expansion sleeves.

\item  	Ground rods should be driven far enough away from the footing and drain tile and also past the roof’s drip edge.
\item 	Ground rods shall be installed into undisturbed soil.
\item  	If it is not practical to install ground rods outside of the building, the ground rods should be installed as close to the building’s walls as practical without damaging the footing.
\item 	The correct ground rod driver adapter should be used to avoid mushrooming or damaging the end of the ground rod.
\item  	If the damage to the ground rod is to severe, the top of the rod may need to be cut off so that the ground rod clamp or exothermic connection can be properly made
\item 	For testing and maintenance, access of each ground electrode should be available.

\end{itemize}

\paragraph{\underline{Ground ring electrode installation}}

\begin{itemize}
\item If required, a ground ring electrode for the lightning protection system shall be installed at least 460mm (18 inches) below earth unless prohibited by ground conditions;
\item 	A ground ring electrode installed for the purposes of electrical grounding shall be installed to a depth of at least 765 mm (30 inches);
\item 	Ground ring electrodes shall be continuous around the structure and connected to all down conductors.  The ground ring electrode shall be installed below the line.

\end{itemize}

\paragraph{\underline{field test}}
Test the grounding system to assure the continuity and that the resistance to ground is not excessive.




\subsection{WEM design}
%\textcolor{red}{RB Sanchez to write here the basic of design such as design criteria, calculation, applied codes, objectives, etc. Try to summary it in table form and cite to correct references including as-built drawings, control philosophy, etc}
%
%\textcolor{red}{A set of conceptual design drawings shall be referred as enclosed in the Appendix}
\section{Bill of Materials}
Based on the recommendations and conceptual design, a high level Bill of Quantity (BOQ) can be generated . The BOQ table includes the condition states and intervention types respectively. %This BOQ will be served as a base for the estimation of life cycle cost.

%\begin{longtable}{c|p{3cm}|p{4cm}|c|c|c|r|r|c|r|p{4cm}}
%\caption{Bill of Quantity (BOQ)} \label{tab:boq} \\
%%	\resizebox{\columnwidth}{!}{%
%%	{\scriptsize
%	\hline \multicolumn{1}{|l|}{No.} & \multicolumn{1}{l|}{Description} & \multicolumn{1}{l|}{Items}& \multicolumn{1}{l|}{CAT}& \multicolumn{1}{l|}{CS}& \multicolumn{1}{l|}{IT}& \multicolumn{1}{l|}{Unit Cost}& \multicolumn{1}{l|}{Quantity}& \multicolumn{1}{l|}{Unit}& \multicolumn{1}{l|}{Total}& \multicolumn{1}{l|}{Remarks}\\
%	 \hline
%	\endfirsthead
%
%	\multicolumn{11}{c}%
%	{{\bfseries \tablename\ \thetable{} -- continued from previous page}} \\
%	\hline \multicolumn{1}{|l|}{No.} & \multicolumn{1}{l|}{Description} & \multicolumn{1}{l|}{Items}& \multicolumn{1}{l|}{CAT}& \multicolumn{1}{l|}{CS}& \multicolumn{1}{l|}{IT}& \multicolumn{1}{l|}{Unit Cost}& \multicolumn{1}{l|}{Quantity}& \multicolumn{1}{l|}{Unit}& \multicolumn{1}{l|}{Total}& \multicolumn{1}{l|}{Remarks}\\
%	\hline
%	\endhead
%	\hline \multicolumn{11}{|r|}{{Continued on next page}} \\ \hline
%	\endfoot
%	\hline
%	\endlastfoot
%	\hline
%	1 & Instrumentation & Pressure gauge & M & 2 & 8 &  2,500.00  & 7 & pcs &  17,500.00  & for replacement of defective PG \\
%	2 & Pump and motor & Bearing grease & M & 2 & 2 &  3,000.00  & 1 & lot &  3,000.00  &  \\
%	3 & Instrumentation & Pressure gauge & M & 2 & 7 &  2,500.00  & 16 & pcs &  40,000.00  & to be tapped near intake/discharge of pump \\
%	4 & Interior lighting & Manual light switch & F & 2 & 6 &  1,000.00  & 1 & pcs &  1,000.00  &  \\
%	5 & Outside lighting & 80 Watts LED street light & F & 2 & 7 &  1,500.00  & 5 & pcs &  7,500.00  &  \\
%	6 & Sound protection & Earmuffs & M & 3 & 10 &  500.00  & 5 & pcs &  2,500.00  & for visitors \\
%	7 & Pump house ventilation & Louver panels (1.2  x 1.5 m) & M & 3 & 6 &  2,500.00  & 64 & pcs &  160,000.00  &  \\
%	8 & Internal Data & Valve/Fitting performance curve & M & - & 11 &  -    & 1 & lot &  -    & from manufacturer \\
%	9 & Controls & VFD control package  & M & 3 & 6 &  75,000.00  & 4 & pcs &  300,000.00  &  \\
%	10 & Instrumentation & Portable UFM & M & 4 & 8 &  100,000.00  & 1 & pcs &  100,000.00  &  \\
%	11 & Instrumentation & Calibration of UFM & M & - & 8 &  50,000.00  & 1 & pcs &  50,000.00  &  \\
%	12 & Instrumentation & Calibration of Pressure gauges & M & - & 8 &  1,000.00  & 7 & pcs &  7,000.00  & Separate calibration fees \\
%	13 & Motor & Grounding cable & X & 4 & 7 &  1,000.00  & 8 & pcs &  8,000.00  &  \\
%	14 & Structural support & Support saddle & S & 4 & 5 &  12,000.00  & 20 & pcs &  240,000.00  &  \\
%	15 & Asset reconditioning & Paint & S & - & 2 &  3,000.00  & 1 & lot &  3,000.00  &  \\
%	16 & Pipe design & Suction strainer & M & 3 & 7 &  5,000.00  & 8 & pcs &  40,000.00  &  \\
%	17 & Internal Data & Asset tags & P & 3 & 5 &  1,000.00  & 1 & lot &  1,000.00  &  \\
%	18 & Housekeeping & Plastic catching basin & M & 3 & 2 &  100.00  & 8 & pcs &  800.00  &  \\
%	19 & Instrumentation & Vibration monitoring probes & M & 4 & 3 &  25,000.00  & 16 & pcs &  400,000.00  &  \\
%	20 & Pipe design & Air vent & P & 3 & 7 &  20,000.00  & 8 & pcs &  160,000.00  &  \\
%	21 & Pump and motor & Alignment bolt and washer & M & 3 & 3 &  300.00  & 32 & pcs &  9,600.00  &  \\
%	22 & Pipe design & Flow conditioner & P & 3 & 7 &  10,000.00  & 6 & pcs &  60,000.00  &  \\
%	23 & Pipe design & Long radius elbow with guide vanes & P & 3 & 6 &  15,000.00  & 6 & pcs &  90,000.00  &  \\
%	24 & Pipe design & Flange type long radius elbow & P & 3 & 5 &  15,000.00  & 6 & pcs &  90,000.00  &  \\
%	25 & Pipe design & Pipe (700mm) & P & 3 & 6 &  30,000.00  & 1 & lot &  30,000.00  &  \\
%	26 & Pipe design & Pipe (600mm) & P & 3 & 6 &  30,000.00  & 1 & lot &  30,000.00  &  \\
%	27 & Pipe design & Thrust block & S & 3 & 7 &  7,500.00  & 8 & pcs &  60,000.00  &  \\
%	28 & Machine foundation & Pile footing & S & 3 & 5 &  30,000.00  & 32 & pcs &  960,000.00  &  \\
%	29 & Machine foundation & Foundation Block RCC (3x1.5x1.25) & S & 3 & 6 &  25,000.00  & 6 & pcs &  150,000.00  &  \\
%	30 & Machine foundation & Foundation Block RCC (3x1.5x1) & S & 3 & 6 &  20,000.00  & 2 & pcs &  40,000.00  &  \\
%	31 & Service reliability & Booster Pump & M & 3 & 7 &  500,000.00  & 1 & pcs &  500,000.00  &  \\
%	32 & Service reliability & Storage Pump & M & 3 & 7 &  400,000.00  & 1 & pcs &  400,000.00  &  \\
%	33 & Service reliability & BP motor & X & 3 & 7 &  500,000.00  & 1 & pcs &  500,000.00  & Necessary for another pump \\
%	34 & Service reliability & SP motor & X & 3 & 7 &  400,000.00  & 1 & pcs &  400,000.00  & Necessary for another pump \\
%	35 & Instrumentation & Vibration analysis & M &  & 8 &  25,000.00  & 1 & lot &  25,000.00  & Monthly Monitoring \\
%	36 & Asset reconditioning & Pump and motor realignment & M &  & 2 &  2,500.00  & 1 & lot &  2,500.00  & Per realignment \\
%	37 & Service reliability & Check valve  & M &  & 7 &  120,000.00  & 2 & pcs &  240,000.00  & Necessary for additional pump \\
%	38 & Service reliability & Butterfly valve & M &  & 7 &  45,000.00  & 4 & pcs &  180,000.00  & Necessary for additional pump \\
%	39 & Service reliability & VFD control package  & M &  & 7 &  75,000.00  & 2 & pcs &  150,000.00  & Necessary for additional pump \\
%	40 & Service reliability & Air valve (for additional pumps) & P &  & 7 &  10,000.00  & 4 & pcs &  40,000.00  & Necessary for additional pump \\
%	41 & Service reliability & Foundation Block RCC (3x1.5x1.25) & S &  & 7 &  25,000.00  & 1 & pcs &  25,000.00  & Necessary for additional pump \\
%	42 & Service reliability & Foundation Block RCC (3x1.5x1) & S &  & 7 &  20,000.00  & 1 & pcs &  20,000.00  & Necessary for additional pump \\
%	43 & Service reliability & Pipe (700mm) & P &  & 7 &  10,000.00  & 1 & lot &  10,000.00  & Necessary for additional pump \\
%	44 & Service reliability & Pipe (600mm) & P &  & 7 &  10,000.00  & 1 & lot &  10,000.00  & Necessary for additional pump \\
%	45 & Service reliability & Pipe (500mm) & P &  & 7 &  10,000.00  & 1 & lot &  10,000.00  & Necessary for additional pump \\
%	46 & Service reliability & Support saddle & S &  & 7 &  12,000.00  & 2 & pcs &  24,000.00  & Necessary for additional pump \\
%	47 & Fire alarm control Panel &  & F & 5 & 4 &  & 1 & set &  -    &  \\
%	48 & Smoke Detectors &  & F & 5 & 4 &  & 18 & units &  -    &  \\
%	49 & Heat Detectors &  & F & 5 & 4 &  & 4 & units &  -    &  \\
%	50 & Manual Pull Station &  & F & 5 & 4 &  & 4 & units &  -    &  \\
%	51 & Horn Strobe Annunciator &  & F & 5 & 4 &  & 4 & units &  -    &  \\
%	52 & Twisted Pair wire 2.5mm2 &  & F & 5 & 4 &  & 275 & lm &  -    &  \\
%	53 & 20mmØ IMC pipe &  & F & 5 & 4 &  & 60 & lengths &  -    &  \\
%	54 & Locknut and Bushing &  & F & 5 & 4 &  & 1 & lot &  -    &  \\
%	55 & 15mmØ FMC Conduit &  & F & 5 & 4 &  & 20 & lm &  -    &  \\
%	56 & FMC Connectors &  & F & 5 & 4 &  & 1 & lot &  -    &  \\
%	57 & Junction Box &  & F & 5 & 4 &  & 1 & lot &  -    &  \\
%	58 & Hangers and Supports &  & F & 5 & 4 &  & 1 & lot &  -    &  \\
%	59 & Testing and Commissioning &  & F & 5 & 4 &  & 1 & lot &  -    &  \\
%	\hline
%%}
%\end{longtable}

\section{Determination of optimal intervention strategies}

This section focuses on determination of intervention strategies for pumps of the PS. Other assets such as electrical assets and assets belong to FDAS system are not part of the analysis because they are utmost critical assets that require to provide adequate level of services at all time. Intervention strategies for electrical assets and FDAS assets are suggested to follow replacement schedule defined by the manufacturers.
%

\subsection{Reliability}
With the historical data given (refer to Table \ref{noofintervention}), it is not possible to precisely quantify the reliability for each pump. Thus, In order to more or less estimate the reliability, obtaining knowledge and experience of the end-users becomes important.

Following tables present the failure statistics on pump components

\begin{table}[h]
	\caption{Failure statistics.}
	\label{ch05_tbl_failuredata}
	{\footnotesize
\begin{tabular}{c|c|c|c}
	\hline
	Order & \multicolumn{3}{c}{Duration (hours)} \\ 
	\hline
	& Mech Seal & Bearing & Coupling \\ 
	\hline
	1 & 1992 & 1992 & 24144 \\ 
	2 & 3408 & 1986 & 2760 \\ 
	3 & 1848 & 624 & 264 \\ 
	4 & 1896 & 3408 &  \\ 
	5 & 840 &  &  \\ 
	6 &  &  &  \\ 
	\hline
\end{tabular}

	}
\end{table}


With so few data points, it is bias to use the data for determine the true parameter values of the Weibull model ($\eta$ and $\beta$). However, for the sake of future implementation, this section still shows the Weibull's parameter values estimated from the data provided in Table \ref{ch05_tbl_failuredata}

Figure \ref{ch05_fig_reliability_coupling} shows the


\begin{figure}[!htb]
	\begin{minipage}[b]{0.5\linewidth}
		\centering
		\includegraphics[width=\textwidth]{figures/ch05_fig_reliability_coupling_parameter}
		\caption*{a - Parameter values}
	\end{minipage}
	\hspace{0.05cm}
	\begin{minipage}[b]{0.5\linewidth}
		\centering
		\includegraphics[width=\textwidth]{figures/ch05_fig_reliability_coupling_curve}
		\caption*{b - Reliability curve}
	\end{minipage}
	\caption{Reliability estimation - Coupling}
	\label{ch05_fig_reliability_coupling}
\end{figure}


\begin{figure}[!htb]
	\begin{minipage}[b]{0.5\linewidth}
		\centering
		\includegraphics[width=\textwidth]{figures/ch05_fig_reliability_mechseal_parameter}
		\caption*{a - Parameter values}
	\end{minipage}
	\hspace{0.05cm}
	\begin{minipage}[b]{0.5\linewidth}
		\centering
		\includegraphics[width=\textwidth]{figures/ch05_fig_reliability_mechseal_curve}
		\caption*{b - Reliability curve}
	\end{minipage}
	\caption{Reliability estimation - Mechanical seal}
	\label{ch05_fig_reliability_mechseal}
\end{figure}



\input{figures/ch05_fig_reliability_bearing}



%As can be seen from the figure,

\subsection{Impacts}
Impacts are costs or loss of benefits incurred to the Client and users. Impacts are incurred by execution of PI or CI and by disruption to the operation of the PS.

Impacts incurred by execution of PI can be estimated based on, for example, conceptual design with the ballpark estimate. However, impacts incurred by execution of a CI is not easy to obtain due to lack of historical data. This is similar to the estimation of impacts such as loss in revenue, reputation, and regulatory.

Fortunately, from mathematical view point, the optimization model presented in subsection \ref{blockreplace} will be only dependent on the ratio between CI and PI. This means that we can make assumption on the ratio between CI and PI based on holistic approach. For example, if the PI is 10 millions PHP and CI is 20 million, the ratio would be CI/PI =2, and the model will determine the optimal time window (T) to execute the PI. This time window T will not change as long as the ratio between CI/PI is the same.

Values of revenues incurred by execution on respective pumps shall be calculated based on assumption on maintainability (e.g. duration of CI execution to fix the pump). Reputation and regulatory are not straightforward to estimate in monetary units, however, they can be assumed to be measured by "Willingness to Pay".

It is assumed in the estimation that the ratio between PI/CI is 0.2, which infers that the when there is a failure, the corrective intervention cost is 5 times higher than that of the preventive intervention.

The discount factor used in the study is 8.5\% annually as confirmed by Maynilad.


\subsection{Optimal Time Window and Impacts}
Figure \ref{ch05_fig_ois} presents a collection of curves representing Optimal Time Window (OTW) and impacts incurred by executing PIs on respective assets.
\begin{figure}[!htb]
	\begin{minipage}[b]{0.5\linewidth}
		\centering
		\includegraphics[width=\textwidth]{figures/ch05_fig_ois_coupling}
		\caption*{a - Coupling}
	\end{minipage}
	\hspace{0.05cm}
	\begin{minipage}[b]{0.5\linewidth}
		\centering
		\includegraphics[width=\textwidth]{figures/ch05_fig_ois_mechseal}
		\caption*{b - Mechanical seal}
	\end{minipage}
	\hspace{0.05cm}
\begin{minipage}[b]{0.5\linewidth}
	\centering
	\includegraphics[width=\textwidth]{figures/ch05_fig_ois_bearing}
	\caption*{c - Bearing}
\end{minipage}
\caption{Impact curves}
\label{ch05_fig_ois}
\end{figure}

The curves are with parabolic shapes, except for that of the coupling, with a minimal point representing the optimal point obtained by the model. Shapes of the curves prove following important conclusions

\begin{itemize}
\item If a PI is executed too often, the failure probability will be decreased, however, at the same time, more money is to spend on intervention activities. It will be a waste of resources to follow a program that triggers PIs;
\item The impact will decrease as the OTW increases to the optimal point and then go upward indicating that if a PI is executed beyond a certain reliability threshold, failure probability will increase, leading to more sudden failures that require to execute CIs and higher loss (e.g. the 3 Rs).
\end{itemize}

Summary of the reliability and LCC analysis are shown in Table \ref{ch05_tbl_ois}.

\begin{table}[h]
	\caption{Impact values (mus).}
	\label{ch05_tbl_ois}
	{\footnotesize
\begin{tabular}{c|c|c|c|c|c}
	\hline
	Assets & $\eta$ & $\beta$ & $PI/CI$ & OTW & Impacts \\ 
	&  &  &  & hours & (mus) \\ 
	\hline
	\multicolumn{1}{l|}{Coupling} & 7'597 & 0.4276 & 0.2 & NA & NA \\ 
	\multicolumn{1}{l|}{Mechanical seal} & 2'275 & 2.203 & 0.2 & 1'139 & $3.31x10^{-4}$ \\ 
	\multicolumn{1}{l|}{Bearing} & 2'354 & 1.477 & 0.2 & 1'675 & $4.18x10^{-4}$ \\ 
	\hline
\end{tabular}
\\
		Note: OTW stands for Optimal Time Window to execute an PI.
	}
\end{table}

Following can be interpreted from the values shown in the Figures and the Table.

\paragraph{\underline{Coupling}}

\begin{itemize}
\item There are only 3 data points recorded to date on coupling for this PS. Values of data points are enormously different, inferring that there has been abnormal events or operation procedure leading to the failure of coupling. In this respect, data points cannot be considered as random and thus it it not recommended to be used for determination of optimal intervention strategy. This is also confirmed by the life cycle cost curve with lowest point goes even beyond 10'000 hours;
\end{itemize}

\paragraph{\underline{Mechanical seal}}
\begin{itemize}
\item The optimal time to execute replacement for the mechanical seal is at 1'139 hours of operation. The corresponding value of lowest impact is $3.31x10^{-4}$ for an hour. %If the value of PI is 200'000 Peso, the CI would be 1'000'000 Peso and the average hourly impact would be $3.31x10^{-4}$*1'000'000=330 Peso. This is equivalent to 2'851'200 Peso annually (330x24x)
\item If the PI is executed at an interval of more than 3'000 hours, the impact becomes high but slowly converge to a constant. This can be understood that after 3'000 hours, it is almost certain that the seal will be failed with high probability and thus it is no different between PI and CI.

\end{itemize}

\paragraph{\underline{Bearing}}
\begin{itemize}
	\item The optimal time to execute replacement for the bearing is at 1'875 hours of operation. The corresponding value of lowest impact is $4.18x10^{-4}$ for an hour. %If the value of PI is 200'000 Peso, the CI would be 1'000'000 Peso and the average hourly impact would be $3.31x10^{-4}$*1'000'000=330 Peso. This is equivalent to 2'851'200 Peso annually (330x24x)
	\item If the PI is executed at an interval of more than 2'000 hours, the impact becomes high but slowly converge to a constant. This can be understood that after 23'000 hours, it is almost certain that the seal will be failed with high probability and thus it is no different between PI and CI.

\end{itemize}

\subsection{Sensitivity analysis}
As a matter of fact, results of the optimization model are subjected to uncertainties with model's parameters and variables. Model's parameters are deterioration parameters $\eta$ and $\beta$ and impacts that should be obtained from historical data.

This subsection provides results of sensitivity analysis (SA) conducted on Weibull parameters $\eta$ and $\beta$ and the ratio between PI and CI. A SA was conducted by running the targeted parameter in a pre-defined range while keeping other parameters constant.

Results of the SA on parameter $\eta$, $\beta$, and $\frac{PI}{CI}$ are shown in Figure \ref{ch05_fig_sa_eta}, Figure \ref{ch05_fig_sa_beta}, and Figure \ref{ch05_fig_sa_pici}.

\begin{figure}[!htb]
	\begin{minipage}[b]{0.5\linewidth}
		\centering
		\includegraphics[width=\textwidth]{figures/ch05_fig_saeta_coupling}
		\caption*{a - Coupling}
	\end{minipage}
	\hspace{0.05cm}
	\begin{minipage}[b]{0.5\linewidth}
		\centering
		\includegraphics[width=\textwidth]{figures/ch05_fig_saeta_mechseal}
		\caption*{b - Mechnical seal}
	\end{minipage}
	\hspace{0.05cm}
	\begin{minipage}[b]{0.5\linewidth}
		\centering
		\includegraphics[width=\textwidth]{figures/ch05_fig_saeta_bearing}
		\caption*{c - Bearing}
	\end{minipage}
	\caption{Sensitivity Analysis ($\eta$)}
	\label{ch05_fig_sa_eta}
\end{figure}

\begin{figure}[!htb]
	\begin{minipage}[b]{0.5\linewidth}
		\centering
		\includegraphics[width=\textwidth]{figures/ch05_fig_sabeta_coupling}
		\caption*{a - Coupling}
	\end{minipage}
	\hspace{0.05cm}
	\begin{minipage}[b]{0.5\linewidth}
		\centering
		\includegraphics[width=\textwidth]{figures/ch05_fig_sabeta_mechseal}
		\caption*{b - Mechnical seal}
	\end{minipage}
	\hspace{0.05cm}
	\begin{minipage}[b]{0.5\linewidth}
		\centering
		\includegraphics[width=\textwidth]{figures/ch05_fig_sabeta_bearing}
		\caption*{c - Bearing}
	\end{minipage}
	\caption{Sensitivity Analysis ($\beta$)}
	\label{ch05_fig_sa_beta}
\end{figure}

\begin{figure}[!htb]
	\begin{minipage}[b]{0.5\linewidth}
		\centering
		\includegraphics[width=\textwidth]{figures/ch05_fig_sapici_coupling}
		\caption*{a - Coupling}
	\end{minipage}
	\hspace{0.05cm}
	\begin{minipage}[b]{0.5\linewidth}
		\centering
		\includegraphics[width=\textwidth]{figures/ch05_fig_sapici_mechseal}
		\caption*{b - Mechnical seal}
	\end{minipage}
	\hspace{0.05cm}
	\begin{minipage}[b]{0.5\linewidth}
		\centering
		\includegraphics[width=\textwidth]{figures/ch05_fig_sapici_bearing}
		\caption*{c - Bearing}
	\end{minipage}
	\caption{Sensitivity Analysis ($\frac{PI}{CI}$)}
	\label{ch05_fig_sa_pici}
\end{figure}


\section{Return on Investment}
It is important to note that in asset management context, the Return on Investment (ROI) is understood different from the ROI used for CAPEX projects. In CAPEX project, ROI is a ratio between the Net Present Value (NPV) of benefits (e.g. positive sum of cash flow) incurred over a pre-defined life cycle of a project (e.g. 20 or 30 years), at which there is a salvage value of the facility.

In asset management context, the ROI encompasses the ROI used in CAPEX project. This can be obviously seen in the impact curves shown in Figure \ref{ch05_fig_ois}. If the owner follows the optimal time window defined to execute a PI on any individual pump, the return on investment compared to other strategy that follow different time window is the difference of the impacts.

If the Owner expects a 5 years intervention program, it equivalents to a program that combines inspection and testing and spare part replacement.

For example, if the Owner believes that the mechanical seal follows the Weibull's model with parameters shown in Table \ref{ch05_tbl_ois}, it means that the optimal replacement would be at 1'139 hours. This infers that within 5 years plan, there will be 38 times to replace mechanical seal ($\frac{5 \times 365 \times 24}{1'139}=38$). Similarly, the numbers of replacement for bearing will be 26 times ($\frac{5 \times 365 \times 24}{1'675}=26$).

Given this background and under maintenance constraint, it might be optimal to combine the two intervention types of coupling and mechanical seal by simply looking at the behavior of the life cycle cost curves. By combining the two curves, the average point can be used to decide the intervention time window for the combination of intervention types. This infers that the OTW will be in between 26 times and 38 times. Figure \ref{ch05_fig_mixIS} shows the combination of the two curves (coupling and bearing), which give the OTW of 1'289 hours, inferring that within 5 years plan, there should be 34 times of replacement for both coupling and bearing (($\frac{5 \times 365 \times 24}{1'289}=34$)).

\begin{figure}[h]
	%	\begin{center}
	\includegraphics[width =\textwidth]{figures/ch05_fig_mixIS}
	%	\end{center}
	\caption{Mix intervention types}
	\label{ch05_fig_mixIS}
\end{figure}

By multiplying the numbers of replacement times in 5 years with total numbers of pumps in operation, the expected number of spare parts to be stocked in 5 years can be straightforward to be estimated.

This exercise gives a complete and scientific way on how to determine the intervention program for 5 years (or for any number of years). It is up to the owner to decide on the next step. However, to the best of the knowledge obtained from limited data, it is optimal to continue monitoring the failures of assets rather than executing an intervention strategy that contains a high level of uncertainty.
%s\section{Discussions}
