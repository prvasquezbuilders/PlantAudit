\chapter{Data of Visual Inspection and Auditing} \label{Chapter4}
%\lipsum[1-1]
\section{Mechanical Audit}
\label{42}
\subsection{Operational based Audit}
Summary of facts and data concerning operational and overall plan reliability is presented in this subsection.

\subsubsection{Normal Operation Scenario}
\begin{itemize}
\item 6 booster pumps are available but only 5 are in operation. The booster pumps deliver to the Cavite side of the distribution system including parts of Pasay. Distribution pipe size is 1400mm. 
\item A separate system for the Reclamation Area is supplied by the storage pumps. The line size is 700mm. Only one storage pump is required for the Reclamation Area distribution system. The other pump serves as standby. 
\item At 10pm-12mn, the MOV to the Reclamation Area distribution system is closed and the 2 storage pumps are used to refill the 2 storage tanks. During this period, the individual consumers are using their own stored water supply for internal use. Normal operations resumes at 12mn.
\end{itemize}

\subsubsection{High Demand Scenario}

\begin{itemize}
	\item all 6 booster pumps are in operation
\end{itemize}

\subsubsection{Low Demand Scenario}

\begin{itemize}
	\item only 5 booster pumps are running
\end{itemize}

\subsubsection{Spares Policy}

\begin{itemize}
	\item The booster pumps do not have a spare during high demand scenario. If 1 pump will be under maintenance during this period, there will be a drop in pressure and some of the consumers in the outermost section of the distribution system will experience low pressure or will not get any water. 
	\item The storage pumps also do not have a spare during the refilling phase. If a failure of one of the storage pumps occurs, the refilling duration will be much longer to complete.
\end{itemize}

\subsubsection{Emergency Situation (loss of electrical power from Meralco)}
\begin{itemize}
	\item Immediately after the power is cut, one of the 2 gensets automatically start and makes electric power available for the booster and/or supply pumps. The operator resets all the pumps before restarting each manually. This may take between 60 seconds to 5 minutes before all the pumps provide enough pressure/flow to the system.
\end{itemize}

\subsubsection{Maintenance}
\begin{itemize}
	\item There is no structured maintenance program in the facilities. The operator makes rounds of the pumps and if something unusual is observed, a text or email is sent to the Control Center for scheduling of maintenance check and repairs.
	\item There is also a maintenance team visiting the site regularly with specific lists of tasks and responsibilities.
\end{itemize}

\subsubsection{Others}
\begin{itemize}
	\item Water Leakages are found in the engineering room during rains.
\end{itemize}

\subsection{Pump discharge and suction pipe - thickness} \label{ch04mech01}
Thickness data on discharge and suction pipes of pumps is presented in Table \ref{ch04_tbl_thickness02} and Table \ref{ch04_tbl_thickness03} .


\begin{table}[h]
	\caption{Thickness data - Booster Pumps (mm).}
	\label{ch04_tbl_thickness02}
	{\footnotesize
\begin{tabular}{l|c|c|c|c|c|c|c|c}
	\hline
	Asset & Position & \multicolumn{7}{c}{Distance} \\ 
	\cline{3-9}
	&  & \multicolumn{4}{c|}{Suction} & \multicolumn{3}{c}{Discharge} \\ 
	\cline{3-9}
	&  & 2.5m & 3.5m & 4.5m & Elbow & 4m & 5m & Elbow \\ 
	\hline
	BP1 & 12 & 4.68 & 4.68 & 4.66 & 4.59 & 4.60 & 4.63 & 3.94 \\ 
	& 3 & 4.62 & 4.75 & 4.79 & 4.48 & 4.69 & 4.67 & 3.92 \\ 
	& 6 & 4.69 & 4.74 & 4.69 & 4.63 & 4.73 & 4.60 & 4.64 \\ 
	& 9 & 4.71 & 4.74 & 4.71 & - & 4.71 & 4.72 & - \\ 
	\hline
	BP2 & 12 & 4.68 & 4.41 & 4.69 & 4.66 & 4.73 & 4.92 & 4.93 \\ 
	& 3 & 4.76 & 4.67 & 4.80 & 4.32 & 4.91 & 4.22 & 4.98 \\ 
	& 6 & 4.69 & 4.46 & 4.76 & 4.69 & 4.70 & 4.94 & 4.62 \\ 
	& 9 & 4.65 & 4.78 & 4.69 & - & 4.88 & 4.90 & - \\ 
	\hline
	BP3 & 12 & 4.61 & 4.67 & 4.65 & 4.69 & 4.71 & 4.69 & 4.62 \\ 
	& 3 & 4.66 & 4.69 & 4.75 & 4.72 & 4.76 & 4.79 & 4.73 \\ 
	& 6 & 4.68 & 4.63 & 4.66 & 4.67 & 4.68 & 4.70 & 4.61 \\ 
	& 9 & 4.65 & 4.69 & 4.64 & - & 4.67 & 4.74 & - \\ 
	\hline
	BP4 & 12 & 4.63 & 4.77 & 4.75 & 4.67 & 4.91 & 4.98 & 4.37 \\ 
	& 3 & 4.65 & 4.75 & 4.70 & 4.69 & 4.90 & 4.92 & 4.97 \\ 
	& 6 & 4.75 & 4.54 & 4.71 & 4.83 & 4.93 & 4.95 & 4.75 \\ 
	& 9 & 4.62 & 4.67 & 4.58 & - & 4.98 & 4.92 & - \\ 
	\hline
	BP5 & 12 & 4.80 & 4.69 & 4.49 & 4.73 & 4.72 & 4.68 & 4.76 \\ 
	& 3 & 4.71 & 4.56 & 4.71 & 4.67 & 4.86 & 4.65 & 4.76 \\ 
	& 6 & 4.75 & 4.75 & 4.60 & 4.67 & 4.75 & 4.64 & 4.86 \\ 
	& 9 & 4.66 & 4.65 & 4.67 & - & 4.67 & 4.74 & - \\ 
	\hline
	BP6 & 12 & 4.64 & 4.66 & 4.77 & 4.09 & 4.95 & 4.96 & 4.92 \\ 
	& 3 & 4.65 & 4.74 & 4.72 & 4.46 & 4.93 & 4.87 & 4.98 \\ 
	& 6 & 4.66 & 4.77 & 4.71 & 4.69 & 4.86 & 4.88 & 4.60 \\ 
	& 9 & 4.75 & 4.63 & 4.70 & - & 4.81 & 4.79 &  \\ 
	\hline
\end{tabular}

	}
\end{table}



\begin{table}[h]
	\caption{Thickness data - Storage Pumps (mm).}
	\label{ch04_tbl_thickness03}
	{\footnotesize
\begin{tabular}{l|c|c|c|c|c|c|c|c|l|l}
	\hline
	Asset & Position & \multicolumn{9}{c}{Distance} \\ 
	\cline{3-11}
	&  & \multicolumn{4}{c|}{Suction} & \multicolumn{5}{c}{Discharge} \\ 
	\cline{3-11}
	&  & 3m & 5m & 6m & Elbow & 4m & 6m & 8m & \multicolumn{1}{c|}{9m} & \multicolumn{1}{c}{Elbow} \\ 
	\hline
	SP1 & 12 &  &  &  & 4.84 & 4.78 & 4.76 & 4.74 & \multicolumn{1}{c|}{4.77} & \multicolumn{1}{c}{4.61} \\ 
	& 3 &  &  &  & 4.54 & 4.61 & 4.78 & 4.64 & \multicolumn{1}{c|}{4.72} & \multicolumn{1}{c}{4.77} \\ 
	& 6 &  &  &  & 4.4 & 4.25 & 4.81 & 4.8 & \multicolumn{1}{c|}{4.64} & \multicolumn{1}{c}{4.66} \\ 
	& 9 &  &  &  &  & 4.73 & 4.88 & 4.65 & \multicolumn{1}{c|}{4.72} & \multicolumn{1}{c}{-} \\ 
	\hline
	SP2 & 12 & 5.4 & 5.11 & 5.35 & 5.43 & 4.6 & 4.75 & 4.97 & \multicolumn{1}{c|}{4.74} & \multicolumn{1}{c}{4.91} \\ 
	& 3 & 5.28 & 5.12 & 5.3 & 5.41 & 4.62 & 4.78 & 4.78 & \multicolumn{1}{c|}{4.66} & \multicolumn{1}{c}{4.64} \\ 
	& 6 & - & 5.09 & 5.45 & 5.4 & 4.43 & 4.18 & 4.73 & \multicolumn{1}{c|}{4.75} & \multicolumn{1}{c}{4.51} \\ 
	& 9 & - & 5.2 & 5.22 & - & 4.4 & 4.75 & 4.45 & \multicolumn{1}{c|}{4.58} & \multicolumn{1}{c}{-} \\ 
	\hline
\end{tabular}

	}
\end{table}

In the table, the positions and the distances for the Ultrasonic Thickness Gauging (UTG) are referred to Figure \ref{ch04_fig_utgbp} and Figure \ref{ch04_fig_utgsp}.

\begin{figure}[!htb]
	\includegraphics[scale=1.3]{figures/ch04_fig_utgbp} \\
	\caption{Positions and distances of UTG - Booster Pump}
	\label{ch04_fig_utgbp} 
\end{figure}

\begin{figure}[!htb]
	\includegraphics[scale=1.3]{figures/ch04_fig_utgsp} \\
	\caption{Positions and distances of UTG - Storage Pump}
	\label{ch04_fig_utgsp} 
\end{figure}


%\begin{table}[h]
%	\caption{Thickness data (mm).}
%	\label{thicknessdata}
%	{\footnotesize
%	\begin{tabular}{l|l|l|l|l}
%		\hline
%		Pumps & \multicolumn{2}{c|}{Suction} & \multicolumn{2}{c}{Discharge} \\ 
%		\cline{2-5}
%		& \multicolumn{1}{c|}{Design} & \multicolumn{1}{c|}{Actual} & \multicolumn{1}{c|}{Design } & \multicolumn{1}{c}{Actual} \\ 
%		\hline
%		BP1 & \multicolumn{1}{c|}{} & \multicolumn{1}{c|}{4.98} & \multicolumn{1}{c|}{} & \multicolumn{1}{c}{3.92} \\ 
%		BP2 & \multicolumn{1}{c|}{} & \multicolumn{1}{c|}{4.32} & \multicolumn{1}{c|}{} & \multicolumn{1}{c}{4.22} \\ 
%		BP3 & \multicolumn{1}{c|}{} & \multicolumn{1}{c|}{4.61} & \multicolumn{1}{c|}{} & \multicolumn{1}{c}{4.61} \\ 
%		BP4 & \multicolumn{1}{c|}{} & \multicolumn{1}{c|}{4.54} & \multicolumn{1}{c|}{} & \multicolumn{1}{c}{4.37} \\ 
%		BP5 & \multicolumn{1}{c|}{} & \multicolumn{1}{c|}{4.49} & \multicolumn{1}{c|}{} & \multicolumn{1}{c}{4.64} \\ 
%		BP6 & \multicolumn{1}{c|}{} & \multicolumn{1}{c|}{4.09} & \multicolumn{1}{c|}{} & \multicolumn{1}{c}{4.60} \\ 
%		SP1 & \multicolumn{1}{c|}{} & \multicolumn{1}{c|}{4.40} & \multicolumn{1}{c|}{} & \multicolumn{1}{c}{4.25} \\ 
%		SP2 & \multicolumn{1}{c|}{} & \multicolumn{1}{c|}{5.09} & \multicolumn{1}{c|}{} & \multicolumn{1}{c}{4.18} \\ 
%		\hline
%	\end{tabular}
%			
%	}
%\end{table}

Detailed measurement data is provided in appendix \ref{appthicknesss}.
%\textcolor{red}{RB Sanchez to write here the summary of raw data collected from visual inspection and testing. Tables shall be used as much as we can. Note that no analysis in this session. This session is purely the high level presentation of data. Raw data can be linked as an Appendix}
\subsection{Pipe, valves, fittings, supports, expansions, and appurtenances} \label{ch04mech02}
%\textcolor{red}{RB Sanchez to write here the summary of raw data collected from visual inspection and testing. Tables shall be used as much as we can. Note that no analysis in this session. This session is purely the high level presentation of data. Raw data can be linked as an Appendix}

\subsubsection{Highlights}

Visual inspection data on pipes, valves, fittings, supports, expansions, and appurtenances is highlighted in Table \ref{ch04_visualinspection01}.

\begin{table}[h]
	\caption{Highlights of visual inspection}
	\label{ch04_visualinspection01}
	\resizebox{\columnwidth}{!}{%
	{\scriptsize
\begin{tabular}{c|p{2.5cm}|c|c|c|p{6cm}|p{1cm}}
	\hline
	No. & Assets/ Description & Code & Status & CS & Remarks & Refer/ Figure \\ 
	\hline
	1 & Valve Leaks &  & 1 &  & Water leakages for moving parts of checked valves of pumps. Evident of prolonged leaks are shown by local corrosion and accumulation of water pools around valves's vincinity &  \\ 
	2 & Pressure gauges &  & 1 &  & Discrepancy in reading between the dial and digital gauges &  \\ 
	&  &  &  &  & Defective bourdon (dial) pressure gauges were found at suction side. No pressure gauses installed near pump suction nozzle and discharge flange inferring no ability to immediately read the head pressure. &  \\ 
	&  &  &  &  & Excessive deterioration/fading of tags making them unreadable &  \\ 
	&  &  &  &  & Some tags have been superimposed with recent data written by inappropriate markers that cause difficulty in reading. &  \\ 
	&  &  &  &  & Some tags found with inconsistency of data &  \\ 
	3 & Alignment bolts &  & 1 &  & without washers &  \\ 
	4 & Grounding cables &  & 1 &  & Not found for any motor &  \\ 
	5 & Vibration monitoring probes &  & 1 &  & Some were found disconnected or/and untended &  \\ 
	6 & Spare pumps &  & 1 &  & Susceptible to false brinelling (e.g. The spare pump located near BP1 may experience minor vibrations due to its location near the almost continuously operating pumps. There is no observed intermittent rotating of the pump shaft by the operators during the visit) &  \\ 
	7 & Piping stability/settlement &  & 1 &  & Possible execcisve level of movement leading to weakness of tensile/compressive strength of materials/fixtures. The entire pump house is suffering from ground settlement and that the alignments of the pumps and the corresponding fittings have been compromised. Mitigating supports have been installed but their functions are doubtful due to observed obscurities. Interviews with operators also confirm serious damages to pumps and increased vibration during operation because of the significant piping movements &  \\ 
	&  &  &  &  & Wall opening provisions for pipe movement due to ground settlement. These local tear downs leave the wall unaesthetic and inconsistently enclosed as some are covered with deteriorating plywood. Some round bars are left protruding and can pause danger &  \\ 
	&  &  &  &  & Crooked and/or drawn concrete saddle strap bolts. Some of the fastening area on the concrete saddle display small to large cracks &  \\ 
	&  &  &  &  & Evidences of axial suction piping movements &  \\ 
	&  &  &  &  & Space gap between saddle arc and pipe &  \\ 
	&  &  &  &  & Doubtful piping supports. The counter action of the U supports (both round and flat type) are doubtful. The U supports function to counter the water thrust on the elbow as immense water volumes pass thru it. The bolts are not turned to tighten the flat bar U support. Space between the U supports and the pipes are observed &  \\ 
	8 & Motion actuated lighting &  &  &  & Interview with the maintenance team revealed that the motion actuated lighting sometimes causes slight nausea due to dim lighting when repair. The minute motions of repair are sometimes not enough to actuate the lights and thus interrupt the work &  \\ 
	9 & Unnecessary objects around the station &  &  &  &  &  \\ 
	&  &  &  &  &  &  \\ 
	\hline
\end{tabular}	
	}}
\end{table}



%Values of CSs presented in the table are determined based on both generic definition of CSs as presented in Table \ref{ch03:cs} and specifically in the Table \ref{ch04:cs}.


%\begin{longtable}{|p{2.5cm}|p{8cm}|p{3cm}|}
%	\caption{Deliverable plan} \label{tab:long} \\
%	
%	\hline \multicolumn{1}{|l|}{\textbf{Items}} & \multicolumn{1}{l|}{\textbf{Task description}} & \multicolumn{1}{l|}{\textbf{Resources}} \\ \hline 
%	\endfirsthead
%	
%	\multicolumn{3}{c}%
%	{{\bfseries \tablename\ \thetable{} -- continued from previous page}} \\
%	\hline \multicolumn{1}{|l|}{\textbf{Items}} & \multicolumn{1}{l|}{\textbf{Task description}} & \multicolumn{1}{l|}{\textbf{Resources}} \\ \hline 
%	\endhead
%	\hline \multicolumn{3}{|r|}{{Continued on next page}} \\ \hline
%	\endfoot
%	\hline \hline
%	\endlastfoot
%	Phase 1 & Current state review &  \\ \hline 
%	Define Status Quo & Project Kick-off through a 3-hour kick off meeting with Maynilad team to review and revise important set of documents (e.g. project plan, objectives, workshop schedule, core team, existing prominent data) & GHD and Maynilad \\ \hline 
%	review & Review and analysis existing prominent data and knowledge, recommend a set of tests to be conducted to further identify the reliability and efficiency of equipment and facilities. & GHD \\ \hline 
%	Stakeholder Engagement Workshop 1 & Run the Workshop 1 to validate the Status Quo and define tasks for the next step (e.g. a concrete list of tests for equipment and facilities) & GHD with Maynilad participants \\ \hline 
%	Deliverable & Report, providing Status Quo, identify and highlight consolidated gaps and challenges & GHD with Maynilad review and approval \\ \hline 
%	Phase 2 & Database development for Data acquisition purposes, System description, and Condition State definition &  \\ \hline 
%	Select program & Define a suitable database program to be implemented for data collection and validation (e.g. PostgreSQL, MS access, or MS excel) & GHD \\ \hline 
%	Data acquisition & Import and migrate data in different formats (e.g. flat file, excel) to the selected database program & GHD \\ \hline 
%	Condition state definition & Identify system description and develop an appropriate set of condition states representing either physical condition and operational condition of items, components, sub-system, and system. & GHD \\ \hline 
%	Stakeholder Engagement Workshop 2 & Run the Workshop 2 to present up-to-date development status for data acquisition, system description, and condition state definition & GHD with Maynilad participants \\ \hline 
%	Deliverable & Report on data acquisition, system description and conditional state definition & GHD with Maynilad review and approval \\ \hline 
%	Phase 3 & System engineering and operational analysis (Modelling) &  \\ \hline 
%	Qualitative risk analysis & Study on risks that can be described as a combination of intensity and consequence. Perform interview to extract useful information learn from end users, engineers and line managers of Maynilad & GHD \\ \hline 
%	Quantitative risk analysis & Conduct reliability study on existing data (incl. estimation for failure rate, reliability, efficiency, availability, maintainability,  & GHD \\ \hline 
%	Operation study & Review and record various parameters on technical and financial operation of items, components, sub-system, and system (e.g. corrective and preventive intervention costs, energy consumption, labor consumption, spare parts) & GHD \\ \hline 
%	Stakeholder Engagement Workshop 3 & Run the Workshop 3 to present up-to-date development status for risk and reliability analysis and operation study & GHD with Maynilad participants \\ \hline 
%	Deliverable & Report on reliability study and operational study & GHD with Maynilad review and approval \\ \hline 
%	Phase 4 & Evaluation &  \\ \hline 
%	Life cycle cost analysis & Perform LCC analysis for items with different set of preventive intervention strategies & GHD \\ \hline 
%	Benchmarking & Benchmark for the optimal set of intervention strategies that yields the minimum LCC, whilst satisfying the requirements of Maynilad & GHD \\ \hline 
%	Consolidation & Consolidate the optimal intervention strategies to form the optimal preventive intervention program for 5 years implementation. & GHD \\ \hline 
%	Stakeholder Engagement Workshop 4 & Run the Workshop 4 to present the results of life cycle cost analysis and optimal preventive intervention program & GHD with Maynilad participants \\ \hline 
%	Deliverable & Report on life cycle cost analysis and optimal preventive intervention program for 5 years plan & GHD with Maynilad review and approval \\ \hline 
%	Phase 5 & Design &  \\ \hline 
%	Design & Perform 4 steps of detailed design for the purpose of procurement and installation & GHD \\ \hline 
%	Stakeholder Engagement Workshop 5 & Run the Workshop 5 to present the results of the detailed design works and conduct value engineering for selection of optimal design if required& GHD with Maynilad participants \\ \hline 
%	Deliverables & Calculation sheet, modelling, basic design report, draft and final versions of all reports and drawings associated with detailed design. & GHD with Maynilad review and approval \\ \hline 
%	Phase 6 & Tender package preparation &  \\ \hline 
%	Writing tender documents & Write and compile documents for tender package (e.g. instruction to bidders, scope of works, contract, program, functional guarantee) & GHD \\ \hline 
%	Stakeholder Engagement Workshop 6 & Run the Workshop 6 to present the content of the tender package & GHD with Maynilad participants \\ \hline 
%	Deliverable & Draft version and final version of the tender package  & GHD with Maynilad review and approval \\ \hline 
%	Finalization &  &  \\ \hline 
%	Stakeholder Engagement Workshop 7 & Run the Workshop 7 to transfer the knowledge to Maynilad team & GHD with Maynilad participants \\ \hline 
%	Deliverable & Report on knowledge transfer workshop  & GHD with Maynilad review and approval \\ \hline 
%	
%\end{longtable}

\subsubsection{Visual inspection data}
Visual inspection data on assets are summarized in tables of this section and also in the Appendix \ref{appvisualinspectionmech} with pictures.

%\paragraph{\textbf{BP1}}

\begin{table}[h]
	\caption{Visual inspection data - BP1}
	\label{ch04_visualinspectionbp1}
	\resizebox{\columnwidth}{!}{%
		{\scriptsize
			\begin{tabular}{c|p{2.5cm}|c|c|c|p{6cm}|p{1cm}}
\hline
No. & Assets/ Description & Code & CS & Status & Remarks & Refer/ Figure \\ 
\hline
1 & SBV & WSO-PAGPS-0051 & - &  & Below ground; not inspected,  &  \\ 
2 & FJ &  & - &  & Below ground; not inspected, &  \\ 
3 & SE &  & 1 &  & Mitered elbows were used (not radius elbows) &  \\ 
4 & PIPE1 &   & 2 &  &  &  \\ 
5 & ECR &  & 1 &  &  &  \\ 
6 & FJ &  & 4 &  &  &  \\ 
7 & CS1 &  & 4 &  &  &  \\ 
8 & CS2 &  & 4 &  &  &  \\ 
9 & CCR &  & 1 &  &  &  \\ 
10 & CV & WSO-PAGPS-0055 & 2 &  &  &  \\ 
11 & EJ &  & 4 &  &  &  \\ 
12 & DBV & WSO-PAGPS-0058 & 1 &  &  &  \\ 
13 & DE &  & 1 &  &  &  \\ 
14 & PAD &  & 1 &  &  &  \\ 
15 & CS3 &  & 1 &  &  &  \\ 
\hline
\end{tabular}

	}}
\end{table}



\begin{table}[h]
	\caption{Visual inspection data - BP2}
	\label{ch04_visualinspectionbp2}
	\resizebox{\columnwidth}{!}{%
		{\scriptsize
			\begin{tabular}{c|p{2.5cm}|c|c|c|p{6cm}|p{1cm}}
				\hline
				No. & Assets/ Description & Code & CS & Status & Remarks & Refer/ Figure \\ 
\hline
1 & SBV & WSO-PAGPS-0061 & - &  & Below ground; not inspected,  &  \\ 
2 & FJ &  & - &  & Below ground; not inspected, &  \\ 
3 & SE &  & 1 &  & Mitered elbows were used (not radius elbows) &  \\ 
4 & PIPE1 &  & 2 &  &  &  \\ 
5 & ECR &  & 1 &  &  &  \\ 
6 & FJ &  & 4 &  &  &  \\ 
7 & CS1 &  & 4 &  &  &  \\ 
8 & CS2 &  & 4 &  &  &  \\ 
9 & CCR &  & 1 &  &  &  \\ 
10 & CV & WSO-PAGPS-0065 & 1 &  &  &  \\ 
11 & EJ &  & 4 &  &  &  \\ 
12 & DBV & WSO-PAGPS-0068 & 1 &  &  &  \\ 
13 & DE &  & 1 &  &  &  \\ 
14 & PAD &  & 1 &  &  &  \\ 
15 & CS3 &  & 1 &  &  &  \\ 
\hline
\end{tabular}
	
	}}
\end{table}

\begin{table}[h]
	\caption{Visual inspection data - BP3}
	\label{ch04_visualinspectionbp3}
	\resizebox{\columnwidth}{!}{%
		{\scriptsize
			\begin{tabular}{c|p{2.5cm}|c|c|c|p{6cm}|p{1cm}}
				\hline
				No. & Assets/ Description & Code & CS & Status & Remarks & Refer/ Figure \\ 
\hline
1 & SBV & WSO-PAGPS-0071 & - &  & Below ground; not inspected &  \\ 
2 & FJ &  & - &  & Below ground; not inspected, &  \\ 
3 & SE &  & 1 &  & Mitered elbows were used (not radius elbows) &  \\ 
4 & PIPE1 &  & 2 &  &  &  \\ 
5 & ECR &  & 1 &  &  &  \\ 
6 & FJ &  & 4 &  &  &  \\ 
7 & CS1 &  & 4 &  &  &  \\ 
8 & CS2 &  & 4 &  &  &  \\ 
9 & CCR &  & 1 &  &  &  \\ 
10 & CV & WSO-PAGPS-0075 & 1 &  &  &  \\ 
11 & EJ &  & 4 &  &  &  \\ 
12 & DBV & WSO-PAGPS-0078 & 1 &  &  &  \\ 
13 & DE &  & 1 &  &  &  \\ 
14 & PAD &  & 1 &  &  &  \\ 
15 & CS3 &  & 1 &  &  &  \\ 
\hline
\end{tabular}
	}}
\end{table}


\begin{table}[h]
	\caption{Visual inspection data - BP4}
	\label{ch04_visualinspectionbp4}
	\resizebox{\columnwidth}{!}{%
		{\scriptsize
			\begin{tabular}{c|p{2.5cm}|c|c|c|p{6cm}|p{1cm}}
				\hline
				No. & Assets/ Description & Code & CS & Status & Remarks & Refer/ Figure \\ 
\hline
1 & SBV & WSO-PAGPS-0081 & - &  & Below ground; not inspected &  \\ 
2 & FJ &  & - &  & Below ground; not inspected, &  \\ 
3 & SE &  & 1 &  & Mitered elbows were used (not radius elbows) &  \\ 
4 & PIPE1 &  & 2 &  &  &  \\ 
5 & ECR &  & 1 &  &  &  \\ 
6 & FJ &  & 4 &  &  &  \\ 
7 & CS1 &  & 4 &  &  &  \\ 
8 & CS2 &  & 4 &  &  &  \\ 
9 & CCR &  & 1 &  &  &  \\ 
10 & CV & WSO-PAGPS-0085 & 2 &  &  &  \\ 
11 & EJ &  & 4 &  &  &  \\ 
12 & DBV & WSO-PAGPS-0088 & 1 &  &  &  \\ 
13 & DE &  & 1 &  &  &  \\ 
14 & PAD &  & 1 &  &  &  \\ 
15 & CS3 &  & 1 &  &  &  \\ 
\hline
\end{tabular}
	}}
\end{table}

\begin{table}[h]
	\caption{Visual inspection data - BP5}
	\label{ch04_visualinspectionbp5}
	\resizebox{\columnwidth}{!}{%
		{\scriptsize
			\begin{tabular}{c|p{2.5cm}|c|c|c|p{6cm}|p{1cm}}
				\hline
				No. & Assets/ Description & Code & CS & Status & Remarks & Refer/ Figure \\ 
				\hline
1 & SBV & WSO-PAGPS-0091 & - &  & Below ground; not inspected &  \\ 
2 & FJ &  & - &  & Below ground; not inspected, &  \\ 
3 & SE &  & 1 &  & Mitered elbows were used (not radius elbows) &  \\ 
4 & PIPE1 &  & 2 &  &  &  \\ 
5 & ECR &  & 1 &  &  &  \\ 
6 & FJ &  & 4 &  &  &  \\ 
7 & CS1 &  & 4 &  &  &  \\ 
8 & CS2 &  & 4 &  &  &  \\ 
9 & CCR &  & 1 &  &  &  \\ 
10 & CV & WSO-PAGPS-0095 & 1 &  &  &  \\ 
11 & EJ &  & 4 &  &  &  \\ 
12 & DBV & WSO-PAGPS-0098 & 1 &  &  &  \\ 
13 & DE &  & 1 &  &  &  \\ 
14 & PAD &  & 1 &  &  &  \\ 
15 & CS3 &  & 1 &  &  &  \\ 
\hline
\end{tabular}
	}}
\end{table}

\begin{table}[h]
	\caption{Visual inspection data - BP6}
	\label{ch04_visualinspectionbp6}
	\resizebox{\columnwidth}{!}{%
		{\scriptsize
			\begin{tabular}{c|p{2.5cm}|c|c|c|p{6cm}|p{1cm}}
				\hline
				No. & Assets/ Description & Code & CS & Status & Remarks & Refer/ Figure \\ 
				\hline
1 & SBV & WSO-PAGPS-0101 & - &  & Below ground; not inspected &  \\ 
2 & FJ &  & - &  & Below ground; not inspected, &  \\ 
3 & SE &  & 1 &  & Mitered elbows were used (not radius elbows) &  \\ 
4 & PIPE1 &  & 2 &  &  &  \\ 
5 & ECR &  & 1 &  &  &  \\ 
6 & FJ &  & 4 &  &  &  \\ 
7 & CS1 &  & 4 &  &  &  \\ 
8 & CS2 &  & 4 &  &  &  \\ 
9 & CCR &  & 1 &  &  &  \\ 
10 & CV & WSO-PAGPS-0105 & 2 &  &  &  \\ 
11 & EJ &  & 4 &  &  &  \\ 
12 & DBV & WSO-PAGPS-0108 & 1 &  &  &  \\ 
13 & DE &  & 1 &  &  &  \\ 
14 & PAD &  & 1 &  &  &  \\ 
15 & CS3 &  & 1 &  &  &  \\ 
\hline
\end{tabular}				
		}}
\end{table}


\begin{table}[h]
	\caption{Visual inspection data - SP1}
	\label{ch04_visualinspectionsp1}
	\resizebox{\columnwidth}{!}{%
		{\scriptsize
			\begin{tabular}{c|p{2.5cm}|c|c|c|p{6cm}|p{1cm}}
				\hline
				No. & Assets/ Description & Code & CS & Status & Remarks & Refer/ Figure \\ 
\hline
1 & SBV & WSO-PAGPS-0021 & - &  & Below ground; not inspected &  \\ 
2 & FJ &  & - &  & Below ground; not inspected, &  \\ 
3 & EJ &  & 2 &  & Mitered elbows were used (not radius elbows) &  \\ 
4 & ECR &  & 1 &  &  &  \\ 
5 & FJ &  & 4 &  &  &  \\ 
6 & CCR &  & 1 &  &  &  \\ 
7 & CV & WSO-PAGPS-0026 & 2 &  &  &  \\ 
8 & DBV & WSO-PAGPS-0029 & 1 &  &  &  \\ 
9 & PIPE2 &  & 2 &  & Repair area where PG is tapped &  \\ 
10 & CS3 &  & 1 &  &  &  \\ 
11 & CS4 &  & 4 &  &  &  \\ 
12 & CS5 &  & 4 &  &  &  \\ 
\hline
\end{tabular}	
	}}
\end{table}


\begin{table}[h]
	\caption{Visual inspection data - SP2}
	\label{ch04_visualinspectionsp2}
	\resizebox{\columnwidth}{!}{%
		{\scriptsize
			\begin{tabular}{c|p{2.5cm}|c|c|c|p{6cm}|p{1cm}}
				\hline
				No. & Assets/ Description & Code & CS & Status & Remarks & Refer/ Figure \\ 
	\hline
	1 & SBV & WSO-PAGPS-0032 & - &  & Below ground; not inspected &  \\ 
	2 & FJ &  & - &  & Below ground; not inspected, &  \\ 
	3 & EJ &  & 2 &  & Mitered elbows were used (not radius elbows) &  \\ 
	4 & ECR &  & 1 &  &  &  \\ 
	5 & FJ &  & 4 &  &  &  \\ 
	6 & CCR &  & 1 &  &  &  \\ 
	7 & CV & WSO-PAGPS-0037 & 2 &  &  &  \\ 
	8 & DBV & WSO-PAGPS-0040 & 1 &  &  &  \\ 
	9 & PIPE2 &  & 1 &  &  &  \\ 
	10 & CS3 &  & 1 &  &  &  \\ 
	11 & CS4 &  & 4 &  &  &  \\ 
	12 & CS5 &  & 4 &  &  &  \\ 
	\hline
\end{tabular}
	}}
\end{table}






%Aside from the CSs observed as presented in the table, we also learn from the inspection that the entire pump house has been suffering from ground settlement and that the alignments of the pumps and the corresponding fittings have been compromised. As shown in Figure \ref{ch04_settlement}, the portion of wall through which the pipe passes thru has been torn down partially to provide allowance for settlement as the pipe levels down.  

%\begin{figure}[!htb]
%	\includegraphics[scale=1.5]{figures/ch04_settlement} \\
%	\caption{Wall opening provision for ground settling}
%	\label{ch04_settlement} 
%\end{figure}

%Damages to the concrete saddle supports appearing as cracks and crooked or completely drawn bolts have been observed as shown in Figure \ref{ch04_settlement01}. 
%
%\begin{figure}[!htb]
%	\includegraphics[scale=1.5]{figures/ch04_settlement01} \\
%	\caption{Support damages due to pipe movement and ground settling}
%	\label{ch04_settlement01} 
%\end{figure}
%
%Furthermore, gaps are also observed between the concrete saddle supports and the straight pipe beneath as shown in Figure \ref{ch04_settlement02}.  
%\begin{figure}[!htb]
%	\includegraphics[scale=1.5]{figures/ch04_settlement02} \\
%	\caption{Support damages due to pipe movement and ground settling}
%	\label{ch04_settlement02} 
%\end{figure}
%
%For the booster pumps, there are two saddle supports located near to each other and are positioned just outside the pump house after the suction elbows as shown in Figure \ref{ch04_piping_supports01}. For the storage pumps, the three saddles are located farther, one before the discharge elbow and   the other still outside the pump house and just after the check valve.
%
%\begin{figure}[!htb]
%	\includegraphics[scale=1.5]{figures/ch04_piping_supports01} \\
%	\caption{Support damages due to pipe movement and ground settling}
%	\label{ch04_piping_supports01} 
%\end{figure}



\section{Pump efficiency} \label{ch04mech03}

Data on flow and head were measured for each pump. However, there was no measured data of motor/pump assembly regarding power ratings of all pumps. 
This was due to the fact that electrical audit is not part of the scope of work. GHD/RBSanchez did verify on the provided electrical audit to see if 
there is available power rating for individual motor/pump assembly. However, the electrical audit does not include measured data for individual pump. 
Thus, cannot be used as a reference to more or less correlate with measured flow and head for computation of pump efficiency as desired. As such, as agreed with Maynilad Team, 
data from control panel specifically VFDs' current and voltage ratings will then be used to get the instantaneous power 
ratings of Pump and Motor during the course of operation.
Such values, on the other hand, might not be equivalent to the values of power during the time of testing thus might affect the result of the analysis.


\subsection{Unit flow measurement} \label{ch04mech04}

Data on measured flow (Q) was recorded with min and max values is shown in Table \ref{ch04_tbl_flow01} for each pumps. %Raw data is provided in the Appendix \ref{appflow}.



\begin{table}[!h]

	\caption{Unit flow measurement (cubic meter per second - cms).}

	\label{ch04_tbl_flow01}

	{\footnotesize

\begin{tabular}{c|c|c|c|c|c|l}

	\hline

	Assets & $\Phi$  & \multicolumn{4}{c|}{Flow Q (cms)} & Remarks \\ 

	\cline{3-6}

	& (mm) & Design & \multicolumn{3}{c|}{Measure} &  \\ 

	\cline{4-6}

	&  &  & Min & Max & Ave. &  \\ 

	\hline

	BP1 & 700 & 0.6365 & 0.3407 & 0.3533 & 0.3470  &  \\ 

	BP2 & 700 & 0.6365 & 0.3028 & 0.3343 & 0.3186  &  \\ 

	BP3 & 700 & 0.6365 & 0.5930 & 0.6119 & 0.6024  &  \\ 

	BP4 & 700 & 0.6365 & 0.5110 & 0.5614 & 0.5362  &  \\ 

	BP5 & 700 & 0.6365 & 0.6939 & 0.7128 & 0.7034  &  \\ 

	BP6 & 700 & 0.6365 & 0.6119 & 0.6561 & 0.6340  &  \\ 

	SP1 & 600 & 0.5324 & 0.3583 & 0.3659 & 0.3621  &  \\ 

	SP2 & 600 & 0.5324 & 0.3785 & 0.3848 & 0.3817  &  \\ 

	\hline

\end{tabular}

	}

\end{table}



\subsection{Pressure measurement} \label{ch04pressure}



Data on measured flow (Q) was recorded with min and max values is shown in Table \ref{ch04_tbl_flow02} for each pumps. %Raw data is provided in the Appendix \ref{appflow}.



\begin{table}[!h]

	\caption{Head ($mH_2O$).}

	\label{ch04_tbl_flow02}

	{\footnotesize

\begin{tabular}{c|c|c|c|l}

	\hline

	Assets & \multicolumn{3}{c|}{Head (H - $mH_2O$)} & Remarks \\ 

	\cline{2-4}

	& Design & Discharge & Suction &  \\ 

	\hline

	BP1 & 40 & 42.1581 & 2.4592 &  \\ 

	BP2 & 40 & 42.1581 & 2.7403 &  \\ 

	BP3 & 40 & 42.1581 & 2.4592 &  \\ 

	BP4 & 40 & 40.7529 & 2.7403 &  \\ 

	BP5 & 40 & 42.1581 & 2.9511 &  \\ 

	BP6 & 40 & 40.7529 & 2.8105 &  \\ 

	SP1 & 50 & 37.8088 & 0.7026 &  \\ 

	SP2 & 50 & 36.5371 & 0.7026 &  \\ 

	\hline

\end{tabular}

	}

\end{table}





\subsection{Efficiency}



Pump efficiency is computed based on the flow/head measurement and the assumed value of power rating (Table \ref{ch05_tbl_efficiency}). 

\begin{table}[!h]

	\caption{Pump efficiency (\%).}

	\label{ch05_tbl_efficiency}

	{\footnotesize


	\begin{tabular}{c|c|c|c|c|c|l|l}
\hline
Assets & Flow & Head & Input Power & Water Power & \multicolumn{3}{c}{Efficiency (\%)} \\ 
\cline{6-8}
 & $m^3/s$ & $mH_2O$ & kW & kW & Tested & Design & Diff. \\ 
\hline
BP1 & 0.347 & 41.68 & 279.41 & 142 & - & \multicolumn{1}{c|}{85.67} & \multicolumn{1}{c}{-} \\ 
BP2 & 0.319 & 41.4 & 280.47 & 129 & - & \multicolumn{1}{c|}{85.67} & \multicolumn{1}{c}{-} \\ 
BP3 & 0.602 & 41.68 & 325.70 & 246 & 75.53 & \multicolumn{1}{c|}{85.67} & \multicolumn{1}{c}{-10.14} \\ 
BP4 & 0.536 & 40 & 325.70 & 210 & 64.48 & \multicolumn{1}{c|}{85.67} & \multicolumn{1}{c}{-21.19} \\ 
BP5 & 0.703 & 41.19 & 295.00 & 242 & - & \multicolumn{1}{c|}{85.67} & \multicolumn{1}{c}{-} \\ 
BP6 & 0.634 & 39.93 & 293.87 & 248 & 84.39 & \multicolumn{1}{c|}{85.67} & \multicolumn{1}{c}{-1.28} \\ 
SP1 & 0.362 & 45.33 & 214.24 & 161 & 75.15 & \multicolumn{1}{c|}{86.89} & \multicolumn{1}{c}{-11.74} \\ 
SP2 & 0.382 & 44.06 & 209.15 & 165 & 78.89 & \multicolumn{1}{c|}{86.89} & \multicolumn{1}{c}{-8.00} \\ 
\hline
\end{tabular}

	}

\end{table}



It is important to note that the values of input power is a measured value, however on a separate day and so different working condition thus might not perfectly reflect 
the actual value during the data gathering which is one of the limitations of the study.
The design pump efficiencies are taken from the test report of
KSB \cite{KSB2010}. 



In Table \ref{ch05_tbl_efficiency}, efficiencies for BP1, BP2 and BP5 were not estimated due to the fact that BP1 and BP2 have been operated in an underflow 
conditions while BP5 in an overflow condition.

These infer that the pumps have been operated inappropriately pursuant to their design specifications. As a consequence, their operations might have already incurred more power 
than they should be. 
If operated continuously on the same manner,high likely that the failure probability will increase for these pumps.

Figure \ref{ch04_efficiencycurves}-a, and \ref{ch04_efficiencycurves}-b presents the efficiency curves for booster pumps and storage pumps, respectively. 
The curves are created based on the recorded data provided in the test record of KSB \cite{KSB2010}. 

\begin{figure}[!htb]

	\begin{minipage}[b]{0.5\linewidth}

		\centering

		\includegraphics[width=\textwidth]{figures/ch04_fig_efficiency01}

		\caption*{a - Booster pumps}% \label{peter1}

	\end{minipage}

	\hspace{0.05cm}

	\begin{minipage}[b]{0.5\linewidth}

		\centering

		\includegraphics[width=\textwidth]{figures/ch04_fig_efficiency02}

		\caption*{b -Storage Pumps} %\label{peter2}

	\end{minipage}

		\caption{Efficiency curves}

		\label{ch04_efficiencycurves}

\end{figure}



Figure \ref{ch04_efficiencycurves}-a shows that BP1 and BP2 deviates away from the curve. In this case, BP1 and BP2 possibly operates at underflow condition, 
and at higher friction loss. These also constitute to the “fair” condition of the pumps with possible cavitation at the suction head. On the other hand, BP3, BP5, and BP6 operate under the 10% tolerance BEP range, thereby
operating at considerably good operating point although BP5 has overflow-measured flow. On the contrary, these pumps were diagnosed to have a “fair” pump condition state. 
This is due to pump possible cavitation from the suction and carried throughout the discharged side of the pump.
BP4 deviates slightly away to the curve with its operating point barely making the 20% tolerance BEP range. The pump operates at
lower head and possibly in underflow condition where its efficiency is lowest compared to BP3 and BP6. On a positive note, the pump have a diagnosis of having a good
pump health that means lower vibration and possible cavitation.

Figure \ref{ch04_efficiencycurves}-b shows that the operating points of both SP1 and SP2 deviate away from the pump curve. These might means that both pumps  
operate at higher friction head and either overflow and underflow. Operating points of both pumps are within the 20\% tolerance range (shaded region) 
based on the best GPM point with a corresponding increase in pressure head. In terms of efficiency, SP1 and SP2  have a value of 75.15\% and 78.89\%, respectively even 
operating at deviated operation conditions, which is a good indication. In the long run however, it would affect the system performance which might be 
the reason also why their initial diagnosis are both "fair" in terms of vibration analysis.

Generally, VFD controlled-pumps deviates greatly away to the curve than the fixed-speed pumps (BP3 to BP6). This could possibly means that the operating duty points 
of the pumps when controlled by VFD is out of the best efficiency range of the pumps, thereby may incur higher power consumptions. 


\paragraph{\underline{Recommendations}}


\begin{itemize}

\item	Proper operating condition shall be established for the VFD Pumps in a given time to avoid them operating beyond their BEP for a long period of time. 
Also include the proper combinations of running pumps in a given time.

\item	Regular pump performance monitoring and analysis shall be conducted to have a profiling of pump in relation to actual operating conditions, a major tool for pump performance
 optimization. In addition, multiple tests and conditions shall be implemented during testing to have a holistic approach on the pump assessment.

\item	Provision for installation of testing points in measuring pump performance shall also be included in the modifications  (refer to the conceptual design in Chapter \ref{Chapter6}).

\end{itemize}




%\section{Electrical Audit}\label{43}
%\subsection{Visual inspection} \label{ch04elec01}
%\textcolor{blue}{APSI to write here the summary of raw data collected from visual inspection and testing. Tables shall be used as much as we can. Note that no analysis in this session. This session is purely the high level presentation of data. Raw data can be linked as an Appendix}
%\subsection{Short circuit calculation} \label{ch04elec02}
%\textcolor{blue}{APSI to write here the summary of raw data collected from visual inspection and testing. Tables shall be used as much as we can. Note that no analysis in this session. This session is purely the high level presentation of data. Raw data can be linked as an Appendix}
%\subsection{Voltage drop calculation} \label{ch04elec03}
%\textcolor{blue}{APSI to write here the summary of raw data collected from visual inspection and testing. Tables shall be used as much as we can. Note that no analysis in this session. This session is purely the high level presentation of data. Raw data can be linked as an Appendix}
%\subsection{Protection coordination study} \label{ch04elec04}
%\textcolor{blue}{APSI to write here the summary of raw data collected from visual inspection and testing. Tables shall be used as much as we can. Note that no analysis in this session. This session is purely the high level presentation of data. Raw data can be linked as an Appendix}
%\subsection{Harmonic analysis} \label{ch04elec05}
%\textcolor{blue}{APSI to write here the summary of raw data collected from visual inspection and testing. Tables shall be used as much as we can. Note that no analysis in this session. This session is purely the high level presentation of data. Raw data can be linked as an Appendix}
%\subsection{Power quality} \label{ch04elec06}
%\textcolor{blue}{APSI to write here the summary of raw data collected from visual inspection and testing. Tables shall be used as much as we can. Note that no analysis in this session. This session is purely the high level presentation of data. Raw data can be linked as an Appendix}
%\subsection{Grounding system} \label{ch04elec07}
%\textcolor{blue}{APSI to write here the summary of raw data collected from visual inspection and testing. Tables shall be used as much as we can. Note that no analysis in this session. This session is purely the high level presentation of data. Raw data can be linked as an Appendix}
%\subsection{Asset registry} \label{ch04elec08}
%\textcolor{blue}{APSI to write here the summary of raw data collected from visual inspection and testing. Tables shall be used as much as we can. Note that no analysis in this session. This session is purely the high level presentation of data. Raw data can be linked as an Appendix}
%\subsection{Electrical system design analysis}\label{ch04elec09}
%\textcolor{blue}{APSI to write here the summary of raw data collected from visual inspection and testing. Tables shall be used as much as we can. Note that no analysis in this session. This session is purely the high level presentation of data. Raw data can be linked as an Appendix}
%
%\subsection{Electrical integrity system} \label{ch04elec11}
%\textcolor{blue}{APSI to write here the summary of raw data collected from visual inspection and testing. Tables shall be used as much as we can. Note that no analysis in this session. This session is purely the high level presentation of data. Raw data can be linked as an Appendix}
%\subsection{Outdoor electrical equipment} \label{ch04elec12}
%\textcolor{blue}{APSI to write here the summary of raw data collected from visual inspection and testing. Tables shall be used as much as we can. Note that no analysis in this session. This session is purely the high level presentation of data. Raw data can be linked as an Appendix}

\section{Fire protection and safety (FDAS) audit} \label{ch04fdas}

\subsection{Fire alarm and detection system} \label{ch04fdas01}
Summary of data and information from FDAS audit is presented in Table \ref{ch04_fdas01} with visual images on as-found devices and panels (Figure \ref{ch04_fig_fdas01}).
\begin{table}[!h]
	\caption{FDAS data highlights.}
	\label{ch04_fdas01}
	{\footnotesize
	\begin{tabular}{c|p{5cm}|c|c|c|p{4cm}|p{1cm}}
		\hline
		No. & Assets & Code & Status & CS & Remarks & Refer/ Figure \\ 
		\hline
		A & Visual check of the fire alarm control panel &  &  &  &  &  \\ 
		1 & Panel Status, installed and location area &  & 1 &  & Power up, located near Engineering office with light indicator &  \\ 
		2 & Power indicator lamp operational &  & 1 &  & Operational &  \\ 
		3 & Devices properly indicated and marked &  & 1 &  & Device installed \@ motor control room and 2nd floor, 3 meters in height &  \\ 
		4 & Panel clear from trouble indicators &  & 0 &  & For verification of fault &  \\ 
		5 & Lamp test indicator operational &  & 0 &  & For verification of fault &  \\ 
		6 & Zones properly indicated and marked &  & 1 &  & Conventional system &  \\ 
		7 & Check if it connected to sprinkler system &  & 0 &  & For verification on testing &  \\ 
		\hline
		B & Checking of installed devices &  &  &  &  &  \\ 
		1 & Check floor plan lay-out and location of the device if accessible/easy to access &  & 1 &  & No FDAS lay-out but devices are accessible &  \\ 
		2 & Heat detectors and / or smoke detectors indicator lamp functioning &  & 1 &  & For verification &  \\ 
		3 & Heat detectors and / or smoke detectors indicator lamp functioning &  & 0 &  & Not all indicator lamp \@ the device are functioning &  \\ 
		4 & Pull station locations acceptable &  & 0 &  & For Verification &  \\ 
		5 & Bells and buzzers operated correctly &  & 0 &  & For verification &  \\ 
		6 & Bells and buzzers audibility &  & 0 &  & For verification &  \\ 
		7 & Strobe lights locations are acceptable &  & 0 &  & Lacking devices at rooms &  \\ 
		8 & Strobe light operated correctly &  & 0 &  & For verification during testing &  \\ 
		9 & Are Fire alarm zones (areas) clearly marked &  & 0 &  &  &  \\ 
		10 & Is there a maintenance and service contract for the fire alarm system &  & 1 &  & None as per interview with the maintenance &  \\ 
		& Does the Fire Alarm System smoke detector, heat detector, manual call point , horn and strobe light working and  have a current inspection tag &  & 0 &  & None &  \\ 
		& Is the fire alarm system if full working order &  & 0 &  & For verification &  \\ 
		\hline
	\end{tabular}
		
	}
\end{table}

\begin{figure}[!htb]
	\includegraphics[scale=1.7]{figures/ch04_fig_fdas01} \\
	\caption{As-found devices and panels}
	\label{ch04_fig_fdas01} 
\end{figure}


\subsection{Lighting protection system} \label{ch04fdas02}
No lightning protection was installed for this PS.
\subsection{Ground-Fault circuit interrupter (GFCI) or electric leakage circuit breaker (ELCB) or Residual circuit devices (RCD)} \label{ch04fdas03}
No ground fault circuit interrupter (GFCI) or earth leakage Circuit breaker (ELCB) protection was installed in the panel for FDAS for this PS.
\subsection{Electrical safety and protective devices} \label{ch04fdas04}

\begin{table}[!h]
	\caption{FDAS data highlights.}
	\label{ch04_fdas01}
	{\footnotesize
		\begin{tabular}{c|p{5cm}|c|c|c|p{4cm}|p{1cm}}
\hline
No. & Assets/Description & Code & Status & CS & Remarks & Refer/Figure \\ 
\hline
1 & Evacuation plan &  & 1 &  &  &  \\ 
2 & Fire extinguishers &  & 1 &  & Green (HCFC)  &  \\ 
&  &  &  &  & FEX signage without physical Fire Extingushers  &  \\ 
&  &  &  &  & As per plan, there should be 6 FEX &  \\ 
3 & Fire exits &  & 1 &  & With exit indication &  \\ 
4 & Fire hose cabine &  & 1 &  & With inspection tag 8/30/18 &  \\ 
5 & Fire sprinkler system &  & 1 &  & No sprinkler system &  \\ 
6 & Emergency exit signages &  & 1 &  & BIGLITE Brand &  \\ 
&  &  &  &  & Exit signage do no have power supply &  \\ 
7 & Emergency lights &  & 1 &  & Firefly brand. Last Inspection 8/30/18 &  \\ 
&  &  &  &  & All emergency lights are with light indicator &  \\ 
8 & PPE cabinet &  & 1 &  & with latest tag inspection as of 8/30/18 &  \\ 
\hline
\end{tabular}	
	}
\end{table}

\begin{figure}[h]
	\begin{minipage}[b]{0.5\linewidth}
		\centering
		\includegraphics[width=\textwidth]{figures/ch04_fig_safety01}
		\caption*{(a - 1st floor)}
		%		\label{ch02_fdas01}
	\end{minipage}
	\hspace{0.05cm}
	\begin{minipage}[b]{0.5\linewidth}
		\centering
		\includegraphics[width=\textwidth]{figures/ch04_fig_safety02}
		\caption*{(b -2nd floor)}
		%		\label{ch02_fdas02}
	\end{minipage}
	\caption{Existing evacuation plan}
	\label{ch04_fig_safety01}
\end{figure}


\begin{figure}[h]

	\begin{minipage}[b]{0.22\linewidth}
		\centering
		\includegraphics[width=\textwidth]{figures/ch04_fig_safety03}
		\caption*{(a - Emergency light)}
		%	\label{ch02_fdas03}
	\end{minipage}
	\hspace{0.03cm}
\begin{minipage}[b]{0.22\linewidth}
	\centering
	\includegraphics[width=\textwidth]{figures/ch04_fig_safety04}
	\caption*{(b-Exit door)}
	%	\label{ch02_fdas03}
\end{minipage}
	\hspace{0.03cm}
\begin{minipage}[b]{0.22\linewidth}
	\centering
	\includegraphics[width=\textwidth]{figures/ch04_fig_safety05}
	\caption*{(c- Cabinet)}
	%	\label{ch02_fdas03}
\end{minipage}
	\hspace{0.03cm}
\begin{minipage}[b]{0.22\linewidth}
	\centering
	\includegraphics[width=\textwidth]{figures/ch04_fig_safety06}
	\caption*{(d - Fire hose)}
	%	\label{ch02_fdas03}
\end{minipage}
	\hspace{0.03cm}
\begin{minipage}[b]{0.22\linewidth}
	\centering
	\includegraphics[width=\textwidth]{figures/ch04_fig_safety07}
	\caption*{(e - HFCF FEX)}
	%	\label{ch02_fdas03}
\end{minipage}
	\hspace{0.03cm}
\begin{minipage}[b]{0.22\linewidth}
	\centering
	\includegraphics[width=\textwidth]{figures/ch04_fig_safety08}
	\caption*{(f-Dry chemical FEX)}
	%	\label{ch02_fdas03}
\end{minipage}
	\hspace{0.03cm}
\begin{minipage}[b]{0.5\linewidth}
	\centering
	\includegraphics[width=\textwidth]{figures/ch04_fig_safety09}
	\caption*{(g - Exit signage)}
	%	\label{ch02_fdas03}
\end{minipage}
	\caption{Existing safety devices}
	\label{ch04_fig_safety02}
\end{figure}



\section{Vibration and structural assessment}
\label{45}
Raw data of vibration measurement is provided in separately digital format. The raw data of each pump is used to generate a set of graphs provided in Appendix \ref{app_vibrationdata}.

\section{Energy management audit}
\label{46}

\subsection{Production and power data}
Production data for this station has been recorded in excel files. Each file represents a month with 24 hours of daily records. Maynilad provided this set of data from 2012 to 2018 per GHD' request. Initial verification on this set was conducted with following conclusions

\begin{itemize}
	\item Data of 2012 and 2013 is not useable due to its incompatibility to the later data. Fundamentally, we found a great number of errors on this data. In addition, the data itself is incomplete and only reflects aggregate value, which makes impossible to compare with the later set;
	\item The structure of data is not homogeneous with many numerical errors. This problem is due to the fact that excel file is not suitable for recording a large volume of data, particularly cells are not set up to reject string and value outside the lower and upper bounds.
	content.
\end{itemize}

When excluding the data of 2012 and 2013, the set used for compilation has following statistics

In order to compile such a huge data set, it is not possible with manual inputting, instead, GHD has developed a hybrid program consisting of Visual Basic (VBA) Code and MySQL code for fast compilation. VBA code is used to add header, fill up missing information in excel file, and ignore rows and columns that should not exist with regard to database structure. MySQL codes are used to eliminate measurement errors and bring together all individual files to one file that allows statistical analysis with R.
\subsection{Measurement errors}
Following measurement errors are with the provided excel files
\begin{itemize}
\item String/text values are found numerous in columns that shall be only numerical values;
\item Extreme values are found numerous;
\item Negative values are found in many places that shall only be positive
\end{itemize}
%\subsection{Summary of statistics}
\subsection{Data compilation for analysis}
Out of all recorded attributes, useful attributes that can be used for energy audit are total production per hour and total power consumption per hour. There is no record on production and power consumption for individual pump.

After data filtering, data correction, and compilation, the obtained set of data includes 27,625 records. Final data set is saved in MySQL server.

\section{Workplace environment management}
\label{47}

\subsection{Temperature and relative humidity}
Data concerning the temperature and relative humidity is presented in Table \ref{ch04_tbl_wem01}. Data was measured at targeted points shown in Figure \ref{ch02_wem01}. Raw data is with the site inspection reports, which will be provided to the Client separately. Persuant to ASHRAE standard, the recommended ranges for temperature and humidity are [72 - 80 $^\circ F$] and [45 - 60 \%], respectively.

\begin{table}[!h]
	\caption{Temperature and relative humidity.}
	\label{ch04_tbl_wem01}
	{\footnotesize
\begin{tabular}{c|l|c|c|c|c}
	\hline
	Points & Description of points & \multicolumn{2}{c|}{Temperature ($^\circ F$)} & \multicolumn{2}{c}{Humidity (\%)} \\ 
	\cline{3-6}
	&  & Actual & Range & Actual & Range \\ 
	\hline
	& A. Inside pump house (outside the office) &  &  &  &  \\ 
	1 & Between BP5 and BP6 away from pump & 91.04 & 72 - 80 & 59.80 & 45 - 60 \\ 
	2 & Between BP5 and BP6 & 95.72 & 72 - 80 & 53.30 & 45 - 60 \\ 
	3 & Between BP5 and BP4 away from pump & 92.12 & 72 - 80 & 53.40 & 45 - 60 \\ 
	4 & Between BP5 and BP4 & 93.38 & 72 - 80 & 52.70 & 45 - 60 \\ 
	5 & Between BP4 and BP3 away from pump & 93.20 & 72 - 80 & 56.30 & 45 - 60 \\ 
	6 & Between BP4 and BP3 & 96.26 & 72 - 80 & 52.30 & 45 - 60 \\ 
	7 & Between BP3 and BP2 away from pump & 94.28 & 72 - 80 & 53.30 & 45 - 60 \\ 
	8 & Between BP3 and BP2 & 95.00 & 72 - 80 & 52.40 & 45 - 60 \\ 
	9 & Between BP2 and BP1 away from pump & 93.74 & 72 - 80 & 56.90 & 45 - 60 \\ 
	10 & Between BP2 and BP1 & 96.08 & 72 - 80 & 53.40 & 45 - 60 \\ 
	11 & Between SP2 and BP1 away from pump & 93.38 & 72 - 80 & 53.90 & 45 - 60 \\ 
	12 & Between SP2 and BP1 & 96.44 & 72 - 80 & 52.40 & 45 - 60 \\ 
	13 & Between SP2 and SP1 away from pump & 93.56 & 72 - 80 & 56.90 & 45 - 60 \\ 
	14 & Between SP2 and SP1 & 94.10 & 72 - 80 & 56.70 & 45 - 60 \\ 
	& Average & 94.16 &  & 54.55 &  \\ 
	\hline
	& B. Outside pump house &  &  &  &  \\ 
	15 & Near Reservoir & 84.44 & 72 - 80 & 64.73 & 45 - 60 \\ 
	16 & Back of the Office & 89.42 & 72 - 80 & 58.97 & 45 - 60 \\ 
	17 & Near Guard House & 88.46 & 72 - 80 & 59.40 & 45 - 60 \\ 
	18 & Near Diesel Tank & 86.66 & 72 - 80 & 62.33 & 45 - 60 \\ 
	& Average & 87.25 &  & 61.36 &  \\ 
	\hline
	& C. Vicinity &  &  &  &  \\ 
	19 & Near Diesel Tank, outside vicinity & 91.28 & 72 - 80 & 56.57 & 45 - 60 \\ 
	20 & Near Guard House, outside vicinity & 90.56 & 72 - 80 & 56.47 & 45 - 60 \\ 
	& Average & 90.92 &  & 56.52 &  \\ 
	\hline
	21 & D. Office & 82.16 & 72 - 80 & 48.80 & 45 - 60 \\ 
	\hline
\end{tabular}

	}
\end{table}




\subsection{Air quality}\label{aq01}

Data concerning the air quality is presented in Table \ref{ch04_tbl_wem03} with value of PM2.5 measured in ppm. Data was measured at targeted points shown in Figure \ref{ch02_wem01}. Raw data is with the site inspection reports, which will be provided to the Client separately. Persuant to XXX standard, the recommended safe ranges for PM2.5 is in [0-35].


\begin{table}[!h]
	\caption{Air quality - PM2.5 (ppm).}
	\label{ch04_tbl_wem03}
	{\footnotesize
\begin{tabular}{c|l|c}
	\hline
	Point & Description of the Point Location & PM2.5 \\ 
	\hline
	& A. Inside pump house (outside the office) &  \\ 
	1 & Between BP5 and BP6 away from pump & 16.00 \\ 
	2 & Between BP5 and BP6 & 16.00 \\ 
	3 & Between BP5 and BP4 away from pump & 12.00 \\ 
	4 & Between BP5 and BP4 & 15.00 \\ 
	5 & Between BP4 and BP3 away from pump & 12.00 \\ 
	6 & Between BP4 and BP3 & 15.00 \\ 
	7 & Between BP3 and BP2 away from pump & 13.00 \\ 
	8 & Between BP3 and BP2 & 15.00 \\ 
	9 & Between BP2 and BP1 away from pump & 14.00 \\ 
	10 & Between BP2 and BP1 & 14.00 \\ 
	11 & Between SP2 and BP1 away from pump & 12.00 \\ 
	12 & Between SP2 and BP1 & 14.00 \\ 
	13 & Between SP2 and SP1 away from pump & 15.00 \\ 
	14 & Between SP2 and SP1 & 14.00 \\ 
	& Average & 14.07 \\ 
	\hline
	& B. Outside pump house &  \\ 
	15 & Near Reservoir & 11.00 \\ 
	16 & Back of the Office & 12.00 \\ 
	17 & Near Guard House & 12.00 \\ 
	18 & Near Diesel Tank & 13.00 \\ 
	& Average & 12.00 \\ 
	\hline
	& C. Vicinity &  \\ 
	19 & Near Diesel Tank, outside vicinity & 14.00 \\ 
	20 & Near Guard House, outside vicinity & 14.00 \\ 
	& Average & 14.00 \\ 
	\hline
	21 & Office & 11.00 \\ 
	\hline
\end{tabular}
	}
\end{table}

%\subsection{Hazards}\label{aq02}
%\textcolor{red}{RB Sanchez to write here the summary of raw data collected from visual inspection and testing. Tables shall be used as much as we can. Note that no analysis in this session. This session is purely the high level presentation of data. Raw data can be linked as an Appendix}
\subsection{Illumination}\label{aq03}

Data concerning the illumination is presented in Table \ref{ch04_tbl_wem04} with the LUX value. Data was measured at targeted points shown in Figure \ref{ch02_wem01}. Raw data is with the site inspection reports, which will be provided to the Client separately. Persuant to RULE 1075.4 of DOLE-OSH standard \cite{DOLE2016}, the recommended minimum for LUX is in 100.


\begin{table}[!h]
	\caption{Illumination (x 100 LUX).}
	\label{ch04_tbl_wem04}
	{\footnotesize
\begin{tabular}{c|l|c|c|c|c}
	\hline
	Point & Description of the points & \multicolumn{3}{c|}{Trials} & Ave. \\ 
	\cline{3-5}
	&  & 1 & 2 & 3 &  \\ 
	\hline
	& A. Inside pump house (outside the office) &  &  &  &  \\ 
	1 & Between BP5 and BP6 away from pump & 15.02 & 12.67 & 15.50 & 14.40 \\ 
	2 & Between BP5 and BP6 & 13.89 & 12.64 & 11.74 & 12.76 \\ 
	3 & Between BP5 and BP4 away from pump & 8.86 & 9.37 & 9.21 & 9.15 \\ 
	4 & Between BP5 and BP4 & 10.79 & 10.50 & 10.40 & 10.56 \\ 
	5 & Between BP4 and BP3 away from pump & 9.43 & 10.21 & 10.09 & 9.91 \\ 
	6 & Between BP4 and BP3 & 10.45 & 10.50 & 9.84 & 10.26 \\ 
	7 & Between BP3 and BP2 away from pump & 10.73 & 10.62 & 9.96 & 10.44 \\ 
	8 & Between BP3 and BP2 & 11.15 & 10.50 & 10.37 & 10.67 \\ 
	9 & Between BP2 and BP1 away from pump & 9.48 & 9.60 & 9.76 & 9.61 \\ 
	10 & Between BP2 and BP1 & 10.52 & 11.13 & 10.51 & 10.72 \\ 
	11 & Between SP2 and BP1 away from pump & 8.97 & 7.98 & 8.01 & 8.32 \\ 
	12 & Between SP2 and BP1 & 10.57 & 10.78 & 10.41 & 10.59 \\ 
	13 & Between SP2 and SP1 away from pump & 6.43 & 5.43 & 5.83 & 5.90 \\ 
	14 & Between SP2 and SP1 & 5.66 & 5.44 & 5.34 & 5.48 \\ 
	& Average & 10.14 & 9.81 & 9.78 & 9.91 \\ 
	\hline
	& B. Outside pump house &  &  &  &  \\ 
	15 & Near Reservoir & 135.70 & 154.10 & 160.20 & 150.00 \\ 
	16 & Back of the Office & 259.00 & 263.00 & 264.00 & 262.00 \\ 
	17 & Near Guard House & 170.60 & 164.40 & 163.90 & 166.30 \\ 
	18 & Near Diesel Tank & 145.70 & 132.70 & 124.00 & 134.13 \\ 
	& Average & 177.75 & 178.55 & 178.03 & 178.11 \\ 
	\hline
	& C. Vicinity &  &  &  &  \\ 
	19 & Near Diesel Tank, outside vicinity & 268.00 & 260.00 & 274.00 & 267.33 \\ 
	20 & Near Guard House, outside vicinity & 259.00 & 269.00 & 263.00 & 263.67 \\ 
	& Average & 263.50 & 264.50 & 268.50 & 265.50 \\ 
	21 & D. Office & 321.00 & 326.00 & 339.00 & 328.67 \\ 
	\hline
\end{tabular}

	}
\end{table}

\subsection{Industrial ventilation}\label{aq04}
There are two ways of ventilation available, natural and mechanical. Natural ventilation is possible through the entrance door in front of Guard House (Door 1) and the other one beside the office (Door 2) which are both always open. Other door, situated at the farthest side of SP1 (Door 3) can also be used as a means of natural ventilation but it is normally closed and not utilized (Figure \ref{ch04_fig_ventilation01}). Thus, mechanical ventilation is the main option as the plant installed supply and exhaust fan for the pump house running most of the time, specifically during hot dry days and when there are on-going maintenance activities. An air conditioned office is also available for the operator monitoring the plant daily operation.

\begin{figure}[!htb]
	\includegraphics[scale=2]{figures/ch04_fig_ventilation01} \\
	\caption{Existing ventilation layout}
	\label{ch04_fig_ventilation01} 
\end{figure}

\subsection{Housekeeping}\label{aq05}
The plant has its own waste segregation policy and an organized documentations procedure (evidenced of the arranged daily monitoring sheet). Standby generator set is situated outside the pump house where its exhaust gases through natural ventilation will not be able to penetrate the pump house (See Appendix XX1). 


\subsection{Noise}\label{aq06}
Data concerning the noise is presented in Table \ref{ch04_tbl_wem02}. Data was measured at targeted points shown in Figure \ref{ch02_wem01}. Raw data is with the site inspection reports, which will be provided to the Client separately. Persuant to 

\begin{table}[!h]
	\caption{Noise (dBA)}
	\label{ch04_tbl_wem02}
	\resizebox{\columnwidth}{!}{%
	{\scriptsize
\begin{tabular}{c|p{4cm}|ccc|ccc|ccc|c}
	\hline
	Point & Description of the Point Location & \multicolumn{10}{c}{Trials} \\ 
	\cline{3-12}
	&  & \multicolumn{3}{c|}{1} & \multicolumn{3}{c|}{2} & \multicolumn{3}{c|}{3} &  \\ 
	\cline{3-11}
	&  & Min & Ave. & max & Min & Ave. & Max & Min & Ave. & Max & Ave. \\ 
	\hline
	& A. Inside pump house (outside the office) &  &  &  &  &  &  &  &  &  &  \\ 
	1 & Between BP5 and BP6 away from pump & 91.30 & 94.40 & 97.50 & 91.50 & 93.55 & 95.60 & 91.30 & 93.15 & 95.00 & 93.70 \\ 
	2 & Between BP5 and BP6 & 90.00 & 92.10 & 94.20 & 90.50 & 92.85 & 95.20 & 90.20 & 92.80 & 95.40 & 92.58 \\ 
	3 & Between BP5 and BP4 away from pump & 91.30 & 93.30 & 95.30 & 91.90 & 93.60 & 95.30 & 90.50 & 92.45 & 94.40 & 93.12 \\ 
	4 & Between BP5 and BP4 & 90.60 & 93.00 & 95.40 & 90.80 & 93.25 & 95.70 & 90.60 & 93.00 & 95.40 & 93.08 \\ 
	5 & Between BP4 and BP3 away from pump & 91.60 & 93.25 & 94.90 & 91.80 & 93.55 & 95.30 & 91.30 & 93.30 & 95.30 & 93.37 \\ 
	6 & Between BP4 and BP3 & 91.10 & 93.20 & 95.30 & 91.50 & 93.35 & 95.20 & 91.90 & 93.50 & 95.10 & 93.35 \\ 
	7 & Between BP3 and BP2 away from pump & 92.20 & 93.90 & 95.60 & 92.50 & 93.90 & 95.30 & 92.10 & 94.05 & 96.00 & 93.95 \\ 
	8 & Between BP3 and BP2 & 91.20 & 92.60 & 94.00 & 90.70 & 92.35 & 94.00 & 90.80 & 92.40 & 94.00 & 92.45 \\ 
	9 & Between BP2 and BP1 away from pump & 91.70 & 93.65 & 95.60 & 91.70 & 93.45 & 95.20 & 92.10 & 93.45 & 94.80 & 93.52 \\ 
	10 & Between BP2 and BP1 & 90.40 & 92.70 & 95.00 & 90.20 & 92.55 & 94.90 & 90.90 & 93.00 & 95.10 & 92.75 \\ 
	11 & Between SP2 and BP1 away from pump & 93.50 & 94.85 & 96.20 & 92.40 & 94.55 & 96.70 & 92.00 & 94.10 & 96.20 & 94.50 \\ 
	12 & Between SP2 and BP1 & 90.70 & 92.95 & 95.20 & 90.60 & 92.85 & 95.10 & 90.60 & 92.70 & 94.80 & 92.83 \\ 
	13 & Between SP2 and SP1 away from pump & 92.10 & 94.30 & 96.50 & 91.90 & 93.85 & 95.80 & 92.20 & 94.00 & 95.80 & 94.05 \\ 
	14 & Between SP2 and SP1 & 91.20 & 93.35 & 95.50 & 90.80 & 92.95 & 95.10 & 90.80 & 91.95 & 93.10 & 92.75 \\ 
	& Average &  & 93.40 &  &  & 93.33 &  &  & 93.13 &  & 93.29 \\ 
	\hline
	& B. Outside pump house &  &  &  &  &  &  &  &  &  &  \\ 
	15 & Near Reservoir & 64.70 & 71.70 & 78.70 & 64.20 & 65.80 & 67.40 & 64.20 & 65.60 & 67.00 & 67.70 \\ 
	16 & Back of the Office & 71.50 & 77.25 & 83.00 & 71.40 & 74.90 & 78.40 & 71.80 & 72.60 & 73.40 & 74.92 \\ 
	17 & Near Guard House & 65.40 & 66.30 & 67.20 & 64.60 & 65.65 & 66.70 & 64.80 & 66.35 & 67.90 & 66.10 \\ 
	18 & Near Diesel Tank & 65.10 & 72.70 & 80.30 & 65.70 & 69.75 & 73.80 & 65.40 & 69.35 & 73.30 & 70.60 \\ 
	& Average &  & 71.99 &  &  & 69.03 &  &  & 68.48 &  & 69.83 \\ 
	\hline
	& C. Vicinity &  &  &  &  &  &  &  &  &  &  \\ 
	19 & Near Diesel Tank, outside vicinity & 65.70 & 69.50 & 73.30 & 66.00 & 68.35 & 70.70 & 65.50 & 72.35 & 79.20 & 70.07 \\ 
	20 & Near Guard House, outside vicinity & 66.80 & 71.75 & 76.70 & 67.00 & 71.75 & 71.00 & 66.90 & 71.75 & 76.00 & 71.07 \\ 
	& Average &  & 70.63 &  &  & 70.05 &  &  & 72.05 &  & 70.91 \\ 
	\hline
	21 & Office & 66.70 & 68.10 & 69.50 & 68.00 & 68.40 & 68.80 & 68.60 & 69.35 & 70.10 & 68.62 \\ 
	\hline
\end{tabular}
	}}
\end{table}




