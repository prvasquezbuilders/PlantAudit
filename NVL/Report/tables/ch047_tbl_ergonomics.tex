\begin{table}[!h]
	\caption{Ergonomics.}
	\label{ch047_tbl_ergonomics}
	{\footnotesize


\begin{tabular}{p{1.5cm}|p{1.5cm}|c|p{8.5cm}}
\hline
Parameters & Sub-parameters & Status & Remarks \\ 
\hline
Posture & Head & 1 & Ceiling height is high enough to cause head injury while sitting or when standing. \\ 
 & Neck & 1 & Neck posture is in good ergonomic condition. \\ 
 &  &  & Consider having an interval for fit-break to avoid neck muscles stiffening. \\ 
 & Back & 1 & Back posture while sitting is in good posture.  \\ 
 &  &  & Consider standing and doing fit-break exercises to relax spine.  \\ 
 & Hands/Wrist  & 0 & Proper hand positioning in the keyboard is not observed. \\ 
 &  &  & Wrist bending is seldom. \\ 
 & Feet & 1 & Feet position is in good posture. \\ 
 &  &  & Good clearance below worktables. \\ 
 & Eyes & 0 & The computer monitor is on eyelevel in a certain operator only. \\ 
 &  &  & Consider adjusting the monitor level comfortable to every operator. \\ 
 &  &  & Look away into distance in order to rest the eyes for every 10 minutes or so. \\ 
\hline
Equipment / Tool &  &  &  \\ 
 & Computer display & 0 & Not adjusted and the operator get used to its current setting. \\ 
 &  &  & Display brightness must be adjustable in the comfortability of the operator-in-charge. \\ 
 &  &  & Consider the use of anti-glare and blue light to reduce the possibility of eyestrain. \\ 
 & Keyboard & 1 & Keyboard position causes poor hand posture that can lead to injury at long exposure. \\ 
 & Mouse & 1 & Mouse usage is average due to monitoring. \\ 
 &  &  & Prolong usage may cause reduced blood flow leading to muscular injury. \\ 
 & Chair & 0 & Consider using ergonomic chair that is capable of back support, height, armrest adjustments. \\ 
 & Table & 0 & Consider use of ergonomic tables to adjust the height of the table in desired position easily without exerting much effort to adjust manually. \\ 
 & Files & 1 & Hard copy file system and location is well observed. Too high or too low file location may require a person to bend his body or force his hand to grip a file in an awkward posture. Frequent situation may lead to MSD. \\ 
\hline
Operations / Maintenance & Illumination & 0 & According to the maintenance team, the motion-activated light is not bright enough to complete their task efficiently at night. Moreover, the light has short on-off delay operation that means that the team must move more often to avoid the light to dim.  \\ 
 &  &  & Consider having a manual switch option to by-pass the motion sensors and le the light on while doing maintenance.  \\ 
 & Noise Exposure & 1 & Noise emitted by the machines in the pump station is high. Consider the use of proper ear protections to reduce the sound intensity. In offices, the sound intensity is acceptable.  \\ 
 & Temperature & 1 & Temperature in the pump station is not acceptable at long exposure. Consider cooling down the body temperature at the designated area (i.e. outside, office). \\ 
\hline
Facility / General Workplace & Layout & 1 & Layout of the pump station is well observed. Distance between pumps is acceptable for well maintenance movement.  \\ 
 & Height clearances & 1 & Height clearances from ceiling to head is very high. Chance of getting head injury is very low. \\ 
\hline
\end{tabular}

	}
\end{table}