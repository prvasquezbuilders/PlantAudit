%VISUAL
\section{Visual Inspection on Pipe, valves, fittings, supports, expansions, and appurtenances}
\label{ch04mech02}

\subsection{Highlights}
\label{ch04mech02_highlight}

Visual inspection data on pipes, valves, fittings, supports, expansions, and appurtenances is highlighted in Table \ref{ch043_tbl_visualinspectionHL}.

\begin{table}[!htb]
	\caption{Highlights of visual inspection}
	\label{ch043_tbl_visualinspectionHL}
%	\resizebox{\columnwidth}{!}{%
	{\scriptsize

\begin{tabular}{c|p{3cm}|p{9.5cm}}

\hline
No. & Items & Remarks \\ 
\hline
1 & Existing suction pipe and fittings & Suction line too short and is jam packed with fittings. Does not promote good flow development. Intake water will be turbulent and is not desired \\ 
2 & Discharge piping and fittings & Pressure gauge near pump discharge is prefered for measuring head of indivual pump \\ 
2 & Pump vibration isolation  & Does not appropriately serve its function to isolate vibration from building and reduce noise \\ 
3 & As built difference & Actual system contain many differences from provided copy of old as-built including valve positions and pipe design \\ 
4 & Pump foundation block & Unitary block with multiple slots and may  \\ 
5 & Space & Modifications for improvement are possible because of available space inside pump room \\ 
6 & Instrumentation and monitoring & Pump instrumentation do not include PLC and other important parameters typically displayed by PLC not monitored \\ 
\hline

\end{tabular}

	}%}
\end{table}

Visual inspections are supported with the photos taken at particular locations or positions in question.

The station houses the pump system on top of the reservoir and is remotely controlled via the control center on an nearby building. See various plant schematic (Figures \ref {plant_schematic} - a, b and c)

\begin{figure} [!htb]
	\begin{minipage}[b]{0.3\linewidth}
		\centering
		\includegraphics[width=\textwidth]{figures/fig_ch043_pumplayout_cad}
		\caption*{a - Cad current schematic }
%		\label{}
	\end{minipage}
	\hspace{0.05cm}
	\begin{minipage}[b]{0.3\linewidth}
		\centering
		\includegraphics[width=\textwidth]{figures/fig_ch043_plant_layout1}
		\caption*{b - Plant Schematic (01)}
%		\label{}
	\end{minipage}
	\hspace{0.05cm}
	\begin{minipage}[b]{0.3\linewidth}
		\centering
		\includegraphics[width=\textwidth]{figures/fig_ch043_plant_layout2}
		\caption*{c - Plant Schematic (02)}
%		\label{}
	\end{minipage}
\caption{Plant Schematics}
\label{plant_schematic}
\end{figure}

The pump station has undergone upgrading last 2009. The pump system is designed for 6 avaialble slots for vertical turbine pumps where 4 are currently installed (Figure \ref{ch01_keyplan}-b). The reservoir has an internal partition where 2 pumps draw water from the right side of the reservoir and 2 on the other. Water levels inside the reservoir are balanced thru a pipe that connects to either side. 

The top of the reservoir contain the pump room where the vertical turbine pumps are located positioning it on top as part of flood proofing. A fraction of the reservoir top is allotted for the pump room where currently 4 pumps of 22.7m head and 23.52 Mld capacity each are fitted. However better utilization of space could have been made by using a wider portion of the reservoir top. This could have given sufficient allowance for future developments.


%this goes to show how important design review is and has consequential impact for future planning.

The pipes join the main discharge header on Y-type connection (Figure \ref{discharge_piping} - a). This helps to avoid higher head loss and strong water thrust that could develop if T junctions were used. However, the area allotted for the placement of the MCC, pumps, pipes and fittings was rather small. Necessary fittings such as the valves and STC’s were installed and thereby only a small portion of straight pipe is left.

The discharge header then elbows down (Figure \ref {discharge_piping} - b) leading the pipe underground to join the supply line. This turn however may result to strong hydraulic thrust as water passes thru and the abrupt change in direction of a considerable volume of water will lead to losses and turbulence. The surface erosion on the elbow may need to check in the future. 

\begin{figure} [!htb]
	\begin{minipage}[b]{0.5\linewidth}
		\centering
		\includegraphics[width=\textwidth]{figures/fig_ch043_y_junction}
		\caption*{a - Discharge header and joining lines}
	%	\label{y_junction}
	\end{minipage}
	\hspace{0.05cm}
	\begin{minipage}[b]{0.5\linewidth}
		\centering
		\includegraphics[width=\textwidth]{figures/fig_ch043_discharge_run_p2}
		\caption*{b - Discharge header continued}
	%	\label{discharge_header_elbow}
	\end{minipage}
\caption{Discharge header}
\label{discharge_piping}
\end{figure}

The surrounding areas around the vicinity currently functions as buffer zones between the residential areas and the station itself. However, this area still makes up considerably wide space and can be utilized if the reservoir had been better positioned. (Figure \ref{unused_spaces} - a and Figure  \ref{unused_spaces} - b)

Similarly, the space on top of the reservoir not covered by the pump room does not show much opportunity for utilization (Figure \ref {unused_spaces} - c and Figure \ref{unused_spaces} - d). If spaces such as these want to be used however, quite enormous changes and construction might be needed such as installing additional columns on the reservoir to help support the load of the pump room as well as repositioning the entire plant reservoir.

\begin{figure} [!htb]
	\begin{minipage}[b]{0.22\linewidth}
		\centering
		\includegraphics[width=\textwidth]{figures/fig_ch043_unused_space3}
		\caption*{a - buffer space (01) \\}
		\label{ch043_unused_space3}
	\end{minipage}
	\hspace{0.05cm}
	\begin{minipage}[b]{0.22\linewidth}
		\centering
		\includegraphics[width=\textwidth]{figures/fig_ch043_unused_space4}
		\caption*{b - buffer space (02) \\}
		\label{ch043_unused_space4}
	\end{minipage}
	\hspace{0.05cm}
	\begin{minipage}[b]{0.22\linewidth}
		\centering
		\includegraphics[width=\textwidth]{figures/fig_ch043_unused_space1}
		\caption*{c - unused space on reservoir top (01)}
		\label{ch043_unused_space1}
	\end{minipage}
	\hspace{0.05cm}
	\begin{minipage}[b]{0.22\linewidth}
		\centering
		\includegraphics[width=\textwidth]{figures/fig_ch043_unused_space2}
		\caption*{d - unused space on reservoir top (02)}
		\label{ch043_unused_space2}
	\end{minipage}
\caption{Discharge side piping}
\label{unused_spaces}
\end{figure}

In contrast to the more prevalent HSC type pumps observed in other stations, the pump station has vertical turbines as its pumps. Relative to the reservoir design, this had the consequence of requiring the pumps be located on top of the reservoir and with it the MCCs. The operator’s office is located on a small building nearby and thus does not directly or immediately access the pump room. 

%insert plant layout picture

The PLC needs to checks and fixed to give complete and accurate readings which as important for consistent plant monitoring. Many values found in the display are either zero (Figure \ref{PLC_displays} - a and Figure \ref{PLC_displays} - b) and do not help the operator at all. 

\begin{figure} [!htb]
	\begin{minipage}[b]{0.5\linewidth}
		\centering
		\includegraphics[width=\textwidth]{figures/fig_ch043_plc1}
		\caption*{a - PLC overview}
		\label{ch043_plc1}
	\end{minipage}
	\hspace{0.05cm}
	\begin{minipage}[b]{0.5\linewidth}
		\centering
		\includegraphics[width=\textwidth]{figures/fig_ch043_plc2}
		\caption*{b - PLC parameters}
		\label{ch043_plc2}
	\end{minipage}
\caption{Discharge side piping}
\label{PLC_displays}
\end{figure}

The water level inside the reservoir partitions are displayed in two separate meters. The inflow and outflow are also likewise displayed by two other meters. (Figures \ref{plant_meters} - a, b and c)

\begin{figure} [!htb]
	\begin{minipage}[b]{0.3\linewidth}
		\centering
		\includegraphics[width=\textwidth]{figures/fig_ch043_water_level_meter}
		\caption*{a - water level meters}
		\label{ch043_water_level_meter}
	\end{minipage}
	\hspace{0.05cm}
	\begin{minipage}[b]{0.3\linewidth}
		\centering
		\includegraphics[width=\textwidth]{figures/fig_ch043_flowmeter1}
		\caption*{b - inflow meter}
		\label{ch043_flowmeter1}
	\end{minipage}
	\hspace{0.05cm}
	\begin{minipage}[b]{0.3\linewidth}
		\centering
		\includegraphics[width=\textwidth]{figures/fig_ch043_flowmeter2}
		\caption*{c - outflow meter}
		\label{ch043_flowmeter2}
	\end{minipage}
\caption{Indicating devices }
\label{plant_meters}
\end{figure}

There is a lack of monitoring devices necessary for remote monitoring such as vibration, temperature, etc. These are important to keep the operating parts health in check. In case of situational bad operating conditions, these may continue for some time before the operator can recognize it. 

This may prove significance in improving CI effectiveness. Consider the following possible root cause as presented in a CI report (Figure \ref{ch043_sample_root_cause_analysis}).

\begin{figure} [!htb]
	%	\begin{center}
	\includegraphics[scale=.7]{figures/fig_ch043_sample_root_cause_analysis} 
	%	\end{center}
	\caption{CI - Root Cause Analysis}
	\label{ch043_sample_root_cause_analysis}
\end{figure}

It can be seen that monitoring devices such as those for vibration are important to easily detect disturbances and immediately call for diagnosis of looseness or mis-alignments to avoid such to escalate to major problems requiring major overhaul and/or replacement.

Local corrosion of pipes and fittings are observed along discharge header (Figure \ref{plant_meters} - a) return lines (Figure \ref{plant_meters} - b) and on the pump headers (Figure \ref{plant_meters} - c).

\begin{figure} [!htb]
	\begin{minipage}[b]{0.225\linewidth}
		\centering
		\includegraphics[width=\textwidth]{figures/fig_ch043_corrosion_1}
		\caption*{a - discharge header}
%		\label{ch043_corrosion_1}
	\end{minipage}
	\hspace{0.05cm}
	\begin{minipage}[b]{0.225\linewidth}
		\centering
		\includegraphics[width=\textwidth]{figures/fig_ch043_corrosion_3}
		\caption*{b - return line}
%		\label{ch043_corrosion_3}
	\end{minipage}
	\hspace{0.05cm}
	\begin{minipage}[b]{0.225\linewidth}
		\centering
		\includegraphics[width=\textwidth]{figures/fig_ch043_corrosion_2}
		\caption*{c - pump header}
%		\label{ch043_corrosion_2}
	\end{minipage}
	\hspace{0.05cm}
	\begin{minipage}[b]{0.225\linewidth}
		\centering
		\includegraphics[width=\textwidth]{figures/fig_ch043_corrosion_4}
		\caption*{d - piping}
%		\label{ch043_corrosion_2}
	\end{minipage}
\caption{Corrosion on parts}
\label{plant_meters}
\end{figure}

Also, the column for suction pressure on the recording sheet should be renamed as supply mainline pressure to avoid ambiguity.

All motors lack frame grounding. Assuming the motor power supplies were grounded at the motor control center (MCC), an assumption that still needs to be verified, current may pass through the bearings should the motor shaft be energized while the motor frame is ungrounded, which may lead to bearing damage, especially among variable speed drive (VFD) units. Bonding of non-current carrying metal components (e.g., motor frames) to the ground system is necessary to create an equipotential plane between the concrete floor and plant personnel who may risk electrocution should the frames parts be energized

%Relative to the station's desired capacity and reservoir size, the area allotted for the pump system is rather very limited (Figure \ref{ch043_psr} - a and Figure \ref{ch043_pump_layout_actual}). 3 HSC pumps each with its motor, valves and pipes and fittings were fitted inside the limited space resulting to a compact arrangement. This led to a few compromises regarding pipe design.

%To increase suction head, the pumps are installed at lower elevation close to bottom level of reservoir. The pump system was recently rehabilitated and new pumps were installed along with its auxillaries. Corresponding foundation blocks were also built beside the old foundations (Figure \ref{ch043_psr} - b).

%\begin{figure}[!htb]
%	\begin{minipage}[b]{0.4\linewidth}
%		\centering
%		\includegraphics[width=\textwidth]{figures/ch043_plant_layout}
%		\caption*{a - plant layout (01)}
%		\label{ch043_plant_layout}
%	\end{minipage}
%	\hspace{0.05cm}
%		\centering
%	\begin{minipage}[b]{0.4\linewidth}
%		\includegraphics[width=\textwidth]{figures/ch043_pump_layout}
%		\caption*{b - pump layout (02)}
%		\label{ch043_pump_layout}
%	\end{minipage}
%\end{minipage}
%\caption{PSR layouts}
%\label{ch043_psr}
%\end{figure}

%\begin{figure}[h]
%	%	\begin{center}
%	\includegraphics[scale=0.5]{figures/ch043_pump_layout_actual} 
%	%	\end{center}
%	\caption{Pump system}
%	\label{ch043_pump_layout_actual}
%\end{figure}




%\begin{figure}[!htb]
%	\begin{minipage}[b]{0.3\linewidth}
%		\centering
%		\includegraphics[width=\textwidth]{figures/ch043_double_elbow_suction1}
%		\caption*{a - double elbow suction (01)}
%		\label{ch043_double_elbow_suction1}
%	\end{minipage}
%	\hspace{0.05cm}
%	\begin{minipage}[b]{0.3\linewidth}
%		\centering
%		\includegraphics[width=\textwidth]{figures/ch043_double_elbow_suction2}
%		\caption*{b - double elbow suction (02)}
%		\label{ch043_double_elbow_suction2}
%	\end{minipage}
%	\hspace{0.05cm}
%	\begin{minipage}[b]{0.3\linewidth}
%		\centering
%		\includegraphics[width=\textwidth]{figures/ch043_double_elbow_suction3}
%		\caption*{c - double elbow suction (03)}
%		\label{ch043_double_elbow_suction3}
%	\end{minipage}
%\caption{suction pipe design}
%\label{ch043_spd}
%\end{figure}

%This is not desired as it will reduce performance and life of pump when flow profile of water entering pump is not relatively developed. Basic simulation of flow predicts show flow profile. Ideally, a fully developed flow entering pump intake is desired to avoid problems in suction by following good suction pipe design practice. Elbow should never be bolted directly to the pumps suction nozzle. This will result to noisy operation loss in efficiency and capacity and heavy end thrust.

%The abrupt connection between the short 200 mm short pipe (just after the elbow) and the 250 mm bellow-type flexible coupling is unideal and will promote turbulence near the pump inlet and further reduce pump performance and life.(Figure \ref{ch043_elbow_to_fj_connection})

%It is standard practice to employ suction-side piping one or two sizes bigger than the pump inlet. Small pipes result in larger friction losses, which means it costs more to run your pumping system. However, for this case, the pipe is but a small length and balancing the cost of larger friction cost and higher cost of a bigger pipe will not be significant, thus the same diameter suction pipe will suffice.

%A reason for this design is to accommodate placing the check valve at pump discharge to later join the discharge header via an elbow and thru a Y junction. 

%A particular pipe section along the discharge section is quite alarming. Pipe deterioration are evident as the protective paint has become brittle and the those that have come off reveal the serious corrosion occurring on the pipe surface. The handle for the pressure control valve have experienced serious galvanic corrosion which resulted to its unaesthetic deterioration. Where the pipe penetrates the wall, substantial corrosion also has been observed and may become a source of leak soon. (reducer should be desired before pump intake)


%Another consequence of this lack of consideration for the piping was that the old pipes are already experiencing heavy corrosion which might soon be replaced to if water quality is desired to be maintained as well as avoiding leaks that could prompt dangerous situations.
%During the rehabilitation, the designer should have considered redesigning the entire pump system including provisions for monitoring 


%\begin{figure}[!htb]
%	\begin{minipage}[b]{0.3\linewidth}
%		\centering
%		\includegraphics[width=\textwidth]{figures/ch043_discharge_pipe_deteriorationD}
%		\caption*{d - pipe corrosion (04)}
%		\label{ch043_discharge_pipe_deteriorationD}
%	\end{minipage}
%	\hspace{0.05cm}
%	\begin{minipage}[b]{0.3\linewidth}
%		\centering
%		\includegraphics[width=\textwidth]{figures/ch043_discharge_pipe_deteriorationE}
%		\caption*{e - pipe corrosion (05)}
%		\label{ch043_discharge_pipe_deteriorationE}
%	\end{minipage}
%	\hspace{0.05cm}
%	\begin{minipage}[b]{0.3\linewidth}
%		\centering
%		\includegraphics[width=\textwidth]{figures/ch043_discharge_pipe_deteriorationF}
%		\caption*{f - pipe corrosion (06)}
%		\label{ch043_discharge_pipe_deteriorationF}
%	\end{minipage}
%\end{figure}



%Although the pipes can withstand a remarkable amount of corrosion before failure, this should not be allowed to simply continue. As part of reliability and good upkeep, all pipes should generally be protected and preserved as much as possible.

%A note on one discharge gate valve located above ground: this placement of valve should be avoided as it is not easily accessible to operator to adjust if needed.

%On a positive note, the pump system incorporates vibration meters which are valuable monitoring tools of the condition of the system.

%A discrepancy in unit is observed between the flow meter and the control panel. This is just a minor error and should be corrected by the operator.


%Other minor observations are listed below:

%\begin {itemize}
%	\item There is discrepancy of units between flow meter and control panel is observed. Use the correct units m3/hr and change the unit for the control panel to avoid further confusion.
%	\item Insert pressure tapping for pressure gauge near pump flange for more accurate measurement of head.
%	\item Suction pipe should have ascending inclination toward pump suction nozzle (preferably 1/100 sloping gradient)
%	\item The column for suction pressure should be renamed as supply mainline pressure to avoid ambiguity.
%\end{itemize}



%Pipe should have descending inclination 
%Insert pressure tapping for pressure gauge
%Insert temperature tapping for thermodynamic efficiency devices


%\begin{figure}[h]
	%	\begin{center}
	%\includegraphics[scale=0.6]{figures/ch04_11_fot_eccentric_reducer} 
	%	\end{center}
	%\caption{Eccentric reducer}
	%\label{ch04_tbl_ch04_11_fot_eccentric_reducer}
%\end{figure}




%The pipe design then has no provisions for flow measurement for conventional meters such as mechanical meters, weigh tanks and ultrasonic/Doppler flow meters, magnetic flow meters. All of these depend on some provisions of straight pipe with relatively developed flow or require space for installation of monitoring devices. 

%Furthermore, common practice to avoid air pockets building up at the suction side of the pump is to use flat-on-top eccentric reducer before the pump suction (Figure \ref{ch04_tbl_ch04_11_fot_eccentric_reducer}). This is applicable for piping coming from below or straight ahead. 

%show flow simulation comparison between two setups


\subsection{Visual inspection data}
Visual inspection data on assets are summarized in tables of this section. %and also in the Appendix \ref{appvisualinspectionmech} with pictures.

%\paragraph{\textbf{BP1}}

\begin{table}[!htb]
	\caption{Visual inspection data - Drawing from Right Section of Reservoir: P1 and P2}
	\label{ch043_tbl_visualinspectionP1P2}
%	\resizebox{\columnwidth}{!}{%
		{\scriptsize
\begin{tabular}{c|l|c|p{9cm}}

\hline
No. & Items & CS & Remarks \\ 
\hline
1 & Elbow after pump discharge & 2 & Need to be supported  \\ 
2 & CV & 1 & No leakage found \\ 
3 & STC with harness & 1 & Reduces hydraulic thrust on header by allowing minute axial displacement along pipe \\ 
4 & Header and joining line & 1 & Joining section via Y connection and long radius miter bend used at later section of discharge header help reduce hydraulic thrust and resistance \\ 
5 & Bypass line & 2 & Not fully bolted at base for bypass line of Pump 2; corrosion of flange and bolts observed \\ 
\hline

\end{tabular}
	}
\end{table}
\begin{table}[!htb]
	\caption{Visual inspection data - Drawing from Left Section of Reservoir: P3 and P4}
	\label{ch043_tbl_visualinspectionP3P4}
%	\resizebox{\columnwidth}{!}{%
		{\scriptsize
\begin{tabular}{c|l|c|p{9cm}}

\hline
No. & Items & CS & Remarks \\ 
\hline
1 & Elbow after pump discharge & 2 & Need to be supported  \\ 
2 & CV & 1 & No leakage found \\ 
3 & STC with harness & 1 & Reduces hydraulic thrust on header by allowing minute axial displacement along pipe \\ 
4 & Header and joining line & 1 & Joining section via Y connection help reduce hydraulic thrust and resistance \\ 
5 & Bypass line & 1 & no leakage observed \\ 
\hline

\end{tabular}
	}
\end{table}


\subsection{Recommendations}

%\begin{itemize}
%\item Corrosion on certain portions of the piping system
%\end{itemize}


\paragraph{\underline{Recommendations}} 
\begin{itemize}
%	\item [$\checkmark$] Modifications need to made to correct these sitaution. Refer to conceptual design for further details.
	\item [$\checkmark$] Clear away oxide scales in affected areas until clean base metal is exposed. Repair or reinforce damaged pipe wall and recoat with a suitable protective coating.
	\item [$\checkmark$] Bond motor frames to the station ground bus; review present grounding design or policy for Noveleta PSR.

\end{itemize}

%\end{document}