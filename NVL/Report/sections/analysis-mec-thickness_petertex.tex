%\begin{document}
\section{Pump discharge and suction pipe - thickness} \label{ch04mech01}
\subsection{Data and measurement}
Thickness data on discharge and suction pipes of pumps is presented in Table \ref{ch04_tbl_thickness02} and Table \ref{ch04_tbl_thickness03} .

\begin{table}[h]
	\caption{Thickness data - Booster Pumps (mm).}
	\label{ch04_tbl_thickness02}
	{\footnotesize
\begin{tabular}{l|c|c|c|c|c|c|c|c}
\hline
Asset & Position & \multicolumn{7}{c}{Distance (m)} \\ 
\cline{3-9}
&  & \multicolumn{4}{c|}{Suction} & \multicolumn{3}{c}{Discharge} \\ 
\cline{3-9}
&  & 2m & 4m & 5m & ELBOW & 2.2 m & 3.2 m & ELBOW \\ 
\hline
BP1 & 12 & 4.84 & 4.76 & 4.7 & 4.74 & 4.75 & 4.76 & 4.81 \\ 
& 3 & 4.77 & 4.74 & 4.76 & 4.86 & 4.71 & 4.75 & 4.63 \\ 
& 6 & 4.85 & 4.74 & 4.74 & 4.79 & 4.68 & 4.66 & 4.76 \\ 
& 9 & 4.84 & 4.77 & 4.69 & - & 4.77 & 4.7 & - \\ 
\hline
BP2 & 12 & 4.84 & 4.84 & 4.93 & 5.02 & 4.76 & 4.76 & 4.68 \\ 
& 3 & 4.88 & 4.47 & 4.91 & 4.91 & 4.66 & 4.7 & 4.77 \\ 
& 6 & 4.83 & 4.94 & 4.84 & 5.04 & 4.65 & 4.57 & 4.77 \\ 
& 9 & 4.74 & 4.78 & 4.9 & - & 4.72 & 4.53 & - \\ 
\hline
&  & 2m & 3m & ELBOW &  & 2m & 3m & ELBOW \\ 
\hline
BP3 & 12 & 4.74 & 4.75 & 4.67 &  & 4.66 & 4.75 & 4.66 \\ 
& 3 & 4.7 & 4.78 & 4.83 &  & 4.65 & 4.8 & 4.75 \\ 
& 6 & 4.81 & 4.76 & 4.75 &  & 4.76 & 4.73 & 4.77 \\ 
& 9 & 4.82 & 4.71 & - &  & 4.65 & 4.74 & - \\ 
\hline
BP4 & 12 & 4.9 & 4.86 & 4.92 &  & 4.79 & 4.92 & 4.92 \\ 
& 3 & 4.79 & 4.89 & 4.75 &  & 4.86 & 4.96 & 4.69 \\ 
& 6 & 4.91 & 4.65 & 4.89 &  & 4.94 & 4.95 & 5.04 \\ 
& 9 & 4.65 & 4.7 & - &  & 4.96 & 4.96 & - \\ 
\hline
\end{tabular}

	}
\end{table}


\begin{table}[h]
	\caption{Thickness data - Storage Pumps (mm).}
	\label{ch04_tbl_thickness03}
	{\footnotesize
\begin{tabular}{l|c|c|c|c|l|l}
	\hline
	Asset & Position & \multicolumn{5}{c}{Distance} \\ 
	\cline{3-7}
	&  & \multicolumn{2}{c|}{Suction} & \multicolumn{3}{c}{Discharge} \\ 
	\cline{3-7}
	&  & 2m & ELBOW & 2m & \multicolumn{1}{c|}{3m} & \multicolumn{1}{c}{ELBOW} \\ 
	\hline
	SP1 & 12 & 4.64 & 4.68 & 4.69 & \multicolumn{1}{c|}{4.64} & \multicolumn{1}{c}{4.39} \\ 
	& 3 & 4.77 & 4.7 & 4.52 & \multicolumn{1}{c|}{3.88} & \multicolumn{1}{c}{4.73} \\ 
	& 6 & 4.76 & 4.69 & 4.05 & \multicolumn{1}{c|}{4.62} & \multicolumn{1}{c}{4.74} \\ 
	& 9 & 4.71 & - & 4.69 & \multicolumn{1}{c|}{4.63} & \multicolumn{1}{c}{-} \\ 
	\hline
	SP2 & 12 & 4.62 & 4.88 & 4.71 & \multicolumn{1}{c|}{4.58} & \multicolumn{1}{c}{4.64} \\ 
	& 3 & 4.76 & 4.8 & 4.63 & \multicolumn{1}{c|}{4.62} & \multicolumn{1}{c}{4.79} \\ 
	& 6 & 4.8 & 4.84 & 4.68 & \multicolumn{1}{c|}{4.45} & \multicolumn{1}{c}{4.71} \\ 
	& 9 & 4.67 & - & 4.69 & \multicolumn{1}{c|}{4.72} & \multicolumn{1}{c}{-} \\ 
	\hline
\end{tabular}


	}
\end{table}

In the table, the positions and the distances for the Ultrasonic Thickness Gauging (UTG) are referred to Figure \ref{ch04_fig_utgbp} and Figure \ref{ch04_fig_utgsp}.

\begin{figure}[!htb]
	\includegraphics[scale=1]{figures/ch04_02_BP_UTG_test_points} \\
	\caption{Positions and distances of UTG - Booster Pump}
	\label{ch04_fig_utgbp} 
\end{figure}

\begin{figure}[!htb]
	\includegraphics[scale=1]{figures/ch04_03_SP_UTG_test_points} \\
	\caption{Positions and distances of UTG - Storage Pump}
	\label{ch04_fig_utgsp} 
\end{figure}


%\begin{table}[h]
%	\caption{Thickness data (mm).}
%	\label{thicknessdata}
%	{\footnotesize
%	\begin{tabular}{l|l|l|l|l}
%		\hline
%		Pumps & \multicolumn{2}{c|}{Suction} & \multicolumn{2}{c}{Discharge} \\ 
%		\cline{2-5}
%		& \multicolumn{1}{c|}{Design} & \multicolumn{1}{c|}{Actual} & \multicolumn{1}{c|}{Design } & \multicolumn{1}{c}{Actual} \\ 
%		\hline
%		BP1 & \multicolumn{1}{c|}{} & \multicolumn{1}{c|}{4.98} & \multicolumn{1}{c|}{} & \multicolumn{1}{c}{3.92} \\ 
%		BP2 & \multicolumn{1}{c|}{} & \multicolumn{1}{c|}{4.32} & \multicolumn{1}{c|}{} & \multicolumn{1}{c}{4.22} \\ 
%		BP3 & \multicolumn{1}{c|}{} & \multicolumn{1}{c|}{4.61} & \multicolumn{1}{c|}{} & \multicolumn{1}{c}{4.61} \\ 
%		BP4 & \multicolumn{1}{c|}{} & \multicolumn{1}{c|}{4.54} & \multicolumn{1}{c|}{} & \multicolumn{1}{c}{4.37} \\ 
%		BP5 & \multicolumn{1}{c|}{} & \multicolumn{1}{c|}{4.49} & \multicolumn{1}{c|}{} & \multicolumn{1}{c}{4.64} \\ 
%		BP6 & \multicolumn{1}{c|}{} & \multicolumn{1}{c|}{4.09} & \multicolumn{1}{c|}{} & \multicolumn{1}{c}{4.60} \\ 
%		SP1 & \multicolumn{1}{c|}{} & \multicolumn{1}{c|}{4.40} & \multicolumn{1}{c|}{} & \multicolumn{1}{c}{4.25} \\ 
%		SP2 & \multicolumn{1}{c|}{} & \multicolumn{1}{c|}{5.09} & \multicolumn{1}{c|}{} & \multicolumn{1}{c}{4.18} \\ 
%		\hline
%	\end{tabular}
%			
%	}
%\end{table}

%Detailed measurement data is provided in appendix \ref{appthicknesss}.
%\textcolor{red}{RB Sanchez to write here the summary of raw data collected from visual inspection and testing. Tables shall be used as much as we can. Note that no analysis in this session. This session is purely the high level presentation of data. Raw data can be linked as an Appendix}

\subsection{Analysis} 
This section provides analysis/discussion on estimation of minimum allowable thickness of pipes and statistics around the measured data collected during inspection and testings.

\subsubsection{Statistics} \label{ch05_statistics}
A summary on statistics regarding the measured thickness for booster pumps and storage pumps is presented in Table \ref{ch05_tbl_thicknesssta}.
\begin{table}[h]
	\caption{Summary of statistics - thickness.}
	\label{ch05_tbl_thicknesssta}
	{\footnotesize
\begin{tabular}{l|l|l|l|l|l|l|l|l}
	\hline
	Statistics & \multicolumn{2}{c|}{Booster suction} & \multicolumn{2}{c|}{Booster discharge} & \multicolumn{2}{c|}{Storage suction} & \multicolumn{2}{c}{Storage discharge} \\ 
	\cline{2-9}
	& \multicolumn{1}{c|}{Straight} & \multicolumn{1}{c|}{Elbow} & \multicolumn{1}{c|}{Straight} & \multicolumn{1}{c|}{Elbow} & \multicolumn{1}{c|}{Straight} & \multicolumn{1}{c|}{Elbow} & \multicolumn{1}{c|}{Straight} & \multicolumn{1}{c}{Elbow} \\ 
	\hline
	Min & \multicolumn{1}{c|}{4.410} & \multicolumn{1}{c|}{4.090} & \multicolumn{1}{c|}{4.220} & \multicolumn{1}{c|}{3.920} & \multicolumn{1}{c|}{5.090} & \multicolumn{1}{c|}{4.400 } & \multicolumn{1}{c|}{4.180} & \multicolumn{1}{c}{4.510 } \\ 
	1st Qua. & \multicolumn{1}{c|}{4.650} & \multicolumn{1}{c|}{4.600} & \multicolumn{1}{c|}{4.690} & \multicolumn{1}{c|}{4.612} & \multicolumn{1}{c|}{5.140} & \multicolumn{1}{c|}{4.615 } & \multicolumn{1}{c|}{4.617} & \multicolumn{1}{c}{4.617} \\ 
	Median & \multicolumn{1}{c|}{4.690} & \multicolumn{1}{c|}{4.670} & \multicolumn{1}{c|}{4.755} & \multicolumn{1}{c|}{4.740 } & \multicolumn{1}{c|}{5.250 } & \multicolumn{1}{c|}{5.120} & \multicolumn{1}{c|}{4.730} & \multicolumn{1}{c}{4.650} \\ 
	Mean & \multicolumn{1}{c|}{4.681} & \multicolumn{1}{c|}{4.608} & \multicolumn{1}{c|}{4.782} & \multicolumn{1}{c|}{4.664} & \multicolumn{1}{c|}{5.252} & \multicolumn{1}{c|}{5.003} & \multicolumn{1}{c|}{4.670} & \multicolumn{1}{c}{4.683 } \\ 
	3rd Qua. & \multicolumn{1}{c|}{4.740} & \multicolumn{1}{c|}{4.690} & \multicolumn{1}{c|}{4.910 } & \multicolumn{1}{c|}{4.905} & \multicolumn{1}{c|}{5.338} & \multicolumn{1}{c|}{5.407} & \multicolumn{1}{c|}{4.772} & \multicolumn{1}{c}{4.742 } \\ 
	Max & \multicolumn{1}{c|}{4.800} & \multicolumn{1}{c|}{4.830 } & \multicolumn{1}{c|}{4.980} & \multicolumn{1}{c|}{4.980} & \multicolumn{1}{c|}{5.450} & \multicolumn{1}{c|}{5.430} & \multicolumn{1}{c|}{4.970} & \multicolumn{1}{c}{4.910} \\ 
	\hline
\end{tabular}
	}
\end{table}

Table \ref{ch05_tbl_thicknessextra} shows the summary of statistics for individual pump with extrados thickness calculated.
\begin{table}[h]
	\caption{Summary of statistics - thickness (individual pump).}
	\label{ch05_tbl_thicknessextra}
	{\footnotesize
\begin{tabular}{l|l|l|l|l|l|l|l|l}
	\hline
	\multicolumn{1}{c|}{Assets} & \multicolumn{4}{c|}{Suction (mm)} & \multicolumn{4}{c}{Discharge (mm)} \\ 
	\cline{2-9}
	\multicolumn{1}{c|}{} & \multicolumn{1}{c|}{Min} & \multicolumn{1}{c|}{Mean} & \multicolumn{1}{c|}{Extrados} & \multicolumn{1}{c|}{Max} & \multicolumn{1}{c|}{Min} & \multicolumn{1}{c|}{Mean} & \multicolumn{1}{c|}{Extrados} & \multicolumn{1}{c}{Max} \\ 
	\hline
	\multicolumn{1}{c|}{BP1} & \multicolumn{1}{c|}{4.480} & \multicolumn{1}{c|}{4.677} & \multicolumn{1}{c|}{4.608} & \multicolumn{1}{c|}{4.790} & \multicolumn{1}{c|}{3.920} & \multicolumn{1}{c|}{4.532} & \multicolumn{1}{c|}{4.346} & \multicolumn{1}{c}{4.730} \\ 
	\multicolumn{1}{c|}{BP2} & \multicolumn{1}{c|}{4.320} & \multicolumn{1}{c|}{4.647} & \multicolumn{1}{c|}{4.554} & \multicolumn{1}{c|}{4.800} & \multicolumn{1}{c|}{4.220} & \multicolumn{1}{c|}{4.794} & \multicolumn{1}{c|}{4.836} & \multicolumn{1}{c}{4.980} \\ 
	\multicolumn{1}{c|}{BP3} & \multicolumn{1}{c|}{4.610} & \multicolumn{1}{c|}{4.671} & \multicolumn{1}{c|}{4.680} & \multicolumn{1}{c|}{4.750} & \multicolumn{1}{c|}{4.610} & \multicolumn{1}{c|}{4.700} & \multicolumn{1}{c|}{4.672} & \multicolumn{1}{c}{4.790} \\ 
	\multicolumn{1}{c|}{BP4} & \multicolumn{1}{c|}{4.540} & \multicolumn{1}{c|}{4.742} & \multicolumn{1}{c|}{4.742} & \multicolumn{1}{c|}{4.830} & \multicolumn{1}{c|}{4.370} & \multicolumn{1}{c|}{4.871} & \multicolumn{1}{c|}{4.796} & \multicolumn{1}{c}{4.980} \\ 
	\multicolumn{1}{c|}{BP5} & \multicolumn{1}{c|}{4.490} & \multicolumn{1}{c|}{4.650} & \multicolumn{1}{c|}{4.674} & \multicolumn{1}{c|}{4.800} & \multicolumn{1}{c|}{4.640} & \multicolumn{1}{c|}{4.735} & \multicolumn{1}{c|}{4.756} & \multicolumn{1}{c}{4.860} \\ 
	\multicolumn{1}{c|}{BP6} & \multicolumn{1}{c|}{4.090} & \multicolumn{1}{c|}{4.534} & \multicolumn{1}{c|}{4.643} & \multicolumn{1}{c|}{4.770} & \multicolumn{1}{c|}{4.600} & \multicolumn{1}{c|}{4.868} & \multicolumn{1}{c|}{4.882} & \multicolumn{1}{c}{4.980} \\ 
	\multicolumn{1}{c|}{SP1} & \multicolumn{1}{c|}{4.400} & \multicolumn{1}{c|}{4.593} & \multicolumn{1}{c|}{4.593} & \multicolumn{1}{c|}{4.840} & \multicolumn{1}{c|}{4.250} & \multicolumn{1}{c|}{4.701} & \multicolumn{1}{c|}{4.710} & \multicolumn{1}{c}{4.880} \\ 
	\multicolumn{1}{c|}{SP2} & \multicolumn{1}{c|}{5.090} & \multicolumn{1}{c|}{5.289} & \multicolumn{1}{c|}{5.398} & \multicolumn{1}{c|}{5.450} & \multicolumn{1}{c|}{4.180} & \multicolumn{1}{c|}{4.644} & \multicolumn{1}{c|}{4.754} & \multicolumn{1}{c}{4.970} \\ 
	\hline
\end{tabular}

	}
\end{table}

Figures \ref{ch05_thickness_suction} and \ref{ch05_thickness_dicharge} show comparative graphs of min, mean, and max values of thickness.


\begin{figure}[!htb]
	\begin{minipage}[b]{0.5\linewidth}
		\centering
		\includegraphics[width=\textwidth]{figures/ch05_thickness_suction}
		\caption*{a - Straight} 
		%		\label{ch05_thickness_suction}
	\end{minipage}
	\hspace{0.05cm}
	\begin{minipage}[b]{0.5\linewidth}
		\centering
		\includegraphics[width=\textwidth]{figures/ch05_thickness_suctione}
		\caption*{b - Elbow} 
		%		\label{ch05_thickness_suction_e}
	\end{minipage}
	\caption{Suction line}
	\label{ch05_thickness_suction}
\end{figure}

\begin{figure}[!htb]
	\begin{minipage}[b]{0.5\linewidth}
		\centering
		\includegraphics[width=\textwidth]{figures/ch05_thickness_discharge}
		\caption*{a - Straight} 
		%		\label{ch05_thickness_suction}
	\end{minipage}
	\hspace{0.05cm}
	\begin{minipage}[b]{0.5\linewidth}
		\centering
		\includegraphics[width=\textwidth]{figures/ch05_thickness_dischargee}
		\caption*{b - Elbow} 
		%		\label{ch05_thickness_suction_e}
	\end{minipage}
	\caption{Discharge line}
	\label{ch05_thickness_dicharge}
\end{figure}

Followings are generic interpretation by examining the tables and graphs
\begin{itemize}
\item Mean value of thickness is above 4.6 mm;
\item Mean and median values are close, inferring a confidence on having less heterogeneity, i.e. distribution of thickness around the pipe is more or less homogeneous;
\item Thickness at elbow is less than that of the straight line;
\item THickness of storage line is likely to be higher than that of the discharge line;
\item BP1 has a value of 3.92 as min at the elbow, which requires attention from time to time.
\end{itemize}

\paragraph{\underline{BP1}}
\begin{itemize}
\item Suction Piping System-This pump is observed to have low noise level during operation. The 4 m mark 3 and 6-o’clock positions have the thinner wall compared to other locations , this may due to the fact that water flows directly towards it. On the other hand, the 2 m mark has considerably small to no thinning;

\item Discharge Piping System- As observed from the data gathered, the flow pattern based on thinning is higher in the lower half of the discharge pipe. It then enters the elbow and approaches towards the middle extrados (4.63 mm). The thickness of the discharge pipe is lower when compared to the suction pipe.

\end{itemize}

\paragraph{\underline{BP2}}
\begin{itemize}
\item Suction Piping System- It is observed to have considerably small or no thinning. However referring to 4m mark, 3-o'clock position where the flow seems to be directed to this area, the thickness has a difference of 0.3mm to 0.4 mm from the rest of the readings. This is possible due to the high velocity of water entering at high momentum. Also possibility that needs to be validated is that abrasions increase the effect of erosions due to sediments making contact in the wall;

\item Discharge Piping System- The thickness gathered is lower compared to the suction line of the same pipeline. This means that the thinning rate at the discharge line is higher. This could be possible due to the cavitation carried out to the discharge side of the pump. The flow pattern is that it enters the elbow having the lower half of the pipe a higher velocity approach. It can also be observed that the thinning occurs greatly at the entrance of the elbow and decreases at the exit of the elbow bend.

\end{itemize}

\paragraph{\underline{BP3}}
\begin{itemize}
	\item Suction Piping System- Upon inspection, this pipe was observed to emit certain water flow sound and crackling sounds. Referring to data, the water flow contacts higher at the exit extrados area and swirls at the 3 and 6-o'clock positions of the 3m mark. It is also seen to have a high backflow rate creating eddies at the 12 and 9-o'clock orientations, thus having a possibility of turbulency. Moreover, 2 m away from the pump, the water tends to swirl from the left side to the right side of the pipe; 
	\item Discharge Piping System- The data gathered may indicate that cavitation is carried over the discharge side of the pump. The cavitation occurs at the upper half of the pipe. This extends to the elbow's extrados.
	
\end{itemize}

\paragraph{\underline{BP4}}
\begin{itemize}
	\item Suction Piping System- This suction pipe inhibits localized wall thinning at the exit elbow extrados. Backflow is also present at the elbow as indicated by the thinning in the intrados area of the bend;
	
	\item Discharge Piping System- Localized wall thinning is not present except in the middle extrados are of the elbow. This thinning indicates that the flow approaches directly in the middle extrados of the elbow and then exits the elbow with less turbulent phase.
	.
\end{itemize}


\paragraph{\underline{SP1}}
\begin{itemize}
\item Suction Piping System- The suction line of the SP's are vertical and not inclined like the setups in the BP's. The thickness confirmed to have a direct flow contacting the extrados surface from the entrance up to the elbow exit. This extends to the 12-o'clock position at the 2m mark. The 2m mark is also near at the elbow. No high backflow rate has been observed;

\item Discharge Piping System- localized wall thinning occurs at the bottom part of the pipe as it is next to a concentric reducers that concentrates the flow. It then continues by swirling to one side of the pipe before entering the elbow. The flow then extends the swirling from the side to the extrados bend entry.
	
\end{itemize}

\paragraph{\underline{SP2}}
\begin{itemize}
	\item Suction Piping System- In this pipeline, the water flow contacted towards the 2m mark and less contacting the entrance elbow part. The thinning data also describes the water pattern of most pattern which travels at the higher half and then flows below half of pipe;
	
	\item Discharge Piping System-Based on the measurements, the discharge side of the pump indicates that low cavitation may be carried out. The flow enters and is concentrated in the lower half of the pipe and swirls to the extrados entry of the pipe. It indicates a backflow at the bend intrados extending to the extrados exit of the bend.

\end{itemize}



\subsubsection{Assumptions}
Following assumptions are used in calculating the required thickness of pipe
\begin{itemize}
\item Maximum Working Head – based on the design drawings and pump nameplate;
\item Pipe Material – assume pipe material is ASTM A570 Grade 33 (market available material for spiral welded pipe);
\item Design Guide – basis used for the simulated calculation is AWWA  Manual M11 – Steel Pipe, A Guide for Design and Installation, 4th Edition. Statement for corrosion allowance is located at Chapter 4, which states \textit{"At one time, it was a general practice to add a fixed, rule-of-thumb thickness to the pipe wall as a corrosion allowance. This was not an applicable solution in the water work field, where standard for coating and lining materials and procedures exists. The design shall be made for the required wall-thickness pipe as determined by the loads imposed, then linings, coatings, and cathodic protection selected to provide the necessary corrosion protection"};
\item Thickness calculation will be based on the internal pressure. External pressure will not be considered because much of the discharge line is not buried.
\item 	Surge Pressure was not considered since there are surge protection along the line. 
\item 	This document will only consider the calculation of the minimum thickness along the discharge line since this is the part of the system where maximum pressure is experience.
\end{itemize}
\subsubsection{Limitations}
As confirmed by Maynilad, there is no available data regarding the design report. Design assumptions herein may be different from what was used by the designer/contractor of this station.

This document will not be able to provide the corrosion/degradation factor of the pipe since there is no available historical data on the thickness of the pipe.

\subsubsection{Parameter values for thickness estimation}
In order to estimate the minimum allowance thickness for pipes in straight line considering material handling ($t_{mh}$) and maximum internal pressure based on AWWA M11 ($t_{sph}$), following equations are used, respectively:

\begin{eqnarray}
&& t_{mh} = \frac{\Phi}{\delta} \label{ch05thickness01}
\end{eqnarray}

\begin{eqnarray}
&& t_{sp} = \frac{\epsilon\times P_{max} \times \Phi}{2 \times S_e} \label{ch05thickness02}
\end{eqnarray}
where $P_{max}$ is maximum internal pressure

\begin{eqnarray}
&& P_{max} = \frac{\rho_{H_2O} \times g \times H_{max}}{1000} \label{ch05thickness03}
\end{eqnarray}

In order to estimate the minimum allowance thickness for pipes at elbows (Miter Bend), only maximum internal pressure is considered:

\begin{eqnarray}
&& t_{mb} = \frac{P_{max} \times \Phi}{2 \times S_e} \times \left[ 1 + \frac{\Phi}{(3 \times R)-(1.5 \times \Phi_d)}\right]\times \epsilon \label{ch05thickness04}
\end{eqnarray}

Paramater values used for computation are given in Table \ref{ch05_tbl_thicknesscalc}

\begin{table}[h]
	\caption{Parameter values for thickness estimation.}
	\label{ch05_tbl_thicknesscalc}
	{\footnotesize
\begin{tabular}{p{4cm}|c|c|c|c|p{4cm}}
	\hline
	Parameters & Symbol & Unit & \multicolumn{2}{c|}{Pumps} & Remarks \\ 
	&  &  & Booster & Storage &  \\ 
	\hline
	Discharge diameter & $\Phi$ & $mm$ & 600 & 600 &  \\ 
	Max flow rate & $Q_{max}$ & $m^3/s$ & 0.64 & 0.53 &  \\ 
	Max pump head & $H_{max}$ & $m$ & 40 & 50 & based on name plate \\ 
	Yield strength of material & $S_y$ & $MPa$ & 227.5 & 227.5 & ASTM A570 Grade 33, spiral welded pipe based on AWWA C200 \\ 
	Allowable stress & $S_e$ & MPa & 113.75 & 113.75 &  \\ 
	Density of water & $\rho_{H_2O}$ & $kg/m^3$ & 1000 & 1000 &  \\ 
	Gravity constant & $g$ & $m/s^2$ & 9.81 & 9.81 &  \\ 
	Safety factor & $\epsilon$ &  & 2 & 2 &  \\ 
	Bulk modulus of compressibility of liquid & $k$ & $Pa$ & 2.1E+09 & 2.1E+09 &  \\ 
	Young's modulus of elasticity of pipe wall & $E$ & $Pa$ & 2.1E+11 & 2.1E+11 &  \\ 
	Radius of Elbow & $R$ & $mm$ & 800 & 800 &  \\ 
	Empirical constant & $\delta$ &  & 288 & 288 &  \\ 
	\hline
\end{tabular}
	}
\end{table}

\subsubsection{Required thickness}

Results of computation for minimum allowable thickness for booster pumps and storage pumps are given in Table \ref{ch05_tbl_thicknesscalcresult}.

\begin{table}[h]
	\caption{Minimum thickness allowance.}
	\label{ch05_tbl_thicknesscalcresult}
	{\footnotesize
		\begin{tabular}{l|l|p{3cm}|p{3cm}|p{3cm}}
			\hline
			Pumps & \multicolumn{1}{c|}{Internal pressure  (Mpa)} & \multicolumn{3}{c}{Minimum allowable thickness (mm)} \\ 
			\cline{3-5}
			& \multicolumn{1}{c|}{$P_{max}$} & \multicolumn{1}{c|}{$t_{mh}$} & \multicolumn{1}{c|}{$t_{sp}$} & \multicolumn{1}{c}{$t_{mb}$} \\ 
			\hline
			Booster & \multicolumn{1}{c|}{0.392} & \multicolumn{1}{c|}{2.080} & \multicolumn{1}{c|}{2.070} & \multicolumn{1}{c}{2.900} \\ 
			Storage & \multicolumn{1}{c|}{0.491} & \multicolumn{1}{c|}{2.080} & \multicolumn{1}{c|}{2.590} & \multicolumn{1}{c}{3.620} \\ 
			\hline
		\end{tabular}
		
	}
\end{table}

If comparing these values with the measured values of thickness shown in subsection \ref{ch05_statistics}, it can be concluded that current thickness of pipes at both suction and discharge still provide adequate level of services as the measured value is about 4.60 mm on average while the required values for booster pumps are less than 3 mm and for storage pump is less than 3.62 mm. 

However, it is important to note that required thickness at the elbow of storage pumps is 3.62 mm, which is not so far off from measured value of 4.60 mm. Especially, elbow section is significant important and shall not be at risk.

%\subsubsection{Deterioration prediction}
\subsubsection{Deterioration}
Given the lack of design data, precise design thickness is unknown, following assumptions are made

\begin{itemize}
\item Maximum measured thickness is considered to be the design thickness;
\item Pipe has been in operation for 9 years since 2010;
\item Deterioration rate is to follow linear function.
\end{itemize}

%Condition states used for pipe are defined in Table \ref{ch04:csthickness}

%\begin{table}[h]
%	\caption{Condition state definition - Pipe thickness.}
%	\label{ch04:csthickness}
%	{\footnotesize
%\begin{tabular}{l|l|l}
%	\hline
%	\multicolumn{1}{c|}{CS} & \multicolumn{2}{c}{Pipe thickness (mm)} \\ 
%	\cline{2-3}
%	\multicolumn{1}{c|}{} & Booster pump & Storage pump \\ 
%	\hline
%	\multicolumn{1}{c|}{1} & \multicolumn{1}{c|}{(4.564 - 4.980]} & (5.084 - 5.450] \\ 
%	\multicolumn{1}{c|}{2} & \multicolumn{1}{c|}{(4.184 - 4.564]} & (4.718 - 5.084] \\ 
%	\multicolumn{1}{c|}{3} & \multicolumn{1}{c|}{(3.732 - 4.184]} & (4.352 - 4.718] \\ 
%	\multicolumn{1}{c|}{4} & \multicolumn{1}{c|}{(3.316 - 3.732]} & (3.986 - 4.352] \\ 
%	\multicolumn{1}{c|}{5} & \multicolumn{1}{c|}{<=3.316} & \multicolumn{1}{c}{<=3.986} \\ 
%	\hline
%\end{tabular}
%	}
%\end{table}

%Note that the range of condition states is mapped from maximum observed thickness to minimum allowable thickness of pipes for booster pumps and storage pumps. 

A simplest way to predict the remaining duration till thickness of pipe reaching alarming level (minimum allowable thickness). This prediction is shown in Table \ref{ch04_thickness_predict}


\begin{figure}[!htb]
	\includegraphics[scale=0.6]{figures/ch04_thickness_predict} \\
	\caption{Thickness prediction}
	\label{ch04_thickness_predict} 
\end{figure}

Inferences from reading the figure are

\begin{itemize}
\item thickness of booster pipes, particularly at the elbow, will reach its minimum allowable thickness at years 18 (or in 2027);
\item thickness of storage pipes, particularly at the elbow, will reach its minimum allowable thickness at years 14 (or in 2023).
\end{itemize}

%It seems critical for pipes of storage pumps as the remaining years till reaching the minimum allowable thickness of 3.62 mm is about 4 years from now.  










\subsubsection{Recommendations}
Given the current thickness of pipe the lack of design information, it is advisable to 
\begin{itemize}
\item Not perform any major intervention on the pipes;
\item Keep regular testing on exact locations using the same type of UTG device. It is important for Maynilad to establish a testing regime for obtaining thickness at exact same location over time (e.g. every year). Information obtained from testing will be then used to compute deterioration rate based on thickness value;
\item Establish an approach to inspect/test the thickness of underground pipe, which is considered to be more vulnerable to leakage and corrosion on external wall;
\item The elbows in the suction and the discharge piping systems must be monitored regularly;
\item It is recommended to have a profiling of the piping systems above and below the ground in order to have a baseline in the analysis of the Maynilad Piping System. In order to have a profiling of pipe thickness at differential time T, additional measurement at similar locations shall be conducted periodically, behavior can then be monitored;
\item Perform coating regularly of the pipe to prevent possible corrosion/errosion and damage that cause by external factors and surrounding condition;
\end{itemize}
%\end{document}