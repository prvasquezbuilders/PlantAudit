% QUALITATIVE
%\begin{document}
\section{Qualitative and Operational Analysis}
\label{42}
\subsection{Facts and Data}

Summary of facts and data concerning operational and overall plan reliability is presented in this subsection.

\subsubsection{Operation Scenario}

\begin{itemize}
	\item Operational since 2011.
	
	\item 3 pumps in operation at 125HP each; Pump 2 is soft start (base load) and 1/3 is with VFD.
	
	\item 4AM – 10PM 2 pumps are operational maintaining 19psi discharge pressure.
	
	\item 10PM – 4AM 1 pump is operational maintaining 12psi discharge pressure.
	
	\item Refilling is done from Pagcor line.
\end{itemize}

%\subsubsection{High Demand Scenario}
%
%\begin{itemize}
%	\item
%	
%	\item
%\end{itemize}
%
%\subsubsection{Low Demand Scenario}
%
%\begin{itemize}
%	\item
%	
%	\item
%
%\end{itemize}

\subsubsection{Spares Policy}

\begin{itemize}
	\item During time of inspection, Pump 4 was under repair/overhaul. %No knowledge when it will be completed.
	\item Switching sequence:
	\begin{itemize}
		\item [$\circ$]	10PM - P2 down; P1 operational
		\item [$\circ$]	12AM - P1 down; P3 operational
		\item [$\circ$] 4AM - P3/P2 operational
		\item [$\circ$] 10PM - P2 down; P3 operational
		\item [$\circ$] 12AM - P3 down; P1 operational
	\end{itemize}
\end{itemize}

\subsubsection{Emergency Situation (loss of electrical power from Meralco)}
\begin{itemize}
	\item Genset on auto-start;
	\item P1/P3 continue operations; P2 on manual start

\end{itemize}

\subsubsection{Maintenance}
\begin{itemize}
	\item For operational problems, operator will call Control Center to report problem.  
	\item Control Center to send contractor within 1-2 hours.
	\item Maintenance contractors conduct a weekly visit to do some maintenance activities.
\end{itemize}

\subsubsection{Current Problems}
\begin{itemize}
	\item P3 VFD fault.  It has been down for the last 4 days and no indication when it will be fixed.  This problem occurred previously and rework has to be scheduled for more repairs of the VFD.
\end{itemize}

\subsection{Recommendations}
In order to ensure the PS to provide adequate level of services around the clock, it is important to establish a good operational scheme that allows optimization of use of pumps to reduce breakdown and to conserve energy. A summary of major recommendations to be considered are:

\begin{itemize}

	\item Monitor installation of P4.  With P4 in service, there will be enough capacity to cater for high demand and unplanned equipment failures and/or extended pump maintenance scenarios.
	
	\item Consider a dedicated duty and a dedicated spare set-up for the pumps.  If this is not acceptable, then consider doing a much longer switch of the storage pumps.  Currently, it is being switched daily to supply 700mm distribution system.  This allows for almost an equal rate of deterioration between the two pumps and if one pump fails due to age-related component failure, the other one is close to a similar failure which may occur before the first pump is fully repaired.  It is suggested that the switch happen once a month or even every 3 months.
	
	\item In place of the longer switching cycle (e.g. every 3 months), there should be a corresponding maintenance program for the standby pump for both booster and storage.
	
	\item Need to know what maintenance activities are done weekly and how the contractors/Maynilad use the information gathered to predict equipment failures.  
	
	\item Develop a more structured discipline in applying routine maintenance work process to ensure that maintenance tasks are given the proper priority in terms of mitigation measures and avoid unplanned shutdown of critical pumps in operation.
		
\end{itemize}

Aside from the above recommendations, we also generate a list of recommendations based on the RCM methodology. This is presented at at the end of the document on Appendix  \ref{app_maintenance}. The list shall be considered as a living program, which requires continuously improvement as part of the total quality management system (refer to Deming cycle presented in GHD's technical proposal).

%\end{document}