%begin{document}

\section{Workplace environment management}
\label{47}

\subsection{Temperature and relative humidity}
\subsubsection{Data}
Data concerning the temperature and relative humidity is presented in Table \ref{ch047_ashrae_psychrometrics}. Data was measured at targeted points shown in Figure \ref{fig_ch02_wem01}. Raw data is with the site inspection reports, which will be provided to the Client separately. Persuant to ASHRAE standard, the recommended ranges for temperature and humidity are [72 - 80 $^\circ F$] and [45 - 60 \%], respectively.

\subsubsection{Data and Analysis}

\begin{table}
	\caption{Temperature and Relative Humidity}
	\label{ch047_tbl_tdb_rh}
	{\footnotesize

\begin{tabular}{l|c|c|c|c}

\hline
Description of Location & \multicolumn{1}{c}{Temperature} &  & \multicolumn{1}{c}{RH} &  \\ 
\cline{2-5}
 & Actual & Range & Actual & Range \\ 
\hline
A &  &  &  &  \\ 
Beside P4 & 84.2 & 72-80 & 66.3 & 45-60 \\ 
At staging area opposite 1 & 84.0 & 72-80 & 66.6 & 45-60 \\ 
Between P3 and P4 & 83.8 & 72-80 & 67.2 & 45-60 \\ 
Near MCC opposite 3 & 83.8 & 72-80 & 67.9 & 45-60 \\ 
Between P2 and P3 & 83.8 & 72-80 & 68 & 45-60 \\ 
Near MCC opposite5 & 83.5 & 72-80 & 66.3 & 45-60 \\ 
Between P1 and P2 & 84.4 & 72-80 & 65.4 & 45-60 \\ 
Near MCC opposite 7 & 84.0 & 72-80 & 65.4 & 45-60 \\ 
Beside P1 & 84.6 & 72-80 & 65.5 & 45-60 \\ 
Near door 3 opposite 9 & 84.6 & 72-80 & 64.4 & 45-60 \\ 
Average & 84.1 & 72-80 & 66.3 & 45-60 \\ 
\hline
B &  &  &  &  \\ 
Open area beside pump room & 33.2 & 72-80 & 56.1 & 45-60 \\ 
Open area beside pump room & 32.1 & 72-80 & 59.3 & 45-60 \\ 
Average & 32.7 & 72-80 & 57.7 & 45-60 \\ 
\hline
C &  &  &  &  \\ 
Comfort room & 30.1 & 72-80 & 62.7 & 45-60 \\ 
Office & 30.5 &  & 64.2 &  \\ 
Average & 30.3 & 72-80 & 63.5 & 45-60 \\ 
\hline
D &  &  &  &  \\ 
Near Guard house & 30.8 & 72-80 & 61.3 & 45-60 \\ 
\hline

\end{tabular}
}
\end{table}

The temperature inside the office registered values outside of the recommended range since the AC is not operated. Back door and windows are opened for natural ventilation inside the office and so readings are similar with that of the out side.

Readings inside the pump room register similar values as the room is equipped with louver panels and is also naturally ventilated. This could still be acceptable since the operators go on hourly rounds to the pump room to record operating parameters and do not stay there for long. They also hydrate regularly to combat the heat and body perspiration. 
%Inside office registered a reading of approximately 73 F. This falls on the lower end of the recommended range. The RH, on the other hand, is about 64 \% and is already outside the recommended range. Although this has minimal impact on comfort of operator staying inside the room, it is best to approach the recommended values to be safe. Note that the Air Conditioning System is installed inside the Office thus the temperature can still be vary based on the thermal comfort needed by the Operators.

%From Table \ref{ch047_tdb_rh.txt}, the temperatures inside the pump house at every measurement points are significant higher than the maximum value of the recommended range (80 F). The average value is around 87 F. High values of temperature compared to the range have also been observed for points outside the pump house. The same is observed for the values of RH.

The temperature and humidity are correlated and as per ASHRAE standard 55 under summer comfort zone, the recommended combination of temperature and humidity shall be within the comfortable zone as shown in Figure \ref{ch047_ashrae_psychrometrics}.

\begin{figure}[!htb]
	\includegraphics[scale=2]{figures/fig_ch047_ashrae_psychrometrics} \\
	\caption{ASHRAE standard 55 : Summer Comfort Zone}
	\label{ch047_ashrae_psychrometrics} 
\end{figure}

\subsubsection{Recommendation}
In order to reduce the negative impacts from high temperature, particularly inside the pump house, the Client shall consider

\begin{itemize}
	\item Establishing a good daily monitoring, exercise, and management considering ergonomic and health and occupational activities (e.g. appropriate time window for break in designated resting area);

\end{itemize}

\subsection{Air quality}\label{aq01}

\subsubsection{Data and analysis}
Data concerning the air quality is presented in Table \ref{ch047_tbl_pm} with value of PM2.5 measured in ppm. Data was measured at targeted points shown in Figure \ref{fig_ch02_wem01}. Raw data is with the site inspection reports, which will be provided to the Client separately. Pursuant to currently applied standard, the recommended safe range for PM2.5 is in [0-35].

\begin{table}
	\caption{Air Quality - PM2.5}
	\label{ch047_tbl_pm}
	{\footnotesize
\begin{tabular}{c|l|c}

\hline
Point & Description of Location & PM2.5 \\ 
\hline
A &  &  \\ 
1 & Beside P4 & 10 \\ 
2 & At staging area opposite 1 & 10 \\ 
3 & Between P3 and P4 & 10 \\ 
4 & Near MCC opposite 3 & 10 \\ 
5 & Between P2 and P3 & 9 \\ 
6 & Near MCC opposite5 & 10 \\ 
7 & Between P1 and P2 & 9 \\ 
8 & Near MCC opposite 7 & 10 \\ 
9 & Beside P1 & 9 \\ 
10 & Near door 3 opposite 9 & 9 \\ 
 & Average & 9.6 \\ 
\hline
B &  &  \\ 
11 & Open area beside pump room & 10 \\ 
12 & Open area beside pump room & 10 \\ 
 & Average & 10.0 \\ 
\hline
C &  &  \\ 
13 & Comfort room & 8 \\ 
14 & Office & 8 \\ 
 & Average & 8.0 \\ 
\hline
D &  &  \\ 
15 & Near Guard house & 10 \\ 
\hline

\end{tabular}
}
\end{table}

The average reading inside and outside the vicinity of the Pump Room are well below danger limits and do not pose any significant risk.


\subsubsection{Recommendation}

Though there is no issue with the air quality, it is anticipated that future problem can incur with a certain low probability, a better management approach is to ensure that all activities/tasks to be executed within the premise of the PS to follow strictly safety and environmental regulation. For example, all employees and staff to wear appropriate dust-proofed masks when working with activities that potentially incurs dusts or other harmful particles.



%\subsection{Hazards}\label{aq02}
%\textcolor{red}{RB Sanchez to write here the summary of raw data collected from visual inspection and testing. Tables shall be used as much as we can. Note that no analysis in this session. This session is purely the high level presentation of data. Raw data can be linked as an Appendix}
\subsection{Illumination}\label{aq03}
\subsubsection{Data and analysis}
Data concerning the illumination is presented in Table \ref{ch047_tbl_illumination} with the LUX value. Data was measured at targeted points shown in Figure \ref{fig_ch02_wem01}. Raw data is with the site inspection reports, which will be provided to the Client separately. Persuant to RULE 1075.4 of DOLE-OSH standard \cite{DOLE2016}, the recommended minimum for LUX is in 100.

\begin{table}
	\caption{Illumination}
	\label{ch047_tbl_illumination}
	{\footnotesize


\begin{tabular}{c|l|c}
\hline
Point & Description of the Point Location & Illumination \\ 
\hline
 & A. &  \\ 
1 & Pump 1  (near storage) & 18 \\ 
2 & Pump 1  (near door) & 106 \\ 
3 & Pump 2 (near storage) & 280 \\ 
4 & Pump 2  (near door) & 18 \\ 
5 & Pump 3  (near storage) & 82 \\ 
6 & Pump 3 (below) & 16 \\ 
7 & Between pump 1 and 2 (below) & 18 \\ 
8 & Between pump 1 and 2 (near storage) & 250 \\ 
 & Average & 94.3 \\ 
\hline
 & B. &  \\ 
9 & Office & 340 \\ 
10 & Electrical room & 320 \\ 
 & Average & 330 \\ 
\hline
 & C. &  \\ 
11 & Parking lobby access (between generator and electrical room) & 6900 \\ 
12 & Payment center outside area & 4200 \\ 
13 & Payment center outside area (left side) & 7500 \\ 
14 & Reservoir left side corner & 1800 \\ 
15 & Reservoir right side corner (near day tank) & 106000 \\ 
16 & Reservoir right side corner (near parking lot) & 7400 \\ 
17 & Outer area near pump room & 2400 \\ 
 & Average & 19457 \\ 
\hline
\end{tabular}
}
\end{table}

The illuminations recorded are well above the standard's minimum values. These mean the illumination inside and outside the pump room is suitable for inpsections, and repairs. Considerable natural lighting come from the room windows (Figure \ref{pump_room_illumination} - a and Figure \ref{pump_room_illumination} - b) during the day.During evenings artificial lighting provides illumination inside the pump room and is also sufficient for inspections and repairs.

\begin{figure}
	\begin{minipage}[b]{0.4\linewidth}
		\centering
		\includegraphics[width=\textwidth]{figures/fig_ch047_wem_illumination1}
		\caption*{a - pump room windows (01)}
		\label{ch047_wem_illumination1}
	\end{minipage}
	\hspace{0.05cm}
	\begin{minipage}[b]{0.4\linewidth}
		\centering
		\includegraphics[width=\textwidth]{figures/fig_ch047_wem_illumination2}
		\caption*{b - pump room windows (02)}
		\label{ch047_wem_illumination2}
	\end{minipage}
\caption{Pump room illumination}
\label{pump_room_illumination}
\end{figure}

The operators office and vicinity are also well lit and above the standard minimum. 

%Illumination of the pump room comes from the interior lighting. There is also additional illumination When the access door is open. 

\subsubsection{Recommendations}

\begin{itemize}
	\item	Use artificial lighting equipment when accessing and conducting activities requiring detailed output at darker specific areas especially the route of operator going to the pump room (stairs and open area outside the pump room)

\end{itemize}


\subsection{Industrial ventilation}\label{aq04}
\subsubsection{Data and analysis}


The room is installed with louver panels that provide natural ventilation (Figure \ref{pump_room_ventilation} - a and Figure \ref{pump_room_ventilation} - b) and is further increased by the open access and back doors. The louver panels have an added screen for better weather proofing (Figure \ref {pump_room_ventilation} - c)The room is situated at higher elevations and thus wind flows into the room more freely.

\begin{figure}
	\begin{minipage}[b]{0.22\linewidth}
		\centering
		\includegraphics[width=\textwidth]{figures/fig_ch047_wem_louverpanel_full1}
		\caption*{a - pump room louvers (01)}
		\label{ch047_wem_louverpanel_full1}
	\end{minipage}
	\hspace{0.05cm}
	\begin{minipage}[b]{0.22\linewidth}
		\centering
		\includegraphics[width=\textwidth]{figures/fig_ch047_wem_louverpanel_full2}
		\caption*{b - pump room louvers (02)}
		\label{ch047_wem_louverpanel_full2}
	\end{minipage}
	\hspace{0.05cm}
	\begin{minipage}[b]{0.22\linewidth}
		\centering
		\includegraphics[width=\textwidth]{figures/fig_ch047_wem_louverpanel_upclose}
		\caption*{c - louver panels with screen}
		\label{ch047_wem_louverpanel_upclose}
	\end{minipage}
	\hspace{0.05cm}
	\begin{minipage}[b]{0.22\linewidth}
		\centering
		\includegraphics[width=\textwidth]{figures/fig_ch047_wem_louverpanel_mcc}
		\caption*{d - louver panels behind MCC}
		\label{ch047_wem_louverpanel_mcc}
	\end{minipage}
\caption{Pump room ventilation}
\label{pump_room_ventilation}
\end{figure}

%Natural ventilation through the access door is not sufficient to attain the minimum Air Changes requirement of the Pump House and so Mechanical Ventilation is utilized most of the time.


%Recommendation for venitalation needed
%\begin{figure}[h]
%\includegraphics[scale=0.6]{figures/ch047_wem_ventilation} \\
%	\caption{Existing ventilation layout}
%	\label{ch047_wem_ventilation} 
%\end{figure}


\subsubsection{Recommendation}

\begin{itemize}
\item The louver panels facing the MCC panel boards (Figure \ref{pump_room_ventilation} - d) need to be repositioned to avoid corrosion of the back boards due to water infiltrating thru the louvers especially during rains with strong winds.


%There is no need to add mechanical ventilation for the pump room however 

\end{itemize}




\subsection{Housekeeping}\label{aq05}
\subsubsection{Documentation}
Following problems are the facts:

\begin{itemize}
\item Current documentation practice is heavily dominated with paper based system, which follows the current practice in Maynilad. There is a large amount/collection of papers that recorded past activities but is of no use and beneficial if data cannot be transformed into digital format for time series analysis, which is an essential part of asset management practice;

\item No proper filing/library system with standardized coding rule that will provide convenience for operators/users to timely find appropriate documents;

\item Daily operation data is crucial information for future analysis but it is recorded in excel based file without relational tables, which makes it from hard to impossible for data compilation, filtering, and mining. Many past data has been recorded with outliers and incorrect data types. 

\end{itemize}

\subsubsection{Waste management and environmental control}
There is no significant issue with management and environmental control as confirmed by the checklist shown in Table \ref{ch047_tbl_housekeeping}

A minor observation that still needs to addressed are lack of appropriate storage. Racks or cabinets could be placed to store miscellaneous items such as cleaning materials, repair tools, assembly parts and/or spares such as those observed during inspection as shown in Figure \ref{pump_room_housekeeping}.

\begin{figure}
	\begin{minipage}[b]{0.3\linewidth}
		\centering
		\includegraphics[width=\textwidth]{figures/fig_ch047_wem_housekeeping6}
		\caption*{a - maintenance clutter (01)}
		\label{ch047_wem_housekeeping6}
	\end{minipage}
	\hspace{0.05cm}
	\begin{minipage}[b]{0.3\linewidth}
		\centering
		\includegraphics[width=\textwidth]{figures/fig_ch047_wem_housekeeping4}
		\caption*{b - maintenance clutter (02)}
		\label{ch047_wem_housekeeping4}
	\end{minipage}
	\hspace{0.05cm}
	\begin{minipage}[b]{0.3\linewidth}
		\centering
		\includegraphics[width=\textwidth]{figures/fig_ch047_wem_housekeeping1}
		\caption*{c - maintenance clutter (03)}
		\label{ch047_wem_housekeeping1}
	\end{minipage}
	\hspace{0.05cm}
	\begin{minipage}[b]{0.3\linewidth}
		\centering
		\includegraphics[width=\textwidth]{figures/fig_ch047_wem_housekeeping2}
		\caption*{d - maintenance clutter (04)}
		\label{ch047_wem_housekeeping2}
	\end{minipage}
	\hspace{0.05cm}
	\begin{minipage}[b]{0.3\linewidth}
		\centering
		\includegraphics[width=\textwidth]{figures/fig_ch047_wem_housekeeping3}
		\caption*{e - maintenance clutter (05)}
		\label{ch047_wem_housekeeping3}
	\end{minipage}
	\hspace{0.05cm}
	\begin{minipage}[b]{0.3\linewidth}
		\centering
		\includegraphics[width=\textwidth]{figures/fig_ch047_wem_housekeeping7}
		\caption*{f - maintenance clutter (06)}
		\label{ch047_wem_housekeeping7}
	\end{minipage}
\caption{Pump room illumination}
\label{pump_room_housekeeping}
\end{figure}

\begin{table}
	\caption{Housekeeping.}
	\label{ch047_tbl_housekeeping}
	{\footnotesize

\begin{tabular}{p{4cm}|c|c|c|p{6cm}}
\hline
Description & Status & CS & IT & Remarks \\ 
\hline
Pump room cleanliness & yes & 1 & 1 & After maintenance or intervention, area should be thoroughly cleaned. Water sumps be dried and grease be wiped \\ 
Sufficient waste segregation assets & yes & 1 & 1 &  \\ 
Waste segregation policy & yes & 1 & 1 &  \\ 
Proper/ appropriate signage & yes & 1 & 1 &  \\ 
Genset emission control & yes &  & 1 &  \\ 

\hline
\end{tabular}
	}
\end{table}


%The plant has its own waste segregation policy and an organized documentations procedure (evidenced of the arranged daily monitoring sheet). Standby generator set is situated outside the pump house where its exhaust gases through natural ventilation will not be able to penetrate the pump house.

\subsubsection{Office arrangement and ergonomic}
Table \ref{ch047_tbl_ergonomics} shows the data concerning parameters associated with office arrangement and ergonomic considerations.

\begin{table}[!h]
	\caption{Ergonomics.}
	\label{ch047_tbl_ergonomics}
	{\footnotesize


\begin{tabular}{p{1.5cm}|p{1.5cm}|c|p{8.5cm}}
\hline
Parameters & Sub-parameters & Status & Remarks \\ 
\hline
Posture & Head & 1 & Ceiling height is high enough to cause head injury while sitting or when standing. \\ 
 & Neck & 1 & Neck posture is in good ergonomic condition. \\ 
 &  &  & Consider having an interval for fit-break to avoid neck muscles stiffening. \\ 
 & Back & 1 & Back posture while sitting is in good posture.  \\ 
 &  &  & Consider standing and doing fit-break exercises to relax spine.  \\ 
 & Hands/Wrist  & 0 & Proper hand positioning in the keyboard is not observed. \\ 
 &  &  & Wrist bending is seldom. \\ 
 & Feet & 1 & Feet position is in good posture. \\ 
 &  &  & Good clearance below worktables. \\ 
 & Eyes & 0 & The computer monitor is on eyelevel in a certain operator only. \\ 
 &  &  & Consider adjusting the monitor level comfortable to every operator. \\ 
 &  &  & Look away into distance in order to rest the eyes for every 10 minutes or so. \\ 
\hline
Equipment / Tool &  &  &  \\ 
 & Computer display & 0 & Not adjusted and the operator get used to its current setting. \\ 
 &  &  & Display brightness must be adjustable in the comfortability of the operator-in-charge. \\ 
 &  &  & Consider the use of anti-glare and blue light to reduce the possibility of eyestrain. \\ 
 & Keyboard & 1 & Keyboard position causes poor hand posture that can lead to injury at long exposure. \\ 
 & Mouse & 1 & Mouse usage is average due to monitoring. \\ 
 &  &  & Prolong usage may cause reduced blood flow leading to muscular injury. \\ 
 & Chair & 0 & Consider using ergonomic chair that is capable of back support, height, armrest adjustments. \\ 
 & Table & 0 & Consider use of ergonomic tables to adjust the height of the table in desired position easily without exerting much effort to adjust manually. \\ 
 & Files & 1 & Hard copy file system and location is well observed. Too high or too low file location may require a person to bend his body or force his hand to grip a file in an awkward posture. Frequent situation may lead to MSD. \\ 
\hline
Operations / Maintenance & Illumination & 0 & According to the maintenance team, the motion-activated light is not bright enough to complete their task efficiently at night. Moreover, the light has short on-off delay operation that means that the team must move more often to avoid the light to dim.  \\ 
 &  &  & Consider having a manual switch option to by-pass the motion sensors and le the light on while doing maintenance.  \\ 
 & Noise Exposure & 1 & Noise emitted by the machines in the pump station is high. Consider the use of proper ear protections to reduce the sound intensity. In offices, the sound intensity is acceptable.  \\ 
 & Temperature & 1 & Temperature in the pump station is not acceptable at long exposure. Consider cooling down the body temperature at the designated area (i.e. outside, office). \\ 
\hline
Facility / General Workplace & Layout & 1 & Layout of the pump station is well observed. Distance between pumps is acceptable for well maintenance movement.  \\ 
 & Height clearances & 1 & Height clearances from ceiling to head is very high. Chance of getting head injury is very low. \\ 
\hline
\end{tabular}

	}
\end{table}


%\begin{table}[h]
    \caption{Housekeeping}
    \label{ch05_tbl_housekeeping}
    \footnotesize{

\begin{tabular}{l|c|c|c|l}
\hline
Description & Status & CS & IT & Remarks \\ 
\hline
- Sufficient waste segregation assets & yes & 1 & 1 &  \\ 
- waste segregation policy & yes & 1 & 1 &  \\ 
- Signage & yes & 1 & 1 &  \\ 
- Genset emission control & yes &  & 1 &  \\ 
\hline
\end{tabular}
}

\end{table}


\subsubsection{Recommendations}
Followings are recommendations

\begin{itemize}
\item Development of a web-based database management system, with appropriate set of relational data tables to record operational data, power consumption data, and intervention data;

\item Development of documentation code and naming for appropriate filing and library/referencing;

\item Applying best practices with regard to ergonomic in combination with interior design and arrangement of office space.

\end{itemize}





\subsection{Noise}\label{aq06}
\subsubsection{Data and analysis}
Data concerning the noise is presented in \ref{ch047_tbl_sound}.Data was measured at targeted points shown in Figure 2.18. Raw data is with the site inspection reports, which will be provided to the Client separately. 

\begin{table}
	\caption{Sound Levels.}
	\label{ch047_tbl_sound}
	{\footnotesize

\begin{tabular}{c|l|c}
\hline
Point & Description of the Point Location & Sound level \\ 
\hline
 & A. &  \\ 
1 & Pump 1  (near storage) & 95.8 \\ 
2 & Pump 1  (near door) & 93.8 \\ 
3 & Pump 2 (near storage) & 93.2 \\ 
4 & Pump 2  (near door) & 94.2 \\ 
5 & Pump 3  (near storage) & 90.5 \\ 
6 & Pump 3 (below) & 94.2 \\ 
7 & Between pump 1 and 2 (below) & 96.9 \\ 
8 & Between pump 1 and 2 (near storage) & 95.7 \\ 
 & Average & 94.3 \\ 
\hline
 & B. &  \\ 
9 & Office & 69.5 \\ 
10 & Electrical room & 70.0 \\ 
 & Average & 69.8 \\ 
\hline
 & C. &  \\ 
11 & Parking lobby access (between generator and electrical room) & 68.0 \\ 
12 & Payment center outside area & 65.4 \\ 
13 & Payment center outside area (left side) & 70.9 \\ 
14 & Reservoir left side corner & 51.7 \\ 
15 & Reservoir right side corner (near day tank) & 59.0 \\ 
16 & Reservoir right side corner (near parking lot) & 69.9 \\ 
17 & Outer area near pump room & 87.6 \\ 
 & Average & 67.5 \\ 
\hline
\end{tabular}
}
\end{table}

Regular operation at 2 pumps running was considered during the sound level measurement and so the reading closely represents the normal daily noise level inside the plant. The average sound level inside the Pump room is at 82.7 dBa which means the operator, maintenance team can inspect or repair inside the room for considerably long without hearing impairment.

Sound levels are recorded around the vicinity of the pump room, office and nearby area of the guard house are below 70 dBa. Sound levels within these areas are considered safe even for prolonged stay.


\subsubsection{Recommendations}

\begin{itemize}
	\item	Continue to use protective hearing equipment when working inside the Pump House to further bring down sound level during repair or maintenance.

\end{itemize}


%end{document}