% Chapter 6
\chapter{Conceptual Design and Reliability Study} % Write in your own chapter title
\label{Chapter6}
%\lhead{Chapter 6. \emph{Conceptual Design}} % Write in your own chapter title to set the page header
%
%%%%%%%%%%%%%%%%%%%%%%%%%%%%
\section{Basis of Design}
%\label{61}
\subsection{As-built drawings}
A collection of as-built drawings are given in A3 print out with electronic files saved both in PDF and CAD formats.

\subsection{Conceptual design}
The conceptual design is provided in the Appendix.



%\subsection{Mechanical design}
%As set of conceptual drawings is given in Appendix %\ref{app_design_mech}.
%\subsection{FDAS design}
%\subsubsection{Fire alarm and detection system}
%\paragraph{\underline{Design criteria}}
%The conceptual design for FDAS has been developed based on findings/results of the audit (refer to subsections \ref{ch04fdas}, design criteria including required code of practice, and the required level of services.
%\begin{itemize}
%\item Individual components shall be compatible with each other and shall be approved and listed by institutions recognized by the relevant authority
%\item The FDAS designer shall have the experience in the proper design, application, installation and testing of FDAS
%\item If the total floor area is more than 8000m2, a semi-addressable system shall be used, otherwise, a conventional system maybe used.
%\item  Automatic detection shall have a complete indoor coverage of building or facilities including all rooms, halls, etc.
%\item  For smoke detectors, the performance characteristics of the detector and the area shall be taken into account when selecting smoke detectors. Smoke detectors shall not be installed in rooms with temperature below 5 degree centigrade, above 45 degree centigrade and with relative humidity above 93\%.
%\item  For heat detectors, temperature rating shall be set at least 11 degree centigrade above maximum expected temperature and is spaces not more than 7.5meters. It shall not be installed in locations where relative humidity is above 93 \% and if the ceiling is more than 4 meters.
%\item  Beam-type smoke detector shall be used if ceilings are more than 6m in height and shall be kept clear of opaque obstacles at all times
%\item  Manual detection is achieved through the manual activation of fire push or pull stations installed at a height of 1.4meters above floor and shall be easily seen and is accessible. It is usually colored red.
%\item  Alarm shall be clearly audible throughout the floor and or/or building where they are installed. It shall have a minimum of 65 dbA or 10db higher than ambient room noise and a maximum of 115 dbA.
%\item  Visual notification shall be used along with audible notification for areas where hearing protection is worn.
%\end{itemize}
%
%\paragraph{\underline{Design}}
%
%%The general layout of the proposed conceptual design is shown in Figure \ref{ch06_fdas_design01}. Relevant drawings are enclosed in the Appendix \ref{app_design_fdas}.
%
%%\begin{figure}[!htb]
%%	\includegraphics[scale=2.5, angle = 90]{figures/ch06_fdas_design01} \\
%%	\caption{Conceptual design - FDAS layout}
%%	\label{ch06_fdas_design01}
%%\end{figure}
%\subsubsection{Lighting protection system}
%\paragraph{\underline{Requirements}}
%
%The design of a lightning protection system needs to: Intercept lightning flash (i.e. create a preferred point of strike) Conduct the lightning current to earth Dissipate current into the earth Create an equipotential bond to prevent hazardous potential differences between LPS, structure and internal elements/circuits
%
%Following points shall be considered
%\begin{itemize}
%\item The lightning protection design shall be in accordance with the requirements of Article 2.90 Protection against Lightning , Philippine Electrical Code 2017 and NFPA 780, standard for the installation of lightning protection systems;
%
%\item The building lightning protection system shall include
%
%\begin{itemize}
%\item[-] Roof mounted copper air terminals;
%\item[-]	Ground rods of lightning protection shall shall be metal clad steel with a diameter of 20mm by 3 meters long connected triangularly with equal distances of 3 meters between two ground rods;
%\item[-]	Down conductor, 50mm2 Bare copper wire and shall be placed in protective conduit (PVC);
%\item[-]	Lightning protection shall have electronically controlled mechanical register which activates registration for every discharge;
%\item[-]	clamps, conduits and auxiliary equipment as required for complete and operational lightning protection system. Materials shall be resistant or protected against  corrosion.
%
%\end{itemize}
%
%\item The company providing the design should have a minimum of 5 years experience and be well versed with Article 2.90 Protection against Lightning , Philippine Electrical Code 2017 and NFPA 780;
%\item 	Lightning protection systems are designed specifically for the building or structures they are intended to protect;
%\item 	The design is not only impacted by the shape and size, but also by building systems and structural components.
%
%\end{itemize}
%
%\paragraph{\underline{Installation}}
%The installation shall be in accordance with the requirements set forth in Article 2.90 Protection against Lightning and NFPA 780 and installed in a neat and workmanlike manner
%
%
%\subsubsection{Ground Electrode Installation and Common Bonding }
%
%\paragraph{\underline{Ground electrode installation and common building}}
%\begin{itemize}
%\item Ground electrodes shall be installed in accordance with NFPA 780 and Article 2.90 of PEC 2017.
%\item 	Common bonding between all building electrode systems shall be installed in accordance with NFPA 780 and PEC 2017
%\item 	Maintain horizontal or vertical runs of ground wires and ensures that all bends have at least 200 mm radius and angle of any bend shall not be less than 90 degrees.
%
%\item  	Lightning carrier cable and down conductor shall be supported every 1.50 meters on center using fabricated copper clamps, bolted to roof slab with plastic expansion sleeves.
%
%\item  	Ground rods should be driven far enough away from the footing and drain tile and also past the roof’s drip edge.
%\item 	Ground rods shall be installed into undisturbed soil.
%\item  	If it is not practical to install ground rods outside of the building, the ground rods should be installed as close to the building’s walls as practical without damaging the footing.
%\item 	The correct ground rod driver adapter should be used to avoid mushrooming or damaging the end of the ground rod.
%\item  	If the damage to the ground rod is to severe, the top of the rod may need to be cut off so that the ground rod clamp or exothermic connection can be properly made
%\item 	For testing and maintenance, access of each ground electrode should be available.
%
%\end{itemize}
%
%\paragraph{\underline{Ground ring electrode installation}}
%
%\begin{itemize}
%\item If required, a ground ring electrode for the lightning protection system shall be installed at least 460mm (18 inches) below earth unless prohibited by ground conditions;
%\item 	A ground ring electrode installed for the purposes of electrical grounding shall be installed to a depth of at least 765 mm (30 inches);
%\item 	Ground ring electrodes shall be continuous around the structure and connected to all down conductors.  The ground ring electrode shall be installed below the line.
%
%\end{itemize}
%
%\paragraph{\underline{field test}}
%Test the grounding system to assure the continuity and that the resistance to ground is not excessive.
%
%
%
%
%%\subsection{WEM design}
%%\textcolor{red}{RB Sanchez to write here the basic of design such as design criteria, calculation, applied codes, objectives, etc. Try to summary it in table form and cite to correct references including as-built drawings, control philosophy, etc}
%%
%%\textcolor{red}{A set of conceptual design drawings shall be referred as enclosed in the Appendix}
%\section{Bill of Materials}
%Based on the recommendations and conceptual design, a high level Bill of Quantity (BOQ) can be generated . The BOQ table includes the condition states and intervention types respectively. %This BOQ will be served as a base for the estimation of life cycle cost.
%

