%\begin{document}
\section{Protection coordination study} \label{ch04_elecaudit_protectioncoordination}
In protection coordination study, the protective devices nearest to the FAULT shall trip first and the remaining of the protective devices shall not be affected. The results were obtained from the ETAB software and shown in Table \ref{ch04_elecaudit_protectioncoordination01}, Table \ref{ch04_elecaudit_protectioncoordination02}, and Table \ref{ch04_elecaudit_protectioncoordination03}.


\begin{table}[!htb]
	\caption{Protective Device Settings - Low Voltage Circuit Breaker with Thermal-Magnetic Trip Device}
	\label{ch04_elecaudit_protectioncoordination01}
	%	\resizebox{\columnwidth}{!}{%
	{\scriptsize
		\begin{tabular}{c|c|c|c|c|c|c|c}
			\hline
			LVCB ID & \multicolumn{1}{l|}{Manufacturer} & \multicolumn{2}{c|}{Breaker} & \multicolumn{2}{c|}{Thermal} & \multicolumn{2}{c|}{Magnetic (Inst.)} \\ 
			\cline{3-8}
			& \multicolumn{1}{l|}{} & Model & Size & Setting & Trip & Setting & Trip \\ 
%			\cline{3-8}
			& \multicolumn{1}{l|}{} &  &  &  & (Amps) &  & (Amps) \\ 
			\hline
			CB4 & \multicolumn{1}{l|}{Fuji Electric} & \multicolumn{1}{l|}{BW400EAG} & 250 & Fixed & 250 & Fixed &  8 xIn \\ 
			CB9 & \multicolumn{1}{l|}{Fuji Electric} & \multicolumn{1}{l|}{BW125JAG} & 100 & Fixed & 100 & Fixed & 8 xIn  \\ 
			CB10 & \multicolumn{1}{l|}{Fuji Electric} & \multicolumn{1}{l|}{BW125JAG} & 100 & Fixed & 100 & Fixed &  8 xIn \\ 
			CB11 & \multicolumn{1}{l|}{Fuji Electric} & \multicolumn{1}{l|}{BW32SAG} & 32 & Fixed & 32 & Fixed & 8 xIn  \\ 
			CB12 & \multicolumn{1}{l|}{Fuji Electric} & \multicolumn{1}{l|}{BW32SAG} & 32 & Fixed & 32 & Fixed & 8 xIn  \\ 
			CB1 & \multicolumn{1}{l|}{Fuji Electric} & \multicolumn{1}{l|}{BW400EAG} & 250 & Fixed & 250 & Fixed & 8 xIn  \\ 
			\hline
		\end{tabular}
		
	}%}
\end{table}



\begin{table}[!htb]
	\caption{Cable-circuit breaker coordination}
	\label{ch04_elecaudit_protectioncoordination02}
		\resizebox{\columnwidth}{!}{%
	{\scriptsize
		
	\begin{tabular}{l|l|l|p{2cm}|p{1cm}|p{1cm}|p{1cm}|p{3cm}|p{2cm}|p{1cm}}
		\hline
		Items & \multicolumn{3}{c|}{Protective Device} & \multicolumn{4}{c|}{Cable Protection} & Max Fault 3Ph-Amps & Refe-rence kV \\ 
		\cline{2-8}
		& Location & ID & Type & Pickup Limit & Ampacity & Damage Curve & Condition &  &  \\ 
		\hline
		Cable1 & Load & CB1 & TM-Magnetic & - & - & Pass & Trip curve protects the damage curve & 2418 & 0.48 \\ 
		&  &  & TM-Thermal & Pass & Pass & Pass & Therm. Trip 250 A is within 302.7 A = Ampacity &  &  \\ 
		&  &  &  &  &  &  & Therm. Trip 250 A is within max. limit of 302.7 A = Ampacity x 100\% &  &  \\ 
		&  &  &  &  &  &  & Trip curve protects the damage curve &  &  \\ 
		\hline
		Cable3 & Load & CB4 & TM-Magnetic & - & - & Pass & Trip curve protects the damage curve & 2391 & 0.48 \\ 
		&  &  & TM-Thermal & Pass & Pass & Pass & Therm. Trip 250 A is within 302.7 A = Ampacity &  &  \\ 
		&  &  &  &  &  &  & Therm. Trip 250 A is within max. limit of 302.7 A = Ampacity x 100\% &  &  \\ 
		&  &  &  &  &  &  & Trip curve protects the damage curve &  &  \\ 
		\hline
	\end{tabular}	
		
	}
}
\end{table}



\begin{table}[!htb]
	\caption{MCCB coordination}
	\label{ch04_elecaudit_protectioncoordination03}
	\resizebox{\columnwidth}{!}{%
		{\scriptsize
			
		\begin{tabular}{l|l|l|l|l|l|l|l|l|l|p{4cm}}
			\hline
			\multicolumn{2}{c|}{Zone} & \multicolumn{2}{c|}{Stream} & \multicolumn{2}{c|}{Max Fault} & Ref.  & Coord. & \multicolumn{2}{c|}{Amp Range} & Condition \\ 
			\cline{1-6}\cline{9-10}
			ID & type & up & down & type & Amp & kV & status & From & To &  \\ 
			&  & PD & PD &  &  &  &  &  &  &  \\ 
			\hline
			Bus4 & Bus & CB4 & CB9 & 3Ph & 2391 & 0.48 & Alert & 2000 & 2000 & Miscoordination, the time gap is smaller than 0.001 sec margin at I=2000 A, Plot Ref. kV=0.48 \\ 
			&  &  &  & L-G &  &  & Warning &  &  & L-G fault coordination is not possible.   \\ 
			&  &  & CB11 & 3Ph & 2391 & 0.48 & Alert & 2000 & 2000 & Miscoordination, the time gap is smaller than 0.001 sec margin at I=2000 A, Plot Ref. kV=0.48 \\ 
			&  &  &  & L-G &  &  & Warning &  &  & L-G fault coordination is not possible.   \\ 
			&  &  & CB12 & 3Ph & 2391 & 0.48 & Alert & 2000 & 2000 & Miscoordination, the time gap is smaller than 0.001 sec margin at I=2000 A, Plot Ref. kV=0.48 \\ 
			&  &  &  & L-G &  &  & Warning &  &  & L-G fault coordination is not possible.   \\ 
			&  &  & CB10 & 3Ph & 2391 & 0.48 & Alert & 2000 & 2000 & Miscoordination, the time gap is smaller than 0.001 sec margin at I=2000 A, Plot Ref. kV=0.48 \\ 
			&  &  &  & L-G &  &  & Warning &  &  & L-G fault coordination is not possible.  \\ 
			\hline
		\end{tabular}
			
		}
	}
\end{table}

Further illustration of the coordination is shown in Figure \ref{fig_ch04_elecaudit_protection_coordination01}. 

\begin{figure}[!htb]
	%	\includepdf[angle = 0]{sections/CHE_1PHSC_with_VFD.pdf}
	\includegraphics[width=\textwidth]{figures/fig_ch04_elecaudit_protection_coordination01.pdf} \\
	\caption{Coordination plot}
	\label{fig_ch04_elecaudit_protection_coordination01} 
\end{figure}

As can be seen from the tables and the figure, there is a mis-coordination at the instantaneous region. Following conclusions can be realized.

\begin{itemize}
\item All trip devices are fixed and can not be adjusted. Hence coordination is deemed to be partial since all branch breaker TCC curves crossed the TCC curve of Main breaker on the instantaneous region. 

\item No ground fault protection provided due to the type of breaker supplied. However , this is allowed under the Philippine Electrical Code;

\end{itemize}

Following is recommendation:

\begin{itemize}
	\item main breaker should be of adjustable and electronic type.
\end{itemize}


