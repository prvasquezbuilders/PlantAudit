\section{Vibration and structural assessment}
\label{45}
\subsection{Measurement and spectrum reading}
Rotating equipment generate vibration waveforms that are mathematical functions of machine dynamics, such as speed, alignment, and rotor balance, among others. Vibration analysis entails measurement and analysis of the amplitude of vibration at certain frequencies to gather useful information relating to the accuracy of shaft alignment and balance, the physical condition of bearings, and the possible effects of structural issues; in the case of Maynilad, the problem of impeller possible cavitation is an added and serious concern.

Three main parameters are measured to determine the severity or amplitude of vibration; namely: displacement, velocity and acceleration. Along with temperature, the vibration level is a primary indicator of the physical condition of a machine. As a generally rule, higher vibration levels indicate greater defects. 

Rotating speeds below 600 rpm (10 Hz) generate minimal acceleration, moderate velocity, but relatively high displacement. Hence, shaft displacement is a critical parameter for slow speed rotors, such as steam turbines. Between 600 – 60000 rpm (10 - 1000 Hz) velocity and acceleration levels provide useful indications of the severity of defects. While velocity as a parameter may indicate the presence or relative magnitude of a problem, it makes no distinction as to the source or cause. This is where an FFT vibration analyzer comes in. A fast Fourier transform algorithm converts acceleration waveforms into functions of frequency in a way suitable-trained humans can distinguish the component sources or causes of the vibration.

By means of a OneProd Falcon high-resolution FFT analyzer equipped with tri-axial accelerometer with a linear frequency range of 2Hz-30kHz, vibration spectral readings were taken from four bearing locations in each motor-pump unit. 

Analysis and results are summarized as follows:

\subsection{Data and analysis}

Raw data of vibration measurement is provided in separately digital format. The raw data of each pump is used to generate a set of graphs provided in Appendix \ref{app_vibrationdata}.

Analytical results on vibration are with the Appendix \ref{app_vibrationdata}. A summary of grading for each pump is given in Table \ref{ch05_tbl_vibration}.

\begin{table}[!h]
	\caption{Pump vibration condition state.}
	\label{ch05_tbl_vibration}
	{\footnotesize
\begin{tabular}{l|l|l|l}
\hline
\multicolumn{1}{c|}{Assets} & Operational issues detected & \multicolumn{2}{c}{Condition} \\ 
\cline{3-4}
\multicolumn{1}{c|}{} &  & \multicolumn{1}{c|}{Motor} & \multicolumn{1}{c}{Pump} \\ 
\hline
\multicolumn{1}{c|}{BP1} & impeller cavitation & \multicolumn{1}{c|}{2} & \multicolumn{1}{c}{3} \\ 
\multicolumn{1}{c|}{} & minor looseness on motor base bolts and/or & \multicolumn{1}{c|}{} & \multicolumn{1}{c}{} \\ 
\multicolumn{1}{c|}{} & minor shaft misalignment & \multicolumn{1}{c|}{} & \multicolumn{1}{c}{} \\ 
\hline
\multicolumn{1}{c|}{BP2} & impeller cavitation & \multicolumn{1}{c|}{2} & \multicolumn{1}{c}{3} \\ 
\hline
\multicolumn{1}{c|}{BP3} & no reading; under repair during time of testing & \multicolumn{1}{c|}{-} & \multicolumn{1}{c}{-} \\ 
\hline
\multicolumn{1}{c|}{BP4} & impeller cavitation & \multicolumn{1}{c|}{3} & \multicolumn{1}{c}{3} \\ 
\multicolumn{1}{c|}{} & shaft misalignment & \multicolumn{1}{c|}{} & \multicolumn{1}{c}{} \\ 
\hline
\multicolumn{1}{c|}{SP1} & early stage motor drive-end (DE) bearing defect & \multicolumn{1}{c|}{3} & \multicolumn{1}{c}{2} \\ 
\hline
\multicolumn{1}{c|}{SP2} & early stage motor drive-end bearing (DE) defect & \multicolumn{1}{c|}{3} & \multicolumn{1}{c}{2} \\ 
\hline
\end{tabular}
	}
\end{table}
It is note that the CS 2, and 3 shown in Table \ref{ch05_tbl_vibration} infers good and fair, respectively \footnote{The CS is slightly different from that defines in Table \ref{ch03:cs}}. 

It can be concluded from the results of vibration study that there is no negative impact of the structural foundation and pads of pumps. 


\subsection{Recommendations}
Pumping station comprising two vertical supply pumps and four vertical booster pumps were constructed with extremely short inlet and outlet pipe sections, leading to inefficient, if not incorrect, operation regimes. Increased power consumption and recurrent damage to impellers and bearings are natural consequences of the said design. From the analysis of vibration spectra, there is a high probability that cavitation has been with the three booster pump units in operation.


Recommendations are shown in Table \ref{ch05_tbl_vibrationre} and Table \ref{ch04_tbl_ch04_11_fot_eccentric_reducer}, which describes the permissible misalignment.


\begin{table}[!h]
	\caption{Recommendation to reduce vibration.}
	\label{ch05_tbl_vibrationre}
	{\footnotesize
		\begin{tabular}{l|l|l|p{8cm}|c}
\hline
\multicolumn{1}{c|}{Assets} & \multicolumn{2}{c|}{Condition} & Recommendations & IT \\ 
\cline{2-3}
\multicolumn{1}{c|}{} & \multicolumn{1}{c|}{Motor} & \multicolumn{1}{c|}{Pump} & \multicolumn{1}{c|}{} &  \\ 
\hline
\multicolumn{1}{c|}{BP1} & \multicolumn{1}{c|}{2} & \multicolumn{1}{c|}{3} & Check motor-pump alignment conditions & 2 \\ 
\multicolumn{1}{c|}{} & \multicolumn{1}{c|}{} & \multicolumn{1}{c|}{} & Mitigate or rectify cavitation ASAP &  \\ 
\multicolumn{1}{c|}{} & \multicolumn{1}{c|}{} & \multicolumn{1}{c|}{} & Follow manufacturer-prescribed dynamic operation to prevent impeller damage &  \\ 
\multicolumn{1}{c|}{} & \multicolumn{1}{c|}{} & \multicolumn{1}{c|}{} & Schedule pump-DE bearing for replacement. &  \\ 
\hline
\multicolumn{1}{c|}{BP2} & \multicolumn{1}{c|}{2} & \multicolumn{1}{c|}{3} & Mitigate or rectify impeller cavitation ASAP & 2 \\ 
\multicolumn{1}{c|}{} & \multicolumn{1}{c|}{} & \multicolumn{1}{c|}{} & Follow manufacturer-prescribed dynamic operation to prevent impeller damage &  \\ 
\multicolumn{1}{c|}{} & \multicolumn{1}{c|}{} & \multicolumn{1}{c|}{} & Schedule pump-DE bearing for replacement &  \\ 
\hline
\multicolumn{1}{c|}{BP3} & \multicolumn{1}{c|}{-} & \multicolumn{1}{c|}{-} & no reading; dismantled under repair & NA \\ 
\hline
\multicolumn{1}{c|}{BP4} & \multicolumn{1}{c|}{3} & \multicolumn{1}{c|}{3} & Align motor and pump shafts per manufacturer’s specifications or Table for permissible shaft alignments for machines & 2 \\ 
\multicolumn{1}{c|}{} & \multicolumn{1}{c|}{} & \multicolumn{1}{c|}{} & Rectify cavitation ASAP;  schedule pump reconditioning to allow inspection of impeller and wear ring for damages &  \\ 
\multicolumn{1}{c|}{} & \multicolumn{1}{c|}{} & \multicolumn{1}{c|}{} & Replace pump-DE bearing &  \\ 
\hline
\multicolumn{1}{c|}{SP1} & \multicolumn{1}{c|}{3} & \multicolumn{1}{c|}{2} & Refresh bearing grease regularly & 2 \\ 
\multicolumn{1}{c|}{} & \multicolumn{1}{c|}{} & \multicolumn{1}{c|}{} & Monitor to assess vibration trend &  \\ 
\hline
\multicolumn{1}{c|}{SP2} & \multicolumn{1}{c|}{3} & \multicolumn{1}{c|}{2} & Refresh bearing grease regularly & 2 \\ 
\multicolumn{1}{c|}{} & \multicolumn{1}{c|}{} & \multicolumn{1}{c|}{} & Monitor vibration regularly to assess trend &  \\ 
\hline
\end{tabular}

	}
\end{table}


\begin{table}
	%	\begin{center}
	\caption{Generic table for permissible misalignment}
	\label{ch04_tbl_ch04_11_fot_eccentric_reducer}
	\includegraphics[scale=0.4]{tables/ch04_21_generic_tolerance_table}
	%	\end{center}
	
\end{table}



%A common problem that occurs to all pumps are issues concerning possible misalignment that can affect the vibration of pumps now and in the future. Some proof of issues are with figures shown in subsection \ref{ch04mech02_highlight}.