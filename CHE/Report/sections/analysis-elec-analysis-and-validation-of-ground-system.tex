\section{Grounding system study} \label{ch04_elecaudit_groundingsystem}
The study has been conducted in accordance with the ITP. Some of representative pictures during measurement are shown in Figure \ref{fig_ch04_elecaudit_groundingsystem}.

\begin{figure}[!h]
	
	\begin{minipage}[b]{0.3\linewidth}
		\centering
		\includegraphics[width=\textwidth]{figures/fig_ch04_elecaudit_grounding_voltage_lapost}
		\caption*{(a - Voltage -LA post)}
	\end{minipage}
	\hspace{0.03cm}
	\begin{minipage}[b]{0.3\linewidth}
	\centering
	\includegraphics[width=\textwidth]{figures/fig_ch04_elecaudit_grounding_barecopperwire}
	\caption*{(b - Bare copper wire)}
\end{minipage}
	\hspace{0.03cm}
\begin{minipage}[b]{0.3\linewidth}
	\centering
	\includegraphics[width=\textwidth]{figures/fig_ch04_elecaudit_grounding_resistance}
	\caption*{(c - Ground resistance measure)}
\end{minipage}
	\hspace{0.03cm}
\begin{minipage}[b]{0.3\linewidth}
	\centering
	\includegraphics[width=\textwidth]{figures/fig_ch04_elecaudit_grounding_voltage_mts}
	\caption*{(d - Voltage measure - MTS)}
\end{minipage}
	\hspace{0.03cm}
\begin{minipage}[b]{0.3\linewidth}
	\centering
	\includegraphics[width=\textwidth]{figures/fig_ch04_elecaudit_grounding_groundpod}
	\caption*{(e - Ground pod connection)}
\end{minipage}
	\hspace{0.03cm}
\begin{minipage}[b]{0.3\linewidth}
	\centering
	\includegraphics[width=\textwidth]{figures/fig_ch04_elecaudit_grounding_bcw}
	\caption*{(f - Voltage on BCW)}
\end{minipage}
	\hspace{0.03cm}
\begin{minipage}[b]{0.3\linewidth}
	\centering
	\includegraphics[width=\textwidth]{figures/fig_ch04_elecaudit_grounding_rod}
	\caption*{(g - Voltage on rod)}
\end{minipage}

	\caption{Grounding system measurement}
	\label{fig_ch04_elecaudit_groundingsystem}
\end{figure}

Results of the study are shown in Table \ref{tbl_ch04_elecaudit_groundsystem} with the following note:

\begin{itemize}
\item The resistance between the main grounding electrode and ground should be no greater than five ohms for large commercial or industrial systems and 1.0 ohm or less for generating or transmission station grounds unless otherwise specified by the owner. (Reference ANSI/IEEE Standard 142) 
\end{itemize}


\begin{table}[!htb]
	\caption{Ground system measurement results}
	\label{tbl_ch04_elecaudit_groundsystem}
		\resizebox{\columnwidth}{!}{%
	{\scriptsize
		\begin{tabular}{p{1.5cm}|p{1.5cm}|l|p{3cm}|p{3cm}|p{2.5cm}|p{3.5cm}}
			\hline
			Locations & Asset/Room & Resistance & Findings & Recommendations & Effects & Risks \\ 
			\hline
			Lightning Arrester Post & Test Point 1 Bare Copper Wire & 1.24 $\Omega$ & Within The 5 $\Omega$ Limit As Per Nfpa And Ieee Standards & (1)Check Tightness Of Connection Of Bcw To Ground Rod  & None & None \\ 
			&  &  &  & (2) Grounding System Electrical And Mechanical Connections Should Be Free Of Corrosion. &  &  \\ 
			&  &  &  & (3) Replace Bcw For Better Conductivity.   &  &  \\ 
			\hline
			Mts Equipment Ground & Test Point 2 Bare Copper Wire & 7.2 V & Measured Voltage In The Bare Copper Wire  & Check And Trace Where The Voltage Is Coming From And Correct The Connection  & Danger To Personnel And Damage To Equipment If Not Immediately Corrected  & Health And Safety Risks For Facilities And Personnel And Damage To Equipment Or Accessories \\ 
			\hline
			Mts Equipment Ground & Test Point 2 Ground Rod & 8.2 V & Same As Mts Equipment & Same As Mts Equipment & Same As Mts Equipment & Same As Mts Equipment \\ 
			\hline
			Genset & Test Point 3 Bare Copper Wire & NA & Connected To  Grounding Busbar Of Mts & Same As Mts Equipment & (1) Unwanted Voltage Maybe Present On Non-Current Carrying Metal Objects  & (1) Incorrect Operation Of Overcurrent Device With Ground Fault Protection  \\ 
			&  &  &  &  &  (2) Equipment Might Be Damaged During A Fault Condition  & (2) Health And Safety Risks For Facilities And Personnel  \\ 
			\hline
		\end{tabular}	
	}
}
\end{table}





