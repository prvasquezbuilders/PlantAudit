\section{Fire protection and safety (FDAS) audit} \label{ch04fdas}


\subsection{Fire alarm and detection system} \label{ch04fdas01}
\subsubsection{Data and analysis}
Summary of data and information from FDAS audit is presented in Table \ref{ch04_tbl_fdas01} with visual images on as-found devices and panels (Figure \ref{ch04_fig_fdas01} and Figure \ref{ch04_fig_fdas02}.%, and Figure \ref{ch04_fig_fdas03}).

\begin{table}[!h]
	\caption{FDAS data highlights 01.}
	\label{ch04_tbl_fdas01}
	{\scriptsize
\begin{tabular}{c|p{6.5cm}|c|p{5.5cm}}
	\hline
	No. & Assets & Status & Remarks \\ 
\hline
A. & VISUAL CHECK OF FIRE ALARM CONTROL PANEL &  &  \\ 
1 & Panel Status, installed and location area & 1 & INSTALLED, LOCATED AT ENTRANCE \\ 
2 & Power indicator lamp operational & 1 &  \\ 
3 & Devices properly indicated and marked & 1 &  \\ 
4 & Panel clear from trouble indicators & 1 &  \\ 
5 & Lamp test indicator operational & 1 &  \\ 
6 & Zones properly indicated and marked & 1 &  \\ 
7 & Check if it’s connected to sprinkler system & 0 & NO SPRINKLER INSTALLED AT SITE \\ 
\hline
B. & CHECKING OF INSTALLED DEVICES &  &  \\ 
1 & Check floor plan lay-out and location of the device if accessible/easy to access & 0 & No floor Plan presented during inspection \\ 
2 & Heat detectors and / or smoke detectors locations acceptable & 1 &  \\ 
3 & Heat detectors and / or smoke detectors indicator lamp functioning & 1 &  \\ 
4 & Pull station locations acceptable & 1 &  \\ 
5 & Bells and buzzers operated correctly &  & FOR VERIFICATION \\ 
6 & Bells and buzzers audibility &  & FOR VERIFICATION \\ 
7 & Strobe lights locations are acceptable & 0 & NO STROBE LIGHT DEVICE \\ 
8 & Strobe light operated correctly & 0 & NO DEVICE \\ 
9 & Are Fire alarm zones (areas) clearly marked & 1 &  \\ 
10 & Is there a maintenance and service contract for the fire alarm system & 1 & INFORMATION SUPPLIED BY OPERATOR AT SITE \\ 
11 & Does the Fire Alarm System smoke detector, heat detector, manual call point , horn and strobe light working and  have a current inspection tag & 0 & NO INSPECTION TAG \\ 
12 & Is the fire alarm system in full working order &  & FOR VERIFICATION \\ 
\hline
\end{tabular}

	}
\end{table}

\begin{figure}
	\begin{minipage}[b]{0.2\linewidth}
		\centering
				\includegraphics[width=\textwidth]{figures/fig_ch04_fdas_firealarmcontrolpannel}
		\caption*{(a - Fire alarm control panel)}
		%		\label{ch02_fdas01}
	\end{minipage}
	\hspace{0.05cm}
	\begin{minipage}[b]{0.2\linewidth}
		\centering
				\includegraphics[width=\textwidth]{figures/fig_ch04_fdas_mpb}
		\caption*{(b -Manual push button)}
		%		\label{ch02_fdas02}
	\end{minipage}
	\hspace{0.05cm}
	\begin{minipage}[b]{0.2\linewidth}
		\centering
			\includegraphics[width=\textwidth]{figures/fig_ch04_fdas_missingbell}
		\caption*{(c -Missing bell)}
		%		\label{ch02_fdas02}
	\end{minipage}
	\hspace{0.05cm}
\begin{minipage}[b]{0.2\linewidth}
	\centering
		\includegraphics[width=\textwidth]{figures/fig_ch04_fdas_sm01}
	\caption*{(d -SM01)}
	%		\label{ch02_fdas02}
\end{minipage}
	\hspace{0.05cm}
\begin{minipage}[b]{0.2\linewidth}
	\centering
		\includegraphics[width=\textwidth]{figures/fig_ch04_fdas_sm02}
	\caption*{(e -SM02)}
	%		\label{ch02_fdas02}
\end{minipage}
	\hspace{0.05cm}
\begin{minipage}[b]{0.2\linewidth}
	\centering
		\includegraphics[width=\textwidth]{figures/fig_ch04_fdas_sm03}
	\caption*{(f -SM03)}
	%		\label{ch02_fdas02}
\end{minipage}
	\hspace{0.05cm}
\begin{minipage}[b]{0.2\linewidth}
	\centering
		\includegraphics[width=\textwidth]{figures/fig_ch04_fdas_sm04}
	%		\label{ch02_fdas02}
\end{minipage}
	\hspace{0.05cm}
\begin{minipage}[b]{0.2\linewidth}
	\centering
		\includegraphics[width=\textwidth]{figures/fig_ch04_fdas_sm05}
	\caption*{(h -SM05)}
	%		\label{ch02_fdas02}
\end{minipage}
	\hspace{0.05cm}
\begin{minipage}[b]{0.2\linewidth}
	\centering
		\includegraphics[width=\textwidth]{figures/fig_ch04_fdas_sm06}
	\caption*{(i -SM06)}
	%		\label{ch02_fdas02}
\end{minipage}
	\caption{As-found devices and panels}
	\label{ch04_fig_fdas01}
\end{figure}

\begin{figure}
	\begin{minipage}[b]{0.25\linewidth}
		\centering
				\includegraphics[width=\textwidth]{figures/fig_ch04_fdas_asv}
		\caption*{(a - ASV booster)}
		%		\label{ch02_fdas01}
	\end{minipage}
	\hspace{0.05cm}
	\begin{minipage}[b]{0.25\linewidth}
		\centering
				\includegraphics[width=\textwidth]{figures/fig_ch04_fdas_fireex01}
		\caption*{(b -Fire extinguisher 01)}
		%		\label{ch02_fdas02}
	\end{minipage}
	\hspace{0.05cm}
	\begin{minipage}[b]{0.25\linewidth}
		\centering
			\includegraphics[width=\textwidth]{figures/fig_ch04_fdas_fireex02}
		\caption*{(c -Fire extinguisher 02)}
		%		\label{ch02_fdas02}
	\end{minipage}
	\caption{Electrical safety}
	\label{ch04_fig_fdas02}
\end{figure}

On the inspection date, it was established that there is Fire Alarm Control with PYGARD brand and the devices are SMOKE DETECTOR “PYGARD” brand. 

It is a conventional fire alarm system which is an early warning system design that detects a fire, that tells the zone/ area of the fire but not the exact location of the fire. The existing design plan did not consider where a specific fire alarm can signal exactly where the fire is occurring.

The existing Fire Detection and Alarm system consists of: 20 pcs of smoke detector, 4 zone indicators, 5 sets manual call point and buzzers, and 4 sets of bells. 

On the testing date, the activity was witnessed by the operator on duty for this pump station.

%Findings are shown in Table \ref{ch04_tbl_fdas02} and Figure \ref{ch04_fig_fdas03}

%\begin{table}[!htb]
	\caption{FDAS findings}
	\label{ch04_tbl_fdas02}
	\includegraphics[width=\textwidth]{tables/ch04_tbl_fdas02.pdf} \\	
	
\end{table}

%\begin{figure}
	%	\includepdf[angle = 0]{sections/CHE_1PHSC_with_VFD.pdf}
	\includegraphics[scale=1]{figures/ch04_fig_fdas03} \\
	\caption{Visual inspection on FDAS and safety devices}
	\label{ch04_fig_fdas03} 
\end{figure}

Highlights of the testing are:

\begin{itemize}
\item 16 out of 20 smoke detector devices were tested, activated after spraying of smoke tester. Location of functioning SD are:
\begin{itemize}
\item the lobby
\item hallway 
\item conference room 
\item file room
\item back room
\item pantry
\item customer care
\item utility room
\item Motor control center room
\item Water network room
\item Water network conference room
\end{itemize}

\item Remaining 4 devices not tested due to unavailability of keys in the IT room, cashier room, and need scaffolding in the pump room;
\item  All manual call point in good condition;
\item Three (3) bells were activated every time the smoke detectors and manual call point were activated;
\item One (1) bell device did not function (not activated-no sound) during testing located at the MCC room;
\item There are no smoke detectors at the chlorine tank house and genset room.
\end{itemize}


%\item Lacking smoke / heat detector devices at substation room and genset room, engineer’s office, pantry and guardhouse. In every close room there should be a device installed to detect heat or smoke. 

%\item Lightning protection system is not in place	



\paragraph{\underline{Recommendations}}

\begin{itemize}
\item Troubleshoot non-functioning bell at the MCC room. Check wiring specially on the termination.
\end{itemize}


\paragraph{\underline{System Testing }}

FDAS shall be subjected to the following tests conforming to the Philippine Electronics Code of 2014 and Philippine Electrical Code of 2017
%\renewcommand{\labelitemi}{$\checkmark$}
\begin{itemize}%[label={\checkmark}]
	\item [$\checkmark$] Testing of insulation resistance and continuity of wires;
	\item [$\checkmark$] Verification of installed devices;
	\item [$\checkmark$] Operation and response of FDAS;
	\item [$\checkmark$] Testing the operation of initiating devices;
	\item [$\checkmark$] Measuring sound pressure level generated by notification devices;
\end{itemize}


\paragraph{\underline{Records }}

Every FDAS system shall keep the following documentations
%\renewcommand{\labelitemi}{$\checkmark$}
\begin{itemize}%[label={\checkmark}]
	\item [$\checkmark$] A complete set of operation and maintenance manuals of the manufacturer covering all equipment used in the system;
	\item [$\checkmark$] A complete set of as-built drawings;
	\item [$\checkmark$] A written sequence of operation;
	\item [$\checkmark$] Record of completion and results of every inspection, testing and maintenance;
	\item [$\checkmark$] Record of components within the database.
\end{itemize}



\subsection{Lighting protection system} \label{ch04fdas02}
\subsubsection{Data and testing}
There is a lightning protection system installed for this in pump station. The strike counter has registered previous strikes (Figure \ref{ch04_fig_fdas_lighting}). 


\begin{figure}
	\begin{minipage}[b]{0.25\linewidth}
		\centering
				\includegraphics[width=\textwidth]{figures/R1P_lightning/grounding.jpg}
		\caption*{(a - Bare copper wire)}
		%		\label{ch02_fdas01}
	\end{minipage}
	\hspace{0.05cm}
	\begin{minipage}[b]{0.25\linewidth}
		\centering
				\includegraphics[width=\textwidth]{figures/R1P_lightning/lightningarrester.jpg}
		\caption*{(b -Lighting arrester rod)}
		%		\label{ch02_fdas02}
	\end{minipage}
	\hspace{0.05cm}
	\begin{minipage}[b]{0.25\linewidth}
		\centering
			\includegraphics[width=\textwidth]{figures/R1P_lightning/lightningrecorder.jpg}
		\caption*{(c -Strike counter)}
		%		\label{ch02_fdas02}
	\end{minipage}
	\caption{Lighting devices}
	\label{ch04_fig_fdas_lighting}
\end{figure}

Testing the connection of the continuity of bare copper wire connecting the lightning arrester indicated acceptable values but already on the marginal side. 




\subsubsection{Recommendations}

\paragraph{\underline{Short term Recommendations}}

It is advisable to continue monitoring the lighting system with future tests possible to detect the statistical trend that the system will fail to provide adequate level of services.
%\renewcommand{\labelitemi}{$\checkmark$}


\paragraph{\underline{Long term Recommendations}}

%\renewcommand{\labelitemi}{$\checkmark$}
\begin{itemize}%[label={\checkmark}]
	\item [$\checkmark$] According to the standard, an inspection should be undertaken during the construction of the structure, after the installation, after alterations or repairs, and when it is known that the structure has been struck by lightning;
	\item [$\checkmark$] It is also recommended that inspections take place “periodically at such intervals as determined with regard to the nature of the structure to be protected”, taking into account the local environment, such as corrosive soils and corrosive atmospheric conditions and the type of protection measures employed;
	\item [$\checkmark$]The inspection comprises checking the technical documentation, visual inspections and test measurements;
	\item [$\checkmark$]Prepare an inspection guide to facilitate the inspection process containing sufficient information on the installation and its components, tests methods and previous inspection/test data;	
	\item [$\checkmark$]During the visual inspection, the following should be checked;	
	\begin{itemize}
		\item [-] the deterioration and corrosion of air-termination elements, conductors and connections
		\item [-]	the corrosion of earth electrodes
		\item [-]	the earthing resistance value for the earth-termination system
		\item [-]	the condition of connections, equipotential bonding and fixings.
		
	\end{itemize}
	
	\item [$\checkmark$] For those parts of an earthing system and bonding network not visible for inspection, tests of electrical continuity should be performed;
	
	\item [$\checkmark$] An inspection report should be prepared detailing the status of the system, any deviations from the technical documentation and the results of any measurements undertaken. Any obvious faults should also be reported.
\end{itemize}

No lightning protection system is 100\% effective. A system designed in compliance with the standard does not guarantee immunity from damage. Lightning protection is an issue of statistical probabilities and risk management. A system designed in compliance with the standard should statistically reduce the risk to below a pre-determined threshold. The IEC 62305-2 risk management process provides a framework for this analysis. An effective lightning protection system needs to control a variety of risks. While the current of the lightning flash creates a number of electrical hazards, thermal and mechanical hazards also need to be addressed. 

Risk to persons (and animals) include: 

\begin{itemize}
\item Direct flash;
\item  Step potential ;
\item Touch potential ;
\item  Side flash ;
\item Secondary effects

\begin{itemize}
	
	 \item[-]  asphyxiation from smoke or injury due to fire 
	\item [-] structural dangers such as falling masonry from  point of strike 
	\item [-] unsafe conditions such as water ingress from roof  penetrations causing electrical or other hazards,  failure or malfunction of processes, equipment and  safety systems

\end{itemize}
\end{itemize}




Risk to structures \& internal equipment include: 

\begin{itemize}
\item Fire and/or explosion triggered by heat of lightning flash,  its attachment point or electrical arcing of lightning  current within structures ;
\item  Fire and/or explosion triggered by ohmic heating of  conductors or arcing due to melted conductors;
\item Punctures of structure roofing due to plasma heat  at lightning point of strike ;
\item Failure of internal electrical and electronic systems ;
\item Mechanical damage including dislodged materials at  point of strike.
\end{itemize}


\subsection{Ground-Fault circuit interrupter (GFCI) or electric leakage circuit breaker (ELCB) or Residual circuit devices (RCD)} \label{ch04fdas03}
\subsubsection{Data and analysis}
No ground fault circuit interrupter (GFCI) or earth leakage Circuit breaker (ELCB) protection was installed in the panel for FDAS for this PS.


%\subsubsection{Recommendations}
%Refer to the conceptual design in Chapter \ref{Chapter6}
\subsection{Electrical safety and protective devices} \label{ch04fdas04}
\subsubsection{Data and analysis}
The pump station in general is well maintained and the area station where the operator stays is free from all obstructions. There were some marginal findings of safety issues but otherwise the station is well managed. The station has observed good housekeeping and cleanliness of the pump station.

Some highlights on fire protection are with the Table \ref{ch04_tbl_fdas_safety} and figure \ref{ch04_fig_fdas_safety}

\begin{table}[!h]
	\caption{Highlights on Fire protection data}
	\label{ch04_tbl_fdas_safety}
	{\scriptsize
\begin{tabular}{c|l|c|l}
	\hline
	Items & Visual inspection on & Status & Remarks \\ 
	\hline
	1 & Evacuation Plan & 1 & Posted  \\ 
	2 & Fire Extinguishers & 1 & Green FEX(HCFC) and Red FEX dry chemical \\ 
	3 & Fire Exits & 1 & All doors have exit signages \\ 
	4 & Fire Hose Cabinet & 0 & Not found \\ 
	5 & Fire Sprinkler System & 0 & No sprinkler system \\ 
	6 & Emergency Exit Signages & 1 & Found in every door \\ 
	7 & Emergency Lights & 1 & Emergency lights are functioning (Aglow) Mosy has no inspection tag \\ 
	8 & Ppe Cabinet & 0 & Not found \\ 
	\hline
\end{tabular}
	}
\end{table}

\begin{figure}
	\begin{minipage}[b]{0.22\linewidth}
		\centering
				\includegraphics[width=\textwidth]{figures/R1P_safety/emergencyexitsignale01.jpg}
		\caption*{(a - Emergency exit 01)}
		%		\label{ch02_fdas01}
	\end{minipage}
	\hspace{0.05cm}
	\begin{minipage}[b]{0.22\linewidth}
		\centering
				\includegraphics[width=\textwidth]{figures/R1P_safety/emergencyexitsignale02.jpg}
		\caption*{(b -Emergency exit 02)}
		%		\label{ch02_fdas02}
	\end{minipage}
	\hspace{0.05cm}
	\begin{minipage}[b]{0.22\linewidth}
		\centering
			\includegraphics[width=\textwidth]{figures/R1P_safety/fex_no_update_tage.jpg}
		\caption*{(c - no update tag)}
	\end{minipage}
	\hspace{0.05cm}
\begin{minipage}[b]{0.22\linewidth}
	\centering
	\includegraphics[width=\textwidth]{figures/R1P_safety/fire_exhtinguig_genset.jpg}
	\caption*{(d -fire extinguisher -genset)}
\end{minipage}
	\hspace{0.05cm}
\begin{minipage}[b]{0.22\linewidth}
	\centering
	\includegraphics[width=\textwidth]{figures/R1P_safety/fireexting_checklisht.jpg}
	\caption*{(e -fire extinguisher -red)}
\end{minipage}
	\hspace{0.05cm}
\begin{minipage}[b]{0.22\linewidth}
	\centering
	\includegraphics[width=\textwidth]{figures/R1P_safety/missing_fex.jpg}
	\caption*{(f -Missing flex)}
\end{minipage}
	\hspace{0.05cm}
\begin{minipage}[b]{0.22\linewidth}
	\centering
	\includegraphics[width=\textwidth]{figures/R1P_safety/r1p_evacuation_plan.jpg}
	\caption*{(g -Evacuation plan)}
\end{minipage}
	\hspace{0.05cm}
\begin{minipage}[b]{0.22\linewidth}
	\centering
	\includegraphics[width=\textwidth]{figures/R1P_safety/unprotected_stair_reservoir.jpg}
	\caption*{(h -Unprotected stair)}
\end{minipage}
	\hspace{0.05cm}
\begin{minipage}[b]{0.22\linewidth}
	\centering
	\includegraphics[width=\textwidth]{figures/R1P_safety/manhole_no_signal.jpg}
	\caption*{(i -No signal manhole)}
\end{minipage}
	\hspace{0.05cm}
\begin{minipage}[b]{0.22\linewidth}
	\centering
	\includegraphics[width=\textwidth]{figures/R1P_safety/manhole_no_warning.jpg}
	\caption*{(j -No warning signale)}
\end{minipage}
	\hspace{0.05cm}
\begin{minipage}[b]{0.22\linewidth}
	\centering
	\includegraphics[width=\textwidth]{figures/R1P_safety/nochecklist_emergengylight.jpg}
	\caption*{(k -No checklist)}
\end{minipage}
	\caption{Safety}
	\label{ch04_fig_fdas_safety}
\end{figure}

Facts obtained from inspection are also presented in Table \ref{ch04_tbl_fdas_safety01}.

%\subsubsection{Recommendations}
%Based on the status of devices, recommendations are with intervention types shown in Table \ref{ch04_tbl_fdas03}. Chapter \ref{Chapter6} further illustrates the recommendation with the conceptual design.







%\subsection{Recommendations}
