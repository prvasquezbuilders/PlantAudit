\section{Electrical Integrity} \label{ch04_elecaudit_integrity}

\subsection{Basic for testing}
Integrity test was conducted based on the following basics:

\begin{itemize}
\item Insulation resistance values should be in accordance with the manufacturers published data. Values less than the insulation resistance less than the manufacturer’s recommendations should be investigated. In the absence of the manufacturer’s data, NETA (National Electrical Testing Association) values for IR were  used to determine suitability of existing apparatus to remain in service.

\item If no evidence of distress or insulation failure is observed by the end of the total time of voltage application during the over-potential test, the test specimen is considered to have passed the test. 

\item The acceptable insulation resistance value for electrical apparatus and systems for a nominal rating of 600 Volts is minimum 100Megohms at a test voltage of 1000 volts DC.

\item Insulation test may be used to establish a trending pattern. Deviations from the baseline information permit evaluation of the insulation.


\end{itemize}

\subsection{Results}
Results of the test are presented in Table \ref{tbl_ch04_elecaudit_integrity}

\begin{table}[!htb]
	\caption{Insulation resistance test- results}
	\label{tbl_ch04_elecaudit_integrity}
	%	\resizebox{\columnwidth}{!}{%
	{\scriptsize
\begin{tabular}{c|p{4cm}|l|c|c|c|c|c|c|l}
	\hline
	No. & Description & Voltage & \multicolumn{6}{c|}{Connectivity ($M\Omega$)} & Remarks \\ 
	\cline{4-9}
	&  &  & L1-L2 & L2-L3 & L3-L1 & L1-G & L2-G & L3-G &  \\ 
	\hline
	1 & Normal Side Incoming power cable from ECB 1250AT outdoor to noraml ACB from ECB 1250AT outdoor to noraml ACB & 1kV & >2000 & >2000 & >2000 & 1423 & 1358 & 1179 & Within limits \\ 
	2 & Emergency side incoming power cable from genset breaker to ATS emergency ACB & 1kV & 868 & 930 & 1044 & 518 & 530 & 531 & Within limits \\ 
	3 & Common load busbar  of ATS 1250AT to lineside of sub-main MCCB 1250AT & 1kV & >2000 & >2000 & >2000 & >2000 & >2000 & >2000 & Within limits \\ 
	4 & Common main busbar of MCC & 1kV & >2000 & >2000 & >2000 & >2000 & >2000 & >2000 & Within limits \\ 
	\hline
\end{tabular}
	}%}
\end{table}

It can be confirmed from the data/results of the insulation test that values at the time of the test were within the limits.

