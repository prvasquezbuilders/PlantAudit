%\begin{document}
\section{Protection coordination study} \label{ch04_elecaudit_protectioncoordination}
In protection coordination study, the protective devices nearest to the FAULT shall trip first and the remaining of the protective devices shall not be affected. The results were obtained from the ETAB software and shown in Table \ref{ch04_elecaudit_protectioncoordination01}, Table \ref{ch04_elecaudit_protectioncoordination02}, and Table \ref{ch04_elecaudit_protectioncoordination03}.

\begin{table}[!htb]
	\caption{Protective Device Settings - Low Voltage Circuit Breaker with Thermal-Magnetic Trip Device}
	\label{ch04_elecaudit_protectioncoordination01}
	%	\resizebox{\columnwidth}{!}{%
	{\scriptsize
	\begin{tabular}{c|c|c|c|c|c|c|c}
		\hline
		LVCB ID & \multicolumn{1}{l|}{Manufacturer} & \multicolumn{2}{c|}{Breaker} & \multicolumn{2}{c|}{Thermal} & \multicolumn{2}{c}{Magnetic (Inst.)} \\ 
		\cline{3-8}
		& \multicolumn{1}{l|}{} & Model & Size & Setting & Trip & Setting & Trip \\ 
		\cline{3-8}
		& \multicolumn{1}{l|}{} &  &  &  & (Amps) &  & (Amps) \\ 
		\hline
		CB8 & \multicolumn{1}{l|}{Schneider} & \multicolumn{1}{l|}{EZC} & 63 & 100\% & 63 & Fixed &  \\ 
		CB9 & \multicolumn{1}{l|}{Schneider} & \multicolumn{1}{l|}{EZC} & 63 & 100\% & 63 & Fixed &  \\ 
		CB7 & \multicolumn{1}{l|}{Schneider} & \multicolumn{1}{l|}{EZC} & 250 & 90\% & 225 & 5X & 1250 \\ 
		CB1 & \multicolumn{1}{l|}{Siemens} & \multicolumn{1}{l|}{JFC SENTRON} & 500 & Fixed & 500 & 5 & 4710 \\ 
		CB3 & \multicolumn{1}{l|}{Siemens} & \multicolumn{1}{l|}{JFC SENTRON} & 500 & Fixed & 500 & 3 & 3860 \\ 
		\hline
	\end{tabular}
	
	}%}
\end{table}

\begin{table}[!htb]
	\caption{Cable-circuit breaker coordination}
	\label{ch04_elecaudit_protectioncoordination02}
		\resizebox{\columnwidth}{!}{
	{\scriptsize
	\begin{tabular}{l|l|l|p{2cm}|p{1cm}|p{1cm}|p{1cm}|p{3cm}|p{2cm}|p{1cm}}
\hline
Items & \multicolumn{3}{c|}{Protective Device} & \multicolumn{4}{c|}{Cable Protection} & Max Fault 3Ph-Amps & Refe-rence kV \\ 
\cline{2-8}
& Location & ID & Type & Pickup Limit & Ampacity & Damage Curve & Condition &  &  \\ 
\hline
Cable1 & Load & CB1 & TM-Thermal & Pass & Warning & Pass & Cable is underutilized due to Therm. Trip 500 A < 18161 A (242.1 A \@ 34.5 kV) = Ampacity x 80\% & 1094 & 0.46 \\ 
Cable1 & Load & CB1 & TM-Thermal & Pass & Warning & Pass & Therm. Trip 500 A is within 22701 A (302.7 A \@ 34.5 kV) = Ampacity & 1094 & 0.46 \\ 
Cable1 & Load & CB1 & TM-Thermal & Pass & Warning & Pass & Therm. Trip 500 A is within max. limit of 22701 A (302.7 A \@ 34.5 kV) = Ampacity x 100\% & 1094 & 0.46 \\ 
Cable1 & Load & CB1 & TM-Thermal & Pass & Warning & Pass & Trip curve protects the damage curve & 1094 & 0.46 \\ 
\hline
Cable2 & Load & CB1 & TM-Thermal & Alert & Alert & Pass & Therm. Trip 500 A above max. limit of 302.7 A = Ampacity x 100\% & 1094 & 0.46 \\ 
Cable2 & Load & CB1 & TM-Thermal & Alert & Alert & Pass & Therm. Trip 500 A is set above 302.7 A = Ampacity & 1094 & 0.46 \\ 
Cable2 & Load & CB1 & TM-Thermal & Alert & Alert & Pass & Trip curve protects the damage curve & 1094 & 0.46 \\ 
\hline
Cable3 & Load & CB3 & TM-Thermal & Alert & Alert & Pass & Therm. Trip 500 A above max. limit of 138.9 A = Ampacity x 100\% & 1086 & 0.46 \\ 
Cable3 & Load & CB3 & TM-Thermal & Alert & Alert & Pass & Therm. Trip 500 A is set above 138.9 A = Ampacity & 1086 & 0.46 \\ 
Cable3 & Load & CB3 & TM-Thermal & Alert & Alert & Pass & Trip curve protects the damage curve & 1086 & 0.46 \\ 
\hline
Cable6 & Load & CB12 & TM-Thermal & Alert & Alert & Pass & Therm. Trip 400 A above max. limit of 138.9 A = Ampacity x 100\% & 1138 & 0.46 \\ 
Cable6 & Load & CB12 & TM-Thermal & Alert & Alert & Pass & Therm. Trip 400 A is set above 138.9 A = Ampacity & 1138 & 0.46 \\ 
Cable6 & Load & CB12 & TM-Thermal & Alert & Alert & Pass & Trip curve protects the damage curve & 1138 & 0.46 \\ 
\hline
Cable7 & Load & CB7 & TM-Magnetic & - & - & Pass & Trip curve protects the damage curve & 1248 & 0.46 \\ 
Cable7 & Load & CB7 & TM-Thermal & Alert & Alert & Pass & Therm. Trip 225 A above max. limit of 91.4 A = Ampacity x 100\% & 1248 & 0.46 \\ 
Cable7 & Load & CB7 & TM-Thermal & Alert & Alert & Pass & Therm. Trip 225 A is set above 91.4 A = Ampacity & 1248 & 0.46 \\ 
Cable7 & Load & CB7 & TM-Thermal & Alert & Alert & Pass & Trip curve protects the damage curve & 1248 & 0.46 \\ 
\hline
Cable8 & Load & CB10 & TM-Thermal & Pass & Warning & Pass & Cable is underutilized due to Therm. Trip 150 A < 222.3 A (111.1 A \@ 0.46 kV) = Ampacity x 80\% & 1862 & 0.23 \\ 
Cable8 & Load & CB10 & TM-Thermal & Pass & Warning & Pass & Therm. Trip 150 A is within 277.9 A (138.9 A \@ 0.46 kV) = Ampacity & 1862 & 0.23 \\ 
Cable8 & Load & CB10 & TM-Thermal & Pass & Warning & Pass & Therm. Trip 150 A is within max. limit of 277.9 A (138.9 A \@ 0.46 kV) = Ampacity x 100\% & 1862 & 0.23 \\ 
Cable8 & Load & CB10 & TM-Thermal & Pass & Warning & Pass & Trip curve protects the damage curve & 1862 & 0.23 \\ 
\hline
	\end{tabular}	
		
	}
}
\end{table}




\begin{table}[!htb]
	\caption{MCCB coordination}
	\label{ch04_elecaudit_protectioncoordination03}
	\resizebox{\columnwidth}{!}{%
		{\scriptsize
			
		\begin{tabular}{l|l|l|l|l|l|l|l|l|l|p{4cm}}
	\hline
	\multicolumn{2}{c|}{Zone} & \multicolumn{2}{c|}{Stream} & \multicolumn{2}{c|}{Max Fault} & Ref.  & Coord. & \multicolumn{2}{c|}{Amp Range} & Condition \\ 
	\cline{1-6}\cline{9-10}
	ID & type & up & down & type & Amp & kV & status & From & To &  \\ 
	&  & PD & PD &  &  &  &  &  &  &  \\ 
	\hline
	Bus4 & Bus & CB3 & CB5 & 3Ph & 1201 & 0.46 & Pass & 317.9 & 1201 & Devices are coordinated between amp range of 317.9 A to 1201A \@ 0.46 kV \\ 
	& Bus & CB3 & CB7 & 3Ph & 1248 & 0.46 & Pass & 282.4 & 1248 & Devices are coordinated between amp range of 282.4 A to 1248A \@ 0.46 kV \\ 
	& Bus & CB3 & CB12 & 3Ph & 1138 & 0.46 & Alert & 528.5 & 1138 & Miscoordination, downstream trip curve is above and right of upstream \\ 
	\hline
	Bus8 & Bus & CB7 & CB9 & 3Ph & 1248 & 0.46 & Alert & 1125 & 1125 & Miscoordination, the time gap is smaller than 0.001 sec margin at I=1125 A, Plot Ref. kV=0.46 \\ 
	& Bus & CB7 & CB8 & 3Ph & 1248 & 0.46 & Alert & 1125 & 1125 & Miscoordination, the time gap is smaller than 0.001 sec margin at I=1125 A, Plot Ref. kV=0.46 \\ 
	\hline
		\end{tabular}
			
		}
	}
\end{table}


Further illustration of the coordination is shown in Figure \ref{fig_ch04_elecaudit_protection_coordination01}. 

\begin{figure}[]
	%	\includepdf[angle = 0]{sections/CHE_1PHSC_with_VFD.pdf}
	\includegraphics[width=\textwidth]{figures/fig_ch04_elecaudit_protection_coordination01.pdf} \\
	\caption{Coordination plot}
	\label{fig_ch04_elecaudit_protection_coordination01} 
\end{figure}

Following remarks/recommendations can be interpreted from values shown in the tables and figure.

\begin{itemize}
\item Trip curve is covered by other protective functions or clipped by fault current.

\item Some of the Circuit breaker trip devices are fixed and can not be adjusted. Hence coordination is deemed to be partial since some of the  branch breaker TCC curves crossed the TCC curve of Main breaker on the instantaneous region.

\item L-G fault coordination is not possible. Trip unit of protective device has no ground fault protection provided due to the type of breaker supplied. However , this is allowed under the Philippine Electrical Code.

\item For better coordination, main breaker should be of adjustable and electronic type.


\end{itemize}


