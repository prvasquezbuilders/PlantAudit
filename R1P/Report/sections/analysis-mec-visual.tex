%\begin{document}

\section{Visual Inspection on Pipe, valves, fittings, supports, expansions, and appurtenances}
\label{ch04mech02}


\subsection{Highlights}
\label{ch04mech02_highlight}


Visual inspection data on pipes, valves, fittings, supports, expansions, and appurtenances is highlighted in Table \ref{ch043_tbl_visualinspectionHL}.

\begin{table}[!htb]
	\caption{Highlights of visual inspection}
	\label{ch043_tbl_visualinspectionHL}
%	\resizebox{\columnwidth}{!}{%
	{\scriptsize

\begin{tabular}{c|p{3cm}|p{9.5cm}}

\hline
No. & Items & Remarks \\ 
\hline
1 & Existing suction pipe and fittings & Suction line too short and is jam packed with fittings. Does not promote good flow development. Intake water will be turbulent and is not desired \\ 
2 & Discharge piping and fittings & Pressure gauge near pump discharge is prefered for measuring head of indivual pump \\ 
2 & Pump vibration isolation  & Does not appropriately serve its function to isolate vibration from building and reduce noise \\ 
3 & As built difference & Actual system contain many differences from provided copy of old as-built including valve positions and pipe design \\ 
4 & Pump foundation block & Unitary block with multiple slots and may  \\ 
5 & Space & Modifications for improvement are possible because of available space inside pump room \\ 
6 & Instrumentation and monitoring & Pump instrumentation do not include PLC and other important parameters typically displayed by PLC not monitored \\ 
\hline

\end{tabular}

	}%}
\end{table}

Visual inspections are supported with the photos taken at particular locations/positions in question.

Relative to the station's desired capacity and reservoir size, the area allotted for the pump system is rather very limited (Figure \ref{ch043_psr} - a and Figure \ref{ch043_pump_layout_actual}). 3 HSC pumps each with its motor, valves and pipes and fittings were fitted inside the limited space resulting to a compact arrangement. This led to a few compromises regarding pipe design.

To increase suction head, the pumps are installed at lower elevation close to bottom level of reservoir. The pump system was recently rehabilitated and new pumps were installed along with its auxillaries. Corresponding foundation blocks were also built beside the old foundations (Figure \ref{ch043_psr} - b).

\begin{figure}[!htb]
	\begin{minipage}[b]{0.4\linewidth}
		\centering
		\includegraphics[width=\textwidth]{figures/ch043_plant_layout}
		\caption*{a - plant layout (01)}
		\label{ch043_plant_layout}
	\end{minipage}
	\hspace{0.05cm}
	\begin{minipage}[b]{0.5\linewidth}
		\centering
		\includegraphics[width=\textwidth]{figures/ch043_pump_layout}
		\caption*{b - pump layout (02)}
		\label{ch043_pump_layout}
	\end{minipage}
%\end{minipage}
\caption{PSR layouts}
\label{ch043_psr}
\end{figure}




\begin{figure}[!htb]
	\begin{minipage}[b]{0.5\linewidth}
		\centering
		\includegraphics[width=\textwidth]{figures/ch043_pump_layout_actual}
		\caption{Pump system}
		\label{ch043_pump_layout_actual}
	\end{minipage}
	\hspace{0.05cm}
	\begin{minipage}[b]{0.5\linewidth}
		\centering
		\includegraphics[width=\textwidth]{figures/ch043_pump_discharge}
		\caption{Pump discharge}
		\label{ch043_pump_discharge}
	\end{minipage}
\end{figure}



The distance of the main discharge header and the pump discharge flange was small and so the design opted for higher pump centerline elevation to allow space for the placement of check valve. As a consequence of increasing the pump centerline elevation, existing pipe connecting the reservoir and the old pump had to reach the new pump centerline and so a series of two elbows was used. (Figure \ref{ch043_pump_discharge})


The water leaves the reservoir and passes thru two successive elbows finally a rubber bellow-type flexible coupling before it enters the pump intake. High turbulence of entering water is expected. Further inconsistency in flow profile may be expected and vacuum spots may occur within the suction piping. (Figures \ref{ch043_spd} - a, \ref{ch043_spd} - b, \ref{ch043_spd} - c)

\begin{figure}[!htb]
	\begin{minipage}[b]{0.3\linewidth}
		\centering
		\includegraphics[width=\textwidth]{figures/ch043_double_elbow_suction1}
		\caption*{a - double elbow suction (01)}
		\label{ch043_double_elbow_suction1}
	\end{minipage}
	\hspace{0.05cm}
	\begin{minipage}[b]{0.3\linewidth}
		\centering
		\includegraphics[width=\textwidth]{figures/ch043_double_elbow_suction2}
		\caption*{b - double elbow suction (02)}
		\label{ch043_double_elbow_suction2}
	\end{minipage}
	\hspace{0.05cm}
	\begin{minipage}[b]{0.3\linewidth}
		\centering
		\includegraphics[width=\textwidth]{figures/ch043_double_elbow_suction3}
		\caption*{c - double elbow suction (03)}
		\label{ch043_double_elbow_suction3}
	\end{minipage}
\caption{suction pipe design}
\label{ch043_spd}
\end{figure}

This is not desired as it will reduce performance and life of pump when flow profile of water entering pump is not relatively developed. Basic simulation of flow predicts show flow profile. Ideally, a fully developed flow entering pump intake is desired to avoid problems in suction by following good suction pipe design practice. Elbow should never be bolted directly to the pumps suction nozzle. This will result to noisy operation loss in efficiency and capacity and heavy end thrust.

The abrupt connection between the short 200 mm short pipe (just after the elbow) and the 250 mm bellow-type flexible coupling is unideal and will promote turbulence near the pump inlet and further reduce pump performance and life.(Figure \ref{ch043_elbow_to_fj_connection})

\begin{figure}[h]
	%	\begin{center}
	\includegraphics[scale=0.6]{figures/ch043_elbow_to_fj_connection} 
	%	\end{center}
	\caption{Abrupt connection}
	\label{ch043_elbow_to_fj_connection}
\end{figure} 


It can be said that such a minor details in the pipe suction design will prove to be a great factor in pump performance and longevity. Pumps, and especially centrifugal pumps, work most efficiently when the fluid is delivered in a  surge-free, smooth, laminar flow. Any form of turbulence reduces efficiency and increases wear and tear on the pump’s bearings, seals and other components. There should be at least 5 pipe diameters’ worth of straight piping connecting to the pump. Never connect an elbow, reducer, valve, or strainer within this final run of pipework. If an elbow is directly connected to the pump flange, the fluid is effectively centrifuged towards the outer curve of  the elbow and not directed into the centre of the impeller. This creates stress on the pump’s bearings and seals which could lead to wear and premature failure.

%It is standard practice to employ suction-side piping one or two sizes bigger than the pump inlet. Small pipes result in larger friction losses, which means it costs more to run your pumping system. However, for this case, the pipe is but a small length and balancing the cost of larger friction cost and higher cost of a bigger pipe will not be significant, thus the same diameter suction pipe will suffice.

%A reason for this design is to accommodate placing the check valve at pump discharge to later join the discharge header via an elbow and thru a Y junction. 


%A particular pipe section along the discharge section is quite alarming. Pipe deterioration are evident as the protective paint has become brittle and the those that have come off reveal the serious corrosion occurring on the pipe surface. The handle for the pressure control valve have experienced serious galvanic corrosion which resulted to its unaesthetic deterioration. Where the pipe penetrates the wall, substantial corrosion also has been observed and may become a source of leak soon. (reducer should be desired before pump intake)




%Another consequence of this lack of consideration for the piping was that the old pipes are already experiencing heavy corrosion which might soon be replaced to if water quality is desired to be maintained as well as avoiding leaks that could prompt dangerous situations.
%During the rehabilitation, the designer should have considered redesigning the entire pump system including provisions for monitoring 

The rehabilitation did not entirely replace all pipes and retain a portion of the discharge line and the pipe connecting reservoir to pump suction. These pipes displayed comparable deterioration with the new pipes and might be attributed to ageing. Pipe deterioration is evident as the protective paint has become brittle and the stripped areas reveal the serious corrosion occurring on the pipe surface. Where the pipe penetrates the wall, substantial corrosion also has occurred and may become a source of leak soon. Furthermore, the handle for the pressure control valve has experienced serious galvanic corrosion which resulted to its unaesthetic deterioration (refer to Figure \ref{fig_ch04_visualinspection_deterioration01}). 

\begin{figure}[!htb]
	\begin{minipage}[b]{0.15\linewidth}
		\centering
		\includegraphics[width=\textwidth]{figures/ch043_discharge_pipe_deteriorationA}
		\caption*{a - pipe (01)}
		\label{ch043_discharge_pipe_deteriorationA}
	\end{minipage}
	\hspace{0.05cm}
	\begin{minipage}[b]{0.2\linewidth}
		\centering
		\includegraphics[width=\textwidth]{figures/ch043_old_suction_pipeB}
		\caption*{b - old suction (08)}
		\label{ch043_discharge_pipe_deteriorationB}
	\end{minipage}
	\hspace{0.05cm}
	\begin{minipage}[b]{0.2\linewidth}
		\centering
		\includegraphics[width=\textwidth]{figures/ch043_discharge_pipe_deteriorationC}
		\caption*{c - pipe (03)}
		\label{ch043_discharge_pipe_deteriorationC}
	\end{minipage}
	\begin{minipage}[b]{0.2\linewidth}
		\centering
		\includegraphics[width=\textwidth]{figures/ch043_discharge_pipe_deteriorationD}
		\caption*{d - pipe (04)}
		\label{ch043_discharge_pipe_deteriorationD}
	\end{minipage}
	\hspace{0.05cm}
	\begin{minipage}[b]{0.2\linewidth}
		\centering
		\includegraphics[width=\textwidth]{figures/ch043_discharge_pipe_deteriorationE}
		\caption*{e - pipe (05)}
		\label{ch043_discharge_pipe_deteriorationE}
	\end{minipage}
	\hspace{0.05cm}
	\begin{minipage}[b]{0.2\linewidth}
		\centering
		\includegraphics[width=\textwidth]{figures/ch043_discharge_pipe_deteriorationF}
		\caption*{f - pipe (06)}
		\label{ch043_discharge_pipe_deteriorationF}
	\end{minipage}
\hspace{0.05cm}
	\begin{minipage}[b]{0.2\linewidth}
	\centering
	\includegraphics[width=\textwidth]{figures/ch043_old_suction_pipeA}
	\caption*{g - suction pipe (07)}
	\label{ch043_old_suction_pipeA}
\end{minipage}
\hspace{0.05cm}
\begin{minipage}[b]{0.2\linewidth}
	\centering
	\includegraphics[width=\textwidth]{figures/ch043_discharge_pipe_deteriorationB}
	\caption*{h - pipe (02)}
	\label{ch043_old_suction_pipeB}
\end{minipage}
\hspace{0.05cm}
\begin{minipage}[b]{0.3\linewidth}
	\centering
	\includegraphics[width=\textwidth]{figures/ch043_prv_corroded_manual_switch}
	\caption*{i - corroded handle (09)}
	\label{ch043_prv_corroded_manual_switch}
\end{minipage}
\caption{Examples of deterioration on pipes and connections}
\label{fig_ch04_visualinspection_deterioration01}
\end{figure}











Although the pipes can withstand a remarkable amount of corrosion before failure, this should not be allowed to simply continue. As part of reliability and good upkeep, all pipes should generally be protected and preserved as much as possible.


A note on one discharge gate valve located above ground: this placement of valve should be avoided as it is not easily accessible to operator to adjust if needed (refer to Figure \ref{ch043_inaccesible_gate_valve}).


On a positive note, the pump system incorporates vibration meters which are valuable monitoring tools of the condition of the system (refer to Figure \ref{ch043_vib_probea} and Figure \ref{ch043_vib_probeb}).

A discrepancy in unit is observed between the flow meter and the control panel. This is just a minor error and should be corrected by the operator.

\begin{figure}[!htb]
		\begin{minipage}[b]{0.3\linewidth}
		\centering
		\includegraphics[width=\textwidth]{figures/ch043_inaccesible_gate_valve}
		\caption{Inacessible gate valve}
		\label{ch043_inaccesible_gate_valve}
	\end{minipage}
	\hspace{0.05cm}
	\begin{minipage}[b]{0.3\linewidth}
		\centering
		\includegraphics[width=\textwidth]{figures/ch043_vib_probeA}
		\caption{Pump/motor DE vibration probe-01}
		\label{ch043_vib_probea}
	\end{minipage}
	\hspace{0.05cm}
	\begin{minipage}[b]{0.3\linewidth}
		\centering
		\includegraphics[width=\textwidth]{figures/ch043_vib_probeB}
		\caption{Pump/motor DE vibration probe-02}
		\label{ch043_vib_probeb}
	\end{minipage}

\end{figure}


Other minor observations are listed below:

\begin {itemize}
	\item There is discrepancy of units between flow meter and control panel is observed. Use the correct units m3/hr and change the unit for the control panel to avoid further confusion.
	\item Insert pressure tapping for pressure gauge near pump flange for more accurate measurement of head.
	\item Suction pipe should have ascending inclination toward pump suction nozzle (preferably 1/100 sloping gradient)
	\item Hydraulic design considerations for pump suction piping would recommend at least 5 to 10 diameters of straight pipe leading to the pump intake to minimize swirls or turbulent flow and facilitate better pumping action. This is to be followed should suction piping be reconfigured.
\end{itemize}



%Pipe should have descending inclination 
%Insert pressure tapping for pressure gauge
%Insert temperature tapping for thermodynamic efficiency devices


%\begin{figure}[h]
	%	\begin{center}
	%\includegraphics[scale=0.6]{figures/ch04_11_fot_eccentric_reducer} 
	%	\end{center}
	%\caption{Eccentric reducer}
	%\label{ch04_tbl_ch04_11_fot_eccentric_reducer}
%\end{figure}




%The pipe design then has no provisions for flow measurement for conventional meters such as mechanical meters, weigh tanks and ultrasonic/Doppler flow meters, magnetic flow meters. All of these depend on some provisions of straight pipe with relatively developed flow or require space for installation of monitoring devices. 

%Furthermore, common practice to avoid air pockets building up at the suction side of the pump is to use flat-on-top eccentric reducer before the pump suction (Figure \ref{ch04_tbl_ch04_11_fot_eccentric_reducer}). This is applicable for piping coming from below or straight ahead. 

%show flow simulation comparison between two setups


\subsection{Visual inspection data}
Visual inspection data on assets are summarized in tables of this section. %and also in the Appendix \ref{appvisualinspectionmech} with pictures.

%\paragraph{\textbf{BP1}}

\begin{table}[!htb]
	\caption{Visual inspection data - P1}
	\label{ch043_tbl_visualinspectionP1}
%	\resizebox{\columnwidth}{!}{%
		{\scriptsize
\begin{tabular}{c|l|c|p{9.5cm}}
\hline
No. & Item & CS & Remarks \\ 
\hline
1 & Double elbow & 3 & Design may promote turbulent flow profile which is not benefitial to long term pump operation \\ 
2 & Suction FC & 1 & Connection is serving its purpose to accommodate minor displacements however connection to adjacent pipe need to be replaced with a diffuser to minimize hydraulic losses\\ 
3 & Discharge FC & 1 & Connection is serving its purpose to accommodate minor displacements \\ 
4 & Check valve & 1 & Check valve is relatively new, no leaks or signs of corrosion, no malfunction observed \\ 
5 & Discharge BV & 1 & No malfunction observed \\ 
6 & Machine foundation & 1 & Damping capacity of foundation based on rule of thumb is adequate, raises machine inertia due to height  \\ 
7 & Vibration probes & 1 & Probes are working properly \\ 
\hline

\end{tabular}
	}
\end{table}
\input {tables/ch043_tbl_visualinspectionP2}
\begin{table}[!htb]
	\caption{Visual inspection data - P3}
	\label{ch043_tbl_visualinspectionP3}
%	\resizebox{\columnwidth}{!}{%
		{\scriptsize
\begin{tabular}{c|l|c|p{9.5cm}}
	\hline
	No. & Items & CS & Remarks \\ 

\hline
1 & Double elbow suction 1 & 3 & Design may promote turbulent flow profile which is not benefitial to long term pump operation \\ 
2 & Suction 2 & 1 & Use of existing pipe line might not be benefitial as it promotes turbulence due to design and is not desired for pump operation \\ 
3 & Suction FC & 1 & Connection is serving its purpose to accommodate minor displacements however connection to adjacent pipe need to be replaced with a diffuser to minimize hydraulic losses\\ 
4 & Discharge FC & 1 & Connection is serving its purpose to accommodate minor displacements \\ 
5 & Check valve & 1 & Check valve is relatively new, no leaks or signs of corrosion, no malfunction observed \\ 
6 & Discharge BV & 1 & No malfunction observed \\ 
7 & Machine foundation & 1 & Damping capacity of foundation based on rule of thumb is adequate, raises machine inertia due to height  \\ 
8 & Vibration probes & 1 & Probes are working properly \\ 
\hline
\end{tabular}
	}
\end{table}


\subsection{Recommendations}

\begin{itemize}
\item The rehabilitation was concerned with the major components of the mechanical system, namely the pumps. However, little changes were made to the pipe design and this could make the flow in the system not conducive for long term pump operation.

\paragraph{\underline{Recommendations}} 
\begin{itemize}
	\item [$\checkmark$] Modifications need to made to correct these sitaution. Refer to conceptual design for further details.
\end{itemize}

\item The old pipes are experiencing corrosion which might need to be replaced to if water quality is desired to be maintained as well as avoiding leaks that could prompt dangerous situations.

\paragraph{\underline{Recommendations}}
\begin{itemize}
	\item [$\checkmark$] Schedule replacement or restoration corroded pipes.
\end{itemize}


\item There are some notable discrepancy between the as-built and the as-found, namely, air valves and STCs were not installed. The air valve is important to rid of air pockets that could prove harmful to pump during operation. 


\paragraph{\underline{Recommendations}}
\begin{itemize}
	\item [$\checkmark$] Modify the pump discharge side piping configuration to correct these situation. Refer to conceptual design for further details.

\end{itemize}


\item A quick check between recommended flow velocity of water pipe versus the pipe diameter would show that the velocity induced provided that the pump operate at rated capacity is over the recommended values. This can also shorten both pipe and pump life considerably if velocity is too fast.



\paragraph{\underline{Recommendations}}
\begin{itemize}
	\item [$\checkmark$] Check the velocity of pipe along suction if it exceeds minimum pipe velocity. Install larger pipe diameters for the discharge side.

\end{itemize}

\end{itemize}

%\end{document}