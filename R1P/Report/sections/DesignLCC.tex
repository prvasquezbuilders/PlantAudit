% Chapter 6
\chapter{Conceptual Design and Reliability Study} % Write in your own chapter title
\label{Chapter6}
%\lhead{Chapter 6. \emph{Conceptual Design}} % Write in your own chapter title to set the page header
%
%%%%%%%%%%%%%%%%%%%%%%%%%%%%
\section{Basis of Design}
%\label{61}
\subsection{As-built drawings}
A collection of as-built drawings are given in A3 print out with electronic files saved both in PDF and CAD formats.
\subsection{Mechanical design}
As set of conceptual drawings is given in Appendix %\ref{app_design_mech}.


%\subsection{WEM design}
%\textcolor{red}{RB Sanchez to write here the basic of design such as design criteria, calculation, applied codes, objectives, etc. Try to summary it in table form and cite to correct references including as-built drawings, control philosophy, etc}
%
%\textcolor{red}{A set of conceptual design drawings shall be referred as enclosed in the Appendix}
\section{Bill of Materials}
Based on the recommendations and conceptual design, a high level Bill of Quantity (BOQ) can be generated . The BOQ table includes the condition states and intervention types respectively. %This BOQ will be served as a base for the estimation of life cycle cost.


\section{Determination of optimal intervention strategies}

This section focuses on determination of intervention strategies for pumps or components of pumps in the station. Other assets such as electrical assets and other assets belong to FDAS system are not part of the analysis because they are utmost critical assets that require to provide adequate level of services at all time. Intervention strategies for electrical assets and FDAS assets are suggested to follow replacement schedule defined by the manufacturers.
%
\subsection{Reliability}
With the historical data given, it is not possible to precisely quantify the reliability for each pump or even each components of pumps. Thus, In order to more or less estimate the reliability, obtaining knowledge and experience of the end-users becomes important.

%In reviewing the activity reports, mostly in 2016, following facts are revealed

%\begin{itemize}
%\item Pumps fail to provide adequate level of services due to aging/wear out of bearing, coupling, and mechanical seals both inside and outside;
%\item Defective and aging check valves from time to time caused the disruption of normal operation;
%\item Strong vibration was observed also due to possible misalignment and experienced of foreign objects.
%\end{itemize}

We assume reliability of an asset to follow a Weibull function. The Weibull function is suitable function to be used in survival/reliability analysis since it considers the memory of assets, i.e. past failure probability will be considered with duration of failure (refer to subsection \ref{ch03:weibullmodel} for further description on Weibull model).

Table \ref{ch05_tbl_weibullpara} shows the assumed values used for model's parameters.

\begin{table}[h]
	\caption{Weibull parameters.}
	\label{ch05_tbl_weibullpara}
	{\footnotesize
\begin{tabular}{c|c|c|l}
	\hline
	Assets & \multicolumn{2}{c|}{Weibull parameters} & Remarks \\ 
	\cline{2-3}
	& $\alpha$ & $\beta$ &  \\ 
	\hline
	BP1 & 6,966 & 2.174 &  \\ 
	BP2 & 7,308 & 2.576 &  \\ 
	BP3 & 7,092 & 2.573 &  \\ 
	\hline
\end{tabular}
	}
\end{table}

Using the parameter values in Table \ref{ch05_tbl_weibullpara}, reliability curves for respective pumps can be drawn (Figure \ref{ch05_fig_reliability}. %Figure \ref{ch05_fig_reliability01}) presents a snapshot of reliability of pumps within 5 years.

\begin{figure}[!htb]
	\begin{minipage}[b]{0.5\linewidth}
		\centering
		\includegraphics[width=\textwidth]{figures/ch05_fig_sur_pump1}
		\caption*{a - BP\#1}
	\end{minipage}
	\hspace{0.05cm}
	\begin{minipage}[b]{0.5\linewidth}
		\centering
		\includegraphics[width=\textwidth]{figures/ch05_fig_sur_pump2}
		\caption*{b - BP\#2}
	\end{minipage}
	\hspace{0.05cm}
	\begin{minipage}[b]{0.5\linewidth}
		\centering
		\includegraphics[width=\textwidth]{figures/ch05_fig_sur_pump3}
		\caption*{c - BP\#3}
	\end{minipage}
	\caption{Reliability}
	\label{ch05_fig_reliability}
\end{figure}

%\begin{figure}[!htb]
%	\includegraphics[width=\textwidth]{figures/ch05_fig_reliability01} \\
%	\caption{Reliability curves}
%	\label{ch05_fig_reliability01}
%\end{figure}

%As can be seen from the figure,

\subsection{Impacts}
Impacts are costs or loss of benefits incurred to the Client and users. Impacts are incurred by execution of PI or CI and by disruption to the operation of the PS.

Impacts incurred by execution of PI can be estimated based on, for example, conceptual design with the ballpark estimate. However, impacts incurred by execution of a CI is not easy to obtain due to lack of historical data. This is similar to the estimation of impacts such as loss in revenue, reputation, and regulatory.

Fortunately, from mathematical view point, the optimization model presented in subsection \ref{blockreplace} will be only dependent on the ratio between CI and PI. This means that we can make assumption on the ratio between PI and CI based on holistic approach. For example, if the PI is 10 millions PHP and CI is 20 million, the ratio would be PI/CI =0.5, and the model will determine the optimal time window (T) to execute the PI. This time window T will not change as long as the ratio between CI/PI is the same.

Table \ref{ch05_tbl_impactvalue01} shows the assumption on impacts incurred by execution of PI and CI as well the the impacts incurred considering the 3Rs (revenue, reputation, and regulatory).

\begin{table}[h]
	\caption{Impact values (mus).}
	\label{ch05_tbl_impactvalue01}
	{\footnotesize
\begin{tabular}{c|c|c|l|l|l}
	\hline
	Assets & \multicolumn{2}{c|}{Weibull parameters} & \multicolumn{1}{c|}{Impacts} & \multicolumn{1}{c|}{Discount (\%)} & Remarks \\ 
	\cline{2-3}
	& $\alpha$ & $\beta$ & \multicolumn{1}{c|}{PI/CI} & \multicolumn{1}{c|}{$\rho$} &  \\ 
	\hline
	BP1 & 6,966 & 2.174 & \multicolumn{1}{c|}{0.1072} & \multicolumn{1}{c|}{8.5} &  \\ 
	BP2 & 7,308 & 2.576 & \multicolumn{1}{c|}{0.4244} & \multicolumn{1}{c|}{8.5} &  \\ 
	BP3 & 7,092 & 2.573 & \multicolumn{1}{c|}{0.2583} & \multicolumn{1}{c|}{8.5} &  \\ 
	BP4 & 7,710 & 2.987 & \multicolumn{1}{c|}{0.3061} & \multicolumn{1}{c|}{8.5} &  \\ 
	SP1 & 6,458 & 2.983 & \multicolumn{1}{c|}{0.3086} & \multicolumn{1}{c|}{8.5} &  \\ 
	SP2 & 6,048 & 2.636 & \multicolumn{1}{c|}{0.2031} & \multicolumn{1}{c|}{8.5} &  \\ 
	\hline
\end{tabular}
\\
Note: R1, R2, R3 are revenue, reputation and regulatory, respectively.
	}
\end{table}

Values of revenues incurred by execution on respective pumps shall be calculated based on assumption on maintainability (e.g. duration of CI execution to fix the pump). Reputation and regulatory are not straightforward to estimate in monetary units, however, they can be assumed to be measured by "Willingness to Pay".

\subsection{Optimal Time Window and Impacts}
Figure \ref{ch05_fig_ois} presents a collection of curves representing Optimal Time Window (OTW) and impacts incurred by executing PIs on respective pumps.
\begin{figure}[!htb]
	\begin{minipage}[b]{0.5\linewidth}
		\centering
		\includegraphics[width=\textwidth]{figures/ch05_fig_ois_pump1}
		\caption*{a - BP\#1}
	\end{minipage}
	\hspace{0.05cm}
	\begin{minipage}[b]{0.5\linewidth}
		\centering
		\includegraphics[width=\textwidth]{figures/ch05_fig_ois_pump2}
		\caption*{b - BP\#2}
	\end{minipage}
	\hspace{0.05cm}
\begin{minipage}[b]{0.5\linewidth}
	\centering
	\includegraphics[width=\textwidth]{figures/ch05_fig_ois_pump3}
	\caption*{c - BP\#3}
\end{minipage}
\caption{Impact curves}
\label{ch05_fig_ois}
\end{figure}

The curves are with parabolic shapes, with a minimal point representing the optimal point obtained by the model. Shapes of the curves prove following important conclusions

\begin{itemize}
\item If a PI is executed too often, the failure probability will be decreased, however, at the same time, more money is to spend on intervention activities. It will be a waste of resources to follow a program that triggers PIs;
\item The impact will decrease as the OTW increases to the optimal point and then go upward indicating that if a PI is executed beyond a certain reliability threshold, failure probability will increase, leading to more sudden failures that require to execute CIs and higher loss (e.g. the 3 Rs).
\end{itemize}


\begin{table}[h]
	\caption{Impact values (mus).}
	\label{ch05_tbl_impactvalue02}
	{\footnotesize
\begin{tabular}{c|c|c}
	\hline
	Assets & OTW (hours) & Minimum Impact (mus) \\ 
	\hline
	BP1 & 2,494 & 8.04E-05 \\ 
	BP2 & 5,731 & 1.27E-04 \\ 
	BP3 & 4,068 & 1.06E-04 \\ 
	BP4 & 4,785 & 9.66E-05 \\ 
	SP1 & 4,014 & 1.17E-04 \\ 
	SP2 & 3,052 & 1.08E-04 \\ 
	\hline
\end{tabular}\\
		Note: OTW stands for Optimal Time Window to execute an PI.
	}
\end{table}

\subsection{Sensitivity analysis}
As a matter of fact, results of the optimization model are subjected to uncertainties with model's parameters and variables. Model's parameters are deterioration parameters $\eta$ and $\beta$ and impacts that should be obtained from historical data.

This subsection provides results of sensitivity analysis (SA) conducted on Weibull parameters $\eta$ and $\beta$ and the ratio between CI and PI. A SA was conducted by running the targeted parameter in a pre-defined range while keeping other parameters constant.

Results of the SA on parameter $\eta$ are shown in Figure \ref{ch05_fig_sa_alpha}, with following discussion points.

\begin{figure}[!htb]
	\begin{minipage}[b]{0.5\linewidth}
		\centering
		\includegraphics[width=\textwidth]{figures/ch05_fig_etasa_pump1}
		\caption*{a - BP\#1}
	\end{minipage}
	\hspace{0.05cm}
	\begin{minipage}[b]{0.5\linewidth}
		\centering
		\includegraphics[width=\textwidth]{figures/ch05_fig_etasa_pump2}
		\caption*{b - BP\#2}
	\end{minipage}
	\hspace{0.05cm}
	\begin{minipage}[b]{0.5\linewidth}
		\centering
		\includegraphics[width=\textwidth]{figures/ch05_fig_etasa_pump3}
		\caption*{c - BP\#3}
	\end{minipage}
	\caption{Sensitivity Analysis ($\eta$)}
	\label{ch05_fig_sa_alpha}
\end{figure}

\begin{itemize}
\item Values of OTW follow either exponential function or log-normal function with decreasing trend and be converged to a minimum year. This is logic as the smaller value of $\alpha$ is, the less failure probability, and therefore the OTW will be deferred longer;
\item Values of impact follow a monotonic increasing function. This is also logic as the higher failure probability infers more CIs to be executed.
\end{itemize}

Results of the SA on parameter $\beta$ are shown in Figure \ref{ch05_fig_sa_m}, with following discussion points.


\begin{figure}[!htb]
	\begin{minipage}[b]{0.5\linewidth}
		\centering
		\includegraphics[width=\textwidth]{figures/ch05_fig_betasa_pump1}
		\caption*{a - BP\#1}
	\end{minipage}
	\hspace{0.05cm}
	\begin{minipage}[b]{0.5\linewidth}
		\centering
		\includegraphics[width=\textwidth]{figures/ch05_fig_betasa_pump2}
		\caption*{b - BP\#2}
	\end{minipage}
	\hspace{0.05cm}
	\begin{minipage}[b]{0.5\linewidth}
		\centering
		\includegraphics[width=\textwidth]{figures/ch05_fig_betasa_pump3}
		\caption*{c - BP\#3}
	\end{minipage}

	\caption{Sensitivity Analysis ($\beta$)}
	\label{ch05_fig_sa_m}
\end{figure}

\begin{itemize}
	\item Values of OTW follow an exponential function function with decreasing trend and be converged to a minimum year. This is logic as the smaller value of $\beta$ is, the less failure probability, and therefore the OTW will be deferred longer;
	\item Values of impact follow a monotonic increasing function. This is also logic as the higher failure probability infers more CIs to be executed.
\end{itemize}

Results of the SA on the ratio $PI/CI$ are shown in Figure \ref{ch05_fig_sa_cipi}, with following discussion points.


\begin{figure}[!htb]
	\begin{minipage}[b]{0.5\linewidth}
		\centering
		\includegraphics[width=\textwidth]{figures/ch05_fig_sacipi_pump1}
		\caption*{a - BP\#1}
	\end{minipage}
	\hspace{0.05cm}
	\begin{minipage}[b]{0.5\linewidth}
		\centering
		\includegraphics[width=\textwidth]{figures/ch05_fig_sacipi_pump2}
		\caption*{b - BP\#2}
	\end{minipage}
	\hspace{0.05cm}
	\begin{minipage}[b]{0.5\linewidth}
		\centering
		\includegraphics[width=\textwidth]{figures/ch05_fig_sacipi_pump3}
		\caption*{c - BP\#3}
	\end{minipage}
	\caption{Sensitivity Analysis ($PI/CI$)}
	\label{ch05_fig_sa_cipi}
\end{figure}

\begin{itemize}
	\item Values of OTW follow an exponential function function with decreasing trend and be converged to a minimum year. This is logic as if the value of CI is not much different from PI, there is no need to perform PI often, the operators can just execute a CI when pumps fail. However, if the ratio is high, it means it will be very costly to let the system fails, hence it is advisable to shorten the time window to execute PIs to prevent possible failures from happening;

	\item Values of impact follow a monotonic increasing function. This is also logic as the higher ratio of CI/PI, the more impacts will be incurred with CI and therefore the annual cost will increase.
\end{itemize}

\section{Return on Investment}
It is important to note that in asset management context, the Return on Investment (ROI) is understood different from the ROI used for CAPEX projects. In CAPEX project, ROI is a ratio between the Net Present Value (NPV) of benefits (e.g. positive sum of cash flow) incurred over a pre-defined life cycle of a project (e.g. 20 or 30 years), at which there is a salvage value of the facility.

In asset management context, the ROI encompasses the ROI used in CAPEX project. This can be obviously seen in the impact curves shown in Figure \ref{ch05_fig_ois}. If the owner follows the optimal time window defined to execute a PI on any individual pump, the return on investment compared to other strategy that follow different time window is the difference of the impacts.

%}
%s\section{Discussions}
