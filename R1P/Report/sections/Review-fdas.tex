\subsection{Fire protection and safety (FDAS) audit}
\label{234}
It was confirmed at site that there is no FACP and the devices are only battery operated smoke detectors "KIDDE" brand. 

This smoke alarm warning device is a “stand-alone” system in which device detects and measure smoke particles and emits an audible sound but is not connected to any of the manual call point and bell that were installed separately in the pump station. There is no monitoring panel to tell where the zone/ area of the fire is and also the exact location of the fire. Table \ref{ch02_tbl_fdas01} lists major assets of the FDAS.

\begin{table}[h]
	\caption{Main parameters to be collected.}
	\label{ch02_tbl_fdas01}
	{\footnotesize
	\begin{tabular}{l|l|l}
		\hline
		Assets & Quantity & Unit \\ 
		\hline
		Smoke detectors & \multicolumn{1}{c|}{8} & \multicolumn{1}{c}{pcs} \\ 
		Call points & \multicolumn{1}{c|}{2} & \multicolumn{1}{c}{pcs} \\ 
		Bell sounder & \multicolumn{1}{c|}{2} & \multicolumn{1}{c}{pcs} \\ 
		\hline
	\end{tabular}
	}
\end{table}

The Emergency and fire fighting equipment consists of the following :
\begin{itemize}
\item Fire extinguishers Dry Chemical (Red color)
\item Fire extinguishers HCFC (Green color)
\item Emergency lighting 
\item Exit signages
\item Exit Doors
\item PPE cabinet
\item Fire hose cabinet
\item Evacuation plan for every floor
\end{itemize}

FDAS audit was conducted in the period from October 30, 2018 to November 16, 2018.


Audit on FDAS has been conducted following sequences

\paragraph{Step 1: Assign (1) one person on the Fire Alarm Control Panel  to operate / accept  the fire alarm activation and another group/person to conduct spraying on the device, communicate using two way radio.}

\paragraph{Step 2: Conduct spray of smoke detector tester (SOLO brand or any) directly on the smoke detector device for not more than 1 sec, repeat action until detector is activated. Note : If detector fails to respond after 3 tries,  device will declared faulty (Figure \ref{ch02_fdas}-a).}

\paragraph{Step 3: Hear and visually check strobe light and sounder every time you activated the smoke detectors.}

\paragraph{Step 4: Remove device and clean, allow particles to disperse. Then return to socket (Figure \ref{ch02_fdas}-b).}

\paragraph{Step 5: Check that strobe light is functioning/ blinking after returning. Note original status if no light is visible. Check that the control panel breaker feeding the device is reset (Figure \ref{ch02_fdas}-c).}

\paragraph{Step 6: Repeat steps 2, 3, and 4 on different locations until all the devices are tested.}

\paragraph{Step 7: Conduct testing for manual call point /manual pull station by pressing the device, hear if the alarm bell / buzzer is activate after you trigger the device (Figure \ref{ch02_fdas}-d).}

\paragraph{Step 8: Check bells and buzzer audibility.}

\paragraph{Step 9: Return Manual Call Point /Manual Pull Station on stand by position. Repeat it on all device.}

\paragraph{Step 10: Make a record for the fault device.}

\paragraph{Step 11: Record the status of FACP and reset the panel until the fault clear on trouble.}

\paragraph{Step 12: Conduct closing of activities to all concerned .}

\begin{figure}[h]
	\begin{minipage}[b]{0.22\linewidth}
		\centering
		\includegraphics[width=\textwidth]{figures/ch02_fdas01}
		\caption*{(a)}
		%		\label{ch02_fdas01}
	\end{minipage}
	\hspace{0.05cm}
	\begin{minipage}[b]{0.22\linewidth}
		\centering
		\includegraphics[width=\textwidth]{figures/ch02_fdas02}
		\caption*{(b)}
		%		\label{ch02_fdas02}
	\end{minipage}
	\hspace{0.05cm}
	\begin{minipage}[b]{0.22\linewidth}
		\centering
		\includegraphics[width=\textwidth]{figures/ch02_fdas03}
		\caption*{(c)}
		%	\label{ch02_fdas03}
	\end{minipage}
	\hspace{0.05cm}
	\begin{minipage}[b]{0.22\linewidth}
		\centering
		\includegraphics[width=\textwidth]{figures/ch02_fdas04}
		\caption*{(d)}
		%	\label{ch02_fdas04}
	\end{minipage}
	\caption{FDAS testing}
	\label{ch02_fdas}
\end{figure}