\section{Electrical system design and analysis} \label{ch04_elecaudit_systemdesign}
\subsection{Basics}
In accordance with Article 1.3 Electrical Plans and specifications of the Philippine Electrical Code 2017 Edition, Electrical design analysis shall be included and submitted separately together with the electrical plans. These includes the followings:

\begin{enumerate}%[label=\alph*)]
\item Branch circuits, sub-feeders, feeders, busways, and service entrance;
\item Types, ratings, and trip settings of overload protective device;
\item Calculation of voltage drops;
\item	Calculation of short circuit current for determining the interrupting capacity of overcurrent protective device for residential, commercial and industrial establishment;
\item	Protection coordination of overcurrent protective devices;
\item	ARC-flash Hazard Analysis to determine the required personal protective equipment (PPE).
\end{enumerate}

ARC flash Hazard Analysis is required and is intended for concerned parties to be informed and made aware of the importance of personal protective equipment (PPE) and its type for the flash hazard risk category determined by the analysis (refer to Table \ref{tbl_ch04_elecaudit_arcflash}).

\begin{table}[!htb]
	\caption{ARC flash hazard risk categories and PPE ratings (Appendix H, PEC 2017)}
	\label{tbl_ch04_elecaudit_arcflash}
	%	\resizebox{\columnwidth}{!}{%
	{\scriptsize
		
\begin{tabular}{c|c|c|p{6cm}}
	\hline
	Risk CAT. & Range of calculated  & Minimum Ppe Rating & Clothing Required \\ 
	& incident energy [cal/$cm^2$] & [$Cal/Cm^2$] &  \\ 
	\hline
	0 & 0 < E $\leq$1.2 & N/A & 4.5-14.0 Oz/Yd2Untreated Cotton \\ 
	1 & 1.2 < E $\leq$ 4 & 4 & Flame Retardant (Fr) Shirt And Pants \\ 
	2 & 4 < E $\leq$ 8 & 8 & Cotton Underclothing Plus Fr Shirt And Pants \\ 
	3 & 8 < E $\leq$ 25 & 25 & Cotton Underclothing Plus Fr Shirt, Pants, Overalls Or Equivalent \\ 
	4 & 25 < E $\leq$40 & 40 & Cotton Underclothing Plus Fr Shirt, Pants, Plus Double Layer Switching Coat And Pants Or Equivalent \\ 
	5 & 40 < E $\leq$ 100 & 100 & Cotton Underclothing Plus Fr Shirt, Pants, Plus Multi-Layer Switching Suit Or Equivalent \\ 
	\hline
	\end{tabular}
		
	}%}
\end{table}


\subsection{Results}
Results are briefly presented in the following subsections. Details reports generated by the software are enclosed as part of the Appendix of this report (can also be an electronic files)

%\subsubsection{Branch circuits, sub-feeders, feeders and service entrance}
%Figure \ref{fig_ch04_elecaudit_SLD_with_VFD} shows the SLD with values associated with each nodes and links.
%
%\begin{figure}
%	%	\includepdf[angle = 0]{sections/CHE_1PHSC_with_VFD.pdf}
%%	\includegraphics[width=\textwidth]{figures/fig_ch04_elecaudit_SLD_with_VFD.pdf} \\
%	\caption{Single line diagram}
%	\label{fig_ch04_elecaudit_SLD_with_VFD} 
%\end{figure}
\subsubsection{Types, ratings, and trip settings of overload protective device}
Types, ratings, and trip settings of overload protective devices are shown in Table \ref{ch04_elecaudit_protectioncoordination01} of subsection \ref{ch04_elecaudit_protectioncoordination}

%\begin{table}[!htb]
	\caption{Protective Device Settings - Low Voltage Circuit Breaker with Thermal-Magnetic Trip Device}
	\label{tbl_ch04_elecaudit_tripsetting}
	%	\resizebox{\columnwidth}{!}{%
	{\scriptsize
		
\begin{tabular}{l|l|l|l|l|l|l|l}
	\hline
	LVCB ID & Manufacturer & \multicolumn{2}{c|}{Breaker} & \multicolumn{2}{c|}{Thermal} & \multicolumn{2}{c}{Magnetic (Inst.)} \\ 
	\cline{3-8}
	&  & Model & Size & Setting & Trip & Setting & Trip \\ 
	&  &  &  &  & (Amps) &  & (Amps) \\ 
\hline
CB3 & Siemens  & Sentron JFC & \multicolumn{1}{c|}{500} & \multicolumn{1}{c|}{Fixed} & \multicolumn{1}{c|}{500} & \multicolumn{1}{c|}{Fixed} & \multicolumn{1}{c}{xIn } \\ 
CB12 & Siemens & Sentron JFC & \multicolumn{1}{c|}{400} & \multicolumn{1}{c|}{Fixed} & \multicolumn{1}{c|}{400} & \multicolumn{1}{c|}{Fixed} & \multicolumn{1}{c}{xIn } \\ 
CB11 & Siemens & CQD & \multicolumn{1}{c|}{100} & \multicolumn{1}{c|}{Fixed} & \multicolumn{1}{c|}{100} & \multicolumn{1}{c|}{Fixed} & \multicolumn{1}{c}{xIn } \\ 
CB6 & GE & QB & \multicolumn{1}{c|}{225} & \multicolumn{1}{c|}{Fixed} & \multicolumn{1}{c|}{225} & \multicolumn{1}{c|}{Fixed} & \multicolumn{1}{c}{xIn } \\ 
CB10 & GE & QJ & \multicolumn{1}{c|}{150} & \multicolumn{1}{c|}{Fixed} & \multicolumn{1}{c|}{32} & \multicolumn{1}{c|}{Fixed} & \multicolumn{1}{c}{xIn } \\ 
CB5 & GE & EDNC & \multicolumn{1}{c|}{200} & \multicolumn{1}{c|}{Fixed} & \multicolumn{1}{c|}{200} & \multicolumn{1}{c|}{Fixed} & \multicolumn{1}{c}{xIn } \\ 
CB7 & Schneider & NS & \multicolumn{1}{c|}{225} & \multicolumn{1}{c|}{Fixed} & \multicolumn{1}{c|}{225} & \multicolumn{1}{c|}{Fixed} & \multicolumn{1}{c}{xIn } \\ 
\hline
\end{tabular}

		
	}%}
\end{table}

\subsubsection{Calculation of voltage drops }
Basic for analysis is per PEC 2017 ARTICLE 2.15.1.2(A)(1)(b)FPN NO.2, which states "Conductors  for feeders, as defined in Article1.1, sized to prevent a voltage drop exceeding three 
(3) percent at the farthest outlet of power, heating and lighting loads, or combinations of such
loads,and where the maximum total voltage drop on both feeders and branch circuits to the  
farthest outlet does not exceed five (5) percent , will provide reasonable efficiency.".

Results of voltage drop are summarized in Table \ref{tbl_ch04_elecaudit_vdc} of subsection \ref{ch04_elecaudit_voltagedropcalculation}
%\begin{table}[!htb]
	\caption{Voltage drop summary}
	\label{tbl_ch04_elecaudit_votagedrop}
		\resizebox{\columnwidth}{!}{%
	{\scriptsize
\begin{tabular}{c|p{2cm}|p{2cm}|p{1.2cm}|c|p{1.2cm}|p{1.2cm}|p{1.2cm}|c|c|p{1.2cm}}
	\hline
	Item & From & To & Wire Size, $Mm^2$ & I & Length Meters & R $\Omega$/305M & X $\Omega$/ 305M &  Vd  & \%Vd & Remarks \\ 
	\hline
	1 & Pole Mounted Transformer 50kVA & Main Disconnect Switch & 250 & 425 & 21 & 0.048 & 0.027 &  2.788  & 0.58 & Within Limits \\ 
	2 & Main Disconnect Switch 400 & ATS PANEL & 250 & 425 & 35 & 0.048 & 0.027 &  4.647  & 0.97 & Within Limits \\ 
	3 & ATS PANEL & MDP-P 460V & 250 & 425 & 7.46 & 0.048 & 0.027 &  0.991  & 0.21 & Within Limits \\ 
	4 & MDP-P 460V & MCC 1 & 250 & 425 & 3.56 & 0.048 & 0.027 &  0.273  & 0.06 & Within Limits \\ 
	5 & MDP-P 460V & MCC 2 & 250 & 425 & 6 & 0.048 & 0.027 &  0.797  & 0.17 & Within Limits \\ 
	6 & MDP-P 460V & ECB 200A & 250 & 425 & 9 & 0.048 & 0.027 &  0.691  & 0.14 & Within Limits \\ 
	7 & ECB 200A & DRY TYPE TRANSFORMER 112.5KVA & 125 & 285.0 & 2 & 0.057 & 0.052 &  0.249  & 0.05 & Within Limits \\ 
	8 & DRY TYPE TRANSFORMER 112.5KVA & MDP-S 230V & 125 & 285 & 20.5 & 0.057 & 0.052 &  2.557  & 1.11 & Within Limits \\ 
	9 & MDP-S 230V & MCC 230V & 125 & 285 & 1 & 0.057 & 0.052 &  0.125  & 0.05 & Within Limits \\ 
	10 & MDP-P 460V & DRY TYPE TRANSFORMER 75KVA & 250 & 425 & 4.5 & 0.063 & 0.051 &  0.879  & 0.22 & Within Limits \\ 
	11 & DRY TYPE TRANSFORMER 75KVA & LP PANEL & 60 & 170 & 9 & 0.1 & 0.054 &  0.986  & 0.43 & Within Limits \\ 
	\hline
\end{tabular}
}
	}
\end{table}


\subsubsection{Calculation of short circuit current 3-PHASE}
Table \ref{tbl_ch04_elecaudit_shortcircuitsummary} show summaries of results on short circuit.


\begin{table}
	\caption{Short circuit Summary}
	\label{tbl_ch04_elecaudit_shortcircuitsummary}
	\includegraphics[width=\textwidth]{tables/tbl_ch04_elecaudit_shortcircuitsummary} \\	
\end{table}


\subsubsection{Protection coordination of overcurrent protective devices}
Results of study on protection coordination are presented in subsection \ref{ch04_elecaudit_protectioncoordination}. With reference to the coordination plot shown in Figure \ref{fig_ch04_elecaudit_protection_coordination01}, it is noted that partial coordination only for CB2 Normal breaker of Automatic Transfer Switch and Main of MCC panel. TCC of Main MCC crosses the TCC of ircuit breaker upstream. It is recommended to have a higher trip unit (Microloggic 6) which will be placed upstream  for better coordination.

\subsubsection{Arc-flash Hazard Analysis }
\begin{figure}
	%	\includepdf[angle = 0]{sections/CHE_1PHSC_with_VFD.pdf}
	\includegraphics[scale=0.8, angle =90]{figures/R1P_systemdesign/fig_ch04_elecaudit_ARC1_SLD_Arc_Flash_Analysis.pdf} \\
	\caption{ARC flash analysis}
	\label{fig_ch04_elecaudit_ARC1_SLD_Arc_Flash_Analysis} 
\end{figure}


%\begin{table}
%	\caption{Incident Energy Summary}
%	\label{tbl_ch04_elecaudit_incidenenergysummary}
%%		\includegraphics[width=\textwidth]{tables/tbl_ch04_elecaudit_arcflashsummary} \\	
%	
%\end{table}


\subsection{Recommendations}
According to the results of the Arc Flash Analysis, Flame retardant shirt and pants should be worn when opening the cover of the MCC Panel  respecting the Arc flash Boundary (AFB) .  Contributors to the arc flash are the motor loads and the VFD’s.

An Arc flash label (refer to Figure \ref{fig_ch04_elecaudit_arflashlable}) should be placed on the MCC as per requirement of the Philippine Electrical code.

\begin{figure}
	\begin{minipage}[b]{0.5\linewidth}
	\centering
	\includegraphics[width=\textwidth]{figures/R1P_systemdesign/fig_ch04_elecaudit_arflashlable01}
\end{minipage}
\hspace{0.03cm}
\begin{minipage}[b]{0.5\linewidth}
	\centering
	\includegraphics[width=\textwidth]{figures/R1P_systemdesign/fig_ch04_elecaudit_arflashlable02}
\end{minipage}
\hspace{0.03cm}
\begin{minipage}[b]{0.5\linewidth}
	\centering
	\includegraphics[width=\textwidth]{figures/R1P_systemdesign/fig_ch04_elecaudit_arflashlable03}
\end{minipage}
\hspace{0.03cm}
\begin{minipage}[b]{0.5\linewidth}
	\centering
	\includegraphics[width=\textwidth]{figures/R1P_systemdesign/fig_ch04_elecaudit_arflashlable04}
\end{minipage}
\hspace{0.03cm}
\begin{minipage}[b]{0.5\linewidth}
	\centering
	\includegraphics[width=\textwidth]{figures/R1P_systemdesign/fig_ch04_elecaudit_arflashlable05}
\end{minipage}
\hspace{0.03cm}
\begin{minipage}[b]{0.5\linewidth}
	\centering
	\includegraphics[width=\textwidth]{figures/R1P_systemdesign/fig_ch04_elecaudit_arflashlable06}
\end{minipage}
\hspace{0.03cm}
\begin{minipage}[b]{0.5\linewidth}
	\centering
	\includegraphics[width=\textwidth]{figures/R1P_systemdesign/fig_ch04_elecaudit_arflashlable07}
\end{minipage}
\hspace{0.03cm}
\begin{minipage}[b]{0.45\linewidth}
	\centering
	\includegraphics[width=\textwidth]{figures/R1P_systemdesign/fig_ch04_elecaudit_arflashlable08}
\end{minipage}
	\caption{Arc flash labels}
	\label{fig_ch04_elecaudit_arflashlable} 
\end{figure}