%\begin{document}
\section{Pump discharge and suction pipe - thickness} \label{ch04mech01}
\subsection{Data and measurement}
Thickness data on discharge and suction pipes of pumps is presented in Table \ref{ch04_tbl_thickness02}.
\begin{table}[h]
	\caption{Thickness data - Booster Pumps (mm).}
	\label{ch04_tbl_thickness02}
	{\footnotesize

\begin{tabular}{l|l|l|l|l|l|l|l|l|l|lllllllllllllllllll}
\cline{1-11}
\multicolumn{1}{c|}{Asset} & \multicolumn{1}{c|}{Position} & \multicolumn{9}{c}{Distance} &  &  &  &  &  &  &  &  &  &  &  &  &  &  &  &  &  &  \\ 
\cline{3-11}
\multicolumn{1}{c|}{} & \multicolumn{1}{c|}{} & \multicolumn{6}{c|}{Suction} & \multicolumn{3}{c}{Discharge} &  &  &  &  &  &  &  &  &  &  &  &  &  &  &  &  &  &  \\ 
\cline{3-11}
\multicolumn{1}{c|}{} & \multicolumn{1}{c|}{} & 0.3 m    & 0.5 m    & \multicolumn{1}{c|}{Elbow 1} & 1.0 m    & \multicolumn{1}{c|}{Elbow 2} & \multicolumn{1}{c|}{1.4m} & \multicolumn{1}{c|}{Elbow} & \multicolumn{1}{c|}{1.5 m   } & \multicolumn{1}{c}{2.0 m   } &  &  &  &  &  &  &  &  &  &  &  &  &  &  &  &  &  &  \\ 
\cline{1-11}
\multicolumn{1}{c|}{BP1} & \multicolumn{1}{c|}{12} & \multicolumn{1}{c|}{7.17} & \multicolumn{1}{c|}{-} & \multicolumn{1}{c|}{7.56} & \multicolumn{1}{c|}{6.66} & \multicolumn{1}{c|}{7.76} & \multicolumn{1}{c|}{7.65} & \multicolumn{1}{c|}{6.46} & \multicolumn{1}{c|}{4.47} & \multicolumn{1}{c}{4.35} &  &  &  &  &  &  &  &  &  &  &  &  &  &  &  &  &  &  \\ 
\multicolumn{1}{c|}{} & \multicolumn{1}{c|}{3} & \multicolumn{1}{c|}{7.79} & \multicolumn{1}{c|}{-} & \multicolumn{1}{c|}{7.62} & \multicolumn{1}{c|}{6.57} & \multicolumn{1}{c|}{7.52} & \multicolumn{1}{c|}{7.4} & \multicolumn{1}{c|}{5.19} & \multicolumn{1}{c|}{5.48} & \multicolumn{1}{c}{4.27} &  &  &  &  &  &  &  &  &  &  &  &  &  &  &  &  &  &  \\ 
\multicolumn{1}{c|}{} & \multicolumn{1}{c|}{6} & \multicolumn{1}{c|}{7.74} & \multicolumn{1}{c|}{-} & \multicolumn{1}{c|}{6.89} & \multicolumn{1}{c|}{6.1} & \multicolumn{1}{c|}{7.37} & \multicolumn{1}{c|}{7.13} & \multicolumn{1}{c|}{5.82} & \multicolumn{1}{c|}{5.48} & \multicolumn{1}{c}{4.18} &  &  &  &  &  &  &  &  &  &  &  &  &  &  &  &  &  &  \\ 
\multicolumn{1}{c|}{} & \multicolumn{1}{c|}{9} & \multicolumn{1}{c|}{7.55} & \multicolumn{1}{c|}{-} & \multicolumn{1}{c|}{6.62} & \multicolumn{1}{c|}{6.73} & \multicolumn{1}{c|}{7.71} & \multicolumn{1}{c|}{7.59} & \multicolumn{1}{c|}{-} & \multicolumn{1}{c|}{4.21} & \multicolumn{1}{c}{4.37} &  &  &  &  &  &  &  &  &  &  &  &  &  &  &  &  &  &  \\ 
\cline{1-11}
\multicolumn{1}{c|}{BP2} & \multicolumn{1}{c|}{12} & \multicolumn{1}{c|}{6.48} & \multicolumn{1}{c|}{-} & \multicolumn{1}{c|}{5.81} & \multicolumn{1}{c|}{5.91} & \multicolumn{1}{c|}{6.94} & \multicolumn{1}{c|}{6.99} & \multicolumn{1}{c|}{6.18} & \multicolumn{1}{c|}{6.46} & \multicolumn{1}{c}{4.41} &  &  &  &  &  &  &  &  &  &  &  &  &  &  &  &  &  &  \\ 
\multicolumn{1}{c|}{} & \multicolumn{1}{c|}{3} & \multicolumn{1}{c|}{6.5} & \multicolumn{1}{c|}{-} & \multicolumn{1}{c|}{6.31} & \multicolumn{1}{c|}{6.67} & \multicolumn{1}{c|}{6.81} & \multicolumn{1}{c|}{6.91} & \multicolumn{1}{c|}{6.35} & \multicolumn{1}{c|}{5.72} & \multicolumn{1}{c}{4.2} &  &  &  &  &  &  &  &  &  &  &  &  &  &  &  &  &  &  \\ 
\multicolumn{1}{c|}{} & \multicolumn{1}{c|}{6} & \multicolumn{1}{c|}{6.75} & \multicolumn{1}{c|}{-} & \multicolumn{1}{c|}{5.92} & \multicolumn{1}{c|}{6.31} & \multicolumn{1}{c|}{6.17} & \multicolumn{1}{c|}{7.16} & \multicolumn{1}{c|}{6.34} & \multicolumn{1}{c|}{5.17} & \multicolumn{1}{c}{4.18} &  &  &  &  &  &  &  &  &  &  &  &  &  &  &  &  &  &  \\ 
\multicolumn{1}{c|}{} & \multicolumn{1}{c|}{9} & \multicolumn{1}{c|}{6.55} & \multicolumn{1}{c|}{-} & \multicolumn{1}{c|}{6.19} & \multicolumn{1}{c|}{6.17} & \multicolumn{1}{c|}{6.15} & \multicolumn{1}{c|}{6.9} & \multicolumn{1}{c|}{-} & \multicolumn{1}{c|}{5.92} & \multicolumn{1}{c}{4.09} &  &  &  &  &  &  &  &  &  &  &  &  &  &  &  &  &  &  \\ 
\cline{1-11}
\multicolumn{1}{c|}{BP3} & \multicolumn{1}{c|}{12} & \multicolumn{1}{c|}{6.32} & \multicolumn{1}{c|}{7.26} & \multicolumn{1}{c|}{7.58} & \multicolumn{1}{c|}{6.25} & \multicolumn{1}{c|}{7.69} & \multicolumn{1}{c|}{-} & \multicolumn{1}{c|}{6.65} & \multicolumn{1}{c|}{4.49} & \multicolumn{1}{c}{9.93} &  &  &  &  &  &  &  &  &  &  &  &  &  &  &  &  &  &  \\ 
 & \multicolumn{1}{c|}{3} & \multicolumn{1}{c|}{6.55} & \multicolumn{1}{c|}{7.57} & \multicolumn{1}{c|}{7.65} & \multicolumn{1}{c|}{6.32} & \multicolumn{1}{c|}{7.71} & \multicolumn{1}{c|}{-} & \multicolumn{1}{c|}{6.62} & \multicolumn{1}{c|}{4.9} & \multicolumn{1}{c}{10.02} &  &  &  &  &  &  &  &  &  &  &  &  &  &  &  &  &  &  \\ 
 & \multicolumn{1}{c|}{6} & \multicolumn{1}{c|}{6.33} & \multicolumn{1}{c|}{7.58} & \multicolumn{1}{c|}{7.36} & \multicolumn{1}{c|}{6.02} & \multicolumn{1}{c|}{7.4} & \multicolumn{1}{c|}{-} & \multicolumn{1}{c|}{7.17} & \multicolumn{1}{c|}{4.67} & \multicolumn{1}{c}{10.01} &  &  &  &  &  &  &  &  &  &  &  &  &  &  &  &  &  &  \\ 
 & \multicolumn{1}{c|}{9} & \multicolumn{1}{c|}{6.8} & \multicolumn{1}{c|}{7.46} & \multicolumn{1}{c|}{7.44} & \multicolumn{1}{c|}{-} & \multicolumn{1}{c|}{7.69} & \multicolumn{1}{c|}{-} & \multicolumn{1}{c|}{-} & \multicolumn{1}{c|}{4.92} & \multicolumn{1}{c}{10.23} &  &  &  &  &  &  &  &  &  &  &  &  &  &  &  &  &  &  \\ 
\cline{1-11}
\multicolumn{1}{l}{} & \multicolumn{1}{l}{} & \multicolumn{1}{l}{} & \multicolumn{1}{l}{} & \multicolumn{1}{l}{} & \multicolumn{1}{l}{} & \multicolumn{1}{l}{} & \multicolumn{1}{l}{} & \multicolumn{1}{l}{} & \multicolumn{1}{l}{} &  &  &  &  &  &  &  &  &  &  &  &  &  &  &  &  &  &  &  \\ 
\multicolumn{1}{l}{} & \multicolumn{1}{l}{} & \multicolumn{1}{l}{} & \multicolumn{1}{l}{} & \multicolumn{1}{l}{} & \multicolumn{1}{l}{} & \multicolumn{1}{l}{} & \multicolumn{1}{l}{} & \multicolumn{1}{l}{} & \multicolumn{1}{l}{} &  &  &  &  &  &  &  &  &  &  &  &  &  &  &  &  &  &  &  \\ 
\multicolumn{1}{l}{} & \multicolumn{1}{l}{} & \multicolumn{1}{l}{} & \multicolumn{1}{l}{} & \multicolumn{1}{l}{} & \multicolumn{1}{l}{} & \multicolumn{1}{l}{} & \multicolumn{1}{l}{} & \multicolumn{1}{l}{} & \multicolumn{1}{l}{} &  &  &  &  &  &  &  &  &  &  &  &  &  &  &  &  &  &  &  \\ 
\multicolumn{1}{l}{} & \multicolumn{1}{l}{} & \multicolumn{1}{l}{} & \multicolumn{1}{l}{} & \multicolumn{1}{l}{} & \multicolumn{1}{l}{} & \multicolumn{1}{l}{} & \multicolumn{1}{l}{} & \multicolumn{1}{l}{} & \multicolumn{1}{l}{} &  &  &  &  &  &  &  &  &  &  &  &  &  &  &  &  &  &  &  \\ 
\multicolumn{1}{l}{} & \multicolumn{1}{l}{} & \multicolumn{1}{l}{} & \multicolumn{1}{l}{} & \multicolumn{1}{l}{} & \multicolumn{1}{l}{} & \multicolumn{1}{l}{} & \multicolumn{1}{l}{} & \multicolumn{1}{l}{} & \multicolumn{1}{l}{} &  &  &  &  &  &  &  &  &  &  &  &  &  &  &  &  &  &  &  \\ 
\multicolumn{1}{l}{} & \multicolumn{1}{l}{} & \multicolumn{1}{l}{} & \multicolumn{1}{l}{} & \multicolumn{1}{l}{} & \multicolumn{1}{l}{} & \multicolumn{1}{l}{} & \multicolumn{1}{l}{} & \multicolumn{1}{l}{} & \multicolumn{1}{l}{} &  &  &  &  &  &  &  &  &  &  &  &  &  &  &  &  &  &  &  \\ 
\multicolumn{1}{l}{} & \multicolumn{1}{l}{} & \multicolumn{1}{l}{} & \multicolumn{1}{l}{} & \multicolumn{1}{l}{} & \multicolumn{1}{l}{} & \multicolumn{1}{l}{} & \multicolumn{1}{l}{} & \multicolumn{1}{l}{} & \multicolumn{1}{l}{} &  &  &  &  &  &  &  &  &  &  &  &  &  &  &  &  &  &  &  \\ 
\multicolumn{1}{l}{} & \multicolumn{1}{l}{} & \multicolumn{1}{l}{} & \multicolumn{1}{l}{} & \multicolumn{1}{l}{} & \multicolumn{1}{l}{} & \multicolumn{1}{l}{} & \multicolumn{1}{l}{} & \multicolumn{1}{l}{} & \multicolumn{1}{l}{} &  &  &  &  &  &  &  &  &  &  &  &  &  &  &  &  &  &  &  \\ 
\multicolumn{1}{l}{} & \multicolumn{1}{l}{} & \multicolumn{1}{l}{} & \multicolumn{1}{l}{} & \multicolumn{1}{l}{} & \multicolumn{1}{l}{} & \multicolumn{1}{l}{} & \multicolumn{1}{l}{} & \multicolumn{1}{l}{} & \multicolumn{1}{l}{} &  &  &  &  &  &  &  &  &  &  &  &  &  &  &  &  &  &  &  \\ 
\multicolumn{1}{l}{} & \multicolumn{1}{l}{} & \multicolumn{1}{l}{} & \multicolumn{1}{l}{} & \multicolumn{1}{l}{} & \multicolumn{1}{l}{} & \multicolumn{1}{l}{} & \multicolumn{1}{l}{} & \multicolumn{1}{l}{} & \multicolumn{1}{l}{} &  &  &  &  &  &  &  &  &  &  &  &  &  &  &  &  &  &  &  \\ 
\end{tabular}

	}
\end{table}

In the table, the positions and the distances for the Ultrasonic Thickness Gauging (UTG) are referred to Figure \ref{ch04_fig_utgbp} and Figure \ref{ch04_fig_utgsp}.

\begin{figure}[!htb]
	\includegraphics[scale=0.3]{figures/ch04_fig_utgbp} \\
	\caption{Test points and distances of UTG – Booster Pumps Suction and Discharge Side}
	\label{ch04_fig_utgbp} 
\end{figure}

\begin{figure}[!htb]
	\includegraphics[scale=0.3]{figures/ch04_fig_utgsp} \\
	\caption{Test points and distances of UTG – Storage Pumps Discharge Side}
	\label{ch04_fig_utgsp} 
\end{figure}


%\begin{table}[h]
%	\caption{Thickness data (mm).}
%	\label{thicknessdata}
%	{\footnotesize
%	\begin{tabular}{l|l|l|l|l}
%		\hline
%		Pumps & \multicolumn{2}{c|}{Suction} & \multicolumn{2}{c}{Discharge} \\ 
%		\cline{2-5}
%		& \multicolumn{1}{c|}{Design} & \multicolumn{1}{c|}{Actual} & \multicolumn{1}{c|}{Design } & \multicolumn{1}{c}{Actual} \\ 
%		\hline
%		BP1 & \multicolumn{1}{c|}{} & \multicolumn{1}{c|}{4.98} & \multicolumn{1}{c|}{} & \multicolumn{1}{c}{3.92} \\ 
%		BP2 & \multicolumn{1}{c|}{} & \multicolumn{1}{c|}{4.32} & \multicolumn{1}{c|}{} & \multicolumn{1}{c}{4.22} \\ 
%		BP3 & \multicolumn{1}{c|}{} & \multicolumn{1}{c|}{4.61} & \multicolumn{1}{c|}{} & \multicolumn{1}{c}{4.61} \\ 
%		BP4 & \multicolumn{1}{c|}{} & \multicolumn{1}{c|}{4.54} & \multicolumn{1}{c|}{} & \multicolumn{1}{c}{4.37} \\ 
%		BP5 & \multicolumn{1}{c|}{} & \multicolumn{1}{c|}{4.49} & \multicolumn{1}{c|}{} & \multicolumn{1}{c}{4.64} \\ 
%		BP6 & \multicolumn{1}{c|}{} & \multicolumn{1}{c|}{4.09} & \multicolumn{1}{c|}{} & \multicolumn{1}{c}{4.60} \\ 
%		SP1 & \multicolumn{1}{c|}{} & \multicolumn{1}{c|}{4.40} & \multicolumn{1}{c|}{} & \multicolumn{1}{c}{4.25} \\ 
%		SP2 & \multicolumn{1}{c|}{} & \multicolumn{1}{c|}{5.09} & \multicolumn{1}{c|}{} & \multicolumn{1}{c}{4.18} \\ 
%		\hline
%	\end{tabular}
%			
%	}
%\end{table}

%Detailed measurement data is provided in appendix \ref{appthicknesss}.
%\textcolor{red}{RB Sanchez to write here the summary of raw data collected from visual inspection and testing. Tables shall be used as much as we can. Note that no analysis in this session. This session is purely the high level presentation of data. Raw data can be linked as an Appendix}

\subsection{Analysis} 
This section provides analysis/discussion on estimation of minimum allowable thickness of pipes and statistics around the measured data collected during inspection and testings.


%Followings are generic interpretation by examining the thickness values.
%\begin{itemize}
%\item Mean value of thickness is above 4.6 mm;
%\item Mean and median values are close, inferring a confidence on having less heterogeneity, i.e. distribution of thickness around the pipe is more or less homogeneous;
%\item Thickness at elbow is less than that of the straight line;
%\item Thickness of storage line is likely to be higher than that of the discharge line;
%\item BP1 has a value of 3.92 as min at the elbow, which requires attention from time to time.
%\end{itemize}

\paragraph{\underline{BP1}}
\begin{itemize}
\item Suction Piping System-The 1.0 m section is a spool connecting the bends and is observed to have the thinnest part overall at the suction side, especially the 6 o’clock section. As observed, bends are situated before and after this spool which might affect its thinning rate due to the momentum of water over the course of operation.

\item Discharge Piping System- It is observed that this pump has the thinnest section at its 6 o'clock position at the 2.0 m section which happens to be a bend, and at the 9 o’clock position at 1.5 m section. These are the sections where the water mixes with the water from the other pumps, thus increasing the velocity and momentum of the water with contributes to the increase of the thinning rate of the pipe wall. It is also observed that the thinnest component is at the bend where in it takes all momentum and change in direction over time, thus subjected to high shearing forces and results to higher thinning rates.

\end{itemize}

\paragraph{\underline{BP2}}
\begin{itemize}
\item Suction Piping System-It is observed that the thinning occurs at the 12 o'clock position of the 1.0 m section of the spool. This might indicate that the water starts to flow swirling from the 6 o'clock past to the 9 o'clock and through the 12 o'clock. Then it continues to the extrados area of the bend at the 0.7 m section with a possibility of creating a vacuum or eddy zone at the intrados area (6 o'clock), thus having the localized thinning zones at the 12 and 6 o'clock positions. It continues the pattern of swirling from 12 o'clock to the 3 o'clock before entering the pump. This pattern is different than of the pump1 and has thinner localized wall at 0.7 m, that may be caused by higher flow rate (thus higher velocity) of water.

\item Discharge Piping System-It is observed that the 6 o'clock position at the 1.5 m section is 0.65 mm thinner than the average in that section. This is due to the discharged water entering the header line which the flow might be directed toward the bottom part of the header. Moreover, due to the turbulency and disturbance made by the entering water from the discharge line, the 3-6-9 o'clock positions of the pipe at the 2 m section experience localized thinning.

\end{itemize}

\paragraph{\underline{BP3}}
\begin{itemize}
	\item Suction Piping System-The thinnest part is the spool at the 1.0m section, especially at the 6 o'clock position which is in coherence with the other suction pipe.
	\item Discharge Piping System-The thinnest part of the discharge side is at the 1.5 m section at the 12 o'clock position which might be due to the eddies formed by the disturbance made by the water entering the header pipe. The extrados and crown of the first bend are observed to have a difference in thickness of 0.55 mm. It is evident that the component with the thinnest part is the 1.5 m section which might due to the eddies and the high velocity coming from the pump. On the contrary, the 2.0 m section of the header pipe has the thickest part overall.
	
\end{itemize}

%\paragraph{\underline{BP4}}
%\begin{itemize}
%	\item Suction Piping System- This suction pipe inhibits localized wall thinning at the exit elbow extrados. Backflow is also present at the elbow as indicated by the thinning in the intrados area of the bend;
	
%	\item Discharge Piping System- Localized wall thinning is not present except in the middle extrados are of the elbow. This thinning indicates that the flow approaches directly in the middle extrados of the elbow and then exits the elbow with less turbulent phase.
	.
%\end{itemize}


%\paragraph{\underline{SP1}}
%\begin{itemize}
%\item Suction Piping System- The suction line of the SP's are vertical and not inclined like the setups in the BP's. The thickness confirmed to have a direct flow contacting the extrados surface from the entrance up to the elbow exit. This extends to the 12-o'clock position at the 2m mark. The 2m mark is also near at the elbow. No high backflow rate has been observed;

%\item Discharge Piping System- localized wall thinning occurs at the bottom part of the pipe as it is next to a concentric reducers that concentrates the flow. It then continues by swirling to one side of the pipe before entering the elbow. The flow then extends the swirling from the side to the extrados bend entry.
	
%\end{itemize}

%\paragraph{\underline{SP2}}
%\begin{itemize}
	%\item Suction Piping System- In this pipeline, the water flow contacted towards the 2m mark and less contacting the entrance elbow part. The thinning data also describes the water pattern of most pattern which travels at the higher half and then flows below half of pipe;
	
%	\item Discharge Piping System-Based on the measurements, the discharge side of the pump indicates that low cavitation may be carried out. The flow enters and is concentrated in the lower half of the pipe and swirls to the extrados entry of the pipe. It indicates a backflow at the bend intrados extending to the extrados exit of the bend.

%\end{itemize}



\subsubsection{Assumptions}
Following assumptions are used in calculating the required thickness of pipe
\begin{itemize}
\item Maximum Working Head – based on the design drawings and pump nameplate;
\item Pipe Material – assume pipe material is ASTM A570 Grade 33 (market available material for spiral welded pipe);
\item Design Guide – basis used for the simulated calculation is AWWA  Manual M11 – Steel Pipe, A Guide for Design and Installation, 4th Edition. Statement for corrosion allowance is located at Chapter 4, which states \textit{"At one time, it was a general practice to add a fixed, rule-of-thumb thickness to the pipe wall as a corrosion allowance. This was not an applicable solution in the water work field, where standard for coating and lining materials and procedures exists. The design shall be made for the required wall-thickness pipe as determined by the loads imposed, then linings, coatings, and cathodic protection selected to provide the necessary corrosion protection"};
\item Thickness calculation will be based on the internal pressure. External pressure will not be considered because much of the discharge line is not buried.
\item 	Surge Pressure was not considered since there are surge protection along the line. 
\item 	This document will only consider the calculation of the minimum thickness along the discharge line since this is the part of the system where maximum pressure is experience.
\end{itemize}
\subsubsection{Limitations}
As confirmed by Maynilad, there is no available data regarding the design report. Design assumptions herein may be different from what was used by the designer/contractor of this station.

This document will not be able to provide the corrosion/degradation factor of the pipe since there is no available historical data on the thickness of the pipe.

\subsubsection{Parameter values for thickness estimation}
In order to estimate the minimum allowance thickness for pipes in straight line considering material handling ($t_{mh}$) and maximum internal pressure based on AWWA M11 ($t_{sph}$), following equations are used, respectively:

\begin{eqnarray}
&& t_{mh} = \frac{\Phi}{\delta} \label{ch05thickness01}
\end{eqnarray}

\begin{eqnarray}
&& t_{sp} = \frac{\epsilon\times P_{max} \times \Phi}{2 \times S_e} \label{ch05thickness02}
\end{eqnarray}
where $P_{max}$ is maximum internal pressure

\begin{eqnarray}
&& P_{max} = \frac{\rho_{H_2O} \times g \times H_{max}}{1000} \label{ch05thickness03}
\end{eqnarray}

In order to estimate the minimum allowance thickness for pipes at elbows (Miter Bend), only maximum internal pressure is considered:

\begin{eqnarray}
&& t_{mb} = \frac{P_{max} \times \Phi}{2 \times S_e} \times \left[ 1 + \frac{\Phi}{(3 \times R)-(1.5 \times \Phi_d)}\right]\times \epsilon \label{ch05thickness04}
\end{eqnarray}

Paramater values used for computation are given in Table \ref{ch05_tbl_thicknesscalc}

\begin{table}[h]
	\caption{Parameter values for thickness estimation.}
	\label{ch05_tbl_thicknesscalc}
	{\footnotesize
		\begin{tabular}{p{4cm}|c|c|c|p{4cm}}
			\hline
			Parameters & Symbol & Unit & Pumps & Remarks \\ 
			&  &  & Booster & \\ 
			\hline
			Discharge diameter & $\Phi$ & $mm$ & 250 &  \\ 
			Max flow rate & $Q_{max}$ & $m^3/s$ & 0.174 &  \\ 
			Max pump head & $H_{max}$ & $m$ & 35 & based on name plate \\ 
			Yield strength of material & $S_y$ & $MPa$ & 227.5 & ASTM A570 Grade 33, spiral welded pipe based on AWWA C200 \\ 
			Allowable stress & $S_e$ & MPa & 113.75 &  \\ 
			Density of water & $\rho_{H_2O}$ & $kg/m^3$ & 1000  &  \\ 
			Gravity constant & $g$ & $m/s^2$ & 9.81 &   \\ 
			Safety factor & $\epsilon$ &  & 2 &   \\ 
			Bulk modulus of compressibility of liquid & $k$ & $Pa$ & 2.1E+09 &   \\ 
			Young's modulus of elasticity of pipe wall & $E$ & $Pa$ & 2.1E+11 &  \\ 
			Radius of Elbow & $R$ & $mm$ & 375 &   \\ 
			Empirical constant & $\delta$ &  & 288 &  \\ 
			\hline
		\end{tabular}
	}
\end{table}

%\subsubsection{Required thickness}

%Results of computation for minimum allowable thickness for booster pumps and storage pumps are given in Table \ref{ch05_tbl_thicknesscalcresult}.

\begin{table}[h]
	\caption{Minimum thickness allowance.}	\label{ch05_tbl_thicknesscalcresult}
	{\footnotesize
		\begin{tabular}{l|l|p{3cm}|p{3cm}|p{3cm}}
			\hline
			Pumps & \multicolumn{1}{c|}{Internal pressure  (Mpa)} & \multicolumn{3}{c}{Minimum allowable thickness (mm)} \\ 
			\cline{3-5}
			& \multicolumn{1}{c|}{$P_{max}$} & \multicolumn{1}{c|}{$t_{mh}$} & \multicolumn{1}{c|}{$t_{sp}$} & \multicolumn{1}{c}{$t_{mb}$} \\ 
			\hline
			Booster & \multicolumn{1}{c|}{0.343} & \multicolumn{1}{c|}{0.868} & \multicolumn{1}{c|}{0.754} & \multicolumn{1}{c}{1.005} \\ 
			%Storage & \multicolumn{1}{c|}{0.491} & %\multicolumn{1}{c|}{2.080} & \multicolumn{1}{c|}{2.590} & %\multicolumn{1}{c}{3.620} \\ 
			\hline
		\end{tabular}
		
	}
\end{table}

\subsubsection{Recommendations}
Given the current thickness of pipe the lack of design information, it is advisable to 
\begin{itemize}
\item Not perform any major intervention on the pipes;
\item Keep regular testing on exact locations using the same type of UTG device. It is important for Maynilad to establish a testing regime for obtaining thickness at exact same location over time (e.g. every year). Information obtained from testing will be then used to compute deterioration rate based on thickness value;
\item Establish an approach to inspect/test the thickness of underground pipe, which is considered to be more vulnerable to leakage and corrosion on external wall;
\item The elbows in the suction and the discharge piping systems must be monitored regularly;
\item It is recommended to have a profiling of the piping systems above and below the ground in order to have a baseline in the analysis of the Maynilad Piping System. In order to have a profiling of pipe thickness at differential time T, additional measurement at similar locations shall be conducted periodically, behavior can then be monitored;
\item Perform coating regularly of the pipe to prevent possible corrosion/errosion and damage that cause by external factors and surrounding condition;
\end{itemize}
%\end{document}