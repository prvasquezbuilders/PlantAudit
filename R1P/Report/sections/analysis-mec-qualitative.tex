%\begin{document}
\section{Qualitative and Operational Analysis}
\label{42}
\subsection{Facts and Data}

Summary of facts and data concerning operational and overall plan reliability is presented in this subsection.

\subsubsection{Operation Scenario}

At present the station follows the operation schedules as shown in Table \ref{operation_schedule}

\begin{table} [!htb]
%    \centering
    \begin{tabular}{l|c|c}
\hline
Time & Set Pressure (psi) & No. of Pumps running \\ 
\hline
4am - 7am & 35 & 2 \\ 
7am - 11am & 38 & 2 \\ 
11am - 12nn & 35 & 2 \\ 
1pm - 9pm & 30 & 1 \\ 
9pm - 10 pm & 25 & 1 \\ 
10pm - 4am & 20 & 1 \\ 
\hline
\end{tabular}
    \caption{Plant Operation Schedule}
    \label{operation_schedule}
\end{table}




%\subsubsection{High Demand Scenario}

%\begin{itemize}
%	\item
%	
%	\item
%\end{itemize}

%\subsubsection{Low Demand Scenario}

%\begin{itemize}
%	\item
%	
%	\item
%
%\end{itemize}

\subsubsection{Spares Policy}

\begin{itemize}
	\item 1 or 2 booster switch every 24 hours (in operation for full 24 hours then switch)
	\item 3 booster switch every weekday operational from 6AM – 1PM
	\item 3 booster switch on weekends operational from 6AM – 2PM
\end{itemize}

\subsubsection{Emergency Situation (loss of electrical power from Meralco)}
\begin{itemize}
	\item Auto start of pumps
\end{itemize}

\subsubsection{Maintenance}
\begin{itemize}
	\item For operational problems, operator will call Control Center to report problem.  CC to send contractor within 1-2 hours.  CC supervisor (Gilbert).  
	\item For non-operational maintenance, operator will call Mark Pascual for action.
\end{itemize}

\subsubsection{Current Problems}
\begin{itemize}
	\item Continuous presence of water at sump area of booster pumps.  Sump pump operation is done manually because if it is auto-start, the pump frequently breaks down. When the water level rises to about 7 inches, the operator manually starts the sump pump.  This happens every 3-4 hours daily.  There is a spare sump pump on standby ready to replace the damaged unit. The in-line pump sump areas also have water ingress especially during the rainy season.  This results in frequent start of the sump pump.  Unfortunately, the sump pump fails about 4 times per year.  The operator suspects the sump pump is of poor quality.
\end{itemize}

\subsection{Recommendations}
In order to ensure the PS to provide adequate level of services around the clock, it is important to establish a good operational scheme that allows optimization of use of pumps to reduce breakdown and to conserve energy. A summary of major recommendations to be considered are

\begin{itemize}
	\item This facility is does not allow for the installation of an additional pump in case the demand increases because there is no more available space in the underground facility to cater for another pump.  The design could be acceptable if there will be no significant increase in demand for the foreseeable future – assuming that the customers will only be limited to Ayala Alabang Village.
	
	\item Consider a dedicated duty and a dedicated spare set-up for the pumps.  If this is not acceptable, then consider doing a much longer switch of the storage pumps.  Currently, it is being switched daily.  This allows for almost an equal rate of deterioration between the two pumps and if one pump fails due to age-related component failure, the other one is close to a similar failure which may occur before the first pump is fully repaired or replaced.  It is suggested that the switch happen once a month or even every 3 months.

	\item In place of the longer switching cycle (e.g. every 3 months), there should be a corresponding maintenance program for the standby pump for both booster and storage.

	\item Know what maintenance activities are done weekly and how the contractors/Maynilad use the information gathered to predict equipment failures. 
	
	\item Develop a more structured discipline in applying routine maintenance work process to ensure that maintenance tasks are given the proper priority in terms of mitigation measures and avoid unplanned shutdown of critical pumps in operation.
\end{itemize}

Aside from the above recommendations, we also generate a list of recommendations based on the RCM methodology. This is presented at at the end of the document on \ref{app_maintenance}. The list shall be considered as a living program, which requires continuously improvement as part of the total quality management system (refer to Deming cycle presented in GHD's technical proposal).

%\end{document}



