%\chapter{}
\section{Workplace environment management}
\label{47}

\subsection{Temperature and relative humidity}
\subsubsection{Data}
Data concerning the temperature and relative humidity is presented in Table \ref{ch04_tbl_wem01}. Data was measured at targeted points shown in Figure \ref{ch02_wem01}. Raw data is with the site inspection reports, which will be provided to the Client separately. Persuant to ASHRAE standard, the recommended ranges for temperature and humidity are [72 - 80 $^\circ F$] and [45 - 60 \%], respectively.

\subsubsection{Data and Analysis}

\begin{table}[h]
	\caption{Temperature and Relative Humidity.}
	\label{ch04_tbl_wem01}
	{\footnotesize
\begin{tabular}{c|l|c|c|c|c}
\hline
Points & Description of points & \multicolumn{2}{c|}{Temperature (F)} & \multicolumn{2}{c}{Humidity (\%)} \\ 
\cline{3-6}
 &  & Actual & Range & Actual & Range \\ 
\hline
  & A. Inside pump house (outside office) &  &  &  &  \\ 
1 & Between BP2 and BP1 & 95.36 & 72 - 80 & 50.6 & 45 - 60 \\ 
2 & Between BP1 and SP2 & 93.74 & 72 - 80 & 53.3 & 45 - 60 \\ 
3 & Between SP2 and SP1 & 91.40 & 72 - 80 & 56.9 & 45 - 60 \\ 
4 & Near the Roll Up Door & 89.60 & 72 - 80 & 61.2 & 45 - 60 \\ 
5 & Near the Office Door & 89.60 & 72 - 80 & 61.5 & 45 - 60 \\ 
6 & Near at the Operators’ Office Window & 87.98 & 72 - 80 & 62.4 & 45 - 60 \\ 
7 & Near BP3 and along the Roll Up Door & 88.70 & 72 - 80 & 64.7 & 45 - 60 \\ 
8 & Between BP3 and BP4 & 93.02 & 72 - 80 & 56 & 45 - 60 \\ 
9 & Between BP4 and Spare 1 & 93.74 & 72 - 80 & 55.1 & 45 - 60 \\ 
10 & Between two Spares & 93.38 & 72 - 80 & 56.9 & 45 - 60 \\ 
11 & Between Suction Pipes of BP3 and BP4 & 93.412 & 72 - 80 & 53.5 & 45 - 60 \\ 
12 & Between Suction Pipes of Spares & 93.74 & 72 - 80 & 57.6 & 45 - 60 \\ 
 & Average & 91.97 & 72 - 80 & 57.5 & 45 - 60 \\ 
\hline
13 & B. Office & 80.96 & 72 - 80 & 52.2 & 45 - 60 \\ 
\hline
 & C. Outside pump house &  &  &  &  \\ 
14 & Guard House & 89.80 & 72 - 80 & 55.8 & 45 - 60 \\ 
15 & At the back of the station & 93.74 & 72 - 80 & 53.1 & 45 - 60 \\ 
16 & Near Substation Gate & 95.8 & 72 - 80 & 50.6 & 45 - 60 \\ 
\hline
\end{tabular}
}
\end{table}


Inside office registered a reading of approximately 81 F near the upper limit of the standard. As for the RH, the reading registered 52.2 \% and is close to the range average. Note that Air Conditioning System is installed inside the Office thus the temperature can still be vary based on the thermal comfort needed by the Operators.

As can be seen from Table \ref{ch04_tbl_wem01}, it is the temperatures inside the pump house at every measurement points are significant higher than the maximum value of the recommended range (80 F). The average value is 94.16 F. The higher values of temperature compared to the range have also been observed for points outside the pump house, in the vicinity, and inside the office. 

As a matter of fact, temperature and humidity is highly correlated and as per ASHRAE standard 55 under summer comfort zone, the recommended combination of temperature and humidity shall be within the comfortable zone as shown in Figure \ref{ch04_fig_wem01}.

\begin{figure}[!htb]
	\includegraphics[scale=2]{figures/ch04_fig_wem01} \\
	\caption{ASHRAE standard 55 : Summer Comfort Zone}
	\label{ch04_fig_wem01} 
\end{figure}

\subsubsection{Recommendations}
In order to reduce the negative impacts from high temperature, particularly inside the pump house, the Client shall consider

\begin{itemize}
	\item Establishing a good daily monitoring, exercise, and management considering ergonomic and health and occupational activities (e.g. appropriate time window for break in designated resting area);
	\item Execute physical intervention to reduce temperature can be with improving ventilation system by natural mean (e.g. installation of weather proofed louvers). This will be reflected in the conceptual design in Chapter \ref{Chapter6}.
\end{itemize}

\subsection{Air quality}\label{aq01}

\subsubsection{Data and analysis}
Data concerning the air quality is presented in \ref{ch04_tbl_wem03} with value of PM2.5 measured in ppm. Data was measured at targeted points shown in \ref{ch02_wem01}. Raw data is with the site inspection reports, which will be provided to the Client separately. Pursuant to currently applied standard, the recommended safe range for PM2.5 is in [0-35].

\begin{table}[h]
	\caption{Air Quality - PM2.5}
	\label{ch04_tbl_wem03}
	\footnotesize{
	
\begin{tabular}{c|l|c}
\hline
Point & Description of the Point Location & PM2.5 \\ 
\hline
 & A. Inside pump house (outside office) &  \\ 
1 & Between BP2 and BP1 & 35 \\ 
2 & Between BP1 and SP2 & 33 \\ 
3 & Between SP2 and SP1 & 30 \\ 
4 & Near the Roll Up Door & 26 \\ 
5 & Near the Office Door & 24 \\ 
6 & Near at the Operators’ Office Window & 24 \\ 
7 & Near BP3 and along the Roll Up Door & 19 \\ 
8 & Between BP3 and BP4 & 23 \\ 
9 & Between BP4 and Spare 1 & 20 \\ 
10 & Between two Spares & 24 \\ 
11 & Between Suction Pipes of BP3 and BP4 & 13 \\ 
12 & Between Suction Pipes of Spares & 17 \\ 
 & Average & 24 \\ 
\hline
13 & B. Office & 16 \\ 
\hline
 & C. Outside pump house &  \\ 
14 & Guard House * & 18 \\ 
15 & At the back of the station & 18 \\ 
16 & Near Substation Gate* & 23 \\ 
\hline
\end{tabular}
\\ * Varies along skyway
}

\end{table}



The average reading inside and outside the vicinity of the Pump Stations and Reservoir as well as the Office is from 13 to 35 and falls under the Excellent Air Quality Level Range. Values of PM2.5 are all within the range of acceptance, inferring no issue with the quality of air. 


\subsubsection{Recommendations}

Though there is no issue with the air quality, it is anticipated that future problem can incur with a certain low probability, a better management approach is to ensure that all activities/tasks to be executed within the premise of the PS to follow strictly safety and environmental regulation. For example, all employees and staff to wear appropriate dust-proofed masks when working with activities that potentially incurs dusts or other harmful particles.



%\subsection{Hazards}\label{aq02}
%\textcolor{red}{RB Sanchez to write here the summary of raw data collected from visual inspection and testing. Tables shall be used as much as we can. Note that no analysis in this session. This session is purely the high level presentation of data. Raw data can be linked as an Appendix}
\subsection{Illumination}\label{aq03}
\subsubsection{Data and analysis}
Data concerning the illumination is presented in Table \ref{ch04_tbl_wem04} with the LUX value. Data was measured at targeted points shown in Figure \ref{ch02_wem01}. Raw data is with the site inspection reports, which will be provided to the Client separately. Persuant to RULE 1075.4 of DOLE-OSH standard \cite{DOLE2016}, the recommended minimum for LUX is in 100.

\begin{table}[h]
    \caption{Illumination.}
    \label{ch04_tbl_wem04}
    \footnotesize{


\begin{tabular}{c|l|c|c|c|c}
\hline
Point & Description of the points & \multicolumn{3}{c|}{Trials} & Ave. \\ 
\cline{3-5}
 &  & 1 & 2 & 3 &  \\ 
\hline
 & A. Inside pump house (outside office - x10) &  &  &  &  \\ 
1 & Between BP2 and BP1 & 45.9 & 49.3 & 59.8 & 51.67 \\ 
2 & Between BP1 and SP2 & 44.5 & 54.7 & 74.8 & 58.00 \\ 
3 & Between SP2 and SP1 & 50.4 & 64.1 & 72 & 62.17 \\ 
4 & Near the Roll Up Door & 56.7 & 69.2 & 65.9 & 63.93 \\ 
5 & Near the Office Door & 43 & 46.1 & 41.6 & 43.57 \\ 
6 & Near at the Operators’ Office Window & 55.8 & 53.8 & 57 & 55.53 \\ 
7 & Near BP3 and along the Roll Up Door & 47 & 46.3 & 44.7 & 46.00 \\ 
8 & Between BP3 and BP4 & 70.2 & 72.4 & 68.8 & 70.47 \\ 
9 & Between BP4 and Spare 1 & 79.3 & 82.4 & 86.3 & 82.67 \\ 
10 & Between two Spares & 73.7 & 76.9 & 76 & 75.53 \\ 
11 & Between Suction Pipes of BP3 and BP4 & 138.6 & 135.2 & 134.7 & 136.17 \\ 
12 & Between Suction Pipes of Spares & 140.7 & 145.9 & 148.6 & 145.07 \\ 
 & Average &  &  &  & 74.23 \\ 
\hline
13 & B. Office (x10) & 109.8 & 110.2 & 110.8 & 110.27 \\ 
\hline
 & C. Outside pump house (x100) &  &  &  &  \\ 
14 & Guard House * & 180.2 & 178.4 & 177.2 & 178.60 \\ 
15 & At the back of the station & 220 & 225 & 223 & 222.67 \\ 
16 & Near Substation Gate* & 270 & 285 & 294 & 283.00 \\ 
\hline
\end{tabular}
}
\end{table}



The illumination of the pump is well above the standard minimum. Illumination is provided by natural means through the Skylights and is augmented by Motion-activated Lighting Systems. At night time, the light provided by Lighting System is for a particular zone only or place where there are motions.

\subsubsection{Recommendations}

\begin{itemize}
	\item	Use artificial lighting equipment when accessing and conducting activities requiring detailed output at darker specific areas especially at night because the existing lighting systems cannot provide adequate lighting or when deemed necessary.
	\item	Such artificial supplementary lightings shall be especially designed for the specific tasks and provided with shading or diffusing devices to prevent glare.
	\item	Periodic cleaning of Skylights and glass windows should be implemented to ensure they are kept clean at all times
	
\end{itemize}


\subsection{Industrial ventilation}\label{aq04}
\subsubsection{Data and analysis}

Natural ventilation through Door 1 and Door 2 is not sufficient to attain the minimum Air Changes requirement of the Pump House and so Mechanical Ventilation is utilized most of the time.

\begin{figure}[h]
	\includegraphics[scale=0.6]{figures/ch04_fig_ventilation01} \\
	\caption{Existing ventilation layout}
	\label{ch04_fig_ventilation01} 
\end{figure}


\subsubsection{Recommendations}

\begin{itemize}
\item Install weather-proof louver at the Pump House. This is to increase the air change inside the pump house by natural ventilation. The purpose is to utilize the Natural wind around the area and at the same time, lessen the Power Consumption of the Supply and Exhaust Fan. 
\end{itemize}

If the PS is rehabilitated with the recommendation, eventually, it will contribute to the reduction of hazards associated with air quality and energy/heat. The recommendation is also reflected in the conceptual design provided in later chapter.


\subsection{Housekeeping}\label{aq05}
\subsubsection{Documentation}
Following problems are the facts:

\begin{itemize}
\item Current documentation practice is heavily dominated with paper based system, which follows the current practice in Maynilad. There is a large amount/collection of papers that recorded past activities but is of no use and beneficial if data cannot be transformed into digital format for time series analysis, which is an essential part of asset management practice;

\item No proper filing/library system with standardized coding rule that will provide convenience for operators/users to timely find appropriate documents;

\item Daily operation data is crucial information for future analysis but it is recorded in excel based file without relational tables, which makes it from hard to impossible for data compilation, filtering, and mining. Many past data has been recorded with outliers and incorrect data types. 

\end{itemize}

\subsubsection{Waste management and environmental control}
There is no issue with regard to waste management and environmental control as confirmed by the checklist shown in Table \ref{ch05_tbl_housekeeping}

\begin{table}[h]
    \caption{Housekeeping}
    \label{ch05_tbl_housekeeping}
    \footnotesize{

\begin{tabular}{l|c|c|c|l}
\hline
Description & Status & CS & IT & Remarks \\ 
\hline
- Sufficient waste segregation assets & yes & 1 & 1 &  \\ 
- waste segregation policy & yes & 1 & 1 &  \\ 
- Signage & yes & 1 & 1 &  \\ 
- Genset emission control & yes &  & 1 &  \\ 
\hline
\end{tabular}
}

\end{table}


%The plant has its own waste segregation policy and an organized documentations procedure (evidenced of the arranged daily monitoring sheet). Standby generator set is situated outside the pump house where its exhaust gases through natural ventilation will not be able to penetrate the pump house.

\subsubsection{Office arrangement and ergonomic}
Table \ref{ch05_tbl_housekeeping} shows the data concerning parameters associated with office arrangement and ergonomic considerations.

\begin{table}[h]
    \caption{Housekeeping}
    \label{ch05_tbl_housekeeping}
    \footnotesize{

\begin{tabular}{l|c|c|c|l}
\hline
Description & Status & CS & IT & Remarks \\ 
\hline
- Sufficient waste segregation assets & yes & 1 & 1 &  \\ 
- waste segregation policy & yes & 1 & 1 &  \\ 
- Signage & yes & 1 & 1 &  \\ 
- Genset emission control & yes &  & 1 &  \\ 
\hline
\end{tabular}
}

\end{table}


\subsubsection{Recommendations}
Followings are recommendations

\begin{itemize}
\item Development of a web-based database management system, with appropriate set of relational data tables to record operational data, power consumption data, and intervention data;

\item Development of documentation code and naming for appropriate filing and library/referencing;

\item Applying best practices with regard to ergonomic in combination with interior design and arrangement of office space.

\end{itemize}





\subsection{Noise}\label{aq06}
\subsubsection{Data and analysis}
Data concerning the noise is presented in \ref{ch04_tbl_wem02}.Data was measured at targeted points shown in Figure 2.18. Raw data is with the site inspection reports, which will be provided to the Client separately. 

\begin{table}[h]
	\caption{Noise.}
	\label{ch04_tbl_wem02}
	%\footnotesize{}
\resizebox{\columnwidth}{!}{%
\begin{tabular}{c|p{4cm}|c|c|c|c|c|c|c|c|c|c}
\hline
Point & Desc. of the point location & \multicolumn{10}{c}{Trials} \\ 
\cline{3-12}
 & Location & \multicolumn{3}{c|}{1} & \multicolumn{3}{c|}{2} & \multicolumn{3}{c|}{3} &  \\ 
\cline{3-12}
 &  & Min & Ave. & max & Min & Ave. & Max & Min & Ave. & Max & Ave. \\ 
\hline
 & A. Inside pump house (outside office) &  &  &  &  &  &  &  &  &  &  \\ 
1 & Between BP2 and BP1 & 91.3 & 93.4 & 95.5 & 91.2 & 93.4 & 95.6 & 91.5 & 93.55 & 95.6 & 93.45 \\ 
2 & Between BP1 and SP2 & 91.1 & 93.05 & 95 & 90.8 & 92.75 & 94.7 & 90.6 & 92.8 & 95 & 92.87 \\ 
3 & Between SP2 and SP1 & 89.5 & 91.65 & 93.8 & 89.5 & 91.65 & 93.8 & 89.7 & 91.8 & 93.9 & 91.70 \\ 
4 & Near the Roll Up Door & 88.5 & 89.1 & 89.7 & 88.3 & 88.9 & 89.5 & 88.4 & 88.7 & 89 & 88.90 \\ 
5 & Near the Office Door & 87.9 & 88.45 & 89 & 87.7 & 88.35 & 89 & 87.9 & 88.5 & 89.1 & 88.43 \\ 
6 & Near the Operator's Room Window & 88.1 & 88.5 & 88.9 & 88.1 & 88.65 & 89.2 & 88.5 & 89 & 89.5 & 88.72 \\ 
7 & Near BP3 and along the Roll Up Door & 88.5 & 89.05 & 89.6 & x & 88.7 & 89.1 & 88.3 & 88.85 & 89.4 & 88.87 \\ 
8 & Between BP3 and BP4 & 92 & 93.8 & 95.6 & 89.2 & 91.05 & 92.9 & 89 & 91.55 & 94.1 & 92.13 \\ 
9 & Between BP4 and Spare 1 & 88.8 & 91.75 & 94.7 & 88.7 & 89.1 & 89.5 & 88.7 & 89.2 & 89.7 & 90.02 \\ 
10 & Between two Spares & 88 & 88.65 & 89.3 & 87.9 & 88.55 & 89.2 & 88.3 & 88.6 & 88.9 & 88.60 \\ 
 & Average &  & 90.74 &  &  & 90.11 &  &  & 90.26 &  & 90.37 \\ 
\hline
 & B. Vicinity &  &  &  &  &  &  &  &  &  &  \\ 
11 & Between Suction Pipes of BP3 and BP4 & 65.8 & 66.7 & 67.6 & 67.3 & 67.8 & 68.3 & 65.1 & 66.65 & 68.2 & 67.05 \\ 
12 & Between Suction Pipes of Spares & 60.2 & 63.85 & 67.5 & 59.1 & 62.2 & 65.3 & 60.6 & 63 & 65.4 & 63.02 \\ 
13 & C. Office & 65.4 & 68.5 & 71.6 & 66.4 & 69.35 & 72.3 & 65.8 & 68.5 & 71.2 & 68.78 \\ 
\hline
\end{tabular}
}
\end{table}


Regular operation at 2 Booster and 2 Storage Pumps running (BP3 at breakdown) was considered during the Sound Level Testing and so the reading closely represents the normal daily noise level inside the Plant. The average sound level inside the Pump House is 90.4 dBA, a little more than the 90 dbA treshhold. 

The sound level at pump station vicinity and inside the office are within safe values.

\subsubsection{Recommendations}

\begin{itemize}
	\item	Use protective hearing equipment when working inside the Pump House and have a scheduled break/rest at designated location. Shall not be exposed at such noise beyond 5 hours in a day.
	\item	Designate location inside the Plant with the minimum noise level - can be the Office, Inside Pump Station and Reservoir, and Outside the Vicinity of Pump Station, if below is not possible.
	\item	 Install Sound Attenuation Device (such as sound-absorbing wall panels and door seals) at the Office to reduce the current 69 dBA to ideal Office level of 50-55 dBA.
	
	\item Purchase a sufficient numbers of electronic noise canceling earphones.
\end{itemize}