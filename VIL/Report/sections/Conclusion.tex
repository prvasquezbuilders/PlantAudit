% Chapter 7
%{\color{blue}
\chapter{Conclusions and Recommendations} % Write in your own chapter title
\label{Chapter7}
%\lhead{Chapter 7. \emph{Conclusions and Recommendations}} % Write in your own chapter title to set the page header
Conclusions and recommendations have already presented in Chapter \ref{Chapter4} for individual assets under investigation. This section is hence only to provide a high level conclusions and recommendations.
\section{Conclusions}
\begin{itemize}
\item Pumps have been operated in non-optimal control approach that might lead to higher failure probability of total system failure in a long run;

\item Daily data on operation and power consumption is poorly recorded in non-relational table structure that is of little use for data analysis, especially when considering the huge amount of data recorded over time. This data shall be considered a very important part of management as it shall be used for energy audit and investigation on reliability of pumps and the system;

\item No real time measurement of production and power consumption per pump. This leads to the impossibility of calculating the reliability of pump over time;

\item No systematic inspection/testing strategy;

\item Piping supports suffer significant deterioration and wear out;

\item Existing FDAS system is not provide adequate level of services;

\item No lighting and grounding protection system in place.
\end{itemize}

\section{Recommendations}
\begin{itemize}
\item Establishment of control philosophy to optimally operate pumps in a manner to protect pumps from being failed;

\item Establishment of a good asset management practice (see recommendation in the Appendix);

\item Establishment of a benchmark value for monitoring energy efficiency;

\item Hiring an asset management expert/specialist to help the IAM to design a good platform to record data concerning daily production and power consumption via web-based system. Note that this is different platform from the current CMMS program. (Hiring a specialist to work with the IAM will eventually save cost for Maynilad in a long run. This can be contracted to individual expert);

\item Establishment of a optimal inspection/testing regime for all assets. This testing regime shall be developed using the concept of optimality based on a solid mathematical modeling approach and best practices learnt from the industry (e.g. case based reasoning methodology);

\item Redesign the FDAS system, lighting and grounding system. This shall be an immediate action to be executed prior to next raining season;

\item Installation of flow meters and power meters to all pumps. This will allow having records on daily production and power per pump. This will enable prediction and detection of abnormal behavior of pump, thus providing background for failure prediction;

\item Upgrade entirely the SCADA system to be fully automated. This is toward the industrial revolution 4.0 which is now a reality. By doing so, energy management statistic can be a real time monitoring system, which eventually supports Maynilad to detect the errors and improve the operation to minimize the energy consumption while still providing adequate level of services.

%\item 
\end{itemize}



%}