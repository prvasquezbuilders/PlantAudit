%\begin{document}
\section{Visual Inspection on Pipe, valves, fittings, supports, expansions, and appurtenances} \label{ch04mech02}


\subsection{Highlights} \label{ch04mech02_highlight}

Visual inspection data on pipes, valves, fittings, supports, expansions, and appurtenances is highlighted in Table \ref{ch04_visualinspection01}.

\begin{table}[!htb]
	\caption{Highlights of visual inspection}
	\label{ch04_visualinspection01}
%	\resizebox{\columnwidth}{!}{%
	{\scriptsize

\begin{tabular}{c|p{3cm}|p{9.5cm}}
	\hline
	No. & Items & Remarks \\ 
	\hline
	1 & Modified pipe design to fit space limitations & Limitation in space prompted compromise in pipe design and not did not follow recommended piping arrangements and instead may promote turbulent flow profiles at pump suction and discharge that can accelerate pump and fittings wear \\ 
	2 & Type of pipe and fitting support & Anchor type piping support makes pipe system rigid. \\ 
	3 & Paint and protective coating of outdoor pipe & Portion of pipe extending outside of pump house and are experiencing surface deterioration. Protective coating and paint falling off exposing some bare metal which are already display corrosion. \\ 
	4 & Local corrosion on indoor pipes & Some indoor pipes have torn paint films and exposed bare metals which display corrosion \\ 
	5 & Valve Leaks & Minor leaks that still contributes to deterioration of parts due to corrosion \\ 
	6 & Pressure gauges & Discrepant measurements between dial and digital gauges up to 10 psi \\ 
	&  & Deterioration of information tags making them unreadable \\ 
	&  & Superimposed tags which make it obscure \\ 
	7 & Motion actuated lighting & Interview with the maintenance team revealed that the motion actuated lighting sometimes causes slight nausea due to dim lighting when repair. The minute motions of repair are sometimes not enough to actuate the lights and thus interrupt the work \\ 
	\hline
\end{tabular}

	}%}
\end{table}

Visual inspections are supported with the photos taken at particular locations/positions in questions.

Relative to the station desired capacity and reservoir size, the space allotted for the pump house is limited. Vertical split case pumps were chosen to fit more pumps and pipe design was modified to fit the working space. Straight pipe provisions were cut short to fit in various valves, expansion joints and reducers and T-type branch take-offs are used. 

The suction pipe design for the SPs resemble that of double elbow at two perpendicular planes and immediately connect to the pump intake via a concentric reducer. 

The suction pipe design for the BPs resembles that of double elbow at two separate planes. The second elbow branches of the header at 45 degrees followed by a short straight pipe then connecting to the pump intake via a concentric reducer (refer to Figure \ref{ch04_piping01}-a and Figure \ref{ch04_piping01}-b).

%It was realized that the entire pump house is suffering from ground settlement and that the alignments of pumps and corresponding fittings have been compromised. For example, as can be shown in Figure \ref{ch04_settlement}-a, the portion of wall through which the pipe passes thru has been torn down partially to provide allowance for settlement as the pipe levels going down.

\begin{figure}[!htb]
	\begin{minipage}[b]{0.4\linewidth}
		\centering
		\includegraphics[width=\textwidth]{figures/ch04_09_SP_suction_piping}
		\caption*{a - suction pipe (01)}
	%	\label{pagcorlocation}
	\end{minipage}
	\hspace{0.05cm}
	\begin{minipage}[b]{0.5\linewidth}
		\centering
		\includegraphics[width=\textwidth]{figures/ch04_10_BP_suction_piping}
		\caption*{b - suction pipe (02)}
	%	\label{ch01_pumpgallery}
	\end{minipage}

%\end{minipage}
\caption{Piping design and alignment}
\label{ch04_piping01}
\end{figure}


Hydraulic design considerations for pump suction piping would recommend at least 5 to 10 diameters of straight pipe leading to the pump intake to minimize swirls or turbulent flow and facilitate better pumping action. The pipe design then has no provisions for flow measurement for conventional meters such as mechanical meters, weigh tanks and ultrasonic/Doppler flow meters, magnetic flow meters. All of these depend on some provisions of straight pipe with relatively developed flow or require space for installation of monitoring devices. 

Furthermore, common practice to avoid air pockets building up at the suction side of the pump is to use flat-on-top eccentric reducer before the pump suction (Figure \ref{ch04_tbl_ch04_11_fot_eccentric_reducer}). This is applicable for piping coming from below or straight ahead. 

\begin{figure}[h]
	%	\begin{center}
	\includegraphics[scale=0.6]{figures/ch04_11_fot_eccentric_reducer} 
	%	\end{center}
	\caption{Eccentric reducer}
	\label{ch04_tbl_ch04_11_fot_eccentric_reducer}
\end{figure}

Supports for pipes and valves can fall on three general categories according to the manner of support they provide. Supports can either function simply as rests, as guides or as anchors. Installation of expansion joints would indicate a provision for minute pipe expansion when flow is high and fluid momentum is great. However, from the visual inspection of the concrete support for check valves and how the valve seats on it, the supports function like anchors (fixed).


\begin{figure}[!htb]
	\begin{minipage}[b]{0.3\linewidth}
		\centering
		\includegraphics[width=\textwidth]{figures/ch04_12_CV_concrete_support}
		\caption*{a - concrete support}
		%	\label{pagcorlocation}
	\end{minipage}
	\hspace{0.05cm}
	\begin{minipage}[b]{0.6\linewidth}
		\centering
		\includegraphics[width=\textwidth]{figures/ch04_13_types_of_supports}
		\caption*{b - type of supports}
		%	\label{ch01_pumpgallery}
	\end{minipage}
	%\end{minipage}
	\caption{Piping support}
	\label{ch04_piping02}
\end{figure}


The position of the check valve becomes fixed and the expansion joint will have to expand to the direction of the pump casing, which may promote misalignment of pump and motor. However, the expansion joint is observed to be short circuited by another set of outer stud rods where they are tightened fixed with inner and outer nuts. The pipe system ends up rigid (Figure \ref{ch04_piping03}-a).

\begin{figure}[!htb]
	\begin{minipage}[b]{0.5\linewidth}
		\centering
		\includegraphics[width=\textwidth]{figures/ch04_14_restraining_joint}
		\caption*{a - outer stud rods}
		%	\label{pagcorlocation}
	\end{minipage}
	\hspace{0.05cm}
	\begin{minipage}[b]{0.5\linewidth}
		\centering
		\includegraphics[width=\textwidth]{figures/ch04_15_concrete_penetrations}
		\caption*{b - Concrete penetration}
		%	\label{ch01_pumpgallery}
	\end{minipage}
	%\end{minipage}
	\caption{Piping support}
	\label{ch04_piping03}
\end{figure}


A similar situation for the SP suction pipe could occur. Pipe penetrating the ground will fix the elbow and the expansion will be to the direction of the pump casing promoting misalignment of pump and motor (Figure \ref{ch04_piping03}-b). 

The protective coating and paint of the outdoor pipes on the side of the pumping station facing the golf course have severely deteriorated. The finishing paint has become brittle and the protective layer has cracked. Some portions of the pipe particularly at the elbow, have the bare metal exposed and has corroded for some time (Figure \ref{ch04_piping04}-a). 

Several dial pressure gauges were not functioning. Operating conditions sometimes reach negative suction pressures. Some suction pressure gauges do not have a vacuum range for negative suction pressures.  A discrepancy between the dial and digital gauges is observed to be up to 10 psi like the gauges seen for BP4 discharge (Figure \ref{ch04_piping04}-b). 

\begin{figure}[!htb]
	\begin{minipage}[b]{0.3\linewidth}
		\centering
		\includegraphics[width=\textwidth]{figures/ch04_16_pipe_protection_deterioration}
		\caption*{a - Outer deterioration}
		%	\label{pagcorlocation}
	\end{minipage}
	\hspace{0.05cm}
	\begin{minipage}[b]{0.7\linewidth}
		\centering
		\includegraphics[width=\textwidth]{figures/ch04_18_discrepant_pressure_gauges}
		\caption*{b - Gauge not provide adequate LOS}
		%	\label{ch01_pumpgallery}
	\end{minipage}
	%\end{minipage}
	\caption{Piping support}
	\label{ch04_piping04}
\end{figure}

During inspection, it was found that there are water leaks in the engineering room during rainstorms.


%
%
%
%
%%Damages to the concrete saddle supports appearing as cracks and crooked or completely drawn bolts have been observed as shown in Figure \ref{ch04_settlement}-b. Furthermore, gaps are also observed between the concrete saddle supports and the straight pipe beneath as shown in \ref{ch04_settlement}-c.  
%
%%For the booster pumps, there are two saddle supports located near to each other and are positioned just outside the pump house after the suction elbows as shown in Figure \ref{ch04_fig_visual1}-a. For the storage pumps, the three saddles are located farther, one before the discharge elbow and  the other still outside the pump house and just after the check valve.
%
%Piping movements parallel to the central axis also have been observed. As shown in Figure \ref{ch04_fig_visual1}-b, a gap has been observed between several counter supports and the elbow.  The counter supports are supposed to handle elbow thrust as immense water volumes pass thru the elbow. 
%
%\begin{figure}[!htb]
%	\begin{minipage}[b]{0.5\linewidth}
%		\centering
%		\includegraphics[width=\textwidth]{figures/ch04_fig_visual01}
%		\caption*{a - Piping supports}
%		%	\label{pagcorlocation}
%	\end{minipage}
%	\hspace{0.05cm}
%	\begin{minipage}[b]{0.5\linewidth}
%		\centering
%		\includegraphics[width=\textwidth]{figures/ch04_fig_visual02}
%		\caption*{b - Piping counter supports}
%		%	\label{ch01_pumpgallery}
%	\end{minipage}
%	\caption{Impacts from supporting system}
%	\label{ch04_fig_visual1}
%\end{figure}

%A common observation is that corrosion appears where there are leaks exist. Figure \ref{ch04_settlement}-d shows water leakage of BP6 check valve and surrounding and SP1 check valve.
%
%Several dial pressure gauges were not functioning (Figure \ref{ch04_fig_visual2}-a). Also it was observed that some suction pressure gauges do not have a vacuum range and are only able to measure positive pressures.  Moreover, it was observed that the readings of dial gauges and digital pressure gauges do not match with differences up to 10 psi as shown in Figure \ref{ch04_fig_visual2}-b.
%
%
%\begin{figure}[!htb]
%	\begin{minipage}[b]{0.5\linewidth}
%		\centering
%		\includegraphics[width=\textwidth]{figures/ch04_fig_visual04}
%		\caption*{a - Defective}
%		%	\label{pagcorlocation}
%	\end{minipage}
%	\hspace{0.05cm}
%	\begin{minipage}[b]{0.5\linewidth}
%		\centering
%		\includegraphics[width=\textwidth]{figures/ch04_fig_visual05}
%		\caption*{b - Digital and dial pressure gause discrepancy}
%		%	\label{ch01_pumpgallery}
%	\end{minipage}
%	\caption{Defective/discrepancy pressure gauges}
%	\label{ch04_fig_visual2}
%\end{figure}
%
%The component tags were also inspected during the visits. As observed, some of the tags are unprotected and erased. Further, some tags are inconsistent in content. Different brands of pressure gauges were installed at similar corresponding tapping points. It was observed that the alignment bolts for the pumps are washerless and that nuts are used instead, as shown in Figure \ref{ch04_fig_visual3}-a. Also, there were no frame grounding found for all the pumps as seen in Figure \ref{ch04_fig_visual3}-b.
%
%\begin{figure}[!htb]
%	\begin{minipage}[b]{0.5\linewidth}
%		\centering
%		\includegraphics[width=\textwidth]{figures/ch04_fig_visual06}
%		\caption*{a - Motor alignment bolts}
%		%	\label{pagcorlocation}
%	\end{minipage}
%	\hspace{0.05cm}
%	\begin{minipage}[b]{0.5\linewidth}
%		\centering
%		\includegraphics[width=\textwidth]{figures/ch04_fig_visual07}
%		\caption*{b - Motor alignment supportsy}
%		%	\label{ch01_pumpgallery}
%	\end{minipage}
%	\caption{Alignment impacts}
%	\label{ch04_fig_visual3}
%\end{figure}
%

%Values of CSs presented in the table are determined based on both generic definition of CSs as presented in Table \ref{ch03:cs} and specifically in the Table \ref{ch04:cs}.


%\begin{longtable}{|p{2.5cm}|p{8cm}|p{3cm}|}
%	\caption{Deliverable plan} \label{tab:long} \\
%	
%	\hline \multicolumn{1}{|l|}{\textbf{Items}} & \multicolumn{1}{l|}{\textbf{Task description}} & \multicolumn{1}{l|}{\textbf{Resources}} \\ \hline 
%	\endfirsthead
%	
%	\multicolumn{3}{c}%
%	{{\bfseries \tablename\ \thetable{} -- continued from previous page}} \\
%	\hline \multicolumn{1}{|l|}{\textbf{Items}} & \multicolumn{1}{l|}{\textbf{Task description}} & \multicolumn{1}{l|}{\textbf{Resources}} \\ \hline 
%	\endhead
%	\hline \multicolumn{3}{|r|}{{Continued on next page}} \\ \hline
%	\endfoot
%	\hline \hline
%	\endlastfoot
%	Phase 1 & Current state review &  \\ \hline 
%	Define Status Quo & Project Kick-off through a 3-hour kick off meeting with Maynilad team to review and revise important set of documents (e.g. project plan, objectives, workshop schedule, core team, existing prominent data) & GHD and Maynilad \\ \hline 
%	review & Review and analysis existing prominent data and knowledge, recommend a set of tests to be conducted to further identify the reliability and efficiency of equipment and facilities. & GHD \\ \hline 
%	Stakeholder Engagement Workshop 1 & Run the Workshop 1 to validate the Status Quo and define tasks for the next step (e.g. a concrete list of tests for equipment and facilities) & GHD with Maynilad participants \\ \hline 
%	Deliverable & Report, providing Status Quo, identify and highlight consolidated gaps and challenges & GHD with Maynilad review and approval \\ \hline 
%	Phase 2 & Database development for Data acquisition purposes, System description, and Condition State definition &  \\ \hline 
%	Select program & Define a suitable database program to be implemented for data collection and validation (e.g. PostgreSQL, MS access, or MS excel) & GHD \\ \hline 
%	Data acquisition & Import and migrate data in different formats (e.g. flat file, excel) to the selected database program & GHD \\ \hline 
%	Condition state definition & Identify system description and develop an appropriate set of condition states representing either physical condition and operational condition of items, components, sub-system, and system. & GHD \\ \hline 
%	Stakeholder Engagement Workshop 2 & Run the Workshop 2 to present up-to-date development status for data acquisition, system description, and condition state definition & GHD with Maynilad participants \\ \hline 
%	Deliverable & Report on data acquisition, system description and conditional state definition & GHD with Maynilad review and approval \\ \hline 
%	Phase 3 & System engineering and operational analysis (Modelling) &  \\ \hline 
%	Qualitative risk analysis & Study on risks that can be described as a combination of intensity and consequence. Perform interview to extract useful information learn from end users, engineers and line managers of Maynilad & GHD \\ \hline 
%	Quantitative risk analysis & Conduct reliability study on existing data (incl. estimation for failure rate, reliability, efficiency, availability, maintainability,  & GHD \\ \hline 
%	Operation study & Review and record various parameters on technical and financial operation of items, components, sub-system, and system (e.g. corrective and preventive intervention costs, energy consumption, labor consumption, spare parts) & GHD \\ \hline 
%	Stakeholder Engagement Workshop 3 & Run the Workshop 3 to present up-to-date development status for risk and reliability analysis and operation study & GHD with Maynilad participants \\ \hline 
%	Deliverable & Report on reliability study and operational study & GHD with Maynilad review and approval \\ \hline 
%	Phase 4 & Evaluation &  \\ \hline 
%	Life cycle cost analysis & Perform LCC analysis for items with different set of preventive intervention strategies & GHD \\ \hline 
%	Benchmarking & Benchmark for the optimal set of intervention strategies that yields the minimum LCC, whilst satisfying the requirements of Maynilad & GHD \\ \hline 
%	Consolidation & Consolidate the optimal intervention strategies to form the optimal preventive intervention program for 5 years implementation. & GHD \\ \hline 
%	Stakeholder Engagement Workshop 4 & Run the Workshop 4 to present the results of life cycle cost analysis and optimal preventive intervention program & GHD with Maynilad participants \\ \hline 
%	Deliverable & Report on life cycle cost analysis and optimal preventive intervention program for 5 years plan & GHD with Maynilad review and approval \\ \hline 
%	Phase 5 & Design &  \\ \hline 
%	Design & Perform 4 steps of detailed design for the purpose of procurement and installation & GHD \\ \hline 
%	Stakeholder Engagement Workshop 5 & Run the Workshop 5 to present the results of the detailed design works and conduct value engineering for selection of optimal design if required& GHD with Maynilad participants \\ \hline 
%	Deliverables & Calculation sheet, modelling, basic design report, draft and final versions of all reports and drawings associated with detailed design. & GHD with Maynilad review and approval \\ \hline 
%	Phase 6 & Tender package preparation &  \\ \hline 
%	Writing tender documents & Write and compile documents for tender package (e.g. instruction to bidders, scope of works, contract, program, functional guarantee) & GHD \\ \hline 
%	Stakeholder Engagement Workshop 6 & Run the Workshop 6 to present the content of the tender package & GHD with Maynilad participants \\ \hline 
%	Deliverable & Draft version and final version of the tender package  & GHD with Maynilad review and approval \\ \hline 
%	Finalization &  &  \\ \hline 
%	Stakeholder Engagement Workshop 7 & Run the Workshop 7 to transfer the knowledge to Maynilad team & GHD with Maynilad participants \\ \hline 
%	Deliverable & Report on knowledge transfer workshop  & GHD with Maynilad review and approval \\ \hline 
%	
%\end{longtable}

\subsection{Visual inspection data}
Visual inspection data on assets are summarized in tables of this section. %and also in the Appendix \ref{appvisualinspectionmech} with pictures.

%\paragraph{\textbf{BP1}}

\begin{table}[!htb]
	\caption{Visual inspection data - BP1 and BP2}
	\label{ch04_visualinspectionbp1}
%	\resizebox{\columnwidth}{!}{%
		{\scriptsize
\begin{tabular}{c|l|c|p{12cm}}
	\hline
	No. & Items & CS & Remarks \\ 
\hline
1 & SE & 1 & Elbow surface is good, paint is consistent \\ 
2 & SBV & 1 & Mitered elbows were used (not radius elbows) \\ 
3 & STR & 1 & No leaks, cracks, deterioration or corrosion  \\ 
4 & AV & 1 & Air valve installed to release trapped air during operation \\ 
5 & CS1 & 2 & Cracks propagating, very small gap observed between pipe and saddle rest \\ 
6 & CCR & 3 & Should be replaced with eccentric reducer to avoid formation of air pockets \\ 
7 & EJ & 2 & Check locking nuts on outer and inner stud bolts, some are bolted on both sides of plate making joint fixed \\ 
8 & CCD & 1 & Used as necessary \\ 
9 & EJ & 2 & Check locking nuts on outer and inner stud bolts, some are bolted on both sides of plate making joint fixed \\ 
10 & CV & 1 & Used as necessary, almost no leaks \\ 
11 & CS2 & 3 & May need redesign as it holds valve fixed \\ 
12 & DBV & 1 & Used as necessary \\ 
13 & DE & 2 & Have portions where surface paint is brittle and some areas where bare metal is exposed to corrosion \\ 
\hline
\end{tabular}
	}
\end{table}


\begin{table}[!h]
	\caption{Visual inspection data - BP3 and BP4}
	\label{ch04_visualinspectionbp4}
%	\resizebox{\columnwidth}{!}{%
		{\scriptsize
\begin{tabular}{c|l|c|p{12cm}}
\hline
No. & Items & CS & Remarks \\ 
\hline
1 & SE & 2 & Exposure to outdoor conditions cause pipe coating to deteriorate; protective coat torn with cracks and finishing paint brittle; bare metal exposed on some portions and already showing corrosion \\ 
2 & SBV & 1 & Mitered elbows were used (not radius elbows) \\ 
3 & STR & 1 & No leaks, cracks, deterioration or corrosion  \\ 
4 & AV & 1 & Air valve installed to release trapped air during operation \\ 
5 & CS1 & 2 & Cracks propagating, very small gap observed between pipe and saddle rest \\ 
6 & CCR & 3 & Should be replaced with eccentric reducer to avoid formation of air pockets \\ 
7 & EJ & 2 & Check locking nuts on outer and inner stud bolts, some are bolted on both sides of plate making joint fixed \\ 
8 & CCD & 1 & Used as necessary \\ 
9 & EJ & 2 & Check locking nuts on outer and inner stud bolts, some are bolted on both sides of plate making joint fixed \\ 
10 & CV & 1 & Used as necessary, almost no leaks \\ 
11 & CS2 & 3 & May need redesign as it holds valve fixed \\ 
12 & DBV & 1 & Used as necessary \\ 
13 & DE & 2 & Have portions where surface paint is brittle and some areas where bare metal is exposed to corrosion \\ 
\hline
\end{tabular}
	}
\end{table}

\begin{table}[!h]
	\caption{Visual inspection data - BP5 and BP6}
	\label{ch04_visualinspectionbp5}
%	\resizebox{\columnwidth}{!}{%
		{\scriptsize
\begin{tabular}{c|l|c|p{12cm}}
\hline
No. & Items & CS & Remarks \\ 
\hline
1 & SE & 2 & Exposure to outdoor conditions cause pipe coating to deteriorate; protective coat torn with cracks and finishing paint brittle; bare metal exposed on some portions and already showing corrosion \\ 
2 & SBV & 1 & Mitered elbows were used (not radius elbows) \\ 
3 & STR & 1 & Not yet in use \\ 
4 & AV & 1 & Services empty pipe \\ 
5 & CS1 & 1 & Supports empty pipe \\ 
\hline
\end{tabular}
	}
\end{table}




\begin{table}[!h]
	\caption{Visual inspection data - SP1 and SP2}
	\label{ch04_visualinspectionsp1}
%	\resizebox{\columnwidth}{!}{%
		{\scriptsize
\begin{tabular}{c|l|c|p{12cm}}
\hline
No. & Items & CS & Remarks \\ 
\hline
1 & SE & 2 & Could be replaced with a Y-type branch take off but with space constraints a replacement of long radius type might not be physically possible \\ 
2 & SBV & 1 & Used as necessary \\ 
3 & EJ & 2 & Check locking nuts on outer and inner stud bolts, some are bolted on both sides of plate making joint fixed \\ 
4 & CCR & 1 & Used as necessary \\ 
5 & CCD & 1 & Used as necessary \\ 
6 & EJ & 2 & Check locking nuts on outer and inner stud bolts, some are bolted on both sides of plate making joint fixed \\ 
7 & CV & 1 & Used as necessary \\ 
8 & AV & 1 & Used as necessary \\ 
9 & DBV & 1 & Used as necessary \\ 
10 & DE & 1 & Follows 45 degree entry into discharge header which better facilitates water flow \\ 
11 & CS & 3 & May need redesign as it holds valve fixed \\ 
\hline
\end{tabular}
	}
\end{table}



\subsection{Observation and recommendations}
\begin{itemize}
\item Pumps have little to no isolation from hydrodynamic forces. Strain due to movement at the inlet and outlet pipe sections are directly transmitted to pump casings. Hydrodynamic forces, along with the static weight of said pipe sections, including the weights of the flexible couplings and the water contained therein, are transmitted to the pump casing due to lack of proper piping support upstream or downstream of the pumps.


\paragraph{\underline{Recommendations}}
\begin{itemize}
	\item [$\checkmark$] Install proper isolation and support of pipe sections upstream and downstream of the pumps. 
%	\item[$\checkmark$]  Medium term: install properly engineered piping supports, equipped with anti-corrosion shields; install cathodic protection on the pipelines.
\end{itemize}

\item The welded tubular frame supports employed in this station are less rigid than traditional base plates. Such construction results in mounting surfaces that are not necessarily coplanar or parallel to each other. Moreover, the vertical design raises the centerline of rotation relative to the floor line, leading to a less stable installation. Thus, PSR motor-pump units are deemed vulnerable to misalignment “creep” during operation.

\paragraph{\underline{Recommendations}}
\begin{itemize}
	\item [$\checkmark$] Conduct regular vibration analysis to assess extent of misalignment and/or its effects on the bearings. 
\end{itemize}

\end{itemize}




%Aside from the CSs observed as presented in the table, we also learn from the inspection that the entire pump house has been suffering from ground settlement and that the alignments of the pumps and the corresponding fittings have been compromised. As shown in Figure \ref{ch04_settlement}, the portion of wall through which the pipe passes thru has been torn down partially to provide allowance for settlement as the pipe levels down.  

%\begin{figure}[!htb]
%	\includegraphics[scale=1.5]{figures/ch04_settlement} \\
%	\caption{Wall opening provision for ground settling}
%	\label{ch04_settlement} 
%\end{figure}

%Damages to the concrete saddle supports appearing as cracks and crooked or completely drawn bolts have been observed as shown in Figure \ref{ch04_settlement01}. 
%
%\begin{figure}[!htb]
%	\includegraphics[scale=1.5]{figures/ch04_settlement01} \\
%	\caption{Support damages due to pipe movement and ground settling}
%	\label{ch04_settlement01} 
%\end{figure}
%
%Furthermore, gaps are also observed between the concrete saddle supports and the straight pipe beneath as shown in Figure \ref{ch04_settlement02}.  
%\begin{figure}[!htb]
%	\includegraphics[scale=1.5]{figures/ch04_settlement02} \\
%	\caption{Support damages due to pipe movement and ground settling}
%	\label{ch04_settlement02} 
%\end{figure}
%
%For the booster pumps, there are two saddle supports located near to each other and are positioned just outside the pump house after the suction elbows as shown in Figure \ref{ch04_piping_supports01}. For the storage pumps, the three saddles are located farther, one before the discharge elbow and   the other still outside the pump house and just after the check valve.
%
%\begin{figure}[!htb]
%	\includegraphics[scale=1.5]{figures/ch04_piping_supports01} \\
%	\caption{Support damages due to pipe movement and ground settling}
%	\label{ch04_piping_supports01} 
%\end{figure}
%\end{document}