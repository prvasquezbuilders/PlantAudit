\section{Pump efficiency} \label{ch04mech03}\section{Pump efficiency} \label{ch04mech03}
Data on flow and head were measured for each pump. However, there was no measured data of motor/pump assembly regarding power ratings of all pumps 
as of this writing since the GHD/APSI testing for the station is yet to be conducted. 
Thus, as agreed with Maynilad Team, data from control panel specifically VFDs' current and voltage ratings will then be used to get the instantaneous power ratings of Pump and Motor during the course of operation.
However, such values might not be equivalent to the values of power during the time of testing thus might affect the result of the analysis.

\subsection{Unit flow measurement} \label{ch04mech04}
Data on measured flow (Q) was recorded with min and max values is shown in Table \ref{ch04_tbl_flow01} for each pumps. %Raw data is provided in the Appendix \ref{appflow}.

\begin{table}[!h]
	\caption{Unit flow measurement (cubic meter per second - cms).}
	\label{ch04_tbl_flow01}
	{\footnotesize
\begin{tabular}{c|c|c|c|c|c|l}
	\hline
Assets & $\Phi$  & \multicolumn{4}{c|}{Flow Q (cms)} & \multicolumn{1}{c}{Remarks} \\ 
\cline{3-6}
 & (mm) & Design & \multicolumn{3}{c|}{Measure} & \multicolumn{1}{c}{} \\ 
\cline{4-6}
 &  &  & Min & Max & Ave. & \multicolumn{1}{c}{} \\ 
\hline
BP1 & 700 & 0.6713 & 0.4858 & 0.4984 & 0.4921 & \multicolumn{1}{c}{} \\ 
BP2 & 700 & 0.6713 & 0.4858 & 0.4984 & 0.4921 & \multicolumn{1}{c}{} \\ 
BP4 & 700 & 0.6713 & 0.5110 & 0.5615 & 0.5363 & \multicolumn{1}{c}{} \\ 
SP1 & 500 & 0.2315 & 0.3584 & 0.3659 & 0.3622 & \multicolumn{1}{c}{} \\ 
SP2 & 500 & 0.2315 & 0.3584 & 0.3659 & 0.3622 & \multicolumn{1}{c}{} \\ 
\hline
\end{tabular}

	}
\end{table}

\subsection{Pressure measurement} \label{ch04pressure}

Data on measured flow (Q) was recorded with min and max values is shown in Table \ref{ch04_tbl_flow02} for each pumps. %Raw data is provided in the Appendix \ref{appflow}.

\begin{table}[!h]
	\caption{Head ($mH_2O$).}
	\label{ch04_tbl_flow02}
	{\footnotesize
\begin{tabular}{c|c|c|c|l}
\hline
Assets & \multicolumn{3}{c|}{Head (H - $mH_2O$)} & Remarks \\ 
\cline{2-4}
 & Design & Discharge & Suction &  \\ 
\hline
BP1 & 50 & 47.0766 & -3.4429 &  \\ 
BP2 & 50 & 48.4819 & -3.0916 &  \\ 
BP4 & 50 & 49.1845 & -3.0916 &  \\ 
SP1 & 60 & 48.4819 & 0.5621 &  \\ 
SP2 & 60 & 48.4819 & 0 &  \\ 
\hline
\end{tabular}

	}
\end{table}


\subsection{Efficiency}

Pump efficiency is computed based on the flow/head measurement and the assumed value of power rating (Table \ref{ch05_tbl_efficiency}). 
\begin{table}[!h]
	\caption{Pump efficiency (\%).}
	\label{ch05_tbl_efficiency}
	{\footnotesize

		
	\begin{tabular}{c|c|c|c|c|c|l|l}
\hline
Assets & Flow & Head & Input Power & Water Power & \multicolumn{3}{c}{Efficiency (\%)} \\ 
\cline{6-8}
 & $m^3/s$ & $mH_2O$ & kW & kW & Tested & Design & Diff. \\ 
\hline
BP1 & 0.492 & 51.581 & 375.5 & 249 & 66.3 & \multicolumn{1}{c|}{80} & \multicolumn{1}{c}{-13.7} \\ 
BP2 & 0.492 & 52.635 & 394.5 & 254 & 64 & \multicolumn{1}{c|}{80} & \multicolumn{1}{c}{-16} \\ 
BP4 & 0.536 & 53.338 & 394.5 & 280 & 71 & \multicolumn{1}{c|}{80} & \multicolumn{1}{c}{-9} \\ 
SP1 & 0.362 & 50.364 & 129.1 & 179 & - & \multicolumn{1}{c|}{83.2} & \multicolumn{1}{c}{-} \\ 
SP2 & 0.362 & 50.926 & 129.1 & 181 & - & \multicolumn{1}{c|}{83.2} & \multicolumn{1}{c}{-} \\ 
\hline
\end{tabular}


	}
\end{table}

It is important to note that the values of input power is a measured value, however on a separate day and so different working condition thus might not perfectly reflect 
the actual value during the data gathering. This case is one of the limitations of the study.
The design pump efficiencies are taken from the test report of Kubota Corporation \cite{Kubota2010}. 

In Table \ref{ch05_tbl_efficiency}, efficiencies for SP1 and SP2 were not estimated due to the fact that the measured flow are beyond the maximum allowable flow of the pumps. 
These might infers that the pumps have been operated inappropriately pursuant to their design specifications, 
a case that could be associated to VFD operated pumps, aside from the fact that the viable measuring points of the flow is near the elbow at the discharge side. 
With that, the operation of the pumps might not be the optimum and as a result incurred more power than they should be. 

Figure \ref{ch04_efficiencycurves}-a, and \ref{ch04_efficiencycurves}-b presents the efficiency curves for booster pumps and storage pumps, respectively. The curves are created based on the recorded data provided in the test record of Kubota Corporation \cite{Kubota2010}. 


\begin{figure}[!htb]
	\begin{minipage}[b]{0.5\linewidth}
		\centering
		\includegraphics[width=\textwidth]{figures/ch04_fig_efficiency01}
		\caption*{a - Booster pumps}% \label{peter1}
	\end{minipage}
	\hspace{0.05cm}
	\begin{minipage}[b]{0.5\linewidth}
		\centering
		\includegraphics[width=\textwidth]{figures/ch04_fig_efficiency02}
		\caption*{b -Storage Pumps} %\label{peter2}
	\end{minipage}
		\caption{Efficiency curves}
		\label{ch04_efficiencycurves}
\end{figure}

Figure \ref{ch04_efficiencycurves}-a 

As shown in Figure \ref{ch04_efficiencycurves}-b, the storage pumps operate at an overflow condition and are beyond the tolerance of 20\% Best Efficiency Point (BEP). Both pumps only run at 1015 rpm at the time of testing, but delivers an overflow and so deviate greatly from the curve. In this operating conditions, the pump might incurs higher power consumption aside from the increase of deterioration rate of its components. Specifically, it will produce high cavitation and high vibration thus associated to the so called operate-to-destruction. This is similar to underflow as it will increase the temperature rise rate and produce noisy operations, cavitation surge and even high vibration scenarios.  

%In addition to these, the results of vibration analysis indicates that both pumps have considerable magnitude of vibration and could incur greater damage if not corrected. Both motors have found out to have early-stage DE bearing defect due to its lubrication; either the bearing lubrication is not maintained/checked regularly or insufficient lubrication while operating.

%Generally, VFD controlled-pumps deviates greatly away to the curve than the fixed-speed pumps (BP3 to BP6). This could possibly means that the operating duty points of the pumps when controlled by VFD is out of the best efficiency range of the pumps, thereby may incur higher power consumptions. 

\paragraph{\underline{Recommendations}}

\begin{itemize}
\item	Proper operating condition shall be established for the VFD pumps in a given time to avoid them operating beyond their BEP for a long period of time. Also include optimization of the proper combinations of running pumps in a given time.
\item	Regular pump performance monitoring and analysis shall be conducted to have a profiling of pump in relation to actual operating conditions, a major tool for pump performance optimization. In addition, multiple tests and conditions shall be implemented during testing to have a holistic approach on the pump assessment.
\item	Provision for installation of testing points in measuring pump performance, such as thermodynamic method, shall also be included in the modifications  (refer to the conceptual design in Chapter \ref{Chapter6}).
\end{itemize}

