%\begin{document}

\section{Qualitative and Operational Analysis}
\label{42}
\subsection{Facts and Data}
Summary of facts and data concerning operational and overall plan reliability is presented in this subsection.

\subsubsection{Normal Operation Scenario}
\begin{itemize}
\item 3 Booster pumps out of 4 are in operation to meet the 80psi discharge pressure;
 
\item The 2 Supply pumps are not in operation.
\end{itemize}

\subsubsection{High Demand Scenario}

\begin{itemize}
	\item If the pressure drops below 80 psi, one of the Supply Pumps is started to increase the discharge pressure above 80psi;
	
	\item Note that the current supply line pressure is at 4psi and future increase in demand (population growth), this will continue to drop causing more issues to the booster pumps.
\end{itemize}

\subsubsection{Low Demand Scenario}

\begin{itemize}
	\item The one of the supply pumps is used to refill the supply tanks from 12mn until 5am;

\item The pressure of the main supply line is only at 4psi and is not high enough to refill the supply tanks through the MOVs.

\end{itemize}

\subsubsection{Spares Policy}

\begin{itemize}
\item Booster pumps: There is no fixed schedule in switching the duty and the spare pumps. By “feel”, the operators rotate the pumps as they see fit;

\item Storage pumps: SP1 and SP2 are switched daily during the filling-up of the supply tanks.

	
\end{itemize}

\subsubsection{Emergency Situation (loss of electrical power from Meralco)}
\begin{itemize}
	\item As soon as the power is cut, one of the 2 Gensets automatically start and makes electric power available for the booster and/or supply pumps. The operator resets all the pumps before restarting each manually. This may take between 60 seconds to 5 minutes before all the pumps provide enough pressure/flow to the system.
\end{itemize}

\subsubsection{Maintenance}
\begin{itemize}
	\item There is no structured maintenance program in the facilities. The operator makes rounds of the pumps and if something unusual is observed, a text or email is sent to the Control Center for scheduling of maintenance check and repairs;
	
	\item There is also a maintenance team visiting the site regularly with specific lists of tasks and responsibilities. 
	
\end{itemize}

\subsubsection{Others}
\begin{itemize}
	\item BP3 is has just gone overhauling with impeller replacement as well as replacement of a damaged bearing. It was reported that this is the second bearing failure in 4 years.
\end{itemize}

\subsection{Recommendations}
In order to ensure the PS to provide adequate level of services around the clock, it is important to establish a good operational scheme that allows optimization of use of pumps to reduce breakdown and to conserve energy. A summary of major recommendations to be considered are

\begin{itemize}

\item Consider an additional booster pump to improve the reliability of the facility. With one pump undergoing major repairs, the arrangement does not allow for another pump to fail, otherwise, there will be a major loss of pressure with only 2 booster pumps running. The use of the other 2 storage pumps can only be done on short-term arrangement because the storage tank has a limited capacity plus the fact that during peak loads, the 4 pumps (2 BP + 2 SP) is not enough to sustain at least 80psi in the discharge line; 

\item Consider a dedicated duty and a dedicated spare set-up for the pumps. If this is not acceptable, then consider doing a much longer switch of the storage pumps. Currently, it is being switched daily to supply the storage tank between midnight and 5am. This allows for almost an equal rate of deterioration between the two pumps and if one pump fails due to age-related component failure, the other one is close to a similar failure which may occur before the first pump is fully repaired. It is suggested that the switch happen once a month or even every 3 months; 

\item In place of the longer switching cycle (e.g. every 3 months), there should be a corresponding maintenance program for the standby pump for both booster and storage; 

\item Develop a more structured discipline in applying routine maintenance work process to ensure that maintenance tasks are given the proper priority in terms of mitigation measures and avoid unplanned shutdown of critical pumps in operation. 

\end{itemize}

Aside from the above recommendations, we also generate a list of recommendations based on the RCM methodology (refer to Appendix \ref{app_maintenance}). The list shall be considered as a living program, which requires continuously improvement as part of the total quality management system (refer to Deming cycle presented in GHD's technical proposal).

%\end{document}