% Chapter 1

\chapter{Introduction} % Write in your own chapter title
\label{Chapter1}
%\lhead{Chapter 1. \emph{Introduction}} % Write in your own chapter title to set the page header
%%%

\section{General introduction} \label{namdario}
The station is located in Villamor area (refer to Figure \ref{pagcorlocation}). It has a total capacity of 676,266 $m^3/day$ and pumps the treated water to a discharge line serving the an area in Villamor, Metro Manila. The schematic diagram showing its connectivity is in Figure \ref{ch01_keyplan}.


\begin{figure}[!htb]
	\begin{minipage}[b]{0.5\linewidth}
		\centering
		\includegraphics[width=\textwidth]{figures/pslocation}
		\caption{Pagcor PS [$14^{\circ}31^{'}40.74^{''}N, 121^{\circ}1^{'}23.10^{''}E$]}
		\label{pagcorlocation}
	\end{minipage}
	\hspace{0.05cm}
	\begin{minipage}[b]{0.5\linewidth}
		\centering
		\includegraphics[width=\textwidth]{figures/ch01_pumpgallery}
		\caption{Pump gallery}
		\label{ch01_pumpgallery}
	\end{minipage}
\end{figure}

\begin{figure}[h]
	%	\begin{center}
	\includegraphics[scale=2.1]{figures/ch04_schematicdiagram} 
	%	\end{center}
	\caption{Schematic diagram}
	\label{ch01_keyplan}
\end{figure}

%\begin{table}[h]
%\caption{Major assets of the PS.}
%\label{assetlist01}
%{\footnotesize
%\begin{tabular}{|l|l|l|l|l|}
%	\hline
%	\multicolumn{1}{|c|}{No.} & Major Asset & Abbre. & Capacity & Unit  \\ 
%	\hline
%	\multicolumn{1}{|c|}{1} & Storage Pump 1 & SP1 &  &   \\ 
%	\hline
%	\multicolumn{1}{|c|}{2} & Storage Pump 2 & SP2 &  &   \\ 
%	\hline
%	\multicolumn{1}{|c|}{3} & Booster Pump 1 & BP1 &  &   \\ 
%	\hline
%	\multicolumn{1}{|c|}{4} & Booster Pump 2 & BP2 &  &   \\ 
%	\hline
%	\multicolumn{1}{|c|}{5} & Booster Pump 3 & BP3 &  &   \\ 
%	\hline
%	\multicolumn{1}{|c|}{6} & Booster Pump 4 & BP4 &  &   \\ 
%	\hline
%	\multicolumn{1}{|c|}{7} & Booster Pump 5 & BP5 &  &    \\ 
%	\hline
%	\multicolumn{1}{|c|}{8} & Booster Pump 6 & BP6 &  &    \\ 
%	\hline
%		\multicolumn{1}{|c|}{9} & SP1 motor & SP1-motor &  &    \\ 
%	\hline
%			\multicolumn{1}{|c|}{10} & SP2 motor & SP2-motor &  &    \\ 
%	\hline
%			\multicolumn{1}{|c|}{11} & BP1 motor & BP1-motor &  &    \\ 
%	\hline
%			\multicolumn{1}{|c|}{12} & BP2 motor & BP2-motor &  &    \\ 
%	\hline
%			\multicolumn{1}{|c|}{13} & BP3 motor & BP3-motor &  &    \\ 
%\hline
%			\multicolumn{1}{|c|}{14} & BP4 motor & BP4-motor &  &    \\ 
%\hline
%			\multicolumn{1}{|c|}{15} & BP5 motor & BP5-motor &  &    \\ 
%\hline
%			\multicolumn{1}{|c|}{16} & BP6 motor & BP6-motor &  &    \\ 
%\hline
%	\multicolumn{1}{|c|}{9} & GENSET &  &  &   \\ 
%	\hline
%	\multicolumn{1}{|c|}{10} & Transformer &  &  &   \\ 
%	\hline
%	\multicolumn{1}{|c|}{11} & MCC &  &  &    \\ 
%	\hline
%\end{tabular}
%		}
%\end{table}

This PS has been reported by the Client to experience considerable degradation with regard to inefficiency of operation, higher power consumption, and more frequent breakdowns of assets leading to the increase in maintainability and decrease in availability. 

The Client has therefore awarded GHD and its sub-consultants (RB Sanchez and APSI) to conduct a plant audit project with an expectation to establish rigorous asset management framework based on reliability study and to determine optimal intervention program for the next five (5) years.

%%%
\section{Objectives}
%%%%%%%%%%%
The objectives of this work are as follows
\begin{itemize}
	\item To evaluate the current operating condition of PS as compared to the original design intent and to provide recommendations for improving the operational efficiency and lowering operating cost;
	\item To be able to determine an optimal intervention program for the PS in the next 5 years with reference to the recommendations from the assessment and audit based on life cycle cost; and equipment efficiency study whether the equipment is subjected to replacement or repair. These equipment are:
	\begin{itemize}
		\item[$\circ$] Pumps;
		\item[$\circ$] Motors;
		\item[$\circ$] Generators;
		\item[$\circ$] Electrical System and Protective Device;
		\item[$\circ$] Substation (Transformer, Switchgears);
		\item[$\circ$] MCC (VFDs, Soft starters, Circuit Breakers, and Protective Devices);
		\item[$\circ$] Motorize Valves.	
	\end{itemize}
\end{itemize}

\section{Scope of Work}
Scope of Work (SOW) has been defined in the Contract Agreement and be in compliance with the GHD technical and financial proposal and the agreements made during a number of project meetings (refer to minutes of meeting of the project). 

{\color{red}
It is important to note that the electrical audit and analysis is not part of this report as the test has been deferred due to operational constraint. Maynilad will decide to give a specific time window to conduct electrical test. Thus, this report will be updated to next revision once the electrical tests are completed. 
}


\section{Limitations}
Results of the study with analysis, conclusion, and recommendations are only within the scope of work and agreements, and particularly under the following major constraints:
\begin{itemize}
\item Operational constraint: It was not possible to shutdown the entire PS for visual inspection of assets, particularly mechanical assets;
\item Incomplete historical data: It was a matter of fact that Maynilad has not established an asset management system, thus data regarding historical intervention is limited and incomplete, leading to non-optimal reliability analysis; 
\end{itemize}



\section{Glossaries}
Following glossaries are defined and used in the report:

\paragraph{\textbf{Level of Services (LOS)}}
A Level of Services (LOS) is any value or expectation of asset managers and beneficiers regarding the functionality and serviability of an asset of a system of assets.

\paragraph{\textbf{Intervention}}
Intervention is a generic and global term used to refer to non-physical and physical activities on assets. It encompasses do-nothing, or do somethings like repair, maintenance, rehabilitation, renewal, investment, and inspection and testing.

\paragraph{\textbf{Corrective Intervention (CI)}}
A Corrective Intervention (CI) is an intervention executed without proper and systematic plan. An CI is often incurred by failure/breakdown of assets. In most of cases, it incurs significant negative impacts (e.g. cost to repair, disruption of service, loss in revenue).

\paragraph{\textbf{Preventive Intervention (PI)}}
A Preventive Intervention (PI) is an intervention executed with proper and systematic plan. Note that an PI is executed on asset that is still working but not provide adequate level of services.

\paragraph{\textbf{Intervention Type}}
An Intervention Type (IT) is a specific and well-defined type of work/task that can be executed on/for an asset (e.g. replacement of a bearing for a pump).

\paragraph{\textbf{Intervention Strategy (IS)}}
An Intervention Strategy (IS) is a set/collection of intervention types.

\paragraph{\textbf{Intervention Program (IP)}}
An Intervention Program (IP) is a set/collection of intervention strategies for one asset or more than one assets of the same system.

\paragraph{\textbf{Work Program (WP)}}
A Work Program (WP) is an execution program consisting of Intervention Program and management program (e.g. project management, procurement) that shall be implemented in order to realize/actualize the Intervention Program.


